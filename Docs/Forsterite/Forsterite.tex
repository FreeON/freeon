
%   This file is part of the APS files in the REVTeX 4 distribution.
%   Version 4.0 of REVTeX, August 2001
%
%   Copyright (c) 2001 The American Physical Society.
%
%   See the REVTeX 4 README file for restrictions and more information.
%
% TeX'ing this file requires that you have AMS-LaTeX 2.0 installed
% as well as the rest of the prerequisites for REVTeX 4.0
%
% See the REVTeX 4 README file
% It also requires running BibTeX. The commands are as follows:
%
%  1)  latex apssamp.tex
%  2)  bibtex apssamp
%  3)  latex apssamp.tex
%  4)  latex apssamp.tex
%
%\documentclass[prb,aps,nobibnotes,twocolumn,doublespace,twocolumngrid,superbib]{revtex4}
%%\documentclass[twocolumn,showpacs,preprintnumbers,amsmath,amssymb]{revtex4}
%\documentclass[preprint,showpacs,preprintnumbers,amsmath,amssymb]{revtex4}

% Some other (several out of many) possibilities
%\documentclass[preprint,aps]{revtex4}
%\documentclass[preprint,aps,draft]{revtex4}
%\documentclass[prb]{revtex4}% Physical Review B

%%\usepackage{amsmath}
%%\usepackage{amssymb}
%%\usepackage{graphicx}% Include figure files
%%\usepackage{dcolumn}% Align table columns on decimal point
%%\usepackage{bm}% bold math

%\documentclass[pre,aps,twocolumn,showpacs,twocolumngrid,superbib]{revtex4}
%\documentclass[prl,aps,twocolumn,showkeys,twocolumngrid,superbib]{revtex4}
%\documentclass[twocolumn,showkeys,showpacs,preprintnumbers,amsmath,amssymb]{revtex4}
\documentclass[pra,twocolumn,twocolumngrid,superbib]{revtex4} %<<< double
%\documentclass[prb,onecolumn,twocolumngrid,superbib]{revtex4}
%\documentclass[prl,preprint,doublespace]{revtex4} %<<< For the journal
%%%\documentclass[prl,showpacs,preprint,superbib]{revtex4}
%\documentclass[showpacs,preprint,superbib]{revtex4}

\usepackage{graphicx}
\usepackage{amsfonts}
\usepackage{amsmath}
\usepackage{bm}
\usepackage{alltt}
\usepackage{fancyhdr}
\usepackage{dcolumn} 

\pagestyle{fancy}

\newcommand{\commentoutB}[1]{#1}
\newcommand{\commentoutA}[1]{}

%\newcommand{\commentoutA}[1]{#1}
%\newcommand{\commentoutB}[1]{}

%\nofiles

\begin{document}

%\preprint{APS/123-QED}
%%%%%%%%%%%%%%%%%%%%%%%%%%%%%%%%%%%%%%%%%%%%%%%%%%%%%%%%%%%%%%%%%%%%%%%%%%%
%Optical Properties of Photonic Crystals
%Solid State Spectroscopies: Basic Principles and Applications
%Solid-State Spectroscopy: An Introduction
%%
%Basic Solid State Chemistry
%Solid State Chemistry
%Structure and Chemistry of Crystalline Solids
%%
%http://www.psusse.de/minabs/newabs/rgr.html
% Uniform ASCII symbols for space groups, point groups, and crystal systems.  by P. SUSSE, Poster presented at the 17th General Meeting of the International Mineralogical Association, Toronto, Canada, Aug. 9 - 14, 1998. s. Abstracts  page A62.
%
%
% TO SEE TO SEE TO SEE TO SEE TO SEE TO SEE TO SEE TO SEE TO SEE TO SEE 
% http://www.vlab.msi.umn.edu/reports/allpublications/
% ``Vibrational and thermodynamic properties of MgSiO3 post-perovskite'', 
% J. Tsuchiya, T. Tsuchiya, and R. M. Wentzcovitch, J. Geophys. Res., 110 (B2), B02204/1-6 (2005). 
% ``Vibrational and thermodynamic properties of forsterite'', 
% Li Li, R. M. Wentzcovitch, D. Weidner, C. R. S. da Silva, J. Geophys. Res., submitted (2006). 
%
%%%%%%%%%%%%%%%%%%%%%%%%%%%%%%%%%%%%%%%%%%%%%%%%%%%%%%%%%%%%%%%%%%%%%%%%%%%
%
%
\title{ Influence of density functional models in calculation of geometrical and 
        elastic parameters of forsterite within the $\Gamma$-point approximation }

\author{Val\'ery Weber}
\email{valeryw@lanl.gov}
\author{Matt Challacombe}%
\affiliation{Los Alamos National Laboratory, Theoretical Division, Los Alamos 87545, New Mexico, USA.}%

\date{\today}% It is always \today, today,
             %  but any date may be explicitly specified


\begin{abstract}
The application of theoretical methods based on density-functional 
theory is known to provide atomic and cell parameters in very good 
agreement with experimental values. Recently, construction of the 
hybrid density functional gradients with respect to atomic positions and cell parameters
within the $\Gamma$-point approximation has been introduced 
[V. Weber et al., J. Chem. Phys. {\bf 124}, 214105, (2006) and 
 V. Weber et al., J. Chem. Phys. {\bf 124}, 224107, (2006)]. 
In this article, the analytical gradients have been used in conjunction with the
QUICCA algorithm [K. N\'emeth and M. Challacombe, J. Chem. Phys. {\bf 121}, 2877, (2004)]
to compute the geometrical parameters and the elastic constants of orthorhombic forsterite 
at the density-functional theory and hybrid Hartree-Fock/density-functional theory levels. 
The influence of the DFT functionals on the calculated atomic positions, cell parameters and
elastic constants is also discussed.
\end{abstract}

%\pacs{Valid PACS appear here}% PACS, the Physics and Astronomy
                             % Classification Scheme.
\keywords{Periodic boundary condition, elastic constants, density functional theory, $\Gamma$-point approximation.}
%display desired

\maketitle

%\footnotetext[1]{Preprint LA-UR-05-3120.}

\section{Introduction}
The Kohn-Sham approach to density functional theory has proven to be a highly
competitive method for a wide range of applications in solid
state physics and chemistry.
The hybrid Hartree-Fock/density functional theory (hybrid-HF/DFT) model chemistries
are an important next step in accuracy beyond the generalized gradient
approximation~\cite{Gill92,Becke93,VBarone96,CAdamo99}.

In preceding papers, we have developed linear scaling quantum chemical methods
for construction of the periodic Coulomb, exchange-correlation~\cite{CTymczak04a}
and the exact Hartree-Fock exchange~\cite{CTymczak04b}
matrices within the $\Gamma$-point approximation.
In this paper, the analytical gradients have been used in conjunction with the
QUICCA algorithm~\cite{KNemeth04} to compute the geometrical parameters and the 
elastic constants of orthorhombic forsterite at the density-functional theory 
and hybrid Hartree-Fock/density-functional theory levels.

Finding crystal structures of condensed systems can
be formulated as a minimization of the total energy
with respect to atomic coordinates and cell vectors.
The problem is then minimization of the total energy with $L$ degrees of freedom, where
$L=3N_{atm}+3$, $N_{atm}$ is the number of atoms, $3N_{atm}-3$ is the number
of independent coordinates after the elimination of translation,
and the number of independent vector elements
after the elimination of cell rotations is 6.
This minimization can be achieved with the help of an
efficient optimizer~\cite{KNemeth04,TBucko05,KNemeth05}
and knowledge of the gradients with respect to atomic
positions and cell parameters.

The magnesium orthosilicate (Mg$_2$SiO$_4$, forsterite) is one of the main
constituent of the middle and lower crust as well as the upper mantle of 
the Earth~\cite{DLAnderson98}. Forsterite has been the subject of 
many experimental studies on crystallographic, thermodynamic and 
spectroscopic properties. First-principle investigations have been performed
on the structural, elastic~\cite{PJochym04,JBrodholt96,CSilva97} 
and vibrational spectra~\cite{YNoel06} of forsterite.
%
%JBrodholt96 Brodholt, J; Patel, A; Refson, K
%AMERICAN MINERALOGIST; JAN-FEB 1996; v.81, no.1-2, p.257-260
%CSilva97 Da Silva, C; Stixrude, L; Wentzcovitch, RM
%Geophysical Research Letters; 1 Aug. 1997; vol.24, no.15, p.1963-6
%
The elastic tensor determines important geophysical properties
of the Earth's interior such as: seismic velocity, Young's and shear moduli, 
Poisson's ratio... These properties play an important role in the interpretation
of sesmic data.

In this paper, the geometry and elastic constants of forsterite 
have been computed for each Hamiltonian
considered (HF, PW91~\cite{JPerdew92}, PBE0~\cite{Perdew_96v77,CAdamo99} 
and B3LYP~\cite{ABecke93,Stephens94}). 
Inner coordinates and cell parameters have been optimized
within an iterative procedure based on the total
energy gradients calculated analytically with respect to the nuclear
coordinates and cell parameters. 
The elastic parameters are determined by appling strains and calculating the
resulting stress tensor. Because strains couple to vibrational modes, the atomic
positions are re-optimized in the deformed lattice.

%... orthorhombic magnesium orthosilicate (Mg$_2$SiO$_4$, forsterite)
%was optimized at the B3LYP~\cite{ABecke93,Stephens94} level of theory.

The remainder of this paper is organized as follows:
In Section~\ref{Sec:ComputMethods}, we summarize briefly the method employed for the
calculation of the elastic tensor, Section~\ref{Sec:ResultsAndDiscusions}
presents our calculated data and compare the present results with previous theoretical and
experimental data. Finally in Section~\ref{Sec:Conclusions} we summarize our results.

\section{Computational Methods}\label{Sec:ComputMethods}

\subsection{How to get the elastic parameters}

The elastic parameters have been obtained by a simple finite deformation approach.
The system of inhomogenous linear equations is solved by the singular value decomposition.

A Taylor expansion of the energy of the unit cell to second order in the strain components
yields
\begin{equation}
  E(\varepsilon)=E(0)+\sum_{i}\frac{\partial E}{\partial \varepsilon_{i}}
  \bigg|_{0}\varepsilon_{i}
  +\frac{1}{2}\sum_{ij}\frac{\partial^2 E}{\partial \varepsilon_{i}\partial \varepsilon_{j}}
  \bigg|_{0}\varepsilon_{i}\varepsilon_{j}+\cdots
\end{equation}

If $E(0)$ refers to the equilibrium conguration the rst derivative is zero, since there is no
force on any atom in equilibrium. The elastic constants of the system can be obtained by
evaluating the energy as a function of deformations of the unit cell parameters.

The elastic constants are second derivatives of the energy 
density with respect to strain components:
\begin{equation}
  c_{ij}=\frac{1}{V}\frac{\partial^2 E}{\partial \varepsilon_{i}\partial \varepsilon_{j}},
\end{equation}
where V is the volume of the cell.

\begin{equation}
  \sigma_{i}=\sum_{j}c_{ij}\varepsilon_{j}
\end{equation}


\subsection{How to run the calculations}

Four different Hamiltonians are considered, namely the Hartree–Fock (HF), 
Density Functional Theory (DFT) both in the nonlocal (PW91) formulation, 
and the hybrid scheme (PBE0 and B3LYP). The PBE0 and B3LYP hybrid functionals 
have been shown to provide excellent frequency values for molecular~\cite{WKoch00}. 
The B3LYP has also been extensively used to compute properties of crystalline 
systems~\cite{PBaranek01,RDovesi98,MCatti00,FPascale02,BCivalleri99,BCivalleri00}.

%???1
%Koch, W.; Holthausen, M. C. A Chemist’s Guide to Density Functional
%Theory, Wiley–VCH Verlag GmbH: Weinheim, 2000.
%???2
%Baranek, Ph.; Lichanot, A.; Orlando, R.; Dovesi, R. Chem Phys Lett 2001, 340, 362
%Civalleri, B.; Zicovich–Wilson, C. M.; Ugliengo, P.; Saunders, V. R.;
%Dovesi, R. Chem Phys Lett 1998, 292, 394.
%Catti, M.; Ugliengo, P.; Civalleri, B. J Phys Chem B 2000, 104, 7259.
%Pascale, F.; Ugliengo, P.; Civalleri, B.; Orlando, R.; D’Arco, Ph.; Dovesi, R. J Chem Phys 2002, 117, 5337.
%Civalleri, B.; Casassa, S.; Garrone, E.; Pisani, C.; Ugliengo, P. J Phys Chem B 1999, 103, 2165.
%Civalleri, B.; Ugliengo, P. J Phys Chem B 2000, 104, 9491.






In this work, density functional theory calculations were
performed using the method ref{...} as
implemented in the {\sc MondoSCF} program~\cite{MondoSCF_1.0-alpha-11}. 
A localized gaussian-type basis set for the expansion of the wave functions has been used.

We used the Perdew-PB parametrization to the GGA


....optimized at the 
HF, PW91~\cite{JPerdew92}, PBE0~\cite{Adamo99} and B3LYP~\cite{ABecke93} level of theories.
The PW91 functional has been obtained from the the Density Functional 
Repository~\cite{DFRepository}.



The {\tt TIGHT} level of numerical accuracy has been used throughout this work.
Thresholds that define the {\tt TIGHT} accuracy level include a matrix
threshold $\tau=10^{-6}$, as well as other numerical thresholds
detailed in Ref.~\cite{CTymczak04a}, which deliver at least $10^{-8}$ of
relative accuracy in the total energy and $10^{-4}$ of absolute accuracy
in the forces.

An all-electron gaussian basis set has been used to expand the Kohn-Sham orbitals.
This basis set is a 8-61G*, 88-31G* and 8-51G* contraction for Mg, Si and O, respectively.
The basis sets have been obtained from~\cite{CrystalLib}. 

We used only the $\Gamma$ point for sampling the Brillouin zone.

During the geometry optimization no constrains on the system have been imposed.
Our simulations were conducted on supercells $2\times1\times2$ and
$3\times1\times2$ corresonding to 16 and 24 Mg$_2$SiO$_4$ units, respectively.


Under a given deformation, the internal
coordinates and unit cell parameters of the forsterite
crystal were determined by minimizing the Hellmann-Feynman force 
on the atoms and cell simultaneously.
The optimizations were carried out with the QUICCA algorithm~\cite{KNemeth04,KNemeth05}.

At standard conditions the forsterite is orthorhombic and there are nine independent elastic
constants. 


\section{Results and Discusions}\label{Sec:ResultsAndDiscusions}

\subsection{Cell Parameters}

In Table~\ref{Tab:Forsterite}, we present the optimized lattice parameters of forsterite 
with the HF, PW91, PBE0 and B3LYP Hamiltonians and the basis set 
8-61G*(Mg)/88-31G*(Si)/8-51G*(O) within the $\Gamma$-point approximation.
The cell parameters for the B3LYP functional have been presented 
elsewhere~\cite{VWeber06b} and are reported here for the sake of comparison.
The total CPU time needed to fully relax the forsterite $2\times 1\times 2$ supercell 
was about 270-400 days, depending on the Hamiltonain used.
For comparison, we report the optimized cell parameters of forsterite
obtained by Brodholt, Patel and Refson~\cite{JBrodholt96} using the density functional code 
{\sc Cetep}~\cite{MPayne92}, the local exchange-correlation potential of 
Perdew and Zunger (PZ)~\cite{JPerdew81}, the \emph{ab initio} norm-conserving nonlocal 
Kleinman and Bylander pseudopotentials~\cite{LKleinman82}, a plane waves cut-off of 600 eV
and the $\Gamma$-point has been used to sample the Brillouin zone.
%4.643 9.988 6.074 % JBrodholt96 %PL=Perdew-Local=Perdew-Zunger % CETEP (Payne et al. 1992)
%ab initio norm-conserving, nonlocal, Kleinman-Bylander-type
%pseudopotentials for the valence-core electron interactions
%-Brodholt, J; Patel, A; Refson, K
% Source: AMERICAN MINERALOGIST; JAN-FEB 1996; v.81, no.1-2, p.257-260
%- J. P. Perdew and A. Zunger, Phys. Rev. B 23, 5048 (1981).
%-Payne, M.C., Teter, M.P., Allan, D.C., Arias, T., and Joannopoulos, J.D.
% (1992) Iterative minimization techniques for ab-initio total energy calculations:
% Molecular dynamics and conjugate gradients. Reviews of Modern Physics, 64, 1045-1097.
%-Kleinman, L., and Bylander, D.M. (1982) Efficacious form for model
% pseudopotentials. Physical Review Letters, 48, 1425-1428.
Jochym, Parlinski and Krzywiec~\cite{PJochym04} reported the lattice parameters
at the PW91 level of theory, utilizing projector augmented wave (PAW) 
pseudopotentials~\cite{PBlochl94,GKresse99}, a plane waves cut-off of 600 eV and 
a $3\times 3\times 3$ $\mathbf{k}$-points net
with the {\sc Vasp}~\cite{Kresse96a,Kresse96b} package.
%-P.E. Blöchl, Phys. Rev. B 50, 17953 (1994).
%-G. Kresse, and J. Joubert, Phys. Rev. B 59, 1758 (1999).
Noel~\emph{et al.}~\cite{YNoel06} also computed the geometrical paramters of forsterite within 
a localized gaussian-type basis set approach implemented in the {\sc Crystal} program.
The B3LYP Hamiltonian with the 8-511G*(Mg), 8-6311G*(Si) and 8-411G*(O) basis sets and
a $4\times 4\times 4$ $\mathbf{k}$-points integration grid have been used to compute
geometrical and elastic parameters.
%-Y. Noel, M. Catti, Ph. D’Arco and R. Dovesi
% Phys Chem Minerals (2006)
%Add crystal results here.
We also report the experimental cell parameters
measured by Yoder and Sahama~\cite{HYoder57}, Hazen~\cite{RHazen76}, 
Fujino~\emph{et al.}~\cite{KFujino81} and Takeuchi~\emph{et al.}~\cite{YTakeuchi84}.
%-YODER, HS; SAHAMA, TG Source: American Mineralogist; July-Aug 1957; v.42, no.7-8, p.475-491
%-Hazen, RM  Source: AMERICAN MINERALOGIST; 1976; v.61, no.11-1, p.1280-1293
%-Fujino, K.; Sasaki, S.; Takeuchi, Y.; Sadanaga, R.
% Source: Acta Crystallographica, Section B (Structural Crystallography and Crystal Chemistry); 
% 15 March 1981; vol.B37, pt.3,
% Acta Cryst. (1981). B37, 513-518  
%-Takeuchi, Y., T. Yamanaka, N. Haga, and M. Hirano, High temperature crystallography
% of olivines and spinels, in Materials Science of the Earth’s Interior, edited by I.
% Sunagawa, pp. 191–231, Terra Scientific Publishing, Tokyo, 1984. 
For the previous \emph{ab initio} calculations, the quality of the lattice parameters
improved from the LDA (PZ), GGA (PW91) to the hybrid functional (B3LYP) where the maximum error
with respect to the experimental values is 2.4\%, 1.2\% and 0.9\%, respectively. 
The {\sc MondoSCF} $\Gamma$-point calculations follow the same tendency, \emph{i.e.} 
??HF?? (), PW91 (1.0\%), B3LYP (0.9\%) and PBE0 (0.3\%) with the maximum 
deviations versus experiments in parenthesis. We can also see that GGA 
and hybrid functional give too large lattice parameters, while LDA too short except 
along the $c$-axis. As a global trend, the quality of the different models used in this work
to compute lattice parameters range, with increassing quality, form HF, PW91, B3LYP and PBE0.

\begin{table}[t]
  \centering
  \caption{\protect
    Calculated and experimental cell parameters of forsterite.
%    Cell parameters $a_0$, $b_0$, $c_0$ and total energy $E_0$
%    for orthorhombic (Mg$_2$SiO$_4$)$_n$ using the periodic $\Gamma$-point
%    B3LYP, PBE0, PW91, HF   /8-61G*(Mg)/88-31G*(Si)/8-51G*(O) level of theory and the {\tt TIGHT} thresholds.
%    The number $n=4,16$ correspond to the (super)cells
%    $1\times 1\times 1$ and $2\times 1\times 2$ respectively.
    Cell parameters and energies are in Angstroms and atomic units respectively.
  }\label{Tab:Forsterite}
  \begin{tabular}{llcccc}
  \toprule
  & Model & $a_0$ & $b_0$ & $c_0$ & $E_0$ \\
  \hline
    {\sc MondoSCF}\footnote[1]{$\Gamma$-point.}
    & HF     &  &  &  & $-$ \\%
    & PW91   & 4.794 & 10.262 & 6.004 & $-$991.076090 \\%
    & PBE0   & 4.761 & 10.187 & 5.963 & $-$990.585407 \\%
    & B3LYP\footnote[2]{$2\times 1\times 2$ supercell.} %($2\times 1\times 2$) 
             & 4.787 & 10.258 & 5.996 & $-$991.271696 \\%
    & B3LYP\footnote[3]{$3\times 1\times 2$ supercell.} %($3\times 1\times 2$)
             & 4.787 & 10.257 & 5.995 & $-$991.271841 \\%
  \hline
%Brodholt, J; Patel, A; Refson, K
%Source: AMERICAN MINERALOGIST; JAN-FEB 1996; v.81, no.1-2, p.257-260
%4.643 9.988 6.074 % JBrodholt96 %PL=Perdew-Local=Perdew-Zunger % CETEP (Payne et al. 1992)
%ab initio norm-conserving, nonlocal, Kleinman-Bylander-type 
%pseudopotentials for the valence-core electron interactions
% J. P. Perdew and A. Zunger, Phys. Rev. B 23, 5048 (1981).
%Payne, M.C., Teter, M.P., Allan, D.C., Arias, T., and Joannopoulos, J.D.
%(1992) Iterative minimization techniques for ab-initio total energy calculations:
%Molecular dynamics and conjugate gradients. Reviews of Modern Physics, 64, 1045-1097.
    {\sc Cetep}\footnote[4]{$\Gamma$-point~\cite{JBrodholt96}.}
    & PZ     & 4.643 & 9.998  & 6.074  & \\
    {\sc Vasp}\footnote[5]{$3\times 3\times 3$ $\mathbf{k}$-points~\cite{PJochym04}.}
    & PW91   & 4.800 & 10.306 & 6.041 & \\
    {\sc Crystal03}\footnote[6]{$4\times 4\times 4$ $\mathbf{k}$-points~\cite{YNoel06}.}
    & B3LYP  & 4.79  & 10.25  & 6.01  & \\
  \hline
    {Exp.}\footnote[7]{Experimental values, T=??? K~\cite{HYoder57}.}
    &        & 4.756 & 10.195 & 5.981 & \\
    {Exp.}\footnote[8]{Experimental values, T=??? K~\cite{RHazen76}.}
    &        & 4.746 & 10.18  & 5.976 & \\
    {Exp.}\footnote[9]{Experimental values, T=??? K~\cite{KFujino81}.}
    &        & 4.753 & 10.190 & 5.978 & \\
    {Exp.}\footnote[10]{Experimental values, T=??? K~\cite{YTakeuchi84}.}
    &        & 4.750 & 10.187 & 5.977 & \\
%Takeuchi, Y., T. Yamanaka, N. Haga, and M. Hirano, High temperature crystallography
%of olivines and spinels, in Materials Science of the Earth’s Interior, edited by I.
%Sunagawa, pp. 191–231, Terra Scientific Publishing, Tokyo, 1984. 
  \botrule
  \end{tabular}
\end{table}

{\bf NEED TO SAY SOMETHING ABOUT GAMMA CONVERGENCE B3LYP 2x1x2 vs 3x1x2}\\
\subsection{Atomic coordinates and bond lengths}

Forsterite has an orthorhombic structure (Pbnm). Magnesium atoms occupy two
distinct octahedral sites (Wyckoff site in paranthesis): 
Mg1 (4a) and Mg2 (4c); silicium atoms occupy the tetrahedral site (4c),
oxygen atoms occupy three distinct sites at tetrahedral corners: 
O1 (4c), O2 (4d) and O3 (8d). The oxygen atoms form a distorted hexagonal 
close-packed arrangement.

In Table~\cite{Tab:FracCoords}, we present the {\sc MondoSCF} and {\sc Crystal03}~\cite{YNoel06} 
optimized atomic coordinates and bond lengths at the same respective level of theories as above.
We also report the experimental data measured by Hazen~\cite{RHazen76}, 
Fujino~\emph{et al.}~\cite{KFujino81} and Takeuchi~\emph{et al.}~\cite{YTakeuchi84}.
The calculated equilibrium geometries, given in Table~\cite{Tab:FracCoords}, are
in good aggrement with experiments. The Si$-$O and Mg$-$O distances are very well
reproduiced with a largest difference smaller than about ?.???\AA, 0.03\AA, 0.02\AA~and
0.02\AA, ~for the HF, PW91, PBE0 and B3LYP modeles, respectively.
Again the quality of the different models increasses from HF, PW91, B3LYP and PBE0.


\begin{table}[t]
  \centering
  \caption{\protect
    Calculated and experimental fractional coordinates and cation-anion distances of forsterite.
    Distances are in Angstroms.
  }\label{Tab:FracCoords}
  \begin{tabular}{lllrrrlcccc}
  \toprule
  &  &  &  &  &  &  &  &  &  & \\
  & Model & \multicolumn{4}{l}{Fractional coordinates} & \multicolumn{5}{l}{Cation-anion distances} \\
  \hline
    {\sc MondoSCF}\footnote[1]{$\Gamma$-point.}
    % 2x1x2
    & HF     & Mg1 &  &  &  & Si$-$O  &       &       &       & \\%
    &        & Mg2 &  &  &  & Mg1$-$O &       &       &       & \\%
    &        & Si  &  &  &  & Mg2$-$O &       &       &       & \\%
    &        & O1  &  &  &  &         &       &       &       & \\%
    &        & O2  &  &  &  &         &       &       &       & \\%
    &        & O3  &  &  &  &         &       &       &       & \\%
%%%%%%%%%%%%%%%%%%%%%%%%%%%%%%%%%%%%%%%%%%%%%%%%%%%%%%%%%%%%%%%%%%%%%%%%%%%%%%%%%%%%%%%%%%%
    % 2x1x2 max(std)=0.00008
% Mg1 x = 0.00000 0.00006  y = 0.00001 0.00003  z = 0.00004 0.00006
% Mg2 x = 0.00682 0.00006  y = 0.27719 0.00003  z = 0.24994 0.00008
% Si  x = 0.42873 0.00004  y = 0.09375 0.00005  z = 0.25003 0.00007
% O1  x = 0.22984 0.00006  y = 0.09090 0.00004  z = 0.24999 0.00007
% O2  x = 0.22388 0.00007  y = 0.44630 0.00005  z = 0.25000 0.00008
% O3  x = 0.27539 0.00007  y = 0.16246 0.00004  z = 0.03135 0.00007
% Si-O  = 1.63720 0.00022  1.68077 0.00029  1.66166 0.00013
% Mg1-O = 2.08251 0.00017  2.07580 0.00021  2.13490 0.00046
% Mg2-O = 2.19040 0.00037  2.05777 0.00029  2.08017 0.00030  2.22258 0.00054
    & PW91   & Mg1 &     0.0000 & 0.0000 & 0.0000 & Si$-$O  & 1.637 & 1.681 & 1.662 & \\%
    &        & Mg2 &  $-$0.0068 & 0.2772 & 0.2499 & Mg1$-$O & 2.082 & 2.076 & 2.135 & \\%
    &        & Si  &     0.4287 & 0.0938 & 0.2500 & Mg2$-$O & 2.190 & 2.058 & 2.080 & 2.222 \\%
    &        & O1  &  $-$0.2298 & 0.0909 & 0.2500 &         &       &       &       & \\%
    &        & O2  &     0.2239 & 0.4463 & 0.2500 &         &       &       &       & \\%
    &        & O3  &     0.2754 & 0.1625 & 0.0314 &         &       &       &       & \\%
%%%%%%%%%%%%%%%%%%%%%%%%%%%%%%%%%%%%%%%%%%%%%%%%%%%%%%%%%%%%%%%%%%%%%%%%%%%%%%%%%%%%%%%%%%%
    % 2x1x2 max(std)=0.00006
% Mg1 x = 0.00001 0.00006  y = 0.00000 0.00003  z = 0.00001 0.00002
% Mg2 x = 0.00691 0.00006  y = 0.27719 0.00003  z = 0.24993 0.00006
% Si  x = 0.42914 0.00005  y = 0.09406 0.00003  z = 0.25004 0.00004
% O1  x = 0.23018 0.00004  y = 0.09106 0.00002  z = 0.25000 0.00003
% O2  x = 0.22257 0.00005  y = 0.44688 0.00002  z = 0.24998 0.00003
% O3  x = 0.27611 0.00005  y = 0.16250 0.00002  z = 0.03196 0.00005
% Si-O  = 1.62232 0.00026  1.66418 0.00021  1.64548 0.00025
% Mg1-O = 2.06970 0.00018  2.06405 0.00029  2.12242 0.00026
% Mg2-O = 2.17379 0.00034  2.04499 0.00018  2.06679 0.00029  2.20703 0.00067
    & PBE0   & Mg1 &    0.0000 & 0.0000 & 0.0000 & Si$-$O  & 1.622 & 1.664 & 1.645 & \\%
    &        & Mg2 & $-$0.0069 & 0.2772 & 0.2499 & Mg1$-$O & 2.070 & 2.064 & 2.122 & \\%
    &        & Si  &    0.4291 & 0.0941 & 0.2500 & Mg2$-$O & 2.174 & 2.045 & 2.067 & 2.207 \\%
    &        & O1  & $-$0.2302 & 0.0911 & 0.2500 &         &       &       &       & \\%
    &        & O2  &    0.2226 & 0.4469 & 0.2500 &         &       &       &       & \\%
    &        & O3  &    0.2761 & 0.1625 & 0.0320 &         &       &       &       & \\%
%%%%%%%%%%%%%%%%%%%%%%%%%%%%%%%%%%%%%%%%%%%%%%%%%%%%%%%%%%%%%%%%%%%%%%%%%%%%%%%%%%%%%%%%%%%
    % 2x1x2 max(std)=0.00018
% Mg1 x = 0.00013 0.00009  y = 0.00005 0.00005  z = 0.00004 0.00003
% Mg2 x = 0.00697 0.00018  y = 0.27772 0.00004  z = 0.24989 0.00015
% Si  x = 0.42891 0.00014  y = 0.09401 0.00003  z = 0.25005 0.00016
% O1  x = 0.23127 0.00014  y = 0.09088 0.00003  z = 0.25000 0.00014
% O2  x = 0.22303 0.00014  y = 0.44740 0.00004  z = 0.24999 0.00013
% O3  x = 0.27671 0.00014  y = 0.16210 0.00005  z = 0.03228 0.00017
% Si-O  = 1.62718 0.00012  1.67066 0.00006  1.65012 0.00017
% Mg1-O = 2.08356 0.00043  2.07260 0.00038  2.13483 0.00049
% Mg2-O = 2.19705 0.00036  2.05965 0.00025  2.07786 0.00030  2.22604 0.00084
    & B3LYP\footnote[2]{$2\times 1\times 2$ supercell.} %($2\times 1\times 2$) 
             & Mg1 &    0.0001 & 0.0000 & 0.0000 & Si$-$O  & 1.627 & 1.671 & 1.650 & \\%
    &        & Mg2 & $-$0.0070 & 0.2777 & 0.2499 & Mg1$-$O & 2.084 & 2.073 & 2.135 & \\%
    &        & Si  &    0.4290 & 0.0940 & 0.2501 & Mg2$-$O & 2.197 & 2.060 & 2.079 & 2.226 \\%
    &        & O1  & $-$0.2313 & 0.0909 & 0.2500 &         &       &       &       & \\%
    &        & O2  &    0.2230 & 0.4474 & 0.2500 &         &       &       &       & \\%
    &        & O3  &    0.2767 & 0.1621 & 0.0323 &         &       &       &       & \\%
%%%%%%%%%%%%%%%%%%%%%%%%%%%%%%%%%%%%%%%%%%%%%%%%%%%%%%%%%%%%%%%%%%%%%%%%%%%%%%%%%%%%%%%%%%%
    % 3x1x2 max(std)=0.00007
% Mg1 x = 0.00008 0.00008  y = 0.00000 0.00003  z = 0.00002 0.00003
% Mg2 x = 0.00702 0.00009  y = 0.27773 0.00003  z = 0.25000 0.00003
% Si  x = 0.42891 0.00005  y = 0.09401 0.00004  z = 0.24999 0.00002
% O1  x = 0.23125 0.00005  y = 0.09087 0.00004  z = 0.24999 0.00002
% O2  x = 0.22303 0.00007  y = 0.44740 0.00003  z = 0.25000 0.00002
% O3  x = 0.27669 0.00005  y = 0.16211 0.00003  z = 0.03232 0.00003
% Si-O  = 1.62733 0.00020  1.67055 0.00012  1.64994 0.00017
% Mg1-O = 2.08367 0.00032  2.07277 0.00024  2.13478 0.00023
% Mg2-O = 2.19689 0.00044  2.05956 0.00022  2.07793 0.00021  2.22581 0.00024
    & B3LYP\footnote[3]{$3\times 1\times 2$ supercell.} %($3\times 1\times 2$) 
             & Mg1 &    0.0001 & 0.0000 & 0.0000 & Si$-$O  & 1.627 & 1.671 & 1.650 & \\%
    &        & Mg2 & $-$0.0070 & 0.2777 & 0.2500 & Mg1$-$O & 2.083 & 2.073 & 2.134 & \\%
    &        & Si  &    0.4289 & 0.0940 & 0.2500 & Mg2$-$O & 2.197 & 2.060 & 2.078 & 2.226\\%
    &        & O1  & $-$0.2313 & 0.0909 & 0.2500 &         &       &       &       & \\%
    &        & O2  &    0.2230 & 0.4474 & 0.2500 &         &       &       &       & \\%
    &        & O3  &    0.2767 & 0.1621 & 0.0323 &         &       &       &       & \\%
%    &        &     &  &  &  &      &       &       &       & \\%
  \hline
    {\sc Crystal03}\footnote[4]{$4\times 4\times 4$ $\mathbf{k}$-points~\cite{YNoel06}.}
    & B3LYP & Mg1 &    0      & 0      & 0      & Si$-$O  & 1.629 & 1.673 & 1.653 & \\%
    &       & Mg2 & $-$0.0084 & 0.2774 & 1/4    & Mg1$-$O & 2.095 & 2.073 & 2.132 & \\%
    &       & Si  &    0.4260 & 0.0938 & 1/4    & Mg2$-$O & 2.195 & 2.063 & 2.083 & 2.222 \\%
    &       & O1  & $-$0.2341 & 0.0911 & 1/4    &         &       &       &       & \\%
    &       & O2  &    0.2247 & 0.4466 & 1/4    &         &       &       &       & \\%
    &       & O3  &    0.2747 & 0.1625 & 0.0324 &         &       &       &       & \\%
  \hline
    {Exp.}\footnote[5]{Experimental values~\cite{RHazen76}.}
    &       & Mg1 &    0      & 0      & 0      & Si$-$O  & 1.616 & 1.649 & 1.633 & \\%
    &       & Mg2 & $-$0.0086 & 0.2772 & 1/4    & Mg1$-$O & 2.085 & 2.069 & 2.126 & \\%
    &       & Si  &    0.4261 & 0.0939 & 1/4    & Mg2$-$O & 2.166 & 2.040 & 2.066 & 2.208 \\%
    &       & O1  & $-$0.2339 & 0.0919 & 1/4    &         &       &       &       & \\%
    &       & O2  &    0.2202 & 0.4469 & 1/4    &         &       &       &       & \\%
    &       & O3  &    0.2777 & 0.1628 & 0.0333 &         &       &       &       & \\%
    {Exp.}\footnote[6]{Experimental values~\cite{KFujino81}.}
    &       & Mg1 &    0      & 0      & 0      & Si$-$O  & 1.613 & 1.654 & 1.637 & \\%
    &       & Mg2 & $-$0.0083 & 0.2777 & 1/4    & Mg1$-$O & 2.084 & 2.068 & 2.131 & \\%
    &       & Si  &    0.4265 & 0.0940 & 1/4    & Mg2$-$O & 2.179 & 2.043 & 2.065 & 2.212 \\%
    &       & O1  & $-$0.2341 & 0.0916 & 1/4    &         &       &       &       & \\%
    &       & O2  &    0.2216 & 0.4471 & 1/4    &         &       &       &       & \\%
    &       & O3  &    0.2775 & 0.1631 & 0.0330 &         &       &       &       & \\%
    {Exp.}\footnote[7]{Experimental values~\cite{YTakeuchi84}.}
    &       & Mg1 &    0      & 0      & 0      & Si$-$O  &  &  &  & \\%
    &       & Mg2 & $-$0.0084 & 0.2775 & 1/4    & Mg1$-$O &  &  &  & \\%
    &       & Si  &    0.4266 & 0.0941 & 1/4    & Mg2$-$O &  &  &  & \\%
    &       & O1  & $-$0.2336 & 0.0914 & 1/4    &         &       &       &       & \\%
    &       & O2  &    0.2215 & 0.4472 & 1/4    &         &       &       &       & \\%
    &       & O3  &    0.2778 & 0.1631 &?0.0066?&         &       &       &       & \\%
  \botrule
  \end{tabular}
\end{table}


\subsection{Elastic parameters}

\begin{table}[t]
  \centering
  \caption{\protect
    Calculated and experimental elastic constant of forsterite.
  }\label{Tab:ElasticConstants}
  \begin{tabular}{llccccccccc}
  \toprule
  & Model & $c_{11}$ & $c_{22}$ & $c_{33}$ & $c_{44}$ & $c_{55}$ 
  & $c_{66}$ & $c_{12}$ & $c_{13}$ & $c_{23}$  \\
  \hline
    {\sc MondoSCF}\footnote[1]{$\Gamma$-point.}
  & HF    & & & & & & & & & \\
  & PW91  & & & & & & & & & \\
  & PBE0  & & & & & & & & & \\
  & B3LYP\footnote[2]{$2\times 1\times 2$ supercell.} %($2\times 1\times 2$)
          & 353.8 & 216.3 & & & & & 69.4 & 73.0 & 79.7 \\
  & B3LYP\footnote[3]{$3\times 1\times 2$ supercell.} %($3\times 1\times 2$) 
          & & & & & & & & & \\
  \hline
    {\sc Vasp}\footnote[4]{$3\times 3\times 3$ $\mathbf{k}$-points~\cite{PJochym04}.}
  & PW91  & 306.0 & 190.8 & 219.8 & 60.8 & 74.6 & 74.0 & 63.3 & 65.5 & 69.3 \\
    {\sc Crystal03}\footnote[5]{$4\times 4\times 4$ $\mathbf{k}$-points~\cite{YNoel06}.}
  & B3LYP & 345.3 & 212.0 & 241.7 & 68.0 & 82.8 & 81.3 & 69.1 & 71.9 & 78.7 \\
  \hline
%%%%%%%%%%%
%this is a science paper, but without all the data!!!! Data taken from PJochym04
%    {Exp. (T=??? K)}\footnote[4]{Experimental values, ultrasonic measurments~\cite{GChen96}.}
%  &       & 330.0 & 200.0 & 235.0 & 68.0 & 81.0 & 81.0 & 68.0 & 69.0 & 72.0 \\
%%%%%%%%%%%
    {Exp. (T=300 K)}\footnote[6]{Experimental values, ??????????????????????~\cite{DIssak89}.}
  &       & 330.0 & 200.3 & 236.2 & 67.1 & 81.6 & 81.2 & 66.2 & 68.0 & 72.2 \\
    {Exp. (T=0 K)}\footnote[7]{Experimental values, 
      obtained by extrapolation of temperature gradients~\cite{YNoel06,DIssak89}.}
  &       & 340.9 & 208.2 & 244.4 & 71.5 & 85.7 & 86.0 & 69.4 & 70.8 & 73.7 \\
  \botrule
  \end{tabular}
\end{table}


\begin{table}[t]
  \centering
  \caption{\protect
    aaa
  }\label{Tab:FracCoords}
  \begin{tabular}{llccccccc}
  \toprule
  &  & $B$ & $\mu$ & $E$ & $\sigma$ & $V_p$ & $V_s$ & $E_g$ \\
  \hline
    {\sc MondoSCF}\footnote[1]{$\Gamma$-point.}
    & HF    &  &  &  &  &  &  & \\%
    & PW91  &  &  &  &  &  &  & \\%
    & PBE0  &  &  &  &  &  &  & 8.4 \\%
    & B3LYP ($2\times 1\times 2$)
            &  &  &  &  &  &  & 7.9 \\%
    & B3LYP ($3\times 1\times 2$)
            &  &  &  &  &  &  & 7.9 \\%
  \hline
    {\sc Vasp}\footnote[2]{$3\times 3\times 3$ $\mathbf{k}$-points~\cite{PJochym04}.}
    & PW91  & 123.6 & 76.5 & 0.244 & 190.2 & 8.494 & 4.945 & \\
    {\sc Crystal03}\footnote[3]{$4\times 4\times 4$ $\mathbf{k}$-points~\cite{YNoel06}.}
    & B3LYP & 137.6 & 85.0 & 0.244 & 211.5 &  &  & \\
  \hline
    {Exp.}\footnote[4]{Experimental values~\cite{????}.}
    &       &  &  &  &  &  &  &\\
    {Exp.}\footnote[5]{Experimental values~\cite{????}.}
    &       &  &  &  &  &  &  &\\
    {Exp.}\footnote[6]{Experimental values~\cite{????}.}
    &       &  &  &  &  &  &  &\\
  \botrule
  \end{tabular}
\end{table}





\section{Conclusions}\label{Sec:Conclusions}
In a previous paper, construction of the analytical
exact Hartree-Fock exchange gradients with respect to
atomic and cell parameters within the $\Gamma$-point
approximation has been introduced.
In this article, the formalism for evaluation of the Density
Functional analytic gradients in $\Gamma$-point approximation for
Cartesian Gaussian-type basis functions was presented and
implemented in the {\sc MondoSCF} package.
We show that, like for the exact HF exchange, the evaluation
of the atomic gradients within the $\Gamma$-point approximation
does not lead to difficulties. The implementation requires mainly
the evaluation of the derivative of the basis function with respect
to atomic positions.
However, complications arise when derivatives of the total energy
are taken with respect to lattice parameters. For the periodic far field Coulomb term,
the derivative of the multipole interaction tensor needs to be carefully handled in
both direct and reciprocal space.

The exchange-correlation energy derivative leads to a surface term, which has its
origin in the derivative of the limits of the integration over the cell.
The analytical atomic and cell gradients have been used
in conjunction with the QUICCA algorithm to optimize a few 1D and 3D periodic systems at
the DFT and hybrid-HF/DFT level of theory.
Convergence of bond lengths, bond angles and
cell parameters within the DFT and hybrid-HF/DFT $\Gamma$-point
super cell approach and under full relaxation with no symmetry
%to the converged large cell $\Gamma$-point approximation
have been demonstrated for 1D and 3D systems to better than 3 digits.
%Although the convergence of the Kohn-Sham $\Gamma$-point total energy to
%its ${\bf k}$-space integration counterpart with respect to cell size is relatively slow,
%the convergence of the geometrical parameters (cell and atomic positions)
%requires much smaller cells.
Thus, we could show that a relative accuracy better
than 3 digits can be already achieved with cubic cells of about $600$\AA$^3$.\\




%%%%%%%%%%%%%%%%%%%%%%%%%%%%%%%%%%%%%%%%%%%%%%%%%%%%%%%%%%%%%%%%
%%%%%%%%%%%%%%%%%%%%%%%%%%%%%%%%%%%%%%%%%%%%%%%%%%%%%%%%%%%%%%%%
%Acknowledgements
\begin{acknowledgments}
The authors would like to thank C. J. Tymczak
for careful reading of the manuscript. 
This work has been supported by the US Department of Energy
under contract W-7405-ENG-36 and the ASCI project.
The Advanced Computing Laboratory of Los Alamos National 
Laboratory is acknowledged.
\end{acknowledgments}  
%%%%%%%%%%%%%%%%%%%%%%%%%%%%%%%%%%%%%%%%%%%%%%%%%%%%%%%%%%%%%%%%
%%%%%%%%%%%%%%%%%%%%%%%%%%%%%%%%%%%%%%%%%%%%%%%%%%%%%%%%%%%%%%%%
\bibliography{mondo_new}
\newpage
%%%%%%%%%%%%%%%%%%%%%%%%%%%%%%%%%%%%%%%%%%%%%%%%%%%%%%%%%%%%%%%%
%%%%%%%%%%%%%%%%%%%%%%%%%%%%%%%%%%%%%%%%%%%%%%%%%%%%%%%%%%%%%%%%
%%%%%%%%%%%%%%%%%%%%%%%%%%%%%%%%%%%%%%%%%%%%%%%%%%%%%%%%%%%%%%%%



%%%%%%%%%%%%%%%%%%%%%%%%%%%%%%%%%%%%%%%%%%%%%%%%%%%%%%%%%%%%%%%%
%%%%%%%%%%%%%%%%%%%%%%%%%%%%%%%%%%%%%%%%%%%%%%%%%%%%%%%%%%%%%%%%
\end{document}
%
% ****** End of file apssamp.tex ******

