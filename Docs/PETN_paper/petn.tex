\renewcommand{\thefootnote}{\fnsymbol{footnote}}

\newcommand{\Bulk}{B_0}
\newcommand{\BPrime}{{B'_0}}

\documentclass[prb,aps,nobibnotes,twocolumn,doublespace,twocolumngrid,superbib]{revtex4}
%\documentclass[prb,aps,nobibnotes,superbib,preprint]{revtex4}

\usepackage{graphicx}
\usepackage{amsfonts}
\usepackage{bm}
\usepackage{alltt}
\usepackage{dcolumn}
\usepackage{amsmath}
\usepackage{graphicx}
\makeatletter
\makeatother

\begin{document}
\title[Short Title]{ 
All-electron density functional studies of hydrostatic compression of 
PETN (pentaerythritol tetranitrate)}
\author{Chee Kwan Gan\footnotemark[1]}
\author{Thomas D. Sewell\footnotemark[2]}
\author{Matt Challacombe\footnotemark[3]}

\affiliation{ Theoretical Division, Los Alamos
              National Laboratory,\\ Los Alamos, New Mexico 87545}
\date{June 30, 2003}

\begin{abstract}
All-electron calculations of the hydrostatic compression of
PETN (pentaerythritol tetranitrate) crystal have been performed using
density functional theory with the PBE functional in
conjunction with the 6-31G** Gaussian basis set.  Full optimizations
of the atomic positions and $c/a$ ratio of lattice parameters for the 
tetragonal crystal were performed for eight volume ratios $0.65 
\le V/V_0 \le 1.00$, where $V_0$ is the equilibrium volume at zero
pressure.  The pressure, the linear compressiblities
of $a$ and $c$, and $c/a$ ratio as functions of volume ratio
are found to be in good agreement with experimental data.
Changes in intramolecular coordinates as well as the intermolecular 
distances were studied as a function of compression.
Predictions of the bulk modulus $\Bulk$ and 
its pressure derivative $\BPrime$ were obtained
using various equation of state fitting forms.

\smallskip
\noindent{\bf Keywords}: Density functional theory, linear scaling, 
Gaussian-Orbital, PETN, hydrostatic compression
\end{abstract}

%\pacs{31.15.-p; 31.15.Ew; 02.60.Jh}
\maketitle

\footnotetext[0]{Preprint LA-UR-.}
\footnotetext[1]{\tt CKGan@LANL.Gov}
\footnotetext[2]{\tt Sewell@LANL.Gov}
\footnotetext[3]{\tt MChalla@LANL.Gov}

\section{Introduction}
Pentaerythritol tetranitrate (PETN, $C(CH_2ONO_2)_4$) is an important 
secondary high explosive.
 At room temperature, PETN crystallizes in a 
tetragonal space group, P$\bar{4}$-2$_1$c,  
with two molecules per unit
cell\cite{Booth_1947v,Trotter_1963v16,Conant_1979}.
A high-temperature 
form denoted PETN-II has also been reported\cite{Cady_1975vB41}.  
PETN has received a lot of attention due to interesting anisotropic, 
non-monotonic shock initiation 
sensitivity\cite{Dick_1984v44,Dick_1991v70,Gallagher_1992v339,Dick_1997v81,Gruzdkov_2000v104,Yoo_2000v88}.
The room temperature linear and volumetric hydrostatic compression
of PETN has been measured by Olinger and 
co-workers\cite{Olinger_1975v62,Olinger_1976} 
using x-ray diffraction for pressures up to 10.45~GPa. 
Dick and von Dreele\cite{Dick_1997} used neutron scattering from deuterated
PETN to study detailed pressure-induced changes in dihedral angles and
molecular orientations in the unit cell for pressures up to 4.28~GPa. 
It is interesting to note that, while the volumetric compression reported
by Dick and von Dreele was consistent with the data of Olinger {\it et al.},
significant differences were observed in the linear compressions along
the $a$ and $c$ axes.

A complete set of isentropic elastic coefficients for PETN was reported 
by Morris\cite{Morris_1976} based on
extensive single crystal sound speed measurements.  The resulting
elastic tensor was of high precision with the exception of element $C_{13}$,
which had an uncertainty over ten times the average for the others.
Winey and Gupta\cite{Winey_2001v90} 
re-analyzed Morris' data, identified the source of the inconsistency in 
$C_{13}$, and reported an internally consistent and precise elastic tensor.

There have been a number of theoretical studies of PETN crystals
based on analytic classical potentials.
Dick and Ritchie\cite{Dick_1994v76}
performed molecular 
mechanics calculations using the AMBER4.0 force field to study the energetics 
associated with slip along specific directions in the crystal, in an effort 
to understand the anisotropic initiation sensitivity in PETN.  Their results 
were largely consistent with experimental data, but were not conclusive due 
to the lack of rate-dependent information.  

Theoretical predictions of PETN crystal structure, thermal expansion, and 
hydrostatic compression were reported by 
Sorescu {\it et al.}\cite{Sorescu_1999v103A,
Sorescu_1999v103}.  They used an intermolecular potential function developed 
in their laboratory in conjunction with a rigid-molecule approximation for 
the molecular structure. Agreement to within one percent of the measured 
crystal lattice parameters was obtained at 300~K, although the calculated 
thermal expansion was three times smaller than the experimental 
value.  Sorescu {\it et al.}\cite{Sorescu_1999v103}
predicted linear compressibilities 
that were in reasonable agreement with experiment for pressures below 
6~GPa, with the onset of significant discrepancies for higher pressures.  (The 
calculated linear compressibilites were systematically lower than measured 
ones, resulting in a larger error in the volumetric compressibility 
over the entire pressure interval; which led to a 41\% error compared to 
experiment in the predicted initial bulk modulus.) The authors attributed 
the errors in their predictions away from ambient conditions to a shortcoming 
of the rigid-molecule approximation.

Bunte and Sun\cite{Bunte_2000v104}
reported a flexible molecule inter- and intramolecular COMPASS force field 
parameterization based on a calibration to experimental data and high level 
gas phase electronic structure results for selected nitrate esters.  
Although the force field training set did not include PETN, predicted 
crystal lattice parameters were in good agreement with experiment.  Bunte 
and Sun reported five of the six elastic coefficients for PETN at zero 
Kelvin, based on energy-minimized lattice parameters as well as for a
constrained calculation in which the lattice parameters were set equal to
experimental values.  Reasonable agreement with experimental data was obtained 
in the latter case, in spite of neglect of explicit thermal effects.  Since 
they did not report the $C_{13}$ elastic coefficient, it is not possible to 
obtain a bulk modulus from their work.

Zaoui and Sekkal\cite{Zaoui_2001v118} reported 
calculations of PETN thermal and mechanical properties using a flexible, 
Tersoff-like force field.  This work is difficult to interpret, as the 
primary simulation cell as described in the article corresponds to a 
non-integral number of molecules, and the predicted longitudinal and 
transverse sound speeds are both unrealistically small and inconsistent
with the bulk modulus they report .  For these reasons, it will not be 
considered further below.

In this contribution, we report condensed phase electronic-structure
calculations of the hydrostatic compression of PETN at absolute zero
and pressures up to 25 GPa.  We employ an all-electron density-functional 
method using the PBE functional in conjunction with the 6-31G** Gaussian basis 
set.   Complete optimization of the cell contents and $c/a$ ratio of the 
lattice 
parameters is performed at each of the eight volumes considered, assuming 
a tetragonal cell (i.e., $a=b \neq c$, $\alpha=\beta=\gamma=90^\circ$).
It should be noted that our all-electron treatment is considerably more
demanding computationally than pseudopotential methods, but it should
be free from potential errors associated with pseudopotentials that would
likely become important at high pressures.

The remainder of this paper is organized as follows: In
Section~\ref{sec:comput} we describe the computational framework and
in Section~\ref{sec:protocol} the 
details of our calculations. In Section~\ref{sec:results} we 
present the results of the calculations and properties derived from them,
with comparsions to experimental and previous theoretical studies. 
Finally, in Section~\ref{sec:conclusions} we summarize the main conclusions
of our work.

\section{Computational Framework}
\label{sec:comput}
We have used a parallelized version of MondoSCF\cite{MondoSCF}, 
a suite of programs
for linear scaling electronic-structure theory and 
{\it ab initio} molecular dynamics, to perform the calculations.  This code employs a
number of advanced $O(N)$ techniques such as the quantum chemistry tree code
(QCTC) for the Coulomb matrix build
\cite{MChallacombe96,MChallacombe96B,MChallacombe97} and
the adaptive Hierarchical Cubature (HiCu) for the exchange-correlation 
matrix build~\cite{MChallacombe00A}. Parallelization of MondoSCF
has been carried out, where an exemplary success is 
the efficient data parallel algorithm for the
HiCu method\cite{CGan03} with equal-time partitioning scheme
which exploits the temporal locality of SCF 
calculations to achieve excellent load
balance. We have
used a cluster of 256 4-CPU HP/Compaq Alphaserver ES45s with the
Quadrics QsNet High Speed Interconnect to perform all the calculations 
in this work.   Most of the calculations were performed using 32 
processors.

\section{Computational Protocol}
\label{sec:protocol}
The results described below are for pure, three-dimensionally periodic 
gradient-corrected density-functional theory (GGA), using the PBE
functional\cite{Perdew_96v77}. A variety of split-valence Gaussian
basis sets were considered, although all of the calculations reported
here were performed with the 6-31G** basis set. 

For $k$-point sampling, we primarily used the $\Gamma$ point, which is 
sufficient for the purpose in this work, as confirmed by repeating some 
calculations with a larger $1 \times 1 \times 2 $ supercell.
Thus, unless otherwise stated, all calculations were performed for a
primary simulation cell corresponding to one crystallographic
unit cell; i.e., a $1\times 1\times 1 $ system containing two molecules.  

Starting with the experimental x-ray crystal structure reported by
Conant {\it et al.},\cite{Conant_1979} we calculated the zero pressure unit cell volume $V_0$ 
under the assumption of a tetragonal lattice ($a=b\neq c$, 
$\alpha=\beta=\gamma=90^\circ$) by computing energy-minimized structures
on a $5\times 5$ grid of $a$ and $c$ lattice parameters, 
and fitting  the results 
to a cubic polynomial in $a$ and $c$.  A subsequent calculation
on a $3 \times 3$ grid centered at the predicted minimum from the preceding step
was used to obtain an improved prediction of the minimum.  These
correspond to $a_0$, $c_0$, and $V_0=a_0^2c_0$.  Verification that
the resulting tetragonal structure was stable to non-tetragonal variations
in cell shape was performed by computing energy-minimized orthorhombic
structures, i.e., $a=b\neq c$, in the neighborhood of the tetragonal minimum; 
these tests indicated that the tetragonal structure is energetically favored 
over nearby orthorhombic ones.  

Energy-minimized structures for a given volume ratio $V/V_0$ 
were obtained by scanning a set of $c/a$ ratios consistent with the 
desired volume $V$.  Initial guesses were obtained by rigid translation 
of the molecular structure obtained from the next larger volume ratio 
via a transformation that preserves the center-of-mass positions in 
crystallographic coordinates.  Typically, seven values of
$c/a$ were considered at a given volume $V=a^2c$, and the results were
fit to a parabola.

\section{Results}
\label{sec:results}
Calculated and measured lattice parameters are compared in 
Table~\ref{tab:table1}, where
we also include the experiemental results of Olinger {\it et al.}\cite{Olinger_1975v62} and Conant {\it et al.}\cite{Conant_1979}; 
and the theoretical predictions of Sorescu {\it et al.}\cite{Sorescu_1999v103}  and
Bunte and Sun\cite{Bunte_2000v104}  for completeness.  Percent errors in 
our calculations 
relative to the experimental results of Conant {\it et al.}\cite{Conant_1979} are 0.51\%, 0.75\%, 
and 1.8\% for $a_0$, $c_0$, and $V_0$, respectively.  It is important to note 
that we have not attempted to correct for thermal expansion, since 
experimental coefficients of thermal expansion for PETN do not extend to 
cryogenic temperatures. 

Calculated intramolecular coordinates and selected intermolecular
distances are compared to the experimental results of 
Conant {\it et al.}\cite{Conant_1979} in Table~\ref{tab:table2}. 
The predicted molecular geometry is in excellent agreement
with experiment.  The 12\% discrepancy in the C-H distances is expected,
since we are comparing to x-ray data.  Otherwise, most of the intramolecular
degrees of freedom are accurately predicted to within a few percent.
There are three intermolecular distances less than 3~\AA\ at equilibrium,
each of which is an O$\cdot\cdot\cdot$H contact.  Predicted values are 2.396~\AA, 2.779~\AA,
and 2.898~\AA, with errors relative to experiment of -3.5\%, -0.7\%, and
0.1\%, respectively.

The volumetric hydrostatic compression is summarized in 
Fig.~\ref{fig:volume_compress}.  The 
variation of the energy with volume ratio is shown in 
Fig.~\ref{fig:volume_compress}(a).
The line is a sixth-degree polynomial fit, from which the pressure
shown in Fig.~\ref{fig:volume_compress}(b) is obtained as $P=-dE/dV$.  
The open circles in Fig.~\ref{fig:volume_compress}(b) 
are the experimental data of Olinger {\it et al.}\/ with
experimental uncertainties smaller than the symbol sizes.  
The calculated volumetric compression
is in remarkably good agreement with experiment.  The effects of finite 
temperature would be expected to ``soften'' the calculated 
isotherm at low pressures; however, the effect should
diminish quickly with increasing compression.

\begin{figure}
\resizebox*{3.5in}{!}{\includegraphics[clip]{VRatio_EDiff_Press.eps}}
\caption{Volumetric hydrostatic compression for PETN.
(a): relative energy difference. The
solid line is a sixth-degree polynomial fit to the $1\times 1\times 1$ 
predictions.
Crosses are calculations for a $1\times 1\times 2$ supercell, and indicate
that finite size scaling effects are minimal.  (b): pressure-volume 
relationship calculated from the fit in (a).  Uncertainties in the 
experimental data of Olinger {\it et al.}\cite{Olinger_1975v62}  are smaller than the symbol 
sizes.
}
\label{fig:volume_compress}
\end{figure}

The linear compression along the $a$ and $c$ crystallographic directions
is shown in Fig.~\ref{fig:linear_compress}.  We show in 
Fig.~\ref{fig:linear_compress}(a) the relative compressions of $a/a_0$  and
$c/c_0$.
Calculated values are indicated by open symbols.
Crosses and stars denote the experimental data of 
Olinger {\it et al.}\cite{Olinger_1975v62}; as before, experimental 
uncertainties are smaller than the symbols.  
In Fig.~\ref{fig:linear_compress}(b) we show the variation
of the $c/a$ ratio as a function of compression.  Open squares are 
calculated values; open circles were derived from the data of Olinger
{\it et al.}\cite{Olinger_1975v62}; uncertainties 
in the experimental ratios indicated by the
error bars are shown for representative cases.  The solid line is based
on polynomial representations of $a=a(V)$ and $c=c(V)$ 
due to Olinger {\it et al.}\cite{Olinger_1975v62}. We observe an
initial decrease in the $c/a$ ratio, followed by an increase 
towards a more ``cubic-like'' lattice structure.  
The minima in the curves are in fairly good agreement, 
and correspond to a volume ratio of 0.8 and a pressure of about 7.5~GPa.


\begin{figure}
\resizebox*{3.5in}{!}{\includegraphics[clip]{CAndA.eps}}
\caption{Linear compression of PETN.  (a): relative linear compression
of the lattice parameters. Squares and circles correspond to the $a$ and $c$ 
crystallographic axes, respectively; dashed lines are simply a guide for
the eye.  Experimental data from Olinger {\it et al.}\cite{Olinger_1975v62}; 
experimental uncertainties
are comparable to the size of the symbols. (b): Ratio of $c$ to $a$.
Squares denote calculated results;
dashed line is a guide for the eye.  Circles are experimental 
data\cite{Olinger_1975v62};
error bars in $c/a$ and $V/V_0$ are representative; solid line is a 
polynomial form taken from Ref.~\cite{Olinger_1975v62}.
}
\label{fig:linear_compress}
\end{figure}

%As expected, most of the change with increasing compression is associated
%with intermolecular packing.  At zero pressure there are only twenty four
%bonds per unit cell with $R_{ij} < $3.0 \AA.  At 25 GPa, this number has
%increased to 140.  

\begin{figure}
\resizebox*{3.5in}{!}{\includegraphics[clip]{VRatio_Bond_Angle_CloseContact.eps}}
\caption{(a) The variation of bond lengths as functions of compression.
The meaning of the symbols is as follows: cicle (CC), square (CO),
diamond (ON), up triangle (NO1), cross(NO2),
down triangle (CH1), plus (CH2). (b) The variation of angles as function of compression. The meaning of symbols is as follows:
cicle (CCC2), square (CON), diamond (CCH2), up triangle (HCO2), cross (CCO)
(c) The close contact distances as functions of compression. The circles, squares and diamonds corresponds to the first, second, and third set of OH distances.
}
\label{fig:intramolecular}
\end{figure}
In Figs.~\ref{fig:intramolecular}(a) and (b),
we show the changes in the bond lengths and angles as PETN is 
compressed. To avoid cluttering of data, we have only included 
results which show bigger changes. It is observed that 
the intramolecular bond lengths and angles remain almost constant
for volume ratio from 1.00 to 0.80. Beyond 0.80, most bond lengths and 
angles change by an appreciable amount. It is interesting to note that
at $V/V_0 = 0.80$, the $c/a$ ratio attains its minimum. 
Fig.~\ref{fig:intramolecular}(c)
shows that  most changes with increasing compression are associated
with intermolecular packing. 
The results in Figs.~\ref{fig:intramolecular}(a), (b), and (c) 
confirm the discussion of Pastine and Bernecker\cite{Pastine_1974v45}, who 
stated that the initial compression of an organic explosive is almost 
entirely due to a reduction of intermolecular distances. At higher compression,
the intermolecular distances become too small that the increase of 
van der Waals repulsion becomes comparable to the intramolecular repulsions 
along the covalent bonds. From the results of our calculations, 
the rigid molecule approximation 
breaks down for pressure higher than about
6.0 GPa (see Fig.~\ref{fig:volume_compress}(b)). 
Indeed, Sorescu {\it et al.}\cite{Sorescu_1999v103} 
have observed that the deviations of the predicted
values from the experimental values increase 
rapidly for pressure above 6.0~GPa,
which they have attributed to the rigid molecule approximation.

We can obtain the bulk modulus $\Bulk=-VdP/dV$ and its initial pressure
derivative $\BPrime =d \Bulk/dP$ from the $PV$ data using equation of state 
fitting forms.  In order to connect with preceding work, we consider
three forms. The first form is the Murnaghan equation\cite{Murnaghan_1951}
\begin{equation}
P=\frac{\Bulk}{\BPrime}\left(\eta^{-\BPrime}-1\right)
\label{eq:Murna}
\end{equation}
where $\eta=V/V_0$.
The second form is the third-order Birch-Murnaghan equation\cite{Poirier_1991} 
used previously in the analysis of the isotherm for the
high explosive 
octahydro-1,3,5,7-tetranitro-1,3,5-7-tetrazocine 
(HMX)\cite{Yoo_1999v111,Menikoff_2001v21,Sewell_2003}
\begin{equation}
P=\frac{3}{2}\Bulk(\eta^{-7/3}-\eta^{-5/3})
     \left[1+\frac{3}{4}(\BPrime -4)(\eta^{-2/3}-1)\right],
\label{eq:BM}
\end{equation}
The third fitting form 
\begin{equation}
P=\frac{(V_0-V)c^2}{[V_0-s(V_0-V)]^2},
\label{eq:cs}
\end{equation}
was used by Olinger {\it et al.}\/ in the analysis of their isotherm data for
PETN\cite{Olinger_1975v62,Olinger_1976} and other explosives including
TATB,\cite{Olinger_1976} $\beta$-HMX and RDX\cite{Olinger_1978}
and nitromethane\cite{Yarger_1986v85}.
In the latter case, the initial bulk modulus and its pressure derivative
are given by $\Bulk=c^2/V_0$ and $\BPrime=4s-1$, 
respectively.

Fits to isotherm data are quite sensitive to the fitting form chosen
and the interval of data used in the fit, particularly if the interest
is in obtainig precise predictions of the initial bulk modulus and
its pressure derivative.  Thus, for the sake of simplicity,
we restrict our fits to pressures %of 10 GPa and
not exceeding 10.54~GPa (the maximum pressure achieved by Olinger {\it et
al.}\cite{Olinger_1975v62})
and do not perform a thorough sensitivity analyis of the
predicted bulk modulus to details of how the fit is performed.

Calculated and measured values for the initial bulk modulus and its 
pressure derivative for PETN are summarized in Table~\ref{tab:table3}.  
The results of fits in the $PV$ plane to our calculated isotherm are fairly 
consistent, with $\Bulk=14.5$ to $16.0$~GPa and $\BPrime=5.2$ to $6.7$.  
Using the same fits to the experimental data 
of Olinger {\it et al.}\cite{Olinger_1975v62}, 
we obtain $\Bulk=9.4$ to $12.2$~GPa and $\BPrime=6.4$ to $11.3$
Sorescu
{\it et al.}\cite{Sorescu_1999v103}  
reported $\Bulk=14.1$~GPa and $\BPrime=10.4 $ for a fit of their
rigid-molecule simulation results to the Murnaghan equation of state 
(Eq.~(\ref{eq:Murna})). 
Olinger {\it et al.}\cite{Olinger_1976} reported $\Bulk=8.8$~GPa 
and $\BPrime=9.9$ from their
analysis of the experimental isotherm.  (Note that these
are revised values of $\Bulk$ and $\BPrime$, and are different from the 
ones initially published in Ref. \cite{Olinger_1975v62} and subsequently
used for comparison by 
Sorescu {\it et al.}\/ in Ref. \cite{Sorescu_1999v103}.)  Another experimental 
value for $\Bulk$
can be obtained from the isentropic elastic tensor reported by Winey and 
Gupta\cite{Winey_2001v90}.  Specifically, 
\begin{equation}
\Bulk=\frac{C_{33}(C_{11}+C_{12})-2C_{13}^2}{C_{11}+C_{12}+2C_{33}-4C_{13}}.
\end{equation}
This results in an {\em isentropic} bulk modulus $\Bulk^{s}=9.8$~GPa, which can 
be transformed to an {\em isothermal} bulk modulus $\Bulk^t$ using the relation 
$\Bulk^t/\Bulk^s=C_V/C_P$.  Using values for $C_V$ and $C_P$ recommended in Ref. 
\cite{Olinger_1976}  yields $\Bulk=9.1$~GPa. It is perhaps unsurprising that the values
of $\Bulk$ estimated from the zero Kelvin isotherm are larger than the values
obtained from room-temperature data.  Likewise, the fact that the value
of $\Bulk$ reported by Sorescu and Thompson is larger than experiment even
though it is based on a room-temperature isotherm can be tentatively 
attributed to the use of a rigid-molecule simulation protocol.

%The result we obtain for the bulk modulus using Eqs. 3--4  is substantially 
%different from the others: $\Bulk=6.1$~GPa and $\BPrime=18.7$.  As has been 
%noted previously, for organic molecular cyrstals one expects significant 
%curvature in the $(U_s,U_p)$ plane for low levels of compression.  This 
%stands in contrast to metals or covalent solids, for which a linear 
%$U_s,U_p$ curve is generally expected.  This is shown in  Fig.~3, where
%we include our results (open symbols connected by dashed lines) and
%a piece-wise fit taken from Olinger {\it et al.}\cite{Olinger_1976}  From the 
%figure it is
%clear that additional calculations at lower levels of compression (smaller
%values of the abscissa) would be required to specify precisely the
%initial slope (proportional to $\BPrime$) and intercept (proportional 
%to $\Bulk^{1/2}$).  Indeed, in their final report on the PETN isotherm, Olinger
%{\it et al.}\/ used an independently measured bulk sound speed to specify the
%intercept in the $(U_s,U_p)$ plane.




\section{Conclusions}
\label{sec:conclusions}
Hydrostatic compression of pentaerythritol tetranitrate (PETN) has
been studied with all-electron calculations with DFT-PBE in
conjunction with 6-31G** basis set. Even though our all-electron 
studies are considerably more expensive than the pseudopotential
studies, the obvious advantage of our approach is that the results obtained are
free from any errors that might result from the pseudopotential approximation,
especially at a highly compressed state. 
The content of the simulation cell is fully relaxed, which should give more 
reliable results than those obtained with a rigid molecule approximation, 
especially at a highly compressed state.
The results of pressure, $a/a_0$, $c/c_0$, and $c/a$ versus 
$V/V_0$ are in good agreement with the experimental results. 
Our calculations predicted that the tetragonal PETN cyrstal should 
approach a cubic one when high hydrostatic pressure is applied, which is again
in agreement with the experimental results.
Our results of the bulk modulus and its pressure derivative are in fairly good
agreement with the experimental results.
Our calculations reveal that the initial compression is accompanied by
an decrease of intermolecular distances up to about 6.00~GPa, and then
followed by the intramolecular changes. This confirms the discussion
by Pastine and Bernecker\cite{Pastine_1974v45}.

The success of our all-electron approach should open the door to the studies
of hydrostatic compression of other materials of interest or other
explosives such as TATB, HMX, and RDX. 
Finally, it would be interesting to perform a systematic assessment of
pseudopotential errors in the regime of high compression, by using
our all-electron results as a benchmark.

\begin{acknowledgments}
This work has been carried out under the auspices of the US Department
of Energy under contract W-7405-ENG-36 and the ASCI project.  Most work 
was perform on the computing resources at the
Advanced Computing Laboratory of Los Alamos National Laboratory, Los
Alamos, NM 87545. 
We thank A.~H.~Strachan for many fruitful discussions.
\end{acknowledgments}




\begin{table}[p]
\begin{center}
\begin{tabular}{llll}
\hline\hline
a(\AA) & c(\AA) & V(\AA$^3$) & Source \\
\hline
9.425  & 6.758  &  600.3  & This work \\
9.383  & 6.711  &  590.8  & Olinger {\it et al.}\cite{Olinger_1975v62}\\
9.3776 & 6.7075 &  589.9  & Conant {\it et al.}\cite{Conant_1979} \\
%9.117  & 6.721  &  558.6  & Bunte and Sun\cite{Bunte_2000v104} \\
9.35   & 6.67   &  583    & Bunte and Sun\cite{Bunte_2000v104} (NPT-MD, 293K) \\
9.3348 & 6.6500 &  579.47 & Sorescu {\it et al.}\cite{Sorescu_1999v103} (NPT-MD, 293K) \\
% wrong values 9.2836 & 6.5995 &  569.79 & Sorescu {\it et al.}\/ (energy minimization, rigid molecules) \\
% 9.3471 & 6.6571 &  581.5  & Sorescu {\it et al.}\/ (NPT-MD, 300K, rigid molecules) \\
\hline\hline
\end{tabular}
\end{center}
\caption{Equilibrium lattice parameters and unit cell volumes for PETN.
}
\label{tab:table1}
\end{table}


\begin{table}[p]
\begin{center}
\begin{tabular}{llllll}
\hline\hline
Source & C-C & C-H & C-O & O-N & N=O  \\
This work & 1.532 (0.1) & 1.098 (12.0) & 1.444 (0.1) &  1.437 (2.4) & 1.219 (1.9) \\
Ref.\cite{Conant_1979} &  1.531   &   0.98  & 1.446  &  1.403 &   1.196 \\
\hline
\\
Source  &  CCH   &        HCH     &      CCO    &       HCO \\
This work & 111.0 (-1.1)  & 108.1  (1.5) &  106.6 (-0.1) & 110.1 (0.5) \\
Ref.\cite{Conant_1979} &  112.2    &  106.5  &  106.7   & 109.6 \\
\hline
\\
Source   &  CON     &       ONO1   &        ONO2     &      ONO \\
This work&  111.8 (-1.2) &  117.4 (-0.3)&  111.9 (-0.1)&  130.7 (0.3)\\
Ref.\cite{Conant_1979} &    113.2   &       117.7   &       112.0   &       130.0\\
\hline\\
Source &    CCON    &       CONO1   &       CONO2    &   CCC1\\
This work &  169.9 (0.3) &   3.4 (0.0) &     176.6 (-0.1) & 108.1 (0.4)\\
Ref.\cite{Conant_1979} &   169.4   &       3.4     &       176.8        & 107.7\\
\hline\\
Source &  CCC2 &  O$\cdot\cdot\cdot$H   &       O$\cdot\cdot\cdot$H   &       O$\cdot\cdot\cdot$H\\
This work &  112.2 (-0.4) & 2.396 (-3.5) &   2.779 (-0.7) & 2.898 (0.1)\\
Ref.\cite{Conant_1979} &   112.7 & 2.482   &       2.798   &       2.894\\
\hline\hline
\end{tabular}
\end{center}
\caption{Comparison of selected intra- and intermolecular parameters for
PETN crystal.  Units are \AA\ and degrees.  Numbers in parentheses
are percent errors.
}
\label{tab:table2}
\end{table}

\begin{table}[p]
\begin{center}
\begin{tabular}{lllll}
\hline\hline
$\Bulk$(GPa) &   $\BPrime$   &   Source  &    Fitting form &      Comment \\
\hline
%14.6 & 5.6 &  This work & Eq.~(\ref{eq:Murna}) & For $ P \le 10.3 $GPa\\
%13.2 & 7.4 &  This work &  Eq.~(\ref{eq:BM})       &                --- \\
%14.8 & 5.4 &  This work &  Eq.~(\ref{eq:cs})    &                  --- \\
% 6.1 &18.7 &  This work &  hugoniot  &                --- ; quadratic fit \\
15.8 & 5.3 & This work &  Eq.~(\ref{eq:Murna}) & For $P \le 10.45$~GPa \\
14.5 & 6.7 & This work & Eq.~(\ref{eq:BM}) & For $P \le 10.45$~GPa \\
16.0 & 5.2 & This work & Eq.~(\ref{eq:cs}) & For $P \le 10.45$~GPa \\
11.7 & 6.8 & Olinger {\it et al.}\cite{Olinger_1975v62} & Eq.~(\ref{eq:Murna}) \\
9.4 & 11.3 & Olinger {\it et al.}\cite{Olinger_1975v62} & Eq.~(\ref{eq:BM}) \\
12.2 & 6.4 & Olinger {\it et al.}\cite{Olinger_1975v62} & Eq.~(\ref{eq:cs}) \\
8.8 & 9.9 &   Olinger {\it et al.}&   &        From Conference proceeding{\bf REF?} \\
 9.1 & --- &   Winey and Gupta\cite{Winey_2001v90} &     & Calculated from isentropic value \\
                                     %  value using $\Bulk_t=\Bulk_s*(C_v/C_p)$
                                     %  with $K_s=9.85$~GPa, $C_v=1.00$~J/g/K,
                                     %  and Cp=1.08 J/g/K.
14.1 &10.4 &Sorescu {\it et al.}\cite{Sorescu_1999v103} & Eq.~(\ref{eq:Murna})&         NPT-MD at 298K\\
\hline\hline
\end{tabular}
\end{center}
\caption{Calculated and measured bulk moduli $\Bulk$ and pressure
derivative $\BPrime$ for PETN.
}
\label{tab:table3}
\end{table}

\bibliographystyle{apsrmp} \bibliography{mondo}

\end{document}

