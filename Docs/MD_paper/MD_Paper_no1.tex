% NOTE TO AIP TYPSETERS: TO CONVERT FROM TWO-COL TO PREPRINT, SWITCH
% COMMENTOUT COMMAND FROM A TO B IE. USE:
%
\newcommand{\commentoutA}[1]{}
\newcommand{\commentoutB}[1]{#1}

%\newcommand{\commentoutA}[1]{#1}
%\newcommand{\commentoutB}[1]{}

\commentoutA{\documentclass[prb,aps,nobibnotes,twocolumn,doublespace,twocolumngrid,superbib,showpacs]{revtex4}}
\commentoutB{\documentclass[prb,aps,nobibnotes,superbib,showpacs,preprint]{revtex4}}

\usepackage{graphicx}
\usepackage{amsfonts}
\usepackage{amsmath}
\usepackage{bm}
\usepackage{alltt}
\usepackage{dcolumn} 
\usepackage{graphicx}
\makeatletter 
\makeatother


\begin{document}

\date{\today}

\title{Linear Scaling ab initio Molecular Dynamics with Gaussian Orbitals:
Non-Orthogonal Density Matrix Dynamics }

\author{C.~J.~Tymczak}
\affiliation{Theoretical Division, Los Alamos National Laboratory, Los Alamos, New Mexico 87545 }
\author{Matt Challacombe}
\affiliation{Theoretical Division, Los Alamos National Laboratory, Los Alamos, New Mexico 87545 }

\author{Chee Kwan Gan}
\affiliation{}


\begin{abstract}
We report on the application of the molecular dynamics method to the linear scaling , electronic structure, 
Gaussian orbital based code, MondoSCF.  When applying the molecular dynamics method to electronic structure 
code which use non-orthogonal local basis functions, a major difficulty is illuminated. Within the  
Born-Openheimer approximation, a systematic drift in the total energy is observed [P. Pulay and G. Fogarasi, 
C. Phys Lett., 386 (2004) p. 272-278]. This energy drift is due to a systematic error in the density matrix 
when the previous density matrix is used to compute a gueuss of the new density matrix. We illustrate this 
problem with two examples, and then we illuminate a simple solution which eliminates it.
\\ \smallskip
\noindent{\bf Keywords}: Self-consistent-field, linear-scaling, periodic systems, ab-initio molecular dynamics 

%
\end{abstract}

\pacs{}

\maketitle

\footnotetext[1]{\tt LA-UR}
\footnotetext[2]{\tt tymczak@lanl.gov}

\section{INTRODUCTION}

\section{Method}

\section{Examples}

\section{Conclusions}
\section*{ACKNOWLEDGMENTS}

We would like to acknowledge Tommy Sewell and Ed Kober for their advice
and support. We would also like to thank Chee Kwan Gan and Anders Niklasson 
for a careful reading of this manuscript.\cite{ANiklasson02A,ANiklasson03}

\bibliographystyle{apsrmp} 
\bibliography{mondo_new} 

\end{document}
