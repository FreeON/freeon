% NOTE TO AIP TYPSETERS: TO CONVERT FROM TWO-COL TO PREPRINT, SWITCH
% COMMENTOUT COMMAND FROM A TO B IE. use
% \newcommand{\commentoutA}[1]{}
% \newcommand{\commentoutB}[1]{#1}
% instead of the following
\newcommand{\commentout}[1]{}
\newcommand{\commentoutA}[1]{#1}
\newcommand{\commentoutB}[1]{}
%\renewcommand{\thefootnote}{\fnsymbol{footnote}}

\commentoutA{\documentclass[prb,aps,twocolumn,twocolumngrid,superbib]{revtex4}}
%\commentoutA{\documentclass[prl,aps,nobibnotes,twocolumn,doublespace,twocolumngrid,superbib,showpacs]{revtex4}}
\commentoutB{\documentclass[prl,aps,nobibnotes,superbib,showpacs,preprint]{revtex4}}

\usepackage{graphicx}
\usepackage{amsfonts}
\usepackage{amsmath}
\usepackage{bm}
\usepackage{alltt}
\usepackage{dcolumn} 
\usepackage{graphicx}
\makeatletter 
\makeatother

\begin{document}

\date{\today}

\title{Linear Scaling ab initio Molecular Dynamics with Gaussian Orbitals:
Non-Orthogonal Density Matrix Dynamics }

\author{C.~J.~Tymczak}
\affiliation{Theoretical Division, Los Alamos National Laboratory, Los Alamos, New Mexico 87545 }
\author{Matt Challacombe}
\affiliation{Theoretical Division, Los Alamos National Laboratory, Los Alamos, New Mexico 87545 }

\begin{abstract}
We report on the application of the molecular dynamics method to the linear scaling , electronic structure, 
Gaussian orbital based code, MondoSCF.  When applying the molecular dynamics method to electronic structure 
code which use non-orthogonal local basis functions, a major difficulty is illuminated. Within the  
Born-Openheimer approximation, a systematic drift in the total energy is observed [P. Pulay and G. Fogarasi, 
C. Phys Lett., 386 (2004) p. 272-278]. This energy drift is due to a systematic error in the density matrix 
when the previous density matrix is used to compute a gueuss of the new density matrix. We illustrate this 
problem with two examples, and then we illuminate a simple solution which eliminates it.
\noindent{\bf Keywords}: 

%
\end{abstract}

\pacs{}

\maketitle

\footnotetext[1]{}
\footnotetext[2]{}

\section{INTRODUCTION}

\section{Method}

\section{Examples}

\section{Conclusions}

\end{document}
