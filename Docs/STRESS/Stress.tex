%
%   This file is part of the APS files in the REVTeX 4 distribution.
%   Version 4.0 of REVTeX, August 2001
%
%   Copyright (c) 2001 The American Physical Society.
%
%   See the REVTeX 4 README file for restrictions and more information.
%
% TeX'ing this file requires that you have AMS-LaTeX 2.0 installed
% as well as the rest of the prerequisites for REVTeX 4.0
%
% See the REVTeX 4 README file
% It also requires running BibTeX. The commands are as follows:
%
%  1)  latex apssamp.tex
%  2)  bibtex apssamp
%  3)  latex apssamp.tex
%  4)  latex apssamp.tex
%
%\documentclass[prb,aps,nobibnotes,twocolumn,doublespace,twocolumngrid,superbib]{revtex4}
%%\documentclass[twocolumn,showpacs,preprintnumbers,amsmath,amssymb]{revtex4}
%\documentclass[preprint,showpacs,preprintnumbers,amsmath,amssymb]{revtex4}

% Some other (several out of many) possibilities
%\documentclass[preprint,aps]{revtex4}
%\documentclass[preprint,aps,draft]{revtex4}
%\documentclass[prb]{revtex4}% Physical Review B

%%\usepackage{amsmath}
%%\usepackage{amssymb}
%%\usepackage{graphicx}% Include figure files
%%\usepackage{dcolumn}% Align table columns on decimal point
%%\usepackage{bm}% bold math

%\documentclass[prb,aps,twocolumn,showpacs,twocolumngrid,superbib]{revtex4}
\documentclass[pra,aps,twocolumn,showkeys,twocolumngrid,superbib]{revtex4}

\usepackage{graphicx}
\usepackage{amsfonts}
\usepackage{amsmath}
\usepackage{bm}
\usepackage{alltt}
\usepackage{fancyhdr}

\pagestyle{fancy}

\def\n{\mathbf{n}}
\def\m{\mathbf{m}}
\def\0{\mathbf{0}}
\def\r{\mathbf{r}}
\def\R{\mathbf{R}}
\def\A{\mathbf{A}}
\def\B{\mathbf{B}}
\def\Tr{{\rm Tr}}
%\nofiles

\begin{document}

%\preprint{APS/123-QED}

\title{Analytical ??gradients?? and stress tensor within the $\Gamma$-point approximation}

\author{Christopher J. Tymczak}%
\author{Matt Challacombe}%
\affiliation{Los Alamos National Laboratory, Theoretical Division, Los Alamos 87545, New Mexico, USA.}%
\author{Val\'ery Weber}
\email{valery.weber@unifr.ch}
\affiliation{Department of Chemistry, University of Fribourg, 1700 Fribourg, Switzerland.}%


\date{\today}% It is always \today, today,
             %  but any date may be explicitly specified

\begin{abstract}
  Recently, construction of the periodic Coulomb cite, Exchange-Correlation cite 
  and exact Hartree-Fock exchange cite matrices within the $\Gamma$-point approximation
  have been introduiced. In this article, the formalism for the evaluation of 
  the analytical Hartree-Fock and Kohn-Sham stress at the $\Gamma$-point approximation is
  presented. 
\end{abstract}

%\pacs{Valid PACS appear here}% PACS, the Physics and Astronomy
                             % Classification Scheme.
\keywords{Analytical stress tensor, periodic systems, Hartree-Fock, 
  density functional theory.}%Use showkeys class option if keyword
                              %display desired
\maketitle

\section{Introduction}

In preceding papers, we have developed linear scaling quantum chemical methods
for construction of the periodic Coulomb~\cite{CTymczak03a}, 
Exchange-Correlation~\cite{CTymczak03a} and 
exact Hartree-Fock exchange~\cite{CTymczak03b} matrices within the $\Gamma$-point approximation. 
In this paper, the implementation of the Hartree-Fock
and Kohn-Sham stress tensor at the $\Gamma$-point is presented.

Stress is an important concept in characterizing the states of condensed
matter. The simulation of system with PBC involves the atomic positions 
in the unit cell and the lattice parameters. The location of an energy minimum 
requires an optimization of both sets of coordinates. This minimization can be
achived with the help of the periodic nuclear gradients and the stress tensor.
The latter represents the derivatives of the energy of a periodic system 
with respect to the strain tensor.

                                  
The first implementation of the Hartree-Fock stress tensor based on GTAO,
were for one dimensional periodic systems~\cite{HTeramae83,HTeramae84}. 
Other groups have also described such implementation for 
one~\cite{DJacquemin99A,DJacquemin99B} or two~\cite{MTobita03} dimensions. 
The analytical stress method of density functional theory using GTAO for 
1D extended systems was implemented by Hirata and Iwata~\cite{SHirata98}.
The three dimensional case has been implemented by Kudin and Scuseria 
~\cite{KKudin00A,KKudin00B}. Their approach for the Coulomb problem is 
based on the direct space fast multipole method.  
Recently Doll, Dovesi and Orlando~\cite{KDoll04} presented 
the implementation of the Hartree-Fock stress tensor into 
the CRYSTAL~\cite{RDovesi00} package for three dimensional systems. 
Their code is based on GTAO and the summation 
of the Coulomb energy is performed with the Ewald method~\cite{PEwald21} 
which is a combination of direct and reciprocal lattice summations.


In this article, we introduice the formalism and the implementaion
of the Hartree-Fock and Kohn-Sham stress tensor at the $\Gamma$-point 
approximation and present results on various systems. 
In the next Section, the integrals and their
derivatives with respect to cell parameters are discussed.....
Finally, some examples are shown to illustrate the method.


\section{Integrals and their derivatives}
$M$ is the $3\times3$ matrix composed of the primitive lattice vectors $\mathbf{a}$, 
$\mathbf{b}$ and $\mathbf{c}$,
\begin{equation}
  M=(\mathbf{a},\mathbf{b},\mathbf{c}).
\end{equation}
The position of the direct lattice is defined as $\mathbf{R}=M\n$,
with $\mathbf{n}=(n_1,n_2,n_3)^T$ being integer numbers.
The position of an atom $A$ in a cell is $\mathbf{A}=M(\mathbf{f}_A+\n)$,
and $\mathbf{f}_A=(f_{A1},f_{A2},f_{A3})^T$ are the fractional coordinates of the atom $A$.

The volume of the cell is given by the triple product 
$V=\mathbf{a}\cdot(\mathbf{b}\times \mathbf{c})=\det M$.
The derivative of the volume of the cell with respect to the cell parameters are given by
\begin{align}
  \frac{\partial V}{\partial M_{i1}}&=\epsilon_{ijk}M_{j2}M_{k3},\\
  \frac{\partial V}{\partial M_{i2}}&=\epsilon_{ijk}M_{j3}M_{k1},\\
  \frac{\partial V}{\partial M_{i3}}&=\epsilon_{ijk}M_{j1}M_{k2},
\end{align}
...a bit of tensorial masturbation...
where the permutation symbol $\epsilon_{ijk}$ has been used.


\subsection{Overlap and kinetic energy integrals}
The overlap matrix, in the periodic $\Gamma$-point regime, is given
\begin{equation}
  \begin{split}
    S_{ab}&=\sum_{\R}\int_{V_\infty}\phi_a(\r-\A)\phi_b(\r-(\B+\R))d^3r\\
    &=\sum_{\n}S_{ab}^{{\0\n}},
  \end{split}
\end{equation}
where the integration is extended to all the space.
The evaluation of its derivative with respect to the cell parameters $M_{ij}$
is 
\begin{equation}\label{Eq:OvDer}
  \frac{\partial S_{ab}}{\partial M_{ij}}=\sum_{\n}(f_{Aj}-(f_{Bj}+n_j))
  \frac{\partial S_{ab}^{{\0\n}}}{\partial A_i},
\end{equation}
where translational invariance cite has been used. 
In the sameway, the kinetic energy integral is
\begin{equation}
  \begin{split}
    T_{ab}&=\sum_{\R}\int_{V_\infty}\phi_a(\r-\A)\left(-\frac{1}{2}\nabla^2\right)
    \phi_b(\r-(\B+\R))d^3r\\
    &=\sum_{\n}T_{ab}^{{\0\n}},
  \end{split}
\end{equation}
and its derivatives
\begin{equation}
  \frac{\partial T_{ab}}{\partial M_{ij}}=\sum_{\n}(f_{Aj}-(f_{Bj}+n_j))
  \frac{\partial T_{ab}^{{\0\n}}}{\partial A_i}.
\end{equation}


\subsection{Hartree-Fock exchange matrix}



%\begin{verbatim}
%  FPQx = PQx*PBC%InvBoxSh%D(1,1)+PQy*PBC%InvBoxSh%D(1,2)+PQz*PBC%InvBoxSh%D(1,3)
%  FPQy = PQy*PBC%InvBoxSh%D(2,2)+PQz*PBC%InvBoxSh%D(2,3)
%  FPQz = PQz*PBC%InvBoxSh%D(3,3)
%  IF(PBC%AutoW%I(1)==1) THEN
%     Dumx=DNINT(FPQx-SIGN(1.D-14,FPQx))
%     FPQx=FPQx-DNINT(FPQx-SIGN(1.D-14,FPQx))
%  ENDIF
%  IF(PBC%AutoW%I(2)==1) THEN
%     Dumy=DNINT(FPQy-SIGN(1.D-14,FPQy))
%     FPQy=FPQy-DNINT(FPQy-SIGN(1.D-14,FPQy))
%  ENDIF
%  IF(PBC%AutoW%I(3)==1) THEN
%     Dumz=DNINT(FPQz-SIGN(1.D-14,FPQz))
%     FPQz=FPQz-DNINT(FPQz-SIGN(1.D-14,FPQz))
%  ENDIF
%  PQx = FPQx*PBC%BoxShape%D(1,1)+FPQy*PBC%BoxShape%D(1,2)+FPQz*PBC%BoxShape%D(1,3)
%  PQy = FPQy*PBC%BoxShape%D(2,2)+FPQz*PBC%BoxShape%D(2,3)
%  PQz = FPQz*PBC%BoxShape%D(3,3)
%\end{verbatim}

%Then we have $PQ_x^{w}=PQx$ and $[f_{PQx}^{nw}]=Dumx$.
%\\

%\begin{equation}
%  I_{acbd}=\sum_{{n,m}}\iint
%  \phi_a({r}_1)\phi_c({r}_1+M{n})
%  \frac{1}{|{r}_1-{r}_2|}
%  \phi_b({r}_2)\phi_d({r}_2+M{m})
%  d^3r_1d^3r_2=\sum_{{n,m}}I_{acbd}^{{0n0m}}
%\end{equation}

%\begin{equation}
%  \begin{split}
%    \frac{\partial I_{acbd}}{\partial M_{ij}}=&\sum_{{n,m}}
%    (f_{Aj}-(f_{Dj}+m_j))\frac{\partial I_{acbd}^{{0n0m}}}{\partial A_{i}}\\
%    &+(f_{Cj}+n_j-(f_{Dj}+m_j))\frac{\partial I_{acbd}^{{0n0m}}}{\partial C_{i}}\\
%    &+(f_{Bj}-(f_{Dj}+m_j))\frac{\partial I_{acbd}^{{0n0m}}}{\partial B_{i}}\\
%    &+2\Omega PQ_i^{w}[f_{PQj}^{nw}] \widetilde I_{acbd}^{{0n0m}}
%  \end{split}
%\end{equation}
In computation of the Hartree-Fock exchange matrix, 
the MIC $\Gamma$-point approximation~\cite{CTymczak03b}
is given 
\begin{equation}
  K_{ab}= -\frac{1}{2}\sum_{\n\m,cd}P_{cd}(ac^\n|bd^\m)
%  \sum_{cd}P_{cd}I_{acbd}
%  \sum_{{n,m}}\iint
% \phi_a({r}_1)\phi_c({r}_1+M{n})
% \frac{1}{|{r}_1-{r}_2|}
% \phi_b({r}_2)\phi_d({r}_2+M{m})
% d^3r_1d^3r_2=
\end{equation}
where the MIC condition is appiled at the primitive level.
The derivative of the integrals $(ac^\n|bd^\m)$, with respect 
to cell parameters $M_{ij}$, is given
\begin{equation}
  \begin{split}
%    \frac{\partial I_{acbd}}{\partial M_{ij}}
    \frac{\partial (ac^\n|bd^\m)}{\partial M_{ij}}
%    =&\sum_{\n,\m}(f_{Aj}-(f_{Dj}+m_j))\frac{\partial (ac(\n)|bd(\m))}{\partial A_{i}}\\
    &=(f_{Aj}-(f_{Dj}+m_j))\frac{\partial (ac^\n|bd^\m)}{\partial A_{i}}\\
    &+(f_{Cj}+n_j-(f_{Dj}+m_j))\frac{\partial (ac^\n|bd^\m)}{\partial C_{i}}\\
    &+(f_{Bj}-(f_{Dj}+m_j))\frac{\partial (ac^\n|bd^\m)}{\partial B_{i}}\\
    &+2\rho PQ_i^{w}[f_{PQj}] (ac^\n|bd^\m)^{(1)},
  \end{split}
\end{equation}
where we have used the translational invariance and 
$\mathbf{f}_{PQ}=M^{-1}\mathbf{PQ}$
is a non-wrapped ``fractional`` distance, 
$\mathbf{PQ}^{w}=\mathbf{PQ}-M[\mathbf{f}_{PQ}-\varepsilon \mathrm{sgn}(\mathbf{f}_{PQ})]$ 
is a wrapped distance, 
$\rho=(\alpha+\gamma)(\beta+\delta)/(\alpha+\gamma+\beta+\delta)$ and 
$(ac^\n|bd^\m)^{(1)}$ is a byproduct in the Obara and Saika~\cite{SObara86} 
reccurence relation when computing first nuclear derivatives.
Where $[\bullet]$ is the nearest integer function.

\subsection{Coulomb matrix}

\begin{equation}
  J_{ab}=J_{ab}^{In}+J_{ab}^{PFF}+J_{ab}^{TF}
\end{equation}


\subsubsection{Periodic far field}

\begin{equation}
  J_{ab}^{PFF}=\sum_{R\in PFF}\iint
  \frac{\rho_{ab}^{\infty}(r_1)\rho_{tot}^{\infty}(r_2)}{|r_1-r_2+R|}
  d^3r_1d^3r_2
\end{equation}

\subsubsection{Tin-Foil}
The tin-foil correction to the Coulomb matrix is
\begin{equation}
  J_{ab}^{TF}=\frac{2\pi}{3V}(QS_{ab}-2\mathbf{d}_{ab}\cdot\mathbf{D})
\end{equation}
where $Q$ is the trace of the system quadrupole, $S_{ab}$ is an element of the overlap matrix, 
$\mathbf{D}$ the dipole moment of the system,
$\mathbf{d}_{ab}$ the dipole moment of the distribution $\rho_{ab}^\infty$.
The derivative of $J_{ab}^{TF}$ with respect to the cell parameter $M_{ij}$ is
\begin{equation}
  \begin{split}
    \frac{\partial J_{ab}^{TF}}{\partial M_{ij}}&=
    -\frac{2\pi}{3V^2}\frac{\partial V}{\partial M_{ij}}(QS_{ab}-2\mathbf{d}_{ab}\cdot\mathbf{D})\\
    &+\frac{2\pi}{3V}(\frac{\partial Q}{\partial M_{ij}}S_{ab}+Q\frac{\partial S_{ab}}{\partial M_{ij}}\\
    &-2\frac{\partial \mathbf{d}_{ab}}{\partial M_{ij}}\cdot\mathbf{D}
    -2\mathbf{d}_{ab}\cdot\frac{\partial \mathbf{D}}{\partial M_{ij}}),
  \end{split}
\end{equation}
where the derivatives ${\partial Q}/{\partial M_{ij}}$, 
${\partial \mathbf{d}_{ab}}/{\partial M_{ij}}$ and 
${\partial \mathbf{D}}/{\partial M_{ij}}$ can be easily evaluated; 
the term ${\partial S_{ab}}/{\partial M_{ij}}$ 
is given in (\ref{Eq:OvDer}).

\subsection{Exchange-correlation matrix}
In calculation with PBC, the numerical integration of the DFT 
exchange-correlation term is carried out over a cuboid 
integration domain $V_{\Box}$ which is equivalent to the unit 
cell volume $V_{UC}$. The exchange-correlation matrix element is thus
\begin{equation}\label{Eq:xc}
  K_{ab}^{xc}=\int_{V_\Box}\rho_{ab}(\r)v_{xc}[\rho,\r]d^3r.
\end{equation}
This simple integration volume should be contrasted with more 
conventional methods for computing the exchange-correlation matrix, involving
the ``Becke weights'' cite, which requirs integration over $V_\infty$.
The derivative of $K_{ab}^{xc}$ with respect to cell parameters is given by
\begin{equation}
  \begin{split}
    \frac{\partial K_{ab}^{xc}}{\partial M_{ij}}&=
    \int_{V_{\Box}}\frac{\partial}{\partial M_{ij}}(\rho_{ab}(\r)v_{xc}[\rho,\r])d^3r\\
    &+\delta_{ij}\int_{S_{j\Box}}\rho_{ab}(\r)v_{xc}[\rho,\r]dS
  \end{split}
\end{equation}
where the intergale on the surface $S_{j\Box}$ arises 
from the derivative of the limits of the integral (\ref{Eq:xc}).



\section{The stress}

A straightforward differentiation of Eq. \ref{Eq:E} with respect to 
the cell parameter $M_{ij}$ gives
\begin{equation}
  \begin{split}
    \frac{\partial E}{\partial M_{ij}}&=\Tr[P\frac{\partial T}{\partial M_{ij}}]
    -\Tr[PFP\frac{\partial S}{\partial M_{ij}}]\\
    &+\Tr[P\frac{\partial J}{\partial M_{ij}}]+\Tr[P\frac{\partial K}{\partial M_{ij}}]\\
    &+\Tr[P\frac{\partial K^{xc}}{\partial M_{ij}}],
  \end{split}
\end{equation}
where all the derivatives have been identify and 
explicitly given in the previous Section.


\section{Numerical Examples}
The evaluation of the analytical Hartree-Fock and Kohn-Sham 
nuclear gradients and stress have been implemented in the 
MondoSCF~\cite{MondoSCF} suite of programs
for linear scaling quantum chemistry. The code was compiled using....
with the ... options and with the GNU C compiler GCC v....
using the -O1 flag. All calculations were carried out on a ...
runing ....\cite{TOPUT}
\\
\\
We may try MgO 1x1x1 to 5x5x5 and compare with CRYSTAL.

\section{Conclusions}
We blablabla...


%%%%%%%%%%%%%%%%%%%%%%%%%%%%%%%%%%%%%%%%%%%%%%%%%%%%%%%%%%%%%%%%
%%%%%%%%%%%%%%%%%%%%%%%%%%%%%%%%%%%%%%%%%%%%%%%%%%%%%%%%%%%%%%%%
%Acknowledgements
\begin{acknowledgments}
 V.W. would like to thank K. Doll for a preprint of his manuscript.
 This work has been supported by the US Department of Energy 
 under contract ???????????? and the ASCI project.  
 The Advanced Computing Laboratory of Los 
 Alamos National Laboratory is acknowledged.
 All the numerical computations have been
 performed on computing resources located at this facility.
\end{acknowledgments}
%%%%%%%%%%%%%%%%%%%%%%%%%%%%%%%%%%%%%%%%%%%%%%%%%%%%%%%%%%%%%%%%
%%%%%%%%%%%%%%%%%%%%%%%%%%%%%%%%%%%%%%%%%%%%%%%%%%%%%%%%%%%%%%%%
\bibliography{mondo_new}
%%%%%%%%%%%%%%%%%%%%%%%%%%%%%%%%%%%%%%%%%%%%%%%%%%%%%%%%%%%%%%%%
%%%%%%%%%%%%%%%%%%%%%%%%%%%%%%%%%%%%%%%%%%%%%%%%%%%%%%%%%%%%%%%%
%%%%%%%%%%%%%%%%%%%%%%%%%%%%%%%%%%%%%%%%%%%%%%%%%%%%%%%%%%%%%%%%
%%%%%%%%%%%%%%%%%%%%%%%%%%%%%%%%%%%%%%%%%%%%%%%%%%%%%%%%%%%%%%%%
%%%%%%%%%%%%%%%%%%%%%%%%%%%%%%%%%%%%%%%%%%%%%%%%%%%%%%%%%%%%%%%%
\end{document}
%
% ****** End of file apssamp.tex ******




@Article{KDoll04,
  author = 	 {K. Doll and R. Dovesi and R. Orlando},
  journal = 	 {Theo. Chim. Acta},
  year = 	 {2004},
  note = 	 {submitted},
}

@Article{PEwald21,
        author = {P. P. Ewald},
        journal = {Ann. Phys. (Leipzig)},
        year = 1921,
        volume = 64,
        pages = {253}
}

@Article{HTeramae83,
  author = 	 {H. Teramae and T. Yamabe and C. Satoko and A. Imamura},
  journal = 	 {Chem. Phys. Lett.},
  year = 	 {1983},
  volume = 	 {101},
  pages = 	 {149},
}

@Article{HTeramae84,
  author = 	 {H. Teramae and T. Yamabe and A. Imamura},
  journal = 	 {J. Chem. Phys.},
  year = 	 {1984},
  volume = 	 {81},
  pages = 	 {3564},
}

@Article{DJacquemin99A,
  author = 	 {D. Jacquemin and J. Andr\'e and Benoît Champagne },
  journal = 	 {J. Chem. Phys.},
  year = 	 {1999},
  volume = 	 {111},
  pages = 	 {5306},
}

@Article{DJacquemin99B,
  author = 	 {D. Jacquemin and J. Andr\'e and Benoît Champagne },
  journal = 	 {J. Chem. Phys.},
  year = 	 {1999},
  volume = 	 {111},
  pages = 	 {5324},
}

@Article{MTobita03,
  author = 	 {M. Tobita and S. Hirata and R. J. Bartlett},
  journal = 	 {J. Chem. Phys.},
  year = 	 {2003},
  volume = 	 {118},
  pages = 	 {5776},
}

@Article{SHirata98,
  author = 	 {S. Hirata and S. Iwata},
  journal = 	 {J. Phys. Chem. A},
  year = 	 {1998},
  volume = 	 {102},
  pages = 	 {8426},
}

@Article{KKudin00A,
  author = 	 {K. N. Kudin and G. E. Scuseria},
  journal = 	 {Phys. Rev. B},
  year = 	 {2000},
  volume = 	 {61},
  pages = 	 {5141},
}

@Article{KKudin00B,
  author = 	 {K. N. Kudin and G. E. Scuseria},
  journal = 	 {Phys. Rev. B},
  year = 	 {2000},
  volume = 	 {61},
  pages = 	 {16440},
}

@Article{TOPUT,
  author = 	 {},
  journal = 	 {},
  year = 	 {},
  volume = 	 {},
  pages = 	 {},
}


