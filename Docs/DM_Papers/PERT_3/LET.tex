\documentclass[12pt]{letter}
%
% Define page margins
%
%\setlength{\oddsidemargin}{0.0cm}
%\setlength{\topmargin}{-3.5cm}
%\setlength{\textwidth}{16.5cm}
%\setlength{\textheight}{26.5cm}
%
% Define the font for the letter : minion
% Helvetica is used for the footer and the logo
%
\usepackage{minion,helvetic}
%
% Allow the specification of the spacing
% Important for the logo
%
\usepackage{setspace}
%
% A couple of font commands
%
\makeatletter \newcommand\larger{\@setfontsize\larger{27.0}{32.56}} \makeatother
\makeatletter \newcommand\smaller{\@setfontsize\smaller{8.65}{10.43}} \makeatother
\makeatother
% 
% This bit does the logo
%
\renewcommand*{\opening}[1]{\ifx\@empty\fromaddress
  \thispagestyle{firstpage}%
    {\raggedleft\par}%
  \else  % home address
    \thispagestyle{fancy}%
    {
\begin{spacing}{0.75}
\noindent{\larger\textbf{\textsf{Los Alamos}}}\\
\textsf{\smaller
\begin{tabular}{@{\hspace{0.2mm}}l@{\hspace{0.9mm}}l@{\hspace{0.9mm}}l@{\hspace{0.9mm}}l@{\hspace{0.9mm}}l@{\hspace{0.9mm}}l@{\hspace{0.9mm}}l@{\hspace{0.9mm}}l@{\hspace{0.9mm}}l@{\hspace{0.9mm}}l@{\hspace{0.9mm}}l@{\hspace{0.9mm}}l@{\hspace{0.9mm}}l@{\hspace{0.9mm}}l@{\hspace{0.9mm}}l@{\hspace{0.9mm}}l@{\hspace{0.9mm}}l@{\hspace{0.9mm}}l@{\hspace{0.9mm}}l}
N & A & T & I & O & N & A & L & & L & A & B & O & R & A & T & O & R & Y \\
\end{tabular}
}
\end{spacing}
\vspace{0.4cm}

%
% This bit does our address and phone numbers
%
\begin{spacing}{1.0}
\small
\noindent{\textbf{Theoretical Chemistry and
    Molecular Physics}\hfill\textit{Date: \today}\\
Group T-12, Mail Stop B268\\
Los Alamos, New Mexico 87545\\}
\noindent{\begin{tabular}{@{}l@{}l}
Tel: &(505) 665-5905\\
Fax: &(505) 665-3909\\
\end{tabular}
}
\end{spacing}
}%
  \fi
  \vspace{2\parskip}%
  {\raggedright \toname \\ \toaddress \par}%
  \vspace{2\parskip}%
  #1\par\nobreak}
\makeatletter
%
% This bit does the footer
%
\usepackage{fancyheadings}
\pagestyle{fancy}
\cfoot{\sffamily\small An Equal Opportunity Employer/Operated by the
  University of California}
\renewcommand{\headrulewidth}{0pt}
\addtolength{\headheight}{2.5pt}
\begin{document}

\begin{letter}{
Professor Ernie Davidson\\
Journal of Chemical Physics
}

\signature{Matt Challacombe, \\
           505-665-5905 \\
           MChalla@LANL.Gov\\
           MS-B268, LANL\\
           Los Alamos, NM 87545}

\opening{Dear Professor Davidson,}
We are resubmitting our paper A4.12.222 with minor revisions, per the referees comments.
We have incorporated the referee's (excellent) comments to the best of our ability, except
for items 1, 5 and 8, which we believe involve simple misunderstanding by the referee.
Changes are as follows:

\begin{itemize}
\item[1] No change, we actually do demonstrate linear scaling for three-dimensional water clusters.
         The linear chains are used only to (a) make a connection with the literature (Ref.~[55])
         and (b) to demonstrate error control.  Since separation is achieved rapidly in one-dimension,
         this is appropriate.
\item[2] We have converted our equations to be overall consistent with the Taylor's expansion.
\item[3] We added a comment on the guess we use to start the CPSCF equations, which is just the core Hamiltonian. 
\item[4] Equation (22) has been fixed.  
\item[5] No change. We give these equations without proof; proof is beyond the scope of this paper. 
\item[6] We have made use of the $n+1$ and $2n+1$ rule clear.

\pagebreak 

\item[7] We point out in the discussion that small gap semiconducting polymers like 
         polymethineimine or polydiacetylene will have a later onset of linear scaling, 
         relative to water chains.  Regardless of where the onset occurs,  the proof of principle
         is clear.   
\item[8] No change. Figure 6 demonstrates the increasing delocalization of the derivative density matrix with 
         order in the perturbation.  This constitutes a central result of this paper, and should be
         rendered in color to show this effect.
\end{itemize}


{\bf COLOR}\\
We would like Figure 6 to appear in color.  This is because
there are overlapping distributions that can only be identified
with color.


{\bf POSITIONING}\\
We would like this article to appear immediately after its companion 
piece ``Non-orthogonal Density Matrix Perturbation Theory'', recently resubmitted by 
Anders Nilasson.

\closing{Sincerely,}

\end{letter}
\end{document}

%%% Local Variables: 
%%% mode: latex
%%% TeX-master: t
%%% End: 

