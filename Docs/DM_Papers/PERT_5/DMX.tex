\documentclass[twocolumn,showpacs,preprintnumbers,amsmath,amssymb]{revtex4}
%\documentclass[preprint,showpacs,preprintnumbers,amsmath,amssymb]{revtex4}
% Some other (several out of many) possibilities
%\documentclass[preprint,aps]{revtex4}
%\documentclass[preprint,aps,draft]{revtex4}
%\documentclass[prb]{revtex4}% Physical Review B

\usepackage{graphicx}% Include figure files
\usepackage{dcolumn}% Align table columns on decimal point
\usepackage{bm}% bold math
\bibliographystyle{apsrev}

%\nofiles

\begin{document}

\preprint{LA-UR 04-6956}

%\title{Non-orthogonal density matrix extrapolation by perturbed projection}
%\title{Recursive non-orthogonal density matrix extrapolation}
%\title{Extrapolation in geometry optimization by non-orthogonal trace correcting density matrix purification}
\title{Trace correcting density matrix extrapolation}


\author{Anders M. N. Niklasson$^{\dagger}$, Karloy Nemeth, and Matt Challacombe}
\affiliation{Theoretical Division, Los Alamos National Laboratory, Los Alamos, New Mexico 87545}

\date{\today}

\begin{abstract}
A non-orthogonal density matrix extrapolation method is proposed for geometry optimization 
and molecular dynamics calculations in linear scaling electronic structure theory.
The technique is based on non-orthogonal trace correcting purification and 
perturbation theory (Niklasson, Phys.\ Rev.\ B {\bf 68}, 155115 (2002); 
Niklasson and Challacombe, Phys.\ Rev.\ Lett.\ {\bf 92}, 193001 (2004);
Niklasson, Weber, and Challacombe cond-mat/ xxx).
The proposed extrapolation method is shown to give approximate total energies 
often an order of magnitude closer to the self-consistent solution compared 
to alternative schemes. The extrapolation is performed in a non-orthogonal
representation and avoids transformations between orthogonal and 
non-orthogonal representations.  For insulators the computational complexity 
of the extrapolation method scales linearly with the size of the perturbed 
region affected by the modified geometry $O(N_{\rm pert.})$. For local 
perturbations the computational cost is therefore independent of the total 
size of the system and scales as $O(1)$.
\end{abstract}

\pacs{02.70.-c, 31.15.-p, 71.15.-m, 71.15.Dx}% PACS, the Physics and Astronomy
                             % Classification Scheme.
\keywords{electronic structure theory, geometry optimization, density matrix, O(N) complexity,
extrapolation, molecular dynamics, response, perturbation theory, overlap matrix,
density matrix extrapolation, linear scaling electronic structure theory, purification, 
sign matrix, O(N), basis set, orthogonal, non-orthogonal, transformation, congruence transform,
matrix inverse, inverse Cholesky decomposition, inverse square root, Cholesky factorization}
\maketitle

\section{Introduction}

In {\it ab initio} electronic structure theory the problem of calculating 
properties of systems consisting of millions of atoms is a great 
computational challange.  During the last decade a new approach has evolved that
enforces techniques with a computational complexity that scales
linearly $O(N)$ with the system size $N$ \cite{Yang91,Galli92,Mauri93,Ordejon93,Li93,Stechel94,Goedecker94,Kim95,Wang95,Abrikosov96,Galli96,Kohn96,Bowler97,Sanches97,Yokojima98,Baer98,Guerra98,Goedecker99,Artacho99,Scuseria99,Ordejon00,Wu02,Yam03}.
For small systems this is usually an inefficient approach, but for large complex 
problems it is pivotal.

An important problem in self-consistent electronic structure theory is the ability to 
efficiently treat modifications in the electronic density due to changes in 
the geometry. These modifications may occur either in molecular dynamics
simulations or in geometry optimizations.  The ability to give an accurate estimate 
of a new approximate density for a new modified structure, by an extrapolation from previous 
densities, often substantially reduces the number of iterations necessary 
to reach convergence in a self-consistent field (SCF) calculation. In this article 
we present an $O(N)$  density matrix extrapolation method 
that can be used, for example, in geometry optimization or molecular dynamics calculations.
The technique is based on the trace correcting purification and perturbation method 
for a non-orthogonal representation \cite{NiklassonTC2,NiklassonPRT1,WeberPRT2,NiklassonNO}.
The linear scaling efficiency of these methods is based on the approximate 
exponential decay of the density matrix elements as a function of interatomic 
separation for non-metallic materials. Matrix operations can therefore (for sufficiently 
large and sparse systems) be performed with a computational cost scaling linearly
with system size \cite{NiklassonTRS4,WeberPRT2}. The computational cost of the proposed density matrix
extrapolation scheme has the same property and for insulators the computational complexity 
scales linearly with the size of the perturbed region affected by the modified
geometry $O(N_{\rm pert.})$. For local perturbations the computational cost 
is therefore independent of the total size of the system and scales as $O(1)$ \cite{NiklassonPRT1}.

\section{Density matrix extrapolation}

The density matrix extrapolation problem for a non-orthogonal density matrix representation 
can be stated as follwos; Find a non-orthogonal ``extrapolated'' density matrix $P_1$ for 
a new modified structure with overlap matrix $S_1$ from the previous non-orthogonal 
density matrix $P_0$ with corresponding overlap matrix $S_0$, such that
\begin{equation}\label{XTRP}
\begin{array}{l}
Tr(P_1S_1) = N_e, \\
P_1S_1P_1  = P_1, ~~{\rm and}\\
P_1 = P_0 \quad {\rm iff} ~~ S_1 = S_0. \\
\end{array}
\end{equation}
Here $N_e$ is the number of occupied states.
This extrapolation problem in Eq.\ (\ref{XTRP}) has an infinite number of solutions. 
An additonal critereon; that of energy minimization and commutation with the new Hamiltonian 
constructed from the extrapolated density matrix, would yield the exact and unique self-consistent 
density matrix. This exact self-consistent solution can in general never be found by the extrapolation, 
which only gives an approximate estimate. The efficiency of density matrix  extrapolation can be 
determined by how close the approximate density matrix is to the exact self-consistent solution.

\subsection{L\"{o}wdin and inverse Cholesky extrapolation}

The trace correcting extrapolation algorithm proposed in this article will be compared with two 
alternative density matrix extrapolation methods solving the extrapolation problem in 
Eq.\ (\ref{XTRP}). The first method, 
the L\"{o}wdin ($L$) extrapolation, is based on the $S_1^{-1/2}$ transformation of the previous 
density matrix in an orthogonal represenation \cite{Lowdin}, $P_0^{\perp} = S_0^{1/2} P_0 S_0^{1/2}$. 
The L\"{o}wdin extrapolation is given by
\begin{equation}\label{Low}
P_1^{(L)} = S_1^{-1/2} S_0^{1/2} P_0 S_0^{1/2} S_1^{-1/2}.
\end{equation}
Because of the square root of the overlap matrix and its inverse this method is 
computationally expensive.
The second approach is based on the inverse Cholesky factorisation ($Z$) of the overlap matrix, where
$Z_0^TS_0Z_0 = I$ and $Z_1^TS_1Z_1 = I$.
In this case we have the density matrix extrapolation
\begin{equation}\label{IC}
P_1^{(Z)} = Z_1 Z_0^{-1} P_0 Z_0^{-T} Z_1^{T}.
\end{equation}
This approach is computationally more efficient, thanks to the fast approximate Cholesky factorization
algorithm \cite{Benzi96,Challa99}. However, the inverse Cholesky factorization is difficult to
parallellize.
It is easy to see that both the L\"{o}wdin extraoplation $P_1^{(L)}$ and 
the Cholesky extrapolation $P_1^{(Z)}$ solves the extrapolation problem in Eq.\ (\ref{XTRP})
provided that $P_0$ is a solution for the previous structure. 

\subsection{Trace correcting extrapolation}

\subsubsection{Global extrapolation}

The extrapolation technique proposed in this article is based on 
the non-orthogonal trace correcting purification 
and perturbation methods recently introduced in linear scaling electronic structure 
theory \cite{NiklassonTC2,NiklassonPRT1,WeberPRT2,NiklassonTRS4,NiklassonNO}.
Assume a global change in the geometry and corresponding overlap matrix, 
from $S_0$ to a new modified geometry with the new ovelap $S_1$.
The density matrix extrapolation scheme, based on non-orthogonal 
trace correcting purification \cite{NiklassonNO}, is then given by
\begin{equation}\begin{array}{ll}\label{XTRPP}
X_{n+1} & = X_n + \sigma (I-X_nS_1)X_n, \\
\sigma & = sign[N_e - Tr(X_nS_1)], \end{array}
\end{equation}
where 
\begin{equation}\begin{array}{ll}
X_0 & = P_0, \\
P_1 & = \lim_{n \rightarrow \infty}.
\end{array}
\end{equation}
Close to convergence, when $Tr[(I-X_nS_1)X_n]$ is small, we chose to alternate the sign $\sigma$
between $+1$ and $-1$ every second iteration to improve convergence \cite{NiklassonNO}.
At convergence, it is easy to see that this scheme solves the density matrix extrapolation
problem given in Eq.\ (\ref{XTRP}).

A computational problem, arising in
the non-orthogonal purification and perturbation theory, where the density matrix
is constructed directly from a Hamiltonian, is the commutation between
the density matrix and the Hamiltonian in a non-orthogonal representation,
\begin{equation}
SPH - HPS = 0.
\end{equation}
Because of this requirement,
a matrix inversion, or inverse factorization, is necessary in the construction of
the initial normalized matrix in the recursive construction of the density matrix \cite{NiklassonNO}.
However, since the commutation relation never can be fullfilled exactely
for the extrapolated density matrix the problem never occurs. We can therefore
chose $X_0 = P_0$.

The extrapolation scheme above is based on trace correcting purification of the density matrix
in a non-orthogonal basis. Almost any other purification method, formulated for a non-orthogonal
representation, can be used alternatively. However, using a grand canonical
method \cite{Palser98,Holas01,NiklassonTC2}, for example the McWeeny purification, 
does not necessarily converge to a density matrix
with the correct occupation.  In addition, trace correcting purification is computationally 
more efficient and simple compared to most alternative purification projection methods \cite{NiklassonTC2,
Mazziotti03,NiklassonIPUR}. 

The proposed extrapolation technique works entirely within a non-orthogonal representation and 
in constrast to the L\"{o}wdin and Cholesky methods it avoids the
transformation between an orthogonal and non-orthogonal representations.
It uses only matrix-matrix multiplications, additions and subtractions, which can be
parallellized, and, as will be seen below in section \ref{EX}, it often gives approximations closer
to the exact self-consistent solution.

\subsubsection{Local extrapolation}

For a local perturbation of the geometry we can construct an equivalent 
local extrapolation scheme with the perturbation given by the change in the overlap matrix,
$dS = S_1 - S_0$. In this case the extrapolation scheme, with $\Delta_0 = 0$, is given by
\begin{equation}\label{XTR} \begin{array}{ll}
\Delta_{n+1} & = \Delta_n + \sigma (U_n - \Delta_n), \\
\sigma & = sign[ Tr([P_0 + \Delta_n)S_1]-N_e ], 
\end{array}
\end{equation}
where
\begin{equation}\label{XTR_T}
U_n =  P_0S_1\Delta_n + \Delta_nS_1P_0 + \Delta_nS_1\Delta_n +P_0dS P_0.
\end{equation}
Each product contains the local perturbations $\Delta_n$ or $dS$ and the computational
complexity therefore scales linearly with the size of the perturbed region $O(N_{pert.})$
\cite{NiklassonPRT1}.  The recursion converges rapidly and at convergence 
the extrapolated density matrix is
\begin{equation}\label{PNEW}
P_1 = P_0 + \lim_{n \rightarrow \infty} \Delta_n.
\end{equation}
The new density matrix $P_1$ is idempotent with the correct trace normalization and is thus a 
solution to the extrapolation problem in Eq.\ (\ref{XTRP}). As in the global extrapolation algortihm, 
in Eq.\ (\ref{XTRPP}), we alter $\sigma$ between $+1$ and $-1$ every second iteration
when $P_0+\Delta_n$  get close to idempotency \cite{NiklassonNO}.  This technique improves 
convergence and stability at large thresholding.

In the density matrix perturbation theory it may be difficult to reach
an efficient quantum emebedding for local perturbations, leading to an efficient
$O(N_{pert.})$ scaling, because of long-ranged dipole interactions. This problem 
is avoided in the local extrapolation scheme since the perturbation occurs 
only in the overlap matrix and not in the Hamiltonian. This advantage will
be demonstrated in section \ref{LOC}.

\section{Examples}\label{EX}

\subsection{Geometry optimization}

\begin{figure}[t]
\resizebox*{3.0in}{!}{\includegraphics[angle=00]{EnergyConv29.eps}}
\caption{\label{FigCaffeine}
The SCF energy convergence as a function of the iteration step
after the first density matrix extrapolation for the three different extrapolation methods.
The L\"{o}wdin extrapolation is based on Eq.\ (\ref{Low}), the inverse Cholesky
method is based on Eq.\ (\ref{IC}), and the trace correcting extrapolation on Eq.\ (\ref{XTRPP}).}
\end{figure}

To illustrate the behavior of the trace correcting extrapolation scheme in Eq.\ (\ref{XTRPP}), 
we have compared it to the L\"{o}wdin and inverse Cholseky extrapolation methods, 
Eqs.\ (\ref{Low}) and (\ref{IC}).  The three methods have been applied
in the quasi-independent curvlinear coordinate approximation for geometry optimization 
recently introduced by Nemeth and Challacombe \cite{Nemeth04,Mondo_SCF}. As a test suite for the
geometry optimization we have chosen the Baker's set of test molecules \cite{Baker93,Nemeth04}.
As the accelleration technique for the convergence to a self-consistent 
solution we have used Pulay mixing, i.e. the so called direct inversion 
in the iterative subspace (DIIS) method \cite{Pulay80}.  Figure \ref{FigCaffeine} 
shows the energy convergence for baker's test system \#29 (Caffeine) in
the geometry optimization after the first density matrix extrapolation. This system
has a typical convergence behavior found in the Baker's test suite.
The energy error $|E_n-E_{min}|$ decreases exponentially for all three of the
initial density matrix extrapolation methods. 
The trace correcting extrapolation method gives a better initial energy, 
which is almost an order of magnitude closer to the self consistent
solution. In practice this means that SCF convergence is typically reach in one iteration
less compared to the two alternatives that performs approximately equally well. As is seen
in Fig.\ \ref{FigCaffeine} the initial difference in error between the methods is 
approximately kept throughout the self-consistent
field (SCF) interations. The inital error is therefore a good measure of the relative efficiency
of the extrapolation methods. Fig.\ \ref{FigError} shows the initial error in the
extrapolation for the first geometry optimization of the Baker's test suite of 30 molecules.
In all cases the trace correcting extrapolation gives a better result, often with one
order of magnitude or better. However, of
main importance in linear scaling electronic structure theory is that trace correcting
extrapolation is fast and easy to implement using efficient parallell sparse matrix 
algebra \cite{Mondo_SCF}. The fact that the trace correcting extrapolation
also gives a more accurate density matrix extrapolation is an additional benefit,
but it is not critital.

\begin{figure}[t]
\resizebox*{3.0in}{!}{\includegraphics[angle=00]{EnergyError.eps}}
\caption{\label{FigError}
The intial error of the energy efter the first density matrix extrapolation 
in the geometry optimization of the Baker's suit of test systems (\#1-30) \cite{Baker93,Nemeth04}
for the three different extrapolation methods.
The L\"{o}wdin extrapolation is based on Eq.\ (\ref{Low}), the inverse Cholesky
method is based on Eq.\ (\ref{IC}), and the trace correcting extrapolation on Eq.\ (\ref{XTRPP}).}
\end{figure}

\subsection{Local extrapolation}\label{LOC}

To investigate the efficiency of the local extrapolation scheme compared to the global 
method we have used a 263 atom large proteine Kinase molecule \cite{Nemeth04}, where a Mg atom
has been displaced 0.1 {\AA}ngstr\"{o}m. 
Figure \ref{FigComp} shows the maximum number matrix elements necessary in the global 
extrapolation,  Eq.\ (\ref{XTRPP}), (in \% of non-zero matrix elements),
v.s. the local scheme, Eq.\ (\ref{XTR}), as a function of a numerical threshold $\tau$.
All matrix elements below $\tau$ are removed after each iteration.
Despite the fairly low sparsity of the density matrix, the local extrapolation scheme can efficiently
explore the locality of the update. For a threshold $\tau = 10^{-4}$ less than 1\% of
the matrix elements are involved in the local extrapolation scheme. For a small
local perturbation the extrapolation cost is therefore independent of the total
system size and scales efficiently as $O(1)$. A dissadvantage
with the local scheme is that it costs one additional matrix-matrix multiplication 
in each iteration compared to the global method. Thus, for global relaxations 
the local method is less efficient.

\section{Discussion}

The two alternative methods used in the comparison, i.e. the L\"{o}wdin and the inverse Cholseky extrapolation,
can also be performed with an $O(N)$ cost for sufficiently sparse systems \cite{Baer98,Challa99,NiklassonIR}, 
and in principle also as $O(N_{pert.})$ for local perturbations. However,
the trace correcting method is very simple and can be efficiently implemented within parallell
sparse matrix algebra \cite{Mondo_SCF}. In addition, as was shown in Fig.\ \ref{FigError},
it gives extrapolated total energies often an order of magnitude closer to the self-consistent solution. 

In the case of large structural perturbations, eigenvalues of $X_n$ may be 
out the range of convergence, which is guaranteed only with eigenvalues $[0,1]$. In practice, however,
the range of convergence is larger.  If we would be out of convergence, 
we can divide the extrapolation into $M$ succesive steps, where a sufficiently small 
perturbation $dS^{(M)} = dS/M$ is applied $M$ times. This was never necessary in
the tested examples.

As an efficient alternative to the numerical threshold used here, a radial cut-off can be applied,
where all matrix elements corresponding to an overlap beyond a specified radius are remowed
and set to zero. An interesting way to chose the cut-off radius in the extrapolation scheme
is to have it determined locally by the range of interaction given by the previous density matrix.

The trace correcting density matrix extrapolation scheme can also be used in 
molecular dynamics simulations. Instead of an interpolation between
previous Hamiltonian operators \cite{Pulay04}, the new scheme makes it possible to efficiently 
extrapolate from previous density matrices. A detailed study will given in a future article.

\begin{figure}[t]
\resizebox*{3.0in}{!}{\includegraphics[angle=00]{Local_Global_XTRP_Comp.eps}}
\caption{\label{FigComp}
The upper bound of the matrix sparsity in \% of non-zero matrix elements above
the numerical threshold $\tau$ in the global extrapolation, Eq.\ (\ref{XTRPP}),
and local extrapolation scheme, Eq.\ (\ref{XTR}). The density matrix extrapolation
is calculated for a 0.1 Angstr\"{o}m displacement of on of the two Mg atoms in a 263 atoms
large proteine Kinase molecule \cite{Nemeth04}, for a self-consistent Hartree-Fock calculation. 
A a small Gaussian basis set was used (STO-2G).}
\end{figure}

\section{Summary}

In summary, we have proposed a non-orthogonal trace correcting density matrix extrapolation 
algorithm that gives an efficient initial density matrix approximation for a new modified
geometry in self-consistent electronic structure calculations. In comparison with alternative 
methods the algorithm is an efficient tool that can be used, for example, in geometry optimization 
algorithms and dynamical simulations.

\section{Acknowledgment}

Discussions with C.\ J. Tymczak are gratefully acknowledged.

\bibliography{PP}

{${\dagger}$ Corresponding author: Anders M.\ N. Niklasson, Email: amn@lanl.gov}                                                                                
\end{document}
