%\documentclass[prl,aps,eqsecnum,twocolumn,showpacs,twocolumngrid,superbib]{revtex4}
\documentstyle[twocolumn,prl,aps]{revtex}
%\documentstyle[preprint,aps]{revtex}

\draft
\tighten

\begin{document}

\title{Quantum Perturbation Theory in $O(N)$}
\title{Non-linear Response; Quantum Perturbation Theory in $O(N)$}
\title{Beyond Linear Response; Quantum Perturbation Theory in $O(N)$}

\author{Anders~M.~N.~Niklasson and Matt~Challacombe}

\address{Theoretical Division, Los Alamos National Laboratory,
Los Alamos, NM 87545, USA}

\date{\today}
\maketitle

\begin{abstract}
{\small \bf We propose the exploration and development of
a new orbital-free quantum perturbation theory for linear 
scaling electronic structure calculations.  The method is 
based on the zero-temperature density matrix and its response 
upon variation of the Hamiltonian. The approach is surprisingly 
simple and efficient. It allows treatment of both embedded 
quantum subsystems and response functions. Traditional
textbook perturbation theory based of wave functions can be 
replaced by a quadratically convergent explicit recursion
that gives the expansion of observables to any order.
The theory provides efficient and unique tools to go 
beyond linear response properties in $O(N)$, and a most
significant outcome of the proposal will be an $O(N)$
vibrational frequency method for the calculation of
the free energy and thermodynamic properties of large
complex materials, both in the molecular gas phase and 
in the periodic condensed phase.}
\end{abstract}~ \\

Response properties, such as elasticity, vibrational frequencies, 
polarizabilities, nuclear magnetic resonance, the magnetic susceptibility, 
and the Raman intensity, are all of fundamental importance in 
materials science.

Today state-of-the-art {\em ab initio} electronic structure
methods can predict linear response properties of systems with
at most a couple of hundred non-equivalent atoms. Beyond linear
response the theory is almost entierly limited to small systems in the
gas phase.  We propose the development of a new unique linear scaling 
theory for the calculation of response properties, including and beyond linear 
response, that can deal with very large complex materials, consisting 
of thousands on non-equivalent atoms, both in the molecular gas 
phase and in the periodic condensed phase. In particular, we
will develop a linear scaling {\em ab initio} method, i.e. a method 
whith a computational complexity scaling linearly O(N) with the
number of atoms, for the computation of the vibrational
frequencies. This is necessary for reliable calculations of 
free energies and the thermodynamic properties of materials,
underpining core activities as LANL. Another key problem
we hope to accomplish is to extend our new perturbation 
theory beyond its adiabatic approximation to enable the calculation
of non-adiabatic time-dependent response properties.

During the last 10 years a significant effort has been devoted to 
the development of electronic structure methods that are sufficiently 
efficient to study very large systems consisting of thousands
of nonequivalent atoms \cite{Goedecker}. The success has been based on 
the linear scaling paradigm, where no single part of a calculation is 
allowed to scale in computational complexity more than linearly as a 
function of the number of atoms $N$. This is in contrast to
conventional schemes that scale with the
cube of the system size $O(N^3)$.  Until recently most attention
was focused on linear scaling $O(N)$ methods for computing the
ground state electronic energy.  However, an important problem
that has been given little attention is the $O(N)$ calculation of materials
response properties, such as vibrational frequencies, the polarizability, 
nuclear magnetic resonance, and Raman intensities. Also of great interest 
is extending $O(N)$ electronic structure theory to quantum embedding 
where the computational complexity scales linearly with the size of a 
locally perturbed region $O(N_{\rm pert.})$ and therefore as $O(1)$ 
with the total system size. This would allow efficient studies of 
quantum subsystems, for example surface chemistries or the catalytic 
domains of proteins, without recalculation of the entire system.

We propose the exploration and development of a new surprisingly 
simple and efficient $O(N_{\rm pert.})$ approach to quantum perturbation 
theory \cite{NiklassonPRT}.  The method makes it possible to study embedded 
quantum subsystems and response functions to any order, within linear scaling effort.
The approach is based on spectral projection purification of the zero-temperature 
density matrix, replacing the conventional eigenvalue problem in tight-binding or 
self-consistent Hartree-Fock and Kohn-Sham theory.

The eigenvalue and orbital-free perturbation theory is based on the density 
matrix and avoids using the more elaborate wavefunction or Green's function 
formalism. The idea is described in some detail in Ref.\ \cite{NiklassonPRT}
and an initial very promising $O(N)$ application is presented in Ref.\ \cite{Weber04}.

The main problem in constructing a density matrix perturbation theory is 
the non-analytic relation between the zero temperature density matrix and 
the Hamiltonian, given by the discontinuous Heaviside step function $\theta$,
with the step formed at the chemical potential $\mu$,
\begin{equation} \label{DM}
P = \theta(\mu I -  H).
\end{equation}
This discontinuity makes an expansion of the density matrix $P$ about 
the Hamiltonian $H$ difficult.
Explicit density matrix derivatives for a perturbed Hamiltonian,
$H = H^{(0)} + \lambda H^{(1)}$, such as
\begin{equation}
P^{(\gamma)} = \frac{\partial^\gamma P}{\partial \lambda^\gamma} = 
\frac{\partial^\gamma \theta(\mu I - H^{(0)} - \lambda H^{(1)})}{\partial \lambda^\gamma},
\end{equation}
in the energy response expansion
\begin{equation}
E = E_0 + \sum_{\gamma} \frac{\lambda^{\gamma}}{\gamma !}Tr(P^{(\gamma)} H^{(1)}),
\end{equation} 
are therefore hard to calculate.

Our new perturbation theory avoids the
discontinuity problem by dividing the perturbation expansion
in sequential steps of purification projections, where perturbations
are performed exactly, or to any order, at each projection level.
Our perturbed projection method yields an explicit and quadratically 
convergent recursive construction of the density matrix response.

In linear scaling purification schemes 
the density matrix is constructed by recursion;
\begin{equation}\label{DM_EXP} \begin{array}{ll}
X_0     & = L(H), \\
X_{n+1} & = F_n(X_n), ~~ n = 0,1,2, \ldots\\
P    & = \lim_{n \rightarrow \infty} X_n. \end{array}
\end{equation}
Here, $L(H) = \alpha(\beta I - H)$ is a linear normalization function 
mapping all eigenvalues of $H$
in reverse order to the interval of convergence $[0,1]$ and $F_{n}(X_n)$ 
is a set of functions projecting the eigenvalues of $X_n$
toward  1 (for occupied states) or 0 (for unoccupied states). 
Purification projection schemes are quadratically convergent, numerically
stable, and can even solve problems with degenerate eigenstates, finite
temperatures, and fractional occupancy \cite{NiklassonSP2,NiklassonSP4,NiklassonIP}.

Thanks to an exponential decay of the density matrix elements
as a function of the interatomic distance for insulators, all operators
have sparse matrix representations and the number of non-zero 
matrix elements above a numerical threshold scales linearly
with the system size. In these cases all matrix-matrix 
operations can be performed using efficient sparse matrix
algebra with $O(N)$ computational effort.

By performing perturbations at each projection level $n$ in Eq.\ (\ref{DM_EXP})
a perturbation in $H$ can be carried through recursively to the next projection 
level, exactly or to any order, in a quadratically convergent 
sequence. This is the key idea that makes it possible to
efficiently calculate density matrix derivatives to any order
or local updates exactly with a computational effort that scales linearly 
with the size of the perturbed region for band-gap materials \cite{NiklassonPRT}.
Because the perturbation expansions inherit properties from their 
generator sequence, they are likewise quadratically convergent with 
iteration, numerically stable, and exact to within accuracy 
of the drop tolerance \cite{NiklassonSP4}.

Figure \ref{Fig1} shows an example of the polarizability of a C60 molecule
with the field along the $z$-axis, calculated with our new perturbation
theory. Figure \ref{Fig2} displays the perturbation
convergence result for a local single site displacement of
a one-dimensional random chain model.  Only a small local updates 
of the density matrix difference $\Delta_n$, where the new density matrix
\begin{equation}
P_{\rm new} = P_{\rm old} + \lim_{n \rightarrow \infty} \Delta_n,
\end{equation}
are performed on each projection level and the computational cost
of the quantum embedding is $O(1)$.

Our new approach to quantum perturbation theory can be applied in many contexts.
For example, a straightforward calculation of the energy difference
due to a small perturbation of a very large system may not be possible because
of the numerical problem in resolving a tiny energy difference between
two large energies. With density matrix perturbation theory,
we work directly with the density matrix difference $\Delta_n$ and the
problem can be avoided, for example, the single particle energy change
$\Delta E = \lim_{n \rightarrow \infty} Tr(H\Delta_n)$. In analogy to incremental 
Fock builds in self-consistent field calculations, the technique
can be used in incremental density matrix builds.
Connecting and disconnecting individual weakly interacting
quantum subsystems can be performed by treating off-diagonal elements of the
Hamiltonian as a perturbation. This should be highly useful in nanoscience
for connecting quantum dots, surfaces, clusters and nanowires, where the different
parts can be calculated separately, provided a connection through a common
chemical potential is given, for example via a surface substrate.
In quantum molecular dynamics, such as quantum mechanical-molecular mechanical QM/MM
schemes, or Monte-Carlo simulations, where only a local part of the system is perturbed
and updated, the new approach is of interest.
Several techniques used within the Green's function context also should
apply for the density matrix. The proposed perturbation approach may
be used for response functions \cite{Weber04}, vibrational frequencies,
impurities, geometry optimization, elasticity, and maybe even effective medium and 
time dependent non-adiabatic response theory.
The grand canonical density matrix perturbation theory is thus a rich field
with potential applications in many areas of materials science, chemistry 
and biology, underpinning core activities at LANL.

So far only a very promising linear scaling scheme for self-consistent
Hartree-Fock calculations of the electric polarizability has been developed \cite{Weber04}.
However, as discussed above, the method opens a great variety of new opportunities in linear 
scaling electronic structure theory that we hope to be able to explore. 

Initially we would like to develop an $O(N)$ frequency response theory based 
on higher order derivatives of the total energy with respect to atomic displacement.
This is necessary for the calculation of the free energy and thermodynamic
properties, which will be one of the most unique, important and highly
useful outcomes if the project is funded. 

The possibility to create nanosystems collected from individual systems, calculated
separately in parallel, is another challenge that can be solved.

A likely outcome of the project is an {\em ab initio} 
Monte-Carlo algorithm, which can explore the ability to efficiently 
calculate local updates of the electronic structure.

It would be interesting to extend the $O(N)$ perturbation 
theory to time-dependent problems. It is not clear at all if this is 
possible in the non-adiabatic limit. If it is possible it would be a 
fundamental achievement of great importance and well worth
a substantial effort.

Can the new perturbation theory be used beyond the single-particle
representation, for example, within a reduced density matrix approach? 
That is another question we would like to explore.

We would also like to examine if the new perturbation theory can be used to 
create a mean field effective medium from a self consistent average of 
local embedded perturbations, similar to what is used in the coherent 
potential approximation and the $O(N)$ local scattering Green's function 
method \cite{Igor}.

It is also important to carefully analyze convergence,
stability, and other numerical and theoretical aspects 
of the new perturbation approach. This may not be of direct 
importance to LANL or result in any key applications, but
it is critical to the understanding and possible future 
development, unpredictable at this very early stage.

The total amount of work necessary to explore our new approach
to $O(N)$ perturbation and response theory is flexible since 
the field is very rich. We have decided to dedicate one staff member
Anders Niklasson to 50 $\%$  and the second staff member
Matt Challacombe to 25 $\%$. In addition we would like
to hire either two students or a Post-Doc (Valery Weber)
full time.

\begin{references}

\bibitem{Goedecker} S. Goedecker,
Rev.\ Mod.\ Phys. {\bf 71}, 1085 (1999).

\bibitem{NiklassonPRT} A.\ M.\ N.\ Niklasson, and M. Challacombe,
{\it Density Matrix Perturbation Theory},
http://arxive.org/abs/cond-mat/0311591.

\bibitem{Weber04} V. Weber, A.\ M.\ N. Niklasson, and M. Challacombe,
{\it Ab initio linear scaling response theory: Electric polarizability 
by perturbed projection}, http://arxive.org/abs/cond-mat/0312634.

\bibitem{NiklassonSP2} A.\ M.\ N.\ Niklasson, 
Phys.\ Rev.\ B {\bf 66}, 155115 (2002). 

\bibitem{NiklassonSP4} A.\ M.\ N. Niklasson, C.\ J.\ Tymczak, and M. Challacombe,
J.\ Chem.\ Phys.\ {\bf 118}, 8611 (2003).

\bibitem{NiklassonIP} A.\ M.\ N. Niklasson,
Phys.\ Rev.\ B. {\bf 68}, 233104 (2003).

\bibitem{Igor} I.\ A. Abrikosov, A.\ M.\ N. Niklasson, S.\ I. Simak,
B. Johansson, A.\ V. Ruban, and H.\ L. Skriver,
Phys.\ Rev.\ Lett. {\bf 76}, 4203 (1996).

\end{references}

\begin{figure}
\caption{\small  The static uncoupled polarizability (up to 10th order in the inset)
for a C60 molecule (RHF/6-31G) with the field aligned along the $z$ axis. 
Odd orders vanish due to symmetry.
\label{Fig1}}
\end{figure}


\begin{figure}
\caption{\small The convergence error for a local perturbation $\Delta_n$ as a function 
of iterations $n$.  The lower inset shows the number of non-zero matrix elements 
above a numerical threshold ($10^{-6}$). Only about 50 matrix elements out of a 
total of $10^4$ are necessary to include in the expansion. The upper inset shows 
the density matrix perturbation.
\label{Fig2}}
\end{figure}

\end{document}
