%\documentclass[prl,aps,eqsecnum,twocolumn,showpacs,twocolumngrid,superbib]{revtex4}
\documentstyle[twocolumn,prl,aps]{revtex}
%\documentstyle[preprint,aps]{revtex}

\draft
\tighten

\begin{document}

\title{Density Matrix Perturbation Theory}

\author{Anders~M.~N.~Niklasson$^\dagger$  and Matt~Challacombe}

\address{Theoretical Division, Los Alamos National Laboratory,
Los Alamos, NM 87545, USA}

\date{\today}
\maketitle

\footnotetext{Preprint LA-UR 03-6452}

\begin{abstract}
{\small \bf A highly efficient quantum perturbation theory 
that avoids using wavefunction or Green's function formalism is introduced.
The method achieves quadratically convergent recursions that yield
the response of the density matrix upon variation of the Hamiltonian. 
The technique allows treatment of embedded quantum subsystems with a 
computational cost scaling linearly with the size of the perturbed 
region, ${\cal O}(N_{\rm pert.})$, and as ${\cal O}(1)$ with the total 
system size. It also allows direct computation of the density matrix response 
functions to any order with linear scaling effort.  Energy expressions to 
4th order based on only first and second order density matrix response are given.}
\\

\keywords{density matrix, perturbation theory, purification, sign matrix,
linear scaling, electronic structure, spectral projection, density-functional theory}
%\pacs{02.70.-c, 71.15.-m, 71.15.Dx}
\end{abstract}

In electronic structure 
theory, significant effort has been devoted to the development of
methods with the computational cost scaling linearly with system size 
\cite{Goedecker_RMP_99,Wu02}. The ability to perform accurate calculations
with reduced-complexity ${\cal O}(N)$ scaling is an important breakthrough 
that opens a variety of new possibilities in materials science, 
chemistry and biology. So far most attention has been focused on 
the ground state electronic energy. However, an important key problem
that has been given little attention is the computation of materials 
response properties such as polarizabilities, chemical shielding, Raman 
intensities, or the magnetic susceptibility.
Of great interest is also to combine efficient ${\cal O}(N)$ theory with 
quantum embedding \cite{Pantelides78,Haydock80,Inglesfield81}, where the computational 
complexity scales linearly with the size of a locally perturbed region
${\cal O}(N_{\rm pert.})$ and therefore as ${\cal O}(1)$ with the total 
system size. This would make it possible to efficiently analyze various
quantum subsystems or impurities, for example in nano devices or large 
protein molecules, but without a complete recalculation 
of the entire system. 

In this letter, we introduce a general and surprisingly simple 
${\cal O}(N_{\rm pert.})$ approach to quantum perturbation theory.
The method makes it possible to study embedded quantum subsystems
and response functions to any order, within linear scaling effort.
The approach is based on recently developed spectral projection schemes
for purification of the density matrix, replacing the conventional
eigenvalue problem in tight-binding or self-consistent Hartree-Fock 
and Kohn-Sham theory
\cite{McWeeny60,Clinton69,Palser98,Holas01,NiklassonWLT,NiklassonSP2,NiklassonSP4,NiklassonIP}.

The perturbation theory is based on the density matrix and avoids using the more 
elaborate wavefunction or Green's function formalism \cite{Haydock80,Inglesfield81}.
In spirit, it is therefore similar to the density matrix perturbation method
proposed by McWeeny \cite{McWeeny_PRT}. The present work 
is likewise related to the recent work of Bowler and Gillan \cite{Bowler02},
who developed a functionally constrained density matrix minimization scheme for embedding.
However, our approach to computation of the density matrix response is quite different from 
previous methods of solutions for the coupled-perturbed self-consistent-field equations.  
In contrast to previous methods that pose solution implicitly through coupled equations 
\cite{Frisch,Dupuis,Ochsenfeld,Larsen}, the new method provides explicit construction of the 
derivative density matrix through recursion.  

The main problem in constructing a density matrix perturbation theory is 
the non-analytic relation between the zero temperature density matrix and 
the Hamiltonian, given by the discontinuous step function \cite{Notation},
\begin{equation} \label{DM}
P = \theta[\mu I -  H].
\end{equation}
This discontinuity makes expansion of $P$ about $H$ difficult.
At finite temperatures the discontinuity disappears and we may use perturbation 
expansions of the analytic Fermi-Dirac distribution \cite{Feynman}. 
However, even at finite temperatures a perturbation expansion based
on the Fermi-Dirac distribution may have slow convergence.

In linear scaling purification schemes 
\cite{McWeeny60,Clinton69,Palser98,Holas01,NiklassonWLT,NiklassonSP2,NiklassonSP4}, 
the density matrix is constructed by recursion;
\begin{equation}\label{DM_EXP} \begin{array}{ll}
X_0     & = L(H), \\
X_{n+1} & = F_n(X_n), ~~ n = 0,1,2, \ldots\\
P    & = \lim_{n \rightarrow \infty} X_n. \end{array}
\end{equation}
Here, $L(H) = \alpha(\beta I - H)$ is a linear normalization function \cite{Notation}
mapping all eigenvalues of $H$
in reverse order to the interval of convergence $[0,1]$ and $F_{n}(X_n)$ 
is a set of functions projecting the eigenvalues of $X_n$
toward  1 (for occupied states) or 0 (for unoccupied states). In one of 
the simplest and most efficient techniques, which requires only 
knowledge of the number of occupied states $N_e$ and no {\em a priori} 
knowledge of $\mu$ \cite{NiklassonSP2}, we have 
\begin{equation} \label{SP2}
F_{n}(X_n) = 
\left\{\begin{array}{ll}
X_n^2, &  Tr(X_n) \geq N_e \\
2X_n - X_n^2, & Tr(X_n) < N_e.
\end{array} \right.
\end{equation}
Purification expansion schemes are quadratically convergent, numerically 
stable, and can even solve problems with degenerate eigenstates, finite 
temperatures, and fractional occupancy \cite{NiklassonSP4,NiklassonIP}. Thanks to an 
exponential decay of the density matrix elements as a function of $|{\bf r-r'}|$ 
for insulating materials, the operators have a sparse matrix representation 
and the number of non-zero matrix elements above a numerical threshold
scales linearly with the system size \cite{Goedecker_RMP_99,Wu02}.  
In these cases the matrix-matrix multiplications, 
which are the most time consuming steps, have an $N$-scaling cost.

Equivalent to the purification
schemes are the sign-matrix expansions \cite{Kenney91,Beylkin99,Karoly}.
The general scheme is the same as in Eq.\ (\ref{DM_EXP}), but
the expansion is performed around a step from $-1$ to $1$ at $x=0$.

Our quantum perturbation theory is based on 
the purification in Eq.\ (\ref{DM_EXP}).
A perturbed Hamiltonian $H = H^{(0)} + H^{(1)}$ gives
the expansion 
\begin{equation}\label{dX}
{X_n} = X_n^{(0)} + \Delta_n, ~ n = 0,1,2,\ldots~,
\end{equation}
where $X_n^{(0)}$ is the unperturbed expansion 
and $\Delta_n$ are the differences due to the perturbation $H^{(1)}$
$(\Delta_0 = L(H)-L(H^{(0)}) = -\alpha H^{(1)})$.  It is then easy to 
show that \begin{equation}\label{DYS_F}\begin{array}{ll}
\Delta_{n+1} &= F_n(X_n^{(0)} + \Delta_n) - F_n(X_n^{(0)})\\
{P} &= P^{(0)} + \lim_{n \rightarrow \infty} \Delta_n. \end{array}
\end{equation}
This is the key result of the present
article and defines our grand canonical density matrix perturbation theory.
Combining Eq.\ (\ref{DYS_F}) with the expansion in Eq.\ (\ref{SP2}) gives 
the recursive expansion \cite{Notation}
\begin{equation} \label{DYS_SP2}
\Delta_{n+1} = 
\left\{\begin{array}{l}
\{ X_n^{(0)},\Delta_n\} + \Delta_n^2, ~~{\rm if} ~~  Tr(X_n^{(0)}) \geq N_e \\
2\Delta_n - \{ X_n^{(0)},\Delta_n\} - \Delta_n^2, ~~ {\rm otherwise}.
\end{array} \right.
%\end{array}
\end{equation}
Other expansions based on, for example, McWeeny, trace conserving or 
trace resetting purification \cite{McWeeny60,Palser98,NiklassonSP4}
can also be included in this quite general approach. However, 
Eq.\ (\ref{DYS_SP2}) is particularly efficient since it only 
requires two matrix multiplications per iteration.
Because the perturbation expansions inherit properties from their 
generator sequence, they are likewise quadratically convergent with 
iteration, numerically stable, and exact to within accuracy 
of the drop tolerance \cite{NiklassonSP4}.

If the perturbed ${X}_0^{(0)}$ has eigenvalues outside the interval 
of convergence $[0,1]$ the expansion could fail. To avoid this problem 
the normalization function $L(H)$ in Eq.\ (\ref{DM_EXP}) can be chosen to contract the
eigenvalues of $X_0$ to $[\delta,1-\delta]$, where $\delta>0$ is sufficiently large.

A major advantage with the expansion in Eq.\ (\ref{DYS_SP2}) is that for band-gap 
materials that are locally perturbed, the $\Delta_n$ are likewise localized as a result 
of nearsightedness \cite{Kohn59,Kohn96}. The matrix products in Eq.\ (\ref{DYS_SP2}) 
can therefore be calculated using only the local regions of $X_n$ that respond to the perturbation.
Given that perturbation does not change the overall decay of the
density matrix, the computational cost of the expansion scales linearly with the
size of the perturbed region ${\cal O}(N_{\rm pert.})$ and as ${\cal O}(1)$ with
the total system size.

Density matrix purification does not necessarily require
prior knowledge of the chemical potential, but once the
initial expansion of the unperturbed system is carried out, the
chemical potential is set. The perturbation expansions of 
Eq.\ (\ref{DYS_F}) are therefore grand canonical \cite{CPRT}.
With this in mind, Eq.~(\ref{DYS_SP2}) may be readily applied 
to embedding schemes that do involve long range charge flow.

The computation of many spectroscopic properties such as the Raman spectra, 
chemical shielding and polarization requires the calculation of density 
matrix derivatives with respect to perturbation.
Grand canonical density matrix perturbation 
theory can be used to compute these response functions.
Assume a perturbation of the Hamiltonian $H^{(0)}$,
\begin{equation}
{H} = H^{(0)} + \lambda H^{(1)},
\end{equation}
in the limit $\lambda \rightarrow 0$.  
The corresponding density matrix is
\begin{equation}
{P} = P^{(0)} + \lambda P^{(1)} + \lambda^2 P^{(2)} + \ldots~,
\end{equation}
where the response functions $P^{(\gamma)}$ (density matrix derivatives) correspond 
to order $\gamma$ in $\lambda$.  Expanding the perturbation as in 
Eq. (\ref{DYS_SP2}), individual response terms may be collected
order by order at each iteration;
\begin{equation}
\Delta_n = \lambda \Delta^{(1)}_n + \lambda^2 \Delta^{(2)}_n + \ldots~.
\end{equation}
Keeping terms through order $m$ in $\lambda$ at each iteration, 
with $\Delta^{(0)}_n = X_n$, the following recursive sequence is obtained 
for $\gamma = m,m-1,\ldots,1$ :
\begin{eqnarray}\label{second1}
\Delta^{(\gamma)}_{n+1} = \left\{ \begin{array}{l}
\sum_{i=0}^{\gamma} \Delta^{(i)}_n \Delta^{(\gamma-i)}_n, \quad {\rm if} ~~ {Tr}(X_n) \geq N_e\\ 
2\Delta^{(\gamma)}_n - \sum_{i=0}^{\gamma} \Delta^{(i)}_n \Delta^{(\gamma-i)}_n, \quad {\rm otherwise.}
\end{array} \right.
\end{eqnarray}
These equations provide an explicit, quadratically convergent solution of the response functions, 
where 
\begin{equation}
P^{(\gamma)} = \lim_{n \rightarrow \infty} \Delta^{(\gamma)}_n.  
\end{equation}
With the same technique it is possible to treat perturbations where 
${H} = H^{(0)} + \lambda_a H^{(1)}_a+\lambda_b H^{(1)}_b+ 
\lambda_a \lambda_b H^{(2)}_{a,b} + \ldots$
to produce a mixed density matrix expansion
${P} = P^{(0)} + \lambda_a P_a^{(1)} + \lambda_b P_b^{(1)} + 
\lambda_a \lambda_b P_{a,b}^{(2)} + \ldots~$.

Equation~(\ref{second1}) provide direct explicit construction of the response 
equations based on well developed linear scaling technologies \cite{NiklassonSP2,NiklassonSP4}.  
This is quite different from earlier approaches\cite{Frisch,Dupuis,Ochsenfeld,Larsen}  
that pose solution implicitly through coupled matrix 
equations, achieving at best linear scaling with iterative solvers.
In a future publication we develop our theory for the solution
of the coupled perturbed self-consistent field equations applied to
linear scaling {\it ab initio} calculations of the polarizability of 
large water clusters \cite{Weber04}. 

Higher order expansions of the energy can be calculated efficiently from low
order density matrix terms. Similar to Wigner's 2n+1 rule for
wavefunctions \cite{Helgaker} we have the energy response 
${E} = E^{(0)} + \lambda E^{(1)} + \lambda^2 E^{(2)} + \lambda^3 E^{(3)} 
+ \lambda^4 E^{(4)}$, where
\begin{eqnarray}\label{Tnp1}
E^{(1)} &=& Tr(P^{(0)} H^{(1)}), ~ E^{(2)} = 0.5 Tr(P^{(1)} H^{(1)}) \nonumber \\
E^{(3)} &=& Tr([P^{(1)},P^{(0)}] P^{(1)} H^{(1)}), \\
E^{(4)} &=& 0.5 Tr([(2I-P^{(0)}) P^{(2)} P^{(0)} P^{(1)} \nonumber \\
&~& - P^{(0)} P^{(1)} P^{(2)} (I+P^{(0)})] H^{(1)}). \nonumber
\end{eqnarray}
The corresponding n+1 rule for $\gamma > 0$ is 
\begin{equation}\label{Onp1}
E^{(\gamma)} = \gamma^{-1} Tr(P^{(\gamma-1)}H^{(1)}).
\end{equation}

To demonstrate the grand canonical density matrix
perturbation theory, we present two examples: the first is based  
on a local perturbation of a model Hamiltonian, and the second
example illustrates the ability to calculate higher order
response functions.

The model Hamiltonian has random
diagonal elements exhibiting exponential decay of the overlap elements 
as a function of site separation on a randomly distorted 
lattice. This model represents a Hamiltonian of an insulator
that might occur, for example, with a Gaussian basis set in 
density functional theory or in various tight-binding schemes. 
A local perturbation is imposed on the model Hamiltonian
by moving the position of one of the lattice sites.
Using the perturbation expansion of Eq.\ (\ref{DYS_SP2}),
a series of perturbations $\Delta_n$ is generated. In each 
step a numerical threshold $\tau = 10^{-6}$ is applied as described in
\cite{NiklassonSP4}.  The lower inset in Fig.  \ref{DX} shows 
the number of elements above the threshold in $\Delta_n$ 
as a function of iteration. The local perturbation
is efficiently represented with only $\sim 50$ elements out of $10^4$.
Figure \ref{DX} also illustrates the quadratic convergence
of the error.  At convergence after 
$M$ iterations the new perturbed density matrix is given by
${P} = P^{(0)} + \Delta_M$. The error 
$||{P} - P_{\rm exact}||_2 = 6.4 \times 10^{-5}$
and the error $|{E} - E_{\rm exact}| = 1.3 \times 10^{-6}$
\cite{first_order}.  The error of the perturbed density matrix 
${P}$ is stable at convergence and close to the numerical error 
for the unperturbed density matrix due to thresholding
$||P^{(0)} - P^{(0)}_{\rm ~exact}||_2 = 1.0 \times 10^{-4}$, and
$|{E}^{(0)} - E^{(0)}_{\rm exact}| = 2.4 \times 10^{-6}$.

The second example, in Fig. \ref{Fig2}, illustrates the perturbation of the
Hamiltonian 
\begin{equation}\label{HPRT}
\begin{array}{ll}
\displaystyle H &= -\nabla^2_x - e^{-(x-X_A)^2}\\
\displaystyle & +k \left( e^{-(x-X_A)^2} - e^{-(x-X_B)^2} \right) ~~~~ k \in [0,1].
\end{array}
\end{equation}
with respect to the field 
parameter k, shifting the Gaussian potential centered at $X_A$ to 
the same potential, but centered at $X_B$. The figure shows
the corresponding change in the density.  The average absolute energy
error for $k\in[0,1]$ and the 2-norm error of the density
matrix response, using Eqs.\ (\ref{second1}) and (\ref{Onp1}), respectively, 
decay exponentially with expansion order, as shown in the upper inset. Note, 
the grand canonical perturbation theory works effortless even up to 11th order
in energy (10th order in density) for this example, since no
states cross the chemical potential for $k \in [0,1]$. For
non-metals with a band gap, and for fairly weak perturbations, 
this is generally the case and the theory should apply.
An extension to {\it ab initio} calculations is given in a future
publication \cite{Weber04}, and a more detailed analysis of 
convergence will be given elsewhere.

The present formulation has been developed in an orthogonal representation.  With 
a $N$-scaling congruence transformation \cite{Benzi96,Challa99},  it is straightforward to 
employ this representation when using a non-orthogonal basis. When using a 
non-orthogonal basis set, change in the inverse overlap matrix  $S^{-1}$  due to
a local perturbation $dS$ is given by a recursive local Dyson equation, 
\begin{equation}
\delta_{n+1} = {S_{0}}^{-1} dS ({S_{0}}^{-1} + \delta_n),
\end{equation}
where $S = S_{0} - dS$, $\delta_{0} = 0$, and
$S^{-1} = {S_{0}}^{-1}+\lim_{n \rightarrow \infty} \delta_n$.
The equation contains only terms with local sparse updates 
and the computational cost scales linearly, ${\cal O}(N_{\rm pert.})$,
with the size of the perturbed region. Similar perturbation schemes 
for the sparse inverse Cholesky or square root factorizations can also 
be used \cite{unpubl}.

Density matrix perturbation theory can be applied in many contexts. 
For example, a straightforward calculation of the energy difference 
due to a small perturbation of a very large system may not be possible because
of the numerical problem in resolving a tiny energy difference between
two large energies. With density matrix perturbation theory, 
we work directly with the density matrix difference $\Delta_n$ and the
problem can be avoided, for example, the single particle energy change
$\Delta E = \lim_{n \rightarrow \infty} Tr(H\Delta_n)$. In analogy to incremental Fock builds
in self-consistent field calculations \cite{Schwengler97}, the technique 
can be used in incremental density matrix builds.
Connecting and disconnecting individual weakly interacting 
quantum subsystems can be performed by treating off-diagonal elements of the
Hamiltonian as a perturbation. This should be highly useful in nanoscience 
for connecting quantum dots, surfaces, clusters and nanowires, where the different 
parts can be calculated separately, provided a connection through a common
chemical potential is given, for example via a surface substrate.
In quantum molecular dynamics, such as quantum mechanical-molecular mechanical QM/MM 
schemes, or Monte-Carlo simulations, where only a local part of the system is perturbed 
and updated, the new approach is of interest. 
Several techniques used within the Green's function context also should 
apply for the density matrix. The proposed perturbation approach may 
be used for response functions \cite{Weber04}, impurities, 
effective medium and local scattering techniques \cite{Haydock80,Inglesfield81,Turek,Igor}.  
The grand canonical density matrix perturbation theory is thus a rich field
with applications in many areas of materials science, chemistry and biology.

In summary, we have introduced a grand canonical perturbation theory 
for the zero temperature density matrix, extending quadratically convergent 
purification techniques to expansions of the density matrix upon variation
of the Hamiltonian.  The perturbation method allows the local adjustment of 
embedded quantum subsystems with a computational cost that scales as ${\cal O}(1)$
for the total system size and as ${\cal O}(N_{\rm pert.})$ for the 
region that respond to the perturbation, as demonstrated in Fig.~\ref{DX}.
A new quadratically convergent $N$-scaling recursive approach to computing 
density matrix response functions has been outlined, as illustrated
in Fig.\ \ref{Fig2}, and energy 
expressions to 4th order in terms of only first and second order density matrix 
response were given.  The proposed quantum perturbation theory is surprisingly 
simple and offers an efficient alternative to several Green's function 
or wavefunction methods and conventional schemes for solution of the coupled 
perturbed self-consistent-field equations.

Discussions with  E.\ Chisolm, S.\ Corish, S.\ Tretiak, C.\ J.\ Tymczak, V.\ Weber, 
and J.\ Wills are gratefully acknowledged.


\begin{references}

\bibitem[\dagger]{address} Corresponding author: amn@lanl.gov

\bibitem{Goedecker_RMP_99} S. Goedecker,
Rev.\ Mod.\ Phys. {\bf 71}, 1085 (1999).

\bibitem{Wu02} S.\ Y. Wu and C.\ S. Jayanthi,
Physics Reports-Review Section of Physics Letters, {\bf 358}, 1 (2002).

\bibitem{Pantelides78} S.\ T. Pantelides,
Rev.\ Mod.\ Phys. {\bf 50}, 797 (1978).

\bibitem{Haydock80} R.\ Haydock, 
Solid State Phys.\ Advances in Research and Applications, {\bf 35}, 215 (1980).

\bibitem{Inglesfield81} J.\ E.\ Inglesfield,
J.\ Phys.\ C, {\bf 14}, 3795 (1981).

\bibitem{McWeeny60} R. McWeeny,
Rev.\ Mod.\ Phys. {\bf 32}, 335 (1960).

\bibitem{Clinton69} W.\ L. Clinton, A.\ J. Galli, and L.\ J. Masa,
Phys.\ Rev. {\bf 177}, 7 (1969).

\bibitem{Palser98} A.\ H.\ R. Palser and D.\ E. Manolopoulos,
Phys.\ Rev.\ B {\bf 58}, 12704 (1998).

\bibitem{Holas01} A.\ Holas,
Chem.\ Phys.\ Lett, {\bf 340}, 552 (2001).

\bibitem{NiklassonWLT} A.\ M.\ N.\ Niklasson, C.\ J.\ Tymczak, and H.\ R\"{o}der,
Phys.\ Rev.\ B {\bf 66}, 155120 (2002).

\bibitem{NiklassonSP2} A.\ M.\ N.\ Niklasson, 
Phys.\ Rev.\ B {\bf 66}, 155115 (2002). 

\bibitem{NiklassonSP4} A.\ M.\ N. Niklasson, C.\ J.\ Tymczak, and M.\ Challacombe,
J.\ Chem.\ Phys.\ {\bf 118}, 8611 (2003).

\bibitem{NiklassonIP} A.\ M.\ N. Niklasson,
Phys.\ Rev.\ B, (to appear).

\bibitem{McWeeny_PRT} R. McWeeny, 
Phys.\ Rev.\ {\bf 126}, 1028 (1962).

\bibitem{Bowler02} D.\ R.\ Bowler and M.\ J.\ Gillan,
Chem.\ Phys.\ Lett. {\bf 355}, 306 (2002).

\bibitem{Frisch} M.\ Frisch, M.\ Head-Gordon, and J. Pople,
Chem.\ Phys.\ {\bf 141}, 306 (1990).

\bibitem{Dupuis} S.\ P.\ Karna and M.\ Dupuis,
J.\ Comput.\ Chem.\ {\bf 12}, 487 (1991).

\bibitem{Ochsenfeld} C.\ Ochsenfeld and M.\ Head-Gordon,
Chem.\ Phys.\ Lett.\ {\bf 270}, 399 (1997).

\bibitem{Larsen} H.\ Larsen, P.\ Jorgensen, and T.\ Helgaker,
J.\ Chem.\ Phys. {\bf 113}, 8908 (2000).

\bibitem{Notation} The notation: $\theta$ is the Heaviside step function,
$\mu$ is the chemical potential, 
$I = \delta({\bf r-r'})$, $H = H({\bf r,r'})$, $P = P({\bf r,r'})$,
$Tr(A) = \int d{\bf r} A({\bf r,r})$, $AB = \int d{\bf r''} A({\bf r,r''})B(\bf r'',r')$,
the density $n({\bf r}) = P({\bf r,r})$, the single particle energy $E = Tr(HP)$,
$N_e$ is the number of occupied states below $\mu$, and 
$N$ is the total number of states corresponding to the system size.
For the projections in Eq.\ (\ref{DM_EXP}) $\beta = H_{\rm max}$ and
$\alpha = (H_{\rm max} - H_{\rm min})^{-1}$, where $H_{\rm max~(min)}$
are estimates of the spectral bound of $H$.  The anti-commutator 
$\{A,B\} = AB+BA$ and the commutator $[A,B] = AB-BA$.

\bibitem{Feynman} R.\ P.\ Feynam, {\em ``Statistical Mechanics''},
ISBN 0-201-36076-4, Addison Wesley Longman, Inc. (1998).

\bibitem{Kenney91} C.\ S.\ Kenney, and A.\ J.\ Laub,
SIAM J. Matrix Anal. {\bf 2}, 273 (1991).

\bibitem{Beylkin99} G. Beylkin, N. Coult, and M. Mohlenkamp,
J.\ Comput.\ Phys. {\bf 152}, 32 (1999).

\bibitem{Karoly} K. Nemeth, and G.\ S. Scuseria,
J.\ Chem.\ Phys.\ {\bf 113}, 6035 (2000).

\bibitem{Kohn59} W.\ Kohn,
Phys.\ Rev. {\bf 115}, 809 (1959).

%\bibitem{Li93} X.\ -P. Li, W. Nunes, and D. Vanderbilt,
%Phys.\ Rev.\ B {\bf 47}, 10891 (1993).

\bibitem{Kohn96} W. Kohn, Phys.\ Rev.\ Lett. {\bf 76}, 3168 (1996).

\bibitem{CPRT} A canonical perturbation theory can be constructed
by allowing the spectral projections $F_n(X_n+\Delta_n)$ to differ
from $F_n(X_n)$ in Eq.\ (\ref{DYS_F}), however, in this case the
locality of the expansion for a local perturbation is lost.

\bibitem{Weber04} W. Weber, A.\ M.\ N Niklasson, and M. Challacombe,
{\it Ab initio linear scaling response theory: Electric polarizability 
by perturbed projection}, http://arxive.org/abs/cond-mat/0312634.

\bibitem{Helgaker} T.\ Helgaker, P.\ Jorgensen, and J.\ Olsen, 
{\em Molecular electronic-structure theory}, ISBN 0471 96755 6, 
(John Wiley \& Sons, West Sussex, England, 2000).

\bibitem{first_order} As a comparison, first-order perturbation theory 
gives an error $|Tr((H^{(0)}+\Delta V)P^{(0)})-E_{\rm exact}| = 8.7 \times 10^{-3}$ .

\bibitem{Benzi96} M.\ Benzi, C.\ D. Meyer, and M.\ Tuma,
SIAM J.\ Sci.\ Compt., {\bf 17}, 1135 (1996).

\bibitem{Challa99} M. Challacombe, 
J.\ Chem.\ Phys. {\bf 110}, 2332 (1999).

%\bibitem{Bowler99} D.\ R.\ Bowler and M.\ J.\ Gillan, 
%Comput.\ Phys.\ Commun.\ {\bf 120}, 95 (1999).

%\bibitem{Daniels99} A.\ D.\ Daniels and G.\ E.\ Scuseria, 
%J.\ Chem.\ Phys. {\bf 110}, 1321 (1999).

%\bibitem{Helgaker2000} T. Helgaker, H. Larsen, J. Olsen, P Jorgensen,
%Chem.\ Phys.\ Lett. {\bf 327}, 397 (2000).

%\bibitem{Head_Gordon2003} M. Head-Gordon, Y.\ H.  Shao, C.  Saravanan, C.\ A. White,
%Mol.\ Phys.\ {\bf 101}, 37 (2003);
%Y.\ H.  Shao, C.  Saravanan, M. Head-Gordon, C.\ A. White,
%J.\ Chem.\ Phys.\ {\bf 118}, 6144 (2003).

%\bibitem{Lowdin51} P.\ O.\ L\"{o}wdin,
%J.\ Chem.\ Phys. {\bf 19}, 1396 (1951).


%\bibitem{Mondo_SCF} M.\ Challacombe, E.\ Schwengler, C.\ J.\ Tymczak,
%C.\ K.\ Gan, K.\ Nemeth, V.\ Weber and A.\ M.\ N.\ Niklasson, {\sc MondoSCF}
%{\it A program suite for massively parallel, linear scaling SCF theory and
%ab initio molecular dynamics (2001)}, URL http://www.t12.lanl.gov/home/mchalla/.


%\bibitem{first} $|Tr((F^{(0)}+\Delta V)P^{(0)})-E_{\rm exact}| = 4.5 \times 10^{-3}$ a.u.,
%where $F^{(0)}$ is the unperturbed Fockian.

\bibitem{unpubl} A.\ M.\ N.\ Niklasson, (unpublished).

\bibitem{Schwengler97} E.\ Schwengler, M.\ Challacombe, and M.\ Head-Gordon,
J.\ Chem.\ Phys. {\bf 106}, 9708 (1997).

\bibitem{Turek}   I.\ Turek, V.\ Drachl, J.\ Kudrnovsky, M.\ Sob, and
P.\ Weinberger, {\it Electronic structure of disordered alloys, surfaces
and interfaces}, Klawer Academic Publishers (1997). 

\bibitem{Igor} I.\ A. Abrikosov, A.\ M.\ N. Niklasson, S.\ I. Simak,
B. Johansson, A.\ V. Ruban, and H.\ L. Skriver,
Phys.\ Rev.\ Lett. {\bf 76}, 4203 (1996).

%\bibitem{Long_range} In the case of strong interaction and self-consistent calculations, 
%long ranged coupling may occur from charge fluctuations and changed dipole field. 
%In this case the perturbation is no longer local.

\end{references}

\begin{figure}
\caption{\small  The 
${\rm Error} = \log_{10}(||X^{(0)}_n + \Delta_n - P_{\rm exact}||_2)$
as a function of iterations $n$ ($N = 100$, $N_e = 50$). The lower 
inset shows the number of non-zero matrix elements of $\Delta_n$ 
above threshold $\tau = 10^{-6}$.  The upper inset shows the density matrix
perturbation.
\label{DX}}
\end{figure}

\begin{figure}
\caption{\small The change in density for the Hamiltonian
given in Eq.\ (\ref{HPRT}) for $k \in [0,1]$. The boundary 
conditions are periodic.
\label{Fig2}}
\end{figure}

\end{document}
