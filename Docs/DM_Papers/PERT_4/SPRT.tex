\documentclass[twocolumn,showpacs,preprintnumbers,amsmath,amssymb]{revtex4}
%\documentclass[preprint,showpacs,preprintnumbers,amsmath,amssymb]{revtex4}
% Some other (several out of many) possibilities
%\documentclass[preprint,aps]{revtex4}
%\documentclass[preprint,aps,draft]{revtex4}
%\documentclass[prb]{revtex4}% Physical Review B

\usepackage{graphicx}% Include figure files
\usepackage{dcolumn}% Align table columns on decimal point
\usepackage{bm}% bold math
\bibliographystyle{apsrev}

%\nofiles

\begin{document}

\preprint{LA-UR 04-5100}

\title{Density matrix perturbation theory for non-orthogonal representations}

\author{Anders M. N. Niklasson$^{(1),(\dagger)}$, Val\'ery Weber$^{(1),(2)}$, and Matt Challacombe$^{(1)}$}
\affiliation{$^{(1)}$Theoretical Division, Los Alamos National Laboratory, Los Alamos, New Mexico 87545}
\affiliation{$^{(2)}$Department of Chemistry, University of Fribourg, 1700 Fribourg, Switzerland}


\date{\today}

\begin{abstract}
Density matrix purification and perturbation theory provides an efficient
framework for linear scaling electronic structure calculations of adiabatic response properties. 
We present a non-orthogonal generalization of the density matrix perturbation theory, which 
recently was introduced by the authors (Niklasson and Challacombe, Phys.\ Rev.\ Lett.  
{\bf 92}, 193001 (2004); Weber, Niklasson and Challacombe, Phys.\ Rev.\ Lett.  
{\bf 92}, 193002 (2004)).  A non-orthogonal formulation includes basis-set 
dependent perturbations, such as geometry perturbations, when using a basis 
set of non-orthogonal local orbitals.  This is an important advantage, 
enabling reduced complexity computation of structural response properties 
such as elasticity, interatomic potentials, piezoelectric coefficients, and vibrational 
spectra (including higher order anharmonicities). A brief review of density matrix
purification and perturbation theory is also presented.
\end{abstract}

\pacs{02.70.-c, 31.15.-p, 71.15.-m, 71.15.Dx}% PACS, the Physics and Astronomy
                             % Classification Scheme.
\keywords{electronic structure theory, response, perturbation theory, density
matrix derivative, density matrix, linear scaling electronic structure theory, 
purification, sign matrix, O(N), basis set, orthogonal, non-orthogonal, transformation, 
matrix inverse, Green's function, recursion, commutation relations, Fermi operator}
\maketitle

\section{Introduction}

During the last decade a new computational paradigm has evolved in electronic structure
theory, where no critical part of a calculation is allowed to increase in complexity more than linearly 
with system size 
\cite{Yang91,Galli92,Mauri93,Ordejon93,Li93,Stechel94,Goedecker94,Silver94,Wang94,Kim95,Wang95,Abrikosov96,Galli96,Kohn96,Bowler97,Sanches97,Yokojima98,Baer98,Guerra98,Palser98,Goedecker99,Artacho99,Scuseria99,Ordejon00,Wu02,NiklassonTC2,Yam03,NiklassonPRT1,WeberPRT2}. 
Linear scaling electronic structure theory extends
conventional tight-binding, Hartree-Fock, and Kohn-Sham schemes
to the study of very large complex systems previously out of reach.
This is critical for applications to new areas of research in biology, medicine, 
and nanoscience, which often requires the ability to handle much larger
problems than previously possible.

Conventional electronic structure theory has in general three major 
computational bottlenecks:
1) the construction of the Hartree-Fock, tight-binding, or Kohn-Sham Hamiltonian, 
2) the self-consistent ground state total energy calculation, and quite often,
3) the computation of response properties.
Here we will focus on the last problem, including calculations of both linear 
and higher order non-linear response. Our main purpose is to present
a non-orthogonal generalization of the density matrix perturbation theory, 
applied in linear scaling computation of adiabtic response properties,
which recently was introduced by the authors \cite{NiklassonPRT1,WeberPRT2}.
A non-orthogonal generalization is important because it naturally includes perturbations
in the overlap matrix, when using a basis set of non-orthogonal local orbitals.  This is 
necessary for the calculation of structural response properties, such as elasticity, 
interatomic potentials, piezoelectric coefficients, and vibrational spectra. 
The corresponding perturbations must also be included when calculating electromagnetic 
response using field-dependent basis functions.

In a non-orthogonal representation the Hartree-Fock or Kohn-Sham
eigenvalue problem takes the generalized form,
\begin{equation}\label{SEq}
{\rm H}\phi_i = \varepsilon_i {S} \phi_i.
\end{equation}
In the case when both the Hamiltonian ${\rm H}$ and the overlap matrix $S$ 
are modified by perturbations ${\rm H'}$ and ${S'}$, respectively, we have 
the generalized perturbed eigenvalue problem
\begin{equation}\label{SEqP}
\left( {\rm H} + {\rm H'}\right) \phi_i = \varepsilon_i \left( {S+S'}\right) \phi_i.
\end{equation}
Our previously formulated density matrix perturbation theory \cite{NiklassonPRT1,WeberPRT2}
was based on the orthogonal, standard eigenvalue problem, and did not include perturbations 
in the overlap matrix ${\rm S'}$.  Changes in the overlap matrix occur because of modifications 
in the non-orthogonal basis set. With a basis set of atom-centered local orbitals this happens 
when the geometry is perturbed. With a field-dependent basis set it occurs when the field is altered.

To extend the density matrix perturbation theory to include perturbations in the basis
we will construct a non-orthogonal generalization of the density matrix perturbation theory.
Non-orthogonal generalizations of linear scaling methods for 
the calculation of the density matrix have previously been constructed by including the overlap 
matrix as a metric tensor in operator products \cite{Nunes94,White97,Palser98}.
We will follow a similar approach for the density matrix perturbation theory.

Our approach to density matrix perturbation theory avoids the costly intermediate calculation
of eigenvalues and perturbed orbitals. It is therefore related
to the orbital-free schemes proposed by Ochsenfeld and Head-Gordon \cite{Ochsenfeld97,Ochsenfeld04} as
well as the exponential parametrization of the density matrix by Larsen {\em et al.} \cite{Larsen01}.
However, in contrast to these methods, which pose the perturbed density matrix
implicitly through a set of commuting Sylvester-like equations, our approach explicitly constructs
the density matrix and its response derivatives through recursion. In this way the density matrix 
response can be calculated to any order using standard sparse matrix algebra.
The new orbital-free perturbation theory is surprisingly simple and linear scaling complexity is achieved
already at fairly small systems sizes even in self-consistent {\em ab initio} calculations of
three-dimensional systems using a well controlled numerical threshold \cite{WeberPRT2}.

This article is outlined as follows: First we give a brief review of the orthogonal 
formulation of density matrix purification and perturbation theory. 
Operators represented in an orthogonal basis set are described by italics ($P$). 
Thereafter we generalize the description to non-orthogonal representations, where 
the matrices are described by normal letters (P). The central result is the non-orthogonal
density matrix perturbation theory presented in section \ref{NOPRT}.
A simple example is given in detail for the expansion of the interatomic pair 
interaction of the diatomic H$_2^+$ molecule up to fourth order.


\section{Orthogonal Density Matrix Purification and Perturbation Theory}

\subsection{Orthogonal purification}

The efficiency of density matrix methods used in linear scaling electronic structure theory
is based on the quantum locality (or nearsightedness) of non-metallic systems
\cite{Kohn59,Kohn96,Baer97,Hastings04}. The locality is manifested in 
the approximate exponential decay of the density matrix with inter-atomic separation.  In a local orbital basis
the density matrix therefore has a sparse matrix representation and the number of non-zero entries
above some numerical threshold scales linearly $O(N)$ with the system size $N$ for sufficiently large 
extended problems.  Using efficient sparse matrix algebra the density matrix can 
therefore be constructed with a computational $N$-scaling cost. There are
several approaches to achieve this.  Density matrix purification theory, from which the
corresponding density matrix perturbation theory is derived, is based on the Fermi operator 
relation at $T=0$ between the density matrix $P$ and the Hamiltonian $H$,
\begin{equation}\label{DM}
P = \theta(\mu I -  H).
\end{equation}
By expanding the step function $\theta$, where the step is formed at the chemical potential $\mu$,
we can construct the density matrix. One way of constructing the density matrix is by approximating the Fermi operator
in terms of a Chebychev expansion \cite{Goedecker94,Silver94,Wang94,Baer97_JCP,Liang04},
\begin{equation}\label{Cheb}
P = \theta(\mu I -H) \approx \sum_{k=0}^m a_kT_k(H),
\end{equation}
where the two-step recurrence relation for the Chebychev polynomials $T_k(H)$ can be used,
\begin{equation}
T_{k+1}(X) = 2XT_k(X)-T_{k-1}(X).
\end{equation}
For a $m$th order Chebychev approximation, using the two-step recurrence relation,
the computational cost scales as $O(m)$ with the polynomial order of the expansion. However, 
the cost can be reduced to $O({\sqrt m})$ by using more efficient partial summations
of the polynomial terms instead of the two-step recurrence relation \cite{Paterson73,Liang04}. 
This reduces the computational cost, but more intermediate memory is required to store
temporary matrices.  The order of the polynomial approximation in Eq.\ (\ref{Cheb}), $m$, must therefore
be kept fairly low. Another problem is the occurrence of rapid oscillations around the step in a finite polynomial 
approximation. A Chebychev expansion gives smaller errors compared to other polynomial
approximations, but the oscillations do not disappear.  

Density matrix purification theory avoids these problems
\cite{McWeeny60,Clinton69,Palser98,Holas01,NiklassonTC2,NiklassonTRS4,NiklassonIPUR,Mazziotti03,NiklassonPRT1}.
It approximates the step function recursively,
\begin{equation}\label{Rec}
 P = \theta(\mu I -H) = \lim_{n \rightarrow \infty} F_n(F_{n-1}(...F_0(H)...)),
\end{equation}
which leads to a very high-order approximation in only a few number of iterations.
For example, using polynomials $F_n$ of second order, we can in only 30 steps reach an
expansion order $m \sim 10^9$.
$F_0(H)$ is an initial linear transform, normalizing
the spectra of $H$ to $[0,1]$ in reverse order. The functions $F_n$ ($n>0$) are low order,
monotonically increasing polynomials, with fix points at $0$ and $1$. Each
polynomial $F_n$ recursively projects the eigenvalues of 
\begin{equation}
X_n = F_{n-1}(X_{n_1})= \ldots = F_{n-1}(...F_0(H)...)
\end{equation}
to $0$ for unoccupied states and to $1$ for occupied states, thereby ``purifying'' the approximate 
density matrix $X_n$.  Because of the monotonically increasing form of the purification projectors $F_n$ any 
finite approximation will also be a monotonically increasing function. Oscillations, subject to numerical
inaccuracy in the Chebychev expansion, are therefore avoided.

After each recursive expansion step we apply numerical thresholding, removing all matrix elements below
some tolerance $\tau$. This often leads to a substantial increase in computational efficiency, but also
to an accumulation of the numerical error in each recursive purification step, which initially increases
exponentially \cite{NiklassonTRS4}. This may first look like a serious problem. However, the exponential 
error accumulation disappears when the eigenvalues of $X_n$ are clustered around $0$ and $1$. Since the 
number of purification steps necessary to reach convergence scales with the logarithm of the inverse
band gap, $\log ( 1/gap)$, the total accumulated error is stable and scales linearly with 
the threshold $\tau$ \cite{NiklassonTRS4}. The accuracy of the calculation is therefore easy
to control.

Density matrix purification methods differ in the way the purification polynomials $F_n(X_n)$
are chosen. In grand canonical schemes \cite{Palser98,Holas01,NiklassonTC2} the initial linear normalization
$X_1 = F_0(H)$ shifts the  eigenvalues such that all occupied eigenvalues are in $[c,1]$ and
all unoccupied eigenvalues in [$0,c]$, where $c$ is some predefined number (typically $c = 0.5$).
Thereafter a purification polynomial with inflection point at $c$ is used, which projects
eigenvalues above (below) $c$ to $1$ ($0$). At convergence the correct occupation is therefore reached, with
\begin{equation}
Tr(P) = \lim_{n \rightarrow \infty} Tr(X_n) = N_e.
\end{equation}
The problem with this approach is that it requires prior knowledge of the 
chemical potential $\mu$, which has to be shifted to the inflection point $c$ in the initial
normalization $X_1 = F_0(H)$.
To avoid the problem with an unknown chemical potential Palser and Manolopoulos (PM) invented a
canonical purification scheme \cite{Palser98} with the projection polynomials chosen
such that the trace, i.e. the occupation, is preserved in each purification projection.
By choosing the initial normalization such that $Tr[F_0(H)] = N_e$ the PM scheme
automatically converges to the correct density matrix, without prior knowledge of 
the chemical potential.  The problem with this method is that it has a very slow
convergence at high or low occupation \cite{Palser98,NiklassonTC2}.
A solution to this problem was given by the introduction of trace correcting purification \cite{NiklassonTC2}
described below.

\subsubsection{Second order trace correcting purification}

Probably the most efficient approach to density matrix purification, which is efficient also at 
high or low occupation, and in addition avoids the problem with an unknown chemical potential, 
is trace correcting purification \cite{NiklassonTC2,NiklassonTRS4,Mazziotti03}. In trace correcting 
purification the polynomials $F_n(X_n)$ are chosen to correct the trace $Tr(X_n)$ and
to expand the step function at the same time. In this case the correct occupation of the density matrix, 
$Tr(P) = \lim_{n \rightarrow \infty} Tr(X_n) = N_e$, is reached at convergence, without prior 
knowledge of the chemical potential. The simplest and most memory efficient form by Niklasson \cite{NiklassonTC2}
is given by
\begin{equation} \label{Pract1}
X_1 = F_0(H) = \frac{H_{\rm max}I-H}{H_{\rm max} - H_{\rm min}},
\end{equation}
\begin{equation}\label{TC2}
X_{n+1} = F_n(X_n) = X_n + \sigma_n (I-X_n)X_n,
\end{equation}
where
\begin{equation}\label{Pract2}
\sigma_n = sign [N_e - Tr(X_n)],
\end{equation}
and
\begin{equation}
P = \lim_{n \rightarrow \infty} X_n.
\end{equation}
The constants $H_{\rm max}$ and $H_{\rm min}$ are upper and lower estimates of the spectral
bounds of $H$, given for example by Gersgorin circles \cite{Palser98}. The initial
normalization $F_0(H)$ thus transforms all eigenvalues of $H$ to the interval $[0,1]$ in 
reverse order.  The sign function $sign(x)$ denotes the sign of $x$. It is $+1$ if
$x>0$, otherwise it is $-1$.  The trace correcting recursion in Eq.\ (\ref{TC2}) is
equivalent to previously published versions of the method \cite{NiklassonTC2,NiklassonPRT1,WeberPRT2},
but it is here formulated in a compact form more closely related to an efficient implementation.

Computationally the most expensive step is the matrix-matrix multiplication. For sufficiently sparse
systems these can be performed with a computational cost scaling linearly with the system size.

In the case of large numerical thresholding, combined with high or low occupation, we may push
eigenvalues out of the range of convergence, which is guaranteed only in $[0,1]$. To avoid this 
potential problem we alternate the sign of the trace correction in every step, with $\sigma_n = -\sigma_{n-1}$, 
when we get close to convergence, i.e. when $Tr[(I-X_n)X_n]$ is fairly small (typically $0.1$),
indicating a clustering of the eigenvalues of $X_n$ around $0$ and $1$. This does not affect the trace at
convergence, but extends the interval of convergence for the eigenvalues.


\subsection{Orthogonal Perturbation Theory}

The main problem with density matrix perturbation theory is the
discontinuous relation between the zero temperature density matrix
and the Hamiltonian, given by the step function in Eq.\ (\ref{DM}). 
If we use an analytic approximation of the step
function, we need a very high order expansion for accurate results. 
At finite temperatures we may use the analytic Fermi-Dirac function, but this involves
the computation of matrix exponentials and requires the prior knowledge
of the chemical potential to high precision.
However, with the recursive purification projection scheme we have an analytic
monotonically increasing and very accurate high-order approximation of the step function.
This expansion can be used in the calculation of the density matrix derivative with respect
to a variation in the Hamiltonian.  The basic idea in our theory is that 
an initial perturbation in $H$ can be carried through at each projection
level, either exactly or to any finite order. The resulting
perturbed projection scheme provides an efficient $O(N)$ technique for
the calculation of density matrix derivatives and materials adiabatic response properties 
\cite{NiklassonPRT1,WeberPRT2}..

\subsubsection{Exact expansion (infinite order)}

Assume a perturbation in the Hamiltonian,
\begin{equation}
H = H^{(0)} + H^{(1)}.
\end{equation}
The recursive expansion of the density matrix
\begin{equation}
F_n(F_{n-1}(\ldots F_1(F_0(H^{(0)}+H^{(1)}))\ldots )) 
\end{equation}
generates the corresponding perturbed sequence,
\begin{equation}\begin{array}{ll}
{\widetilde X}_n &= X_n + \Delta_n,\\
{\widetilde X}_{n+1} &= F_{n}({\widetilde X}_n),\end{array}
\end{equation}
where $X_n$ is the unperturbed sequence generated from $X_{n+1} = F_{n}(X_n)$
with $X_0 = H^{(0)}$. The initial perturbation is given by $\Delta_0 = H^{(1)}$.
The perturbed density matrix is given by
\begin{equation}
P = P^{(0)} + \lim_{n \rightarrow \infty} \Delta_n.
\end{equation}
It is easy to see that the perturbations $\Delta_n$ can be generated recursively by the perturbed projections 
\begin{equation}\label{PRT}
\Delta_{n+1} = F_{n}(X_n + \Delta_n) - F_{n}(X_n).
\end{equation}
This recursive equation forms the basis of our perturbation theory.
Combined with the trace correcting purification projection, $F_n$ in Eq.\ (\ref{TC2}), we derive
the following scheme for the density matrix perturbation:
\begin{equation} \label{DYS_SP2}
\Delta_{n+1} =
\left\{\begin{array}{ll}
\{ X_n,\Delta_n\} + \Delta_n^2, & {\rm Occ.} \geq N_e \\
2\Delta_n - \{ X_n,\Delta_n\} - \Delta_n^2, & {\rm Occ.} < N_e ,
\end{array} \right.
\end{equation}
where we use the anti-commutator notation $\{A,B\}=AB+BA$ and the occupation ${\rm Occ.}  = Tr(X_n)$.
The initial perturbation is
\begin{equation}
\Delta_1 = - H^{(1)}/(H_{\rm max}-H_{\rm min}).
\end{equation}
Since the perturbation $\Delta_n$ changes quadratically in each iteration the
expansion order is in practice infinite at convergence and therefore exact.
For insulators the computational cost scales linearly with the size of the
perturbed region $O(N_{\rm pert.})$ since the recursion only involves terms with the response
factors $\Delta_n$. For a local perturbation the computational cost is therefore
independent of system size, i.e. it scales as $O(1)$ \cite{NiklassonPRT1}.

The perturbation theory is grand canonical since the expansion of the perturbation is performed
at a fixed chemical potential determined by the unperturbed system. For sufficiently large perturbations, 
states may cross the chemical potentials $\mu$. In this case $\lim_{n \rightarrow \infty} Tr(\Delta_n) 
\neq 0$ and the system is no longer neutral.

\subsubsection{Finite perturbation expansion}

Assume a perturbation expansion of the Hamiltonian,
\begin{equation}
H = H^{(0)} + \lambda H^{(1)} + \lambda^2 H^{(2)} + \ldots ~ .
\end{equation}
This perturbation generates the corresponding perturbed sequence
\begin{equation}
{\widetilde X}_n = X_n^{(0)} + \lambda \Delta^{(1)}_n + \lambda^2 \Delta^{(2)}_n + \ldots ~ ,
\end{equation}
where the separate $m$th order perturbations $\Delta^{(m)}_n$ can be collected
order by order. Using the second order trace correcting scheme and
keeping terms through order $m$ in $\lambda$ at each iteration,
with $\Delta^{(0)}_n \equiv X^{(0)}_n$, the following explicit recursive sequence is obtained
for $m = m_{\rm max},m_{\rm max}-1,\ldots ,1,0$:
\begin{equation}\label{second1}
\Delta^{(m)}_{n+1} =
\left\{ \begin{array}{ll}
\displaystyle \sum_{i=0}^{m} \Delta^{(i)}_n \Delta^{(m-i)}_n, &
{\rm Occ.} \geq N_e\\
\displaystyle 2\Delta^{(m)}_n - \sum_{i=0}^{m} \Delta^{(i)}_n \Delta^{(m-i)}_n.
& {\rm Occ.} < N_e\\
\end{array} \right.
\end{equation}
The occupation ${\rm Occ.} = {Tr}(\Delta^{(0)}_n) \equiv {Tr}(X^{(0)}_n)$.
The density matrix derivatives are given by
\begin{equation}
\frac{1}{m !}\frac{\partial^m P}{\partial \lambda^m} \bigg|_{\lambda = 0} = P^{(m)}=
\lim_{n\rightarrow\infty} \Delta^{(m)}_{n},
\end{equation}
such that
\begin{equation}
P = P^{(0)} + \lambda P^{(1)} + \lambda^2 P^{(2)} + \ldots ~ .
\end{equation}
These equations provide an explicit and rapidly convergent recursive solution
of the density matrix response functions to any order with a minimal amount
of formalism \cite{NiklassonPRT1,WeberPRT2}.  Mixed terms for multiple independent 
perturbations can also be included.


\section{Non-orthogonal Purification and Perturbation Theory}

If we use a non-orthogonal representation of the operators, the ``standard''
eigenvalue problem in quantum mechanics is replaced by
the generalized matrix eigenvalue problem,
\begin{equation}\label{SES}
{\rm H}\phi_i = \varepsilon_i {S} \phi_i, \quad  \varepsilon_1 \leq \varepsilon_2 \leq \ldots ,
\end{equation}
where ${S}$ is the overlap matrix. As mentioned in the introduction, we use straight normal letters (H)
to distinguish the non-orthogonal operator representations from the corresponding
orthogonal representations described by italics ($H$). In the non-orthogonal
case we can adapt the density matrix purification scheme to a non-orthogonal form. 
The goal is to extend the density matrix perturbation theory to
include perturbations occurring in the overlap matrix of the generalized
eigenvalue problem. This is important for response calculation where the basis set 
is affected by the perturbation.

\subsection{Non-orthogonal Purification}

In the non-orthogonal case the necessary criteria for the orthogonal density matrix ${\rm P}$ are
\begin{equation}\label{CRIT_S}\begin{array}{l}
[{\rm P,H}]_{S} = S{\rm PH-HP}S = 0, \\
Tr({\rm P}S) = N_e, \\
{\rm P = P}S{\rm P}. \\
\end{array}
\end{equation}
Combined with the aufbau principle, i.e. occupying the $N_e$ lowest eigenstates, 
the density matrix is determined for the non-orthogonal representation. The density matrix
purification and perturbation theory can be modified to these new criteria by
starting with an initial normalization X$_1 = {\rm F}_0({\rm H})$ such that 
\begin{equation}\label{INIT}
S{\rm X}_1{\rm H  - HX}_1{S} = 0.
\end{equation}
We also need to replace all orthogonal matrix-matrix multiplications ($Z=XY$) in 
the spectral projections  $F_n(X_n)$ by the non-orthogonal multiplications, 
including the overlap matrix $S$ as a metric tensor in each multiplication, i.e.
\begin{equation}\label{Metric}
Z = XY ~~ {\rightarrow} ~~ {\rm Z} = {\rm X}S{\rm Y}.
\end{equation}
This also preserves the covariant (or contravariant) form after each purification projection \cite{White97}.
The main problem in this approach is the initial normalization of X$_1$, which has to
commute with ${\rm H}$ in Eq.\ (\ref{INIT}). Once this normalization
is performed the commutation is automatically preserved as long as
\begin{equation}
S{\rm F}_n({\rm X_n}){\rm H}-{\rm H}{\rm F}_n({\rm X_n})S = 
S{\rm X_n}{\rm H}-{\rm H}{\rm X_n}S, 
\end{equation}
which is true for the purification projections ${\rm F}_n({\rm X_n})$ discussed in this paper.
There are several options for the initialization:
\begin{eqnarray}
\label{INIT1}
\displaystyle {\rm X}_1 = {\rm F}_0({\rm H}) = \alpha(\beta S^{-1} - S^{-1}{\rm H}S^{-1}), \\
\label{INIT2}
\displaystyle {\rm X}_1 = {\rm F}_0({\rm H}) = \alpha({\rm H}^{-1} + \beta S^{-1}), \\
\label{INIT3}
\displaystyle {\rm X}_1 = {\rm F}_0({\rm H}) = \alpha({\rm H} - \beta S)^{-1},
\end{eqnarray}
where $\alpha$ and $\beta$ are chosen to guarantee convergence, i.e. such that the
corresponding transformed generalized eigenvalue problem in Eq.\ (\ref{SES}) 
has its eigenvalues ${\varepsilon}_i \in [0,1]$.
The first case, Eq.\ (\ref{INIT1}), which is analogous to the initial guess suggested by Palser and Manolopoulos 
\cite{Palser98} for their non-orthogonal grand canonical purification scheme, corresponds to a 
congruence transform of the generalized eigenvalue problem to a standard from
by multiplying Eq.\ (\ref{SES}) by $S^{-1}$ from the left. An additional (but not necessary)
multiplications of $S^{-1}$ from the right hand side of ${\rm H}$ is also included, which keeps
the matrices symmetric. Because of this multiplication the additional overlap matrix must be 
included in the purification projections, which may
seem unnecessary. The second case, Eq.\ (\ref{INIT2}), is of minor interest since it involves 
calculations of both ${\rm H}^{-1}$ and $S^{-1}$. The third normalization, Eq.\ (\ref{INIT3}), 
is the most interesting and useful of the initializations.  
With $\alpha = 1$ and $\beta = \varepsilon_{\rm min} -1$, where $\varepsilon_{\rm min}$
is a lower bound of the eigenvalues $\varepsilon_i$ in Eq.\ (\ref{SES}), this is a
Green's function ,
\begin{equation} \label{GNorm}
{\rm G}(\beta) = ({\rm H} -  \beta S)^{-1},
\end{equation}
with the correct normalization to guarantee convergence. 
At low occupation the Green's function initialization is equivalent to a single 
implicit purification step \cite{NiklassonIPUR}, projecting high energy states ($\varepsilon_i \gg 1$), 
toward $0$ by inversion ($1/\varepsilon_i$).

\subsubsection{Computation and refinement of ${\rm X}_1$}

The Green's function resolvent ${\rm G}(\beta)$ can be calculated with linear scaling complexity
for sufficiently large and sparse systems using several techniques, such as Schulz iterations \cite{Schulz33},
the sparse approximate inverse \cite{Benzi96,Challa99}, or other methods \cite{Ozaki01}. 
In a self-consistent calculation, where the Hamiltonian is changed in each iteration, or in
a quantum molecular dynamics simulation, where both the overlap and the Hamiltonian is modified, 
we can efficiently update the new initialization 
\begin{equation}
{\rm X}_{1({\rm new})}  = ({\rm H_{new}} -  \beta S_{\rm new})^{-1}
\end{equation}
from the previous iteration. If  $X_{1({\rm old})}$ and ${\rm X}_{1({\rm new})}$ are sufficiently close
the following scheme, based on Schulz's method \cite{Schulz33}, rapidly converges
to the new normalization:
\begin{equation}\begin{array}{ll}
{\rm Y}_0 &= X_{1({\rm old})},\\
{\rm Y}_{n+1} &= 2{\rm Y}_{n} - {\rm Y}_{n}({\rm H}_{\rm new} -  \beta_{\rm new} S_{\rm new}){\rm Y}_{n},\\
{\rm X}_{1({\rm new})} &= {\rm G}_{\rm new}(\beta) = \lim_{n \rightarrow \infty} {\rm Y}_{n}.
\end{array}
\end{equation}
In this way we can reduce the cost by using Schulz's method as an efficient iterative
refinement technique. 

\subsubsection{Non-orthogonal trace correcting purification}

Following the necessary modifications, including the new metrics for non-orthogonal
representations, the second order trace correcting purification scheme is given by

\begin{equation} \label{Pract1S}
{\rm X}_1 = {\rm G}(\beta)  = \left[{\rm H} -  (\varepsilon_{\rm min} -1) S\right]^{-1},\\
\end{equation}
\begin{equation}\label{TC2S}
{\rm X}_{n+1} = {\rm F}_n({\rm X}_n) = {\rm X}_n + \sigma_n (I-{\rm X}_nS){\rm X}_n,
\end{equation}
where
\begin{equation}\label{Pract2S}
\sigma_n = sign [N_e - Tr(S{\rm X}_n)],
\end{equation}
and
\begin{equation}
{\rm P} = \lim_{n \rightarrow \infty} {\rm X}_n.
\end{equation}
Notice that any of the initializations in Eqs.\ (\ref{INIT1})-(\ref{INIT3}) can be used
to calculate ${\rm X}_1$ equivalently. 

\subsection{Non-orthogonal Perturbation Theory}\label{NOPRT}

Based on the previous non-orthogonal formulation of the density matrix purification,
an extension of the perturbation theory to non-orthogonal representations is straightforward.
This non-orthogonal generalization is the key purpose of this paper. It enables reduced complexity
calculations of structural response properties such as interatomic potentials, the Born-effective
charge, the vibrational spectra, including higher order anharmonicity, and piezoelectric
coefficients. It can also be used for calculations of electromagnetic response where
the basis set is field dependent. First we show how the exact perturbation expansion work, which is useful
in the case of local structural relaxations \cite{NiklassonPRT1}. Thereafter we show how 
density matrix response to a finite order can be calculated.

\subsubsection{Exact expansion (infinite order)}

For the exact perturbation expansion we replace 
$X_n$ and $F_n(X_n)$ in Eq.\ (\ref{PRT}) by
the non-orthogonal purification projections ${\rm X}_n$ and projectors ${\rm F}_n({\rm X}_n)$, 
such that \begin{equation}\label{PRTS}
\Delta_{n+1} = {\rm F}_n({\rm X}_n+\Delta_n) -  {\rm F}_n({\rm X}_n),
\end{equation}
and
\begin{equation}\label{InitS}
\Delta_1 = {\rm F}_0({\rm H}^{(0)}+{\rm H}^{(1)}) - {\rm F}_0({\rm H^{(0)}}).
\end{equation}
With ${\rm F}_0({\rm H})$ given by one of the normalizations in Eqs.\ (\ref{INIT1})-(\ref{INIT3}) and 
${\rm F}_{n}({\rm X}_n)$ by Eq.\ (\ref{TC2S}) we have
\begin{equation}
\Delta_{n+1} = \left\{\begin{array}{l}
\{\Delta_n,{\rm X}_n\}_S + \Delta_nS\Delta_n, ~~ {\rm Occ.} \geq N_e \\
2\Delta_n - \{\Delta_n,{\rm X}_n\}_S - \Delta_nS\Delta_n, ~ {\rm Occ.} < N_e. \end{array} \right.
\end{equation}
Here we use the non-orthogonal anti-commutator notation
\begin{equation}
\{\Delta_n,{\rm X}_n\}_S = \Delta_nS{\rm X}_n+{\rm X}_nS\Delta_n,
\end{equation}
and the occupation ${\rm Occ.} = Tr(S{\rm X}_n)$.
Sine the perturbation theory is based on the perturbed projections, given as the difference
between the purification of the perturbed and unperturbed sequence, the covariant 
(or contravariant) form of the perturbation $\Delta_n$ is preserved in each step. This holds true also 
for the finite perturbation expansion described below. At convergence
\begin{equation}
{\rm P} = {\rm P}^{(0)} + \lim_{n \rightarrow \infty} \Delta_n.
\end{equation}

\subsubsection{Finite order perturbation}

The perturbation expansion to finite order follows equivalently with the non-orthogonal
generalization of the exact expansion. Assume a perturbation of ${\rm H}^{(0)}$ and $S^{(0)}$, 
where
\begin{equation}
{\rm H} = {\rm H}^{(0)} + \lambda {\rm H}^{(1)} + \lambda^2 {\rm H}^{(2)} + \ldots ~,
\end{equation}
and
\begin{equation}
S = S^{(0)} + \lambda S^{(1)} + \lambda^2 S^{(2)} + \ldots ~.
\end{equation}
The initial perturbation $\Delta_1$ in Eq.\ (\ref{InitS}) can be calculated and expanded to
any order in $\lambda$,
\begin{equation}
{\Delta}_1 = \lambda \Delta^{(1)}_1 + \lambda^2 \Delta^{(2)}_1 + \ldots ~,
\end{equation}
using any of the initial normalizations in Eqs.\ (\ref{INIT1})-(\ref{INIT3}).
However, using the Green's function approach in Eq.\ (\ref{INIT3}) makes the
expansion of $\Delta_1$ particularly easy. The separate terms are given by 
\begin{equation}\label{START}
\begin{array}{ll}
\Delta^{(0)}_1 &= {\rm X}^{(0)}_1 = {\rm G}^{(0)} \\
\Delta^{(1)}_1 &= -{\rm G}^{(0)} {\rm T}^{(1)} {\rm G}^{(0)}, \\
\Delta^{(2)}_1 &= -{\rm G}^{(0)} {\rm T}^{(2)} {\rm G}^{(0)} + {\rm G}^{(0)} {\rm T}^{(1)} {\rm G}^{(0)} {\rm T}^{(1)} {\rm G}^{(0)}, \\
\Delta^{(3)}_1 &= -{\rm G}^{(0)} {\rm T}^{(3)} {\rm G}^{(0)} + {\rm G}^{(0)} {\rm T}^{(1)} {\rm G}^{(0)} {\rm T}^{(2)} {\rm G}^{(0)} + \ldots ~,
\end{array}
\end{equation}
where
\begin{equation}
{\rm T}^{(m)} = ({\rm H}^{(m)}-\beta S^{(m)}),
\end{equation}
and
\begin{equation}\label{STARTG}
{\rm G}^{(0)} = ({\rm H}^{(0)}-\beta S^{(0)})^{-1}.
\end{equation}
This initialization to various order, $\Delta^{(m)}_1$, in Eq.\ (\ref{START}) was
derived from the Dyson series
\begin{equation}
{\rm G} = {\rm G}^{(0)} - {\rm G}^{(0)}\left[{\rm T}^{(1)} + {\rm T}^{(2)} + \ldots \right]{\rm G}^{(0)} + \ldots,
\end{equation}
which can be used to collect ${\rm G}$ order by order in $\lambda$. A generalization to any order
is straightforward.
After the initialization of $\Delta^{(m)}_1$ we have for $m = m_{\rm max}, m_{\rm max}-1 , \ldots ,1, 0$:
\begin{equation}\label{PRT_S}
\displaystyle \Delta^{(m)}_{n+1} = \left\{\begin{array}{ll}
\sum_{i + j + k = m} \Delta^{(i)}_n S^{(j)} \Delta^{(k)}_n, & {\rm Occ.} \geq N_e \\
 2\Delta^{(m)}_n - \sum_{i + j + k = m} \Delta^{(i)}_n S^{(j)} \Delta^{(k)}_n, & {\rm Occ.} < N_e
\end{array} \right.
\end{equation}
where $\Delta^{(0)}_n \equiv {\rm X}^{(0)}_n$. The sum is taken over all combinations of $i,j$ and $k$ such that
$i+j+k= m$. The occupation is given by ${\rm Occ.} = Tr(S^{(0)}\Delta^{(0)}_n) \equiv Tr(S^{(0)} X^{(0)}_n)$.
At convergence the density matrix derivatives are given by
\begin{equation}\label{DMD}
\frac{1}{m !}\frac{\partial^m {\rm P}}{\partial \lambda^m} \bigg|_{\lambda = 0} = {\rm P}^{(m)}=
\lim_{n\rightarrow\infty} \Delta^{(m)}_{n},
\end{equation}
such that the density matrix perturbation expansion in a non-orthogonal representation is 
\begin{equation}
{\rm P} = {\rm P}^{(0)} + \lambda {\rm P}^{(1)} + \lambda^2 {\rm P}^{(2)} + \ldots ~.
\end{equation}
This method for the calculation of density matrix response, including perturbations in the overlap matrix
for a non-orthogonal representation, composes the central result of this paper.

\section{Example}

To illustrate the non-orthogonal perturbation theory we have chosen a diatomic
hydrogen ion H$_2^+$ described in a basis set of two hydrogenic 1s-orbitals \cite{Atkins}.
The overlap matrix $S(R)$ as a function of inter-atomic distance $R$ (in units of Bohr radius $a_0$) 
is given by
\begin{equation}\begin{array}{l}
S_{1,1} = S_{2,2} = 1\\
S_{1,2} = S_{2,1} = (1+R+(1/3)R^2)e^{-R}.
\end{array}
\end{equation}
The matrix elements of the Hamiltonian H$(R)$ are
\begin{equation}\begin{array}{l}
{\rm H}_{1,1} = {\rm H}_{2,2} = E_0 -R^{-1}(1-(1+R)e^{-2R}) + \kappa R^{-1}\\
{\rm H}_{1,2} = {\rm H}_{2,1} = (E_0 + \kappa R^{-1})S_{1,2}-\kappa{a_0}^{-1}(1+R)e^{-R},
\end{array}
\end{equation}
where $\kappa = e^2/(4\pi \varepsilon_0)$ (set to $1$ in the calculation).
By expanding H and $S$ in $r  = (R-R_0)$ around the equilibrium distance $R_0$ (or any other point) we have
\begin{equation}\begin{array}{ll}
{\rm H} &= {\rm H}^{(0)} + r {\rm H}^{(1)} + r^2 {\rm H}^{(2)} + \ldots ~,\\
S &= S^{(0)} + r S^{(1)} + r^2 S^{(2)} + \ldots ~,
\end{array}
\end{equation}
where
\begin{equation}\begin{array}{lll}
{\rm H}^{(m)} &= (m!)^{-1}\partial^m{\rm H}/\partial R^m & {\rm at} \quad R = R_0,\\
S^{(m)} &= (m!)^{-1}\partial^mS/\partial R^m & {\rm at} \quad R = R_0.
\end{array}
\end{equation}
The initial perturbations $\Delta^{(m)}_1$ are given by Eq.\ (\ref{START})
and the recursive expansion is calculated as described in Eq.\ (\ref{PRT_S}).
At convergence the normalized density matrix derivatives ${\rm P}^{(m)}$ are given by Eq.\ (\ref{DMD}).
The expansion of the energy is given by collecting the energy
\begin{equation}\label{Energy}
E = Tr\left[ ({\rm H}^{(m)} + r {\rm H}^{(1)} + \ldots )({\rm P}^{(0)} + r {\rm P}^{(1)} + \ldots)\right]
\end{equation}
in orders of $r$. Figure \ref{H2} shows the interaction potential $E(R)$
as a function of inter-atomic distance in comparison to perturbation expansions up to 4th order 
at the equilibrium distance and at 4th order at a non-equilibrium inter-atomic distance. 

                                                                                               
\begin{figure}[t]
\resizebox*{3.0in}{!}{\includegraphics[angle=00]{H2_Ion_4_noneq.eps}}
\caption{\label{H2}
The expansion of the energy $E(R)$, Eq.\ (\ref{Energy}), as a function of inter-atomic 
distance $R$ for the H$_2^+$ molecule.
The non-orthogonal perturbed projection expansion of the density matrix derivatives was
performed as described in Eq.\ (\ref{PRT_S}) with the initialization given in
Eq.\ (\ref{START}). The analytic expansion is performed around equilibrium distance
up to 4th order at the equilibrium distance and at 4th order at a
non-equilibrium inter-atomic distance.}
\end{figure}

\section{Conclusions}

In this paper we have shown how density matrix perturbation theory based on recursive purification 
can be generalized to include basis-set dependent perturbations. This makes
it possible, for example, to calculate structural response properties 
using local atomic-centered orbitals within a reduced complexity formalism. 
Some key features of importance are: 1) orbital-free density matrix formulation,
which avoids the calculation of eigenfunctions and eigenvalues, 2) very high order, monotonically 
increasing analytic approximation of the step function, 3) initial normalization of the Hamiltonian 
to fulfill the non-orthogonal commutation relation, which is preserved after each purification, 
and 4) the ability to collect perturbations recursively, exactly (infinite order) or to any finite 
order, at each level of purification.

The example for the H$_2^+$ molecule illustrates the extension of the orbital-free density matrix
perturbation theory to non-orthogonal representations. We have also applied the non-orthogonal
method to the calculation of the polarizability of molecular clusters with results
identical to previous calculations \cite{WeberPRT2}. Since 
only matrix-matrix operations are used, the computational cost 
scales linearly with system size for sufficiently large non-metallic systems, 
as has been shown previously for a perturbed projection scheme in an orthogonalized representation 
\cite{WeberPRT2}.  The non-orthogonal density matrix perturbation theory can therefore efficiently 
be applied in calculations of geometry response properties for large complex systems. 

Apart from linear scaling calculations it is important to note that the density
matrix perturbation theory provides an efficient approach valuable also for dense 
matrix problems. Because of the ability to efficiently parallelize matrix-matrix
operations the computational overhead is fairly small. The limiting factor is 
the existence of a gap at the chemical potential,
since the number of purification iterations necessary to reach convergence scales with the logarithm 
of the inverse band gap \cite{NiklassonTC2}.

In the Green's function initialization of the purification, high energy states are projected 
toward $0$.  This may be a great advantage when using very large basis sets, where states at very high energies
($\varepsilon_i \gg 1$) are efficiently projected to $0$ (as 1/$\varepsilon_i$). This avoids a linear rescaling of 
the eigenstates that may lead to
a very small normalized band gap on $[0,1]$. If this band gap is small more
iterations are needed to reach convergence \cite{NiklassonTC2,NiklassonTRS4}.
The Green's function initialization may thus be
useful also for orthogonal representations when $S=I$. 
A practical generalization of the
Green's function initialization in Eq.\ (\ref{GNorm}) is given by
\begin{equation}
{\rm X}_1 = {\rm G}(z) = z^{-1}\left[({\rm H} - (\varepsilon_{\rm min} - z^{-1})S\right]^{-1},
\end{equation}
which is stable for all $z > 0$. The value of $z$ can be tuned to improve convergence and computational
efficiency by optimizing the size of the band gap of the normalized spectra of $X_1$. 
The purification projections are stable with respect to a complex generalization and the
constant $z$ can therefore be extended to regions of the complex plane, in analogy to 
Green's functions for complex energies. There are several other interesting connections to Green's functions
that we may explore in future studies \cite{Niklasson_Unpubl}.

If an ill-conditioned non-orthogonal basis set is used we may run into numerical problems if we chose to
transform the generalized eigenvalue problem to an orthogonal representation. 
With the present formulation for non-orthogonal purification and perturbation theory, this congruence transform 
is avoided. Instead it is replaced by the calculation of ${\rm G}(z)$.  However, if the condition number of 
${\rm G}(z)$ is smaller compared to the condition number of $S$, the numerical accuracy is improved.
In addition, the back-transform from the orthogonal density matrix representation to the atomic orbital 
representation, which is necessary to calculate the electronic density expressed in the atomic
orbital basis, is avoided within a purely non-orthogonal formalism.

The second order trace correcting purification scheme was applied to the
non-orthogonal purification and perturbation theory in this article. However, the formalism
can be directly transformed also to other purification methods, such as grand canonical McWeeny 
purification, implicit purification, matrix sign function expansions, or any other related method. 


\section{Acknowledgment}

Discussions with C.\ J. Tymczak and J. Wills are gratefully acknowledged.

\bibliography{PP}

{${\dagger}$ Corresponding author: Anders M.\ N. Niklasson, Email: amn@lanl.gov}

\end{document}
