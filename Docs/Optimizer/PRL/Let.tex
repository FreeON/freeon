\documentstyle[12pt,letterpaper]{letter}
\begin{document}
\pagestyle{empty}

\signature{K{\'a}roly N{\'e}meth}

\address{
Dr.~K{\'a}roly N{\'e}meth \\
(knemeth@lanl.gov) \\
Theoretical Division \\
Group T-12, MS B268 \\
Los Alamos National Laboratory \\
Los Alamos, NM 87545 }

\begin{letter}{
Professor Donald Levy, Editor\\
Journal of Chemical Physics\\
Department of Chemistry\\
5735 S. Ellis Ave\\
The University of Chicago\\
Chicago, IL 60637}

\opening{Dear Professor Levy,}

Enclosed, please find two copies of 
{\em 
A New View on Geometry Optimization: the Quasi-Independent 
Curvilinear Coordinate Approximation
} by N{\'e}meth and Challacombe,
which I am submitting for publication in the Journal of Chemical Physics (JCP). This 
article builds 
on our JCP publication {\em Linear scaling algorithm for the coordinate transformation problem of molecular geometry optimization., 
Journal of Chemical Physics; 8 Oct. 2000; vol.113, no.14, p.5598-603}.

The present article describes a new view on the geometry optimization
of molecules. Our new concept is based on the observation, that
internal coordinate gradients exhibit certain trends during the 
optimization. These trends can be formalized by curve-fitting. 
The roots of the fitted curve then serve as an improved guess
for the optimum geometry. Our new approach is unprecedented in the 
literature and competes with the well established Hessian matrix update
techniques. In addition to its high efficiency, our new
algorithm is very simple to implement and its operation count scales
linearly with system size. Thus it is very well suited for the
optimization of large molecules, as demonstarated by
the optimization of a large enzyme-fragment.

We would like to kindly ask you to consider our 
submission for publication in the Journal of Chemical Physics.

Our suggestion for reviewers iclude
Prof. Peter Pulay at University of Arkansas (Department of Chemistry and Biochemistry, University of Arkansas, Fayetteville, Arkansas 72701; pulay@uark.edu),  
Prof. Bernhard Schlegel at Wayne State University (Rm 371 Chemistry 
Bldg, Wayne State University, Detroit, MI, USA, 48202-3489; 
\newline
hbs@chem.wayne.edu), 
Prof. G{\'e}za Fogarasi at E{\"o}tv{\"o}s University (Lorand Eotvos Univ, Dept Theoret Chem, H-1518 Budapest, Hungary; fg@para.chem.elte.hu) and 
Prof. Jose-Maria Bofill at University of \newline Barcelona (Univ Barcelona, Fac Quim, Dept Quim Organ, 
Barcelona 08028, Spain; jmbofill@qo.ub.es).
We think that these scientists are probably the most well known
authorities today in the field of molecular geometry optimization.

\closing{With the very best regards,}
\end{letter}
\end{document}

