\documentstyle[12pt,letterpaper]{letter}
\begin{document}
\pagestyle{empty}

\signature{K{\'a}roly N{\'e}meth}

\address{
Dr.~K{\'a}roly N{\'e}meth \\
(knemeth@lanl.gov) \\
Theoretical Division \\
Group T-12, MS B268 \\
Los Alamos National Laboratory \\
Los Alamos, NM 87545 }

\begin{letter}{
Editorial Supervisor, \\
Journal of Chemical Physics, \\
American Institute of Physics, \\
Suite 1NO1, 2 Huntington Quadrangle, \\
Melville, NY 11747-4502 }

\opening{List of modifications in the proof copy of 309430JCP (A4.04.233)}

1. Modify title to : \\
"The quasi-independent curvilinear coordinate approximation for geometry optimization". This will avoid claims of priority and still reflects
properly our intention.
\\ \\
2. Everywhere in text, modify "quasiindependent" to "quasi-independent".
Let me know if you do not agree. I have looked up the word "quasi"
in Merriam Webster on-line dictionary (http://www.m-w.com/dictionary.htm)
. Similar expressions, like "quasi-judical", "quasi-legislative"
appear with "-". 
Also the spelling program "ispell" recommends "quasi-independent" 
or "quasi independent". 
Yet, I am not absolutely sure about what the correct form of this expression is. 
I rely on what your decision is, but if it is not certain, let's use
"quasi-independent".
\\ \\
3. Size of all Figures. \\
I would like to have the size of all Figures bigger. So that ratios
within figures remain the same, but each Figure occupies the maximum
available horizontal place. In the proof-copy, Figures do not occupy
the maximum available horizontal dimension (size of the column).
I understand, that shrinking Figures downscales the size of characters
in the Figure. But, e.g. in an other article in the same Journal,
by Kudin et.al. (Fig 2, JCP Vol 114, No 7, page 2921) I see much larger
letters published.
Of course, if necessary, I can generate the eps files
with smaller characters and e-mail them to you.
\\ \\
4. On page 2 of the Proof Copy replace "geometric DIIS (GDIIS) (Ref. 16) algorithm of Pulay" by
"geometric version of the direct inversion in the iterative
subspace algorithm (GDIIS) (Ref. 16) of Pulay" 
\\ \\
5. "SLATEC" and "POLFIT" is OK on page 4 of the Proof Copy.
\\ \\
6. Replace "ATP" by "adenosine triphosphate (ATP)"
(page 7 of the proof copy). 
\\ \\
7. On page 8 of the proof copy, in the "Conclusions", I would like 
to repeat the whole name for "QUICCA" as in the original version,
so write "QUasi Independent Curvilinear Coordinate Approximation, 
or QUICCA", instead of "the QUICCA". Doing so may help 
summarizing the results.
\\ \\
8. On page 8 of the proof copy, the address of the publisher is "New York".
\\ \\
9. The most important correction, that has to be done 
is in TABLE I of the proof copy. The table is missing a few lines.
Altogether data of thirty molecules should be present. The missing lines are
due to printing error that happened when I uploaded the manuscript 
after the referee report. The uploaded Latex file, however, contained
all necessary lines.\\
Also, I would like to put "QUICCA", "Bakken", "Eckert", "Lindh" and 
"Baker" aligned. I will enclose a copy of the whole table, as in my
original Latex file, in order to provide information about the missing 
rows.
\\ \\ 
10. Replace "PBE" in FIG. 1. by "Perdew-Burke-Ernzerhof (PBE)".
Also, cite here the "PBE" papers and add the corresponding reference
to the literature list :: \\
J. P. Perdew, K. Burke, and M. Ernzerhof, Phys. Rev. Lett. {\bf 77},
3865 (1996); {\bf 78}, 1396 (1997).
\\ \\
11. In Ref 20., replace "Chem. Phys. Phys. Chem." by
"Phys. Chem. Chem. Phys.". This refers to the journal "Physical 
Chemistry Chemical Physics".
\\ \\
12. The ${\cal{O}}(N)$-s have been changed to $O(N)$ in the proof copy.
Please change them back, since the original typesetting has
the mathematical meaning of "ordo". Other JCP papers contain 
${\cal{O}}$, too, for similar expressions.



Thank you very much for considering all these modifications!


\closing{With the very best regards,}
\end{letter}
\end{document}

