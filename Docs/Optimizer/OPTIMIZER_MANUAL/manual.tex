\documentclass[prl,aps,preprint,superbib,12pt]{revtex4}
%\documentclass[12pt]{article}
\oddsidemargin 2 cm
\topmargin -1 cm
\textwidth 14 cm
\textheight 22.0 cm

\usepackage{graphics}
\usepackage{amsfonts}
\usepackage{amsmath}
\usepackage{bm}
\usepackage{alltt}
\usepackage{fancyhdr}
\newcommand{\bms}[1]{{\boldsymbol #1}}
\renewcommand{\thefootnote}{\fnsymbol{footnote}}

\begin{document}
%\pagestyle{empty}

\section{Manual to Using the QUICCA Optimizer in MONDO}

\subsection{Default operation}
The input line to turn on the default options of the QUICCA optimizer
looks like:
\begin{itemize}
\item[{\tt Grad}]{\tt =(Optimize,PrimInt,NoBackTr,BiSect,NoGDIIS) }
\end{itemize}

\noindent
The meaning of these options is the following:
\begin{itemize}
\item[{\tt Optimize}] : Find a local minimum on the potential energy surface of the actual molecule
\item[{\tt PrimInt}] : Use primitive internal coordinates to carry out the optimization
\item[{\tt NoBackTr}] : Do NOT do any energy-based backtracking during
the optimization. The lack of this input-option automatically
turns on the backtracking.
\item[{\tt BiSect}] : This option turns on the QUICCA optimizer. An alternative of it is DiagHess, a simpler optimizer that uses a very rough 
diagonal Hessian.
\item[{\tt NoGDIIS}] : Disables Cartesian geometric-DIIS. The lack of this 
option turns on GDIIS. The use of Cartesian coordinates based 
GDIIS is not recommended.
\end{itemize}

\subsection{Minimization vs Stationary points}
\begin{itemize}
\item[{\tt DoGradNorm}] :
\end{itemize}
The option {\tt DoGradNorm} when placed in the {\tt Grad} line
will result in finding the nearest stationary point close to the
input structure, independent from the type of the stationary point
(minimum or saddle point). This is due to the fact that
{\tt DoGradNorm} forces the optimization of the gradient norm
while the default QUICCA optimizes the gradient curve so that 
the gradient curve will have a positiv derivative at the location of the
optimized structure. This requirement on the slope of the gradient
curve is omitted when {\tt DoGradNorm} is used.
 
\subsection{Constraints}

For gas phase molecules, constraints can be applied to 
Cartesian position, internal coordinates as well as to their 
combinations.
\\
For crystals, constraints can be applied to fractional coordinates,
lattice parameters, internal coordinates and to their combinations.

\subsubsection{Cartesian/fractional constraints}
The input file contains Cartesian or fractional coordinates of atoms.
If putting " C " at the end of the line that contains the coordinates
of a certain atom the coordinate will be frozen during the optimization.
\\
E.g. the line
\\
{\tt N    3.500  -0.500   0.000 C}
\\
has the meaning of keeping a "N" atom in the Cartesian position
of "3.500  -0.500   0.000" during the optimization. 
\\
The same line, when
being part of a crystal structure definition, means that
the fractional coordinates of the same "N" atom will be kept fixed
while the absolute Cartesian position is allowed to change as the 
lattice parameters change. This is independent from whether 
the input is given in "AtomCoord" or "FracCoord".

\subsubsection{Input section for other constraints}
To define other types of constraints the input section
\\
{\tt <BEGIN\_ADD\_INTERNALS>} and {\tt <END\_ADD\_INTERNALS>}
\\
must be defined, similar to {\tt <BeginGeometry>} and
{\tt <EndGeometry>}.

\subsubsection{Constraining lattice parameters}
An example of how to constrain lattice parameters is given
below. All these extra input lines must be entered between
\\
{\tt <BEGIN\_ADD\_INTERNALS>} and {\tt <END\_ADD\_INTERNALS>}
\\
The meaning of the following lines
\\
{\tt STRE\_A 12.6}
\\
{\tt STRE\_B 18.9}
\\
{\tt STRE\_B 10.3}
\\
{\tt ALPHA 120.0}
\\
{\tt BETA  96.0}
\\
{\tt GAMMA 106.0}
\\
is that the length of the lattice vectors A, B and C is constrained to
be 12.6, 18.9 and 10.3 respectively. Also, the lattice angles ALPHA, BETA
and GAMMA are constrained to be 120.0, 96.0 and 106.0.
\\
Lattice constraints can be applied to only a few or all lattice parameters,
in an arbitrary combination.

\subsubsection{Constraining internal coordinates}
The constraints on internal coordinates are very similar to those
on the lattice parameters.
Internal coordinate constraints can be defined for the following types of coordinates:
STRE (stretching), BEND (bending), TORS (torsion), OUTP (out-of-plane), LINB 
(linear bending).
Definitions of the constraits must also be placed between the lines of
{\tt <BEGIN\_ADD\_INTERNALS>} and {\tt <END\_ADD\_INTERNALS>}.
\\
For example
\\
{\tt STRE 15 26 1.3} 
\\
means that the streching coordinate between atoms 15 and 26 must take 
the value of 1.3 angstroems at the end of the optimization.
Other constraints like
\\
{\tt BEND 72 89 31   122.4}
\\
{\tt TORS 72 89 31 21  95.7}
\\
{\tt OUTP  1 25 65 43  0.15}
\\
{\tt LINB  1 5 16  8  180.0}
\\
have similar meaning: the BEND of atoms 72 89 31 (central atom 89) 
must become 122.4 degrees,
the TORS of atoms 72 89 31 21 (central bond between 89 and 31) must 
become 95.7, the OUTP of 1 25 65 43 (with 25 as the central atom, 
and 1 as the 'apex' atom) must become 0.15 degrees,
the LINB of 1 5 16 (5 is the central atom) must take the value of 180 degrees
in the plane defined by atoms 8 16 and 5, after the 1 5 16 angle is projected 
onto this plane.

Note that if the above definitions of internal coordinates are entered
without any constraint value (the last number) then they will be used as 
a flexible internal coordinate during the optimization.

In case of crystal optimizations internal coordinates bridging different
cells can also be defined, similarly to the molecular case, but
in addition the cell indices of atoms must be given.
For example
\\
{\tt TORS 72 89 31 21 CELL 0 0 0  1 0 0  1 0 1  1 1 1  95.7}
\\
defines a TORS coordinate whose 1st atom is in the central cell,
the 2nd atom is in the [1 0 0] cell, the 3rd in the [1 0 1] and
the 4th in the [1 1 1] cell. And this coordinate is constrained to
be 95.7 degrees. Leaving out 95.7 from the end, the coordinate will be
flexible during the optimization.

\subsubsection{Constraining individual Cartesian components}
Individual components of the Cartesian coordinates can be constrained 
exactly the same way as internal coordinates, by using the labels
CARTX, CARTY, CARTZ. Eg.
\\
CARTX 34 1145.98
\\
will constrain the X component of the atom 34 to 1145.98 angstroems.
In case of crystal optimizations, the corresponding fractional
coordinate will be fixed.

\subsection{Recognition of weak bonds}
Weak bonds, such as the Van der Waals bonds can be recognized difficultly,
because it is not clear in what situation they should be used and in what 
not.
The Van der Waals radii of atoms are used in a careful way in the default 
recognition algorithm, meaning that they are scaled dow to reduce the occurance
of overly large number of internal coordinates.
However, in cases when the molecule or crystal has a very large structural
contribution from Van der Waals interaction (eg. sulphure crystal) it is recommended to increase the Van der Waals radii of atoms for the internal coordinate 
recognition. This can be done by using the option {\tt VDWFact}.
Eg. 
\\
{\tt VDWFact=1.25}
\\ 
multiplies the values of default VDW radii by 1.25 . The factor 1.25 activates 
the full length of tabulated atomic Van der Waals radii to be applied.
The default VDW radii values are smaller to avoid the formation of an unnecessarily dense connectivity network.

\subsection{Stepsize control}
The options "MaxStre" and "MaxAngle" allow external control over the maximum
stepsize during an internal coordinate optimization.
Eg.
\\
{\tt MaxStre=0.1} 
\\
sets the maximum allowed stepsize to 0.1 angstroems on stretches,
while
\\
{\tt MaxAngle=5.0}
\\
sets the maximum allowed stepsize on BEND, TORS, OUTP and LINB-s of the 
molecule to 5.0 degrees.






\end{document}

