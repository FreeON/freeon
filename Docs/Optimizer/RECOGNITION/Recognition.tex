% NOTE TO AIP TYPSETERS: TO CONVERT FROM TWO-COL TO PREPRINT, SWITCH
% COMMENTOUT COMMAND FROM A TO B IE. USE:
%
%%\documentclass[prl,twocolumn,showpacs,twocolumngrid,superbib]{revtex4}
%\documentclass[prl,twocolumn,twocolumngrid,superbib]{revtex4}
%\newcommand{\commentoutB}[1]{}
%\newcommand{\commentoutA}[1]{#1}
%
% INSTEAD OF THE FOLLOWING:
%
\newcommand{\commentoutB}[1]{#1}
\newcommand{\commentoutA}[1]{}
%\documentclass[prl,aps,preprint,showpacs,superbib]{revtex4}
\documentclass[prl,aps,preprint,superbib,12pt]{revtex4}
\usepackage{graphicx}
\usepackage{amsfonts}
\usepackage{amsmath}
\usepackage{bm}
\usepackage{alltt}
\usepackage{fancyhdr}
\newcommand{\bms}[1]{{\boldsymbol #1}}
\renewcommand{\thefootnote}{\fnsymbol{footnote}}

%\draft
%\tighten
\pagestyle{fancy}

\begin{document}

\title{On the recognition of internal coordinates in various 
chemical systems}


\author{K\'aroly N\'emeth\footnotemark[1]}
\author{Matt Challacombe}

\affiliation{Theoretical Division,\\ Los Alamos National Laboratory, \\ Los Alamos, NM 87545, USA}

\date{\today}

\begin{abstract}
This paper presents several considerations for the appropriate choice 
of internal coordinates in various chemical systems. The appropriate
and black box recognition of internal coordinates is of fundamental 
importance
for the extension of internal coordinate algorithms to all fields where
previously Cartesian coordinates were the preferred means of geometry 
manipulations. Such fields range from local and global geometry 
optimizations to molecular dynamics as applied to a wide variety of
chemical systems. We peresent a robust algorithm that is capable 
to quickly determine the appropriate choice of internal coordinates
in a wide range of atomic arrangements.
\end{abstract}

%\pacs{31.15.-p,31.15.Ne,02.60.Pn, 45.10.Db, 02.40.Hw} 

\maketitle

\footnotetext[1]{\tt KNemeth@LANL.Gov}

\section{Introduction}
The earliest use of curvilinear internal coordinates 
in computational chemistry
is connected to vibrational analysis \cite{EWilson55}. 
It has been recognized in vibrational analysis 
that most molecular vibrations
are fairly well localized on internal coordinates that reflect
chemical concepts, such as chemical bonds, valence angles or
dihedral torsions. In fact, vibrational coupling between
appropriately chosen internal coordinates is typically
an order of magnitude smaller than in the corresponding Cartesian
representation \cite{PPulay69,GFogarasi79,GFogarasi92,PPulay77}.
This observation holds not only for harmonic but more importantly
also for anharmonic vibrational couplings.
The recognition of this reduced vibrational coupling lead to the
development of internal coordinates based geometry optimization
\cite{PPulay69,PPulay77,GFogarasi79,HSchlegel82,HSchlegel03} which
is now the standard mean of local optimization in most quantum
chemistry software packages. Internal coordinate geometry optimization
reduces the number of optimization steps typically by a factor of 2-10
for small or medium sized molecules, as compared to Cartesian
conjugate gradient algorithms \cite{TBucko05}. Another field of
potentially great use of internal coordinates is represented
by molecular dynamics \cite{PPulay02}, where research aims on developing
efficient simulation of long time-scale dynamics of large molecules.

In order to define a suitable internal coordinate set, first
the molecular connectivity has to be determined.
The connectivity of atoms is usually recognized on the basis of 
overlapping spheres of atomic or Van der Waals radii 
\cite{VBakken02,TBucko05,KNemeth04}. Atomic radii are often not suitable
to recognize bonding in e.g. ionic salts, since typical ionic radii
may be 2-3 times larger or smaller then atomic ones. The application
of Van der Waals radii usually results in a huge number of 
connectivities of which most ones are neither chemically relevant nor 
helpful for the optimization. In fact an overly large number
of internal coordinates can substantially decrease the efficiency of
an internal coordinate optimizer. On the other hand, missing internal 
coordinates can lead to inefficient or non-convergent optimizations.

Besides the case of ionic systems or random clusters of atoms
also fragmented sytems can cause problems. Consider, for example
a system of isolated molecules that interact only by very long range
forces but the intermolecular interaction cannot be described
in terms of well defined bonds such as in covalently bound systems.
E.g. two water molecules at a separation of 10{\AA} at a random 
orientation, where the dominant interaction is the dipole-dipole one.
In principle the interaction of these isolated units could be treated
in terms or the Cartesian coordinates of their centers of masses
and the appropriate rotations around them, however use of 
relative (internal) coordinates could provide a substantial advance
in the description of the relative motions of these units.

One of the disadvantages of internal coordinates over Cartesian ones
is that while Cartesian coordinates are always readily and
obviously at hand,
internal ones are often somewhat arbitrarily defined and thus
can provide a definitions based performance difference. Unfortunately,
a general algorithm that would overcome this inconvenience in the use
of internal coordinates has not yet been developed. This is probably
the reason why internal coordinates based geometry manipulation
is still often cosidered an art rather then science, especially
in the physics community where the use of chemical concepts in 
computation is less accepted. Indeed, the problem is that in many of the
problematic systems the boundaries of these chemical concepts are 
reached while the universality of Cartesian coordinates is always 
clearly undoubted. Thus, at one hand internal coordinates offer
great advantage in geometry manipulations on the other hand their
use seems to be much less deterministic then the one of Cartesians.
In the present paper we attempt to bridge the coordinate recognition
gap between the attractivity of internal coordinates and the 
troblesomeness of their definition.



\section{Conclusions} \label{Conclusions}

\bibliography{../../Bib/mondo_new}
\end{document}
