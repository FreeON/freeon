% NOTE TO AIP TYPSETERS: TO CONVERT FROM TWO-COL TO PREPRINT, SWITCH
% COMMENTOUT COMMAND FROM A TO B IE. USE:
%
%%\documentclass[prl,twocolumn,showpacs,twocolumngrid,superbib]{revtex4}
%\documentclass[prl,twocolumn,twocolumngrid,superbib]{revtex4}
%\newcommand{\commentoutB}[1]{}
%\newcommand{\commentoutA}[1]{#1}
%
% INSTEAD OF THE FOLLOWING:
%
\newcommand{\commentoutB}[1]{#1}
\newcommand{\commentoutA}[1]{}
%\documentclass[prl,aps,preprint,showpacs,superbib]{revtex4}
\documentclass[prl,aps,preprint,superbib,12pt]{revtex4}

\usepackage{graphicx}
\usepackage{amsfonts}
\usepackage{amsmath}
\usepackage{bm}
\usepackage{alltt}
\usepackage{fancyhdr}
\newcommand{\bms}[1]{{\boldsymbol #1}}
\renewcommand{\thefootnote}{\fnsymbol{footnote}}

%\draft
%\tighten
\pagestyle{fancy}

\begin{document}

\title{Geometry Optimization of Crystals using the Quasi-\\
       Independent Curvilinear Coordinate Approximation\footnotemark[1]}

\author{K\'aroly N\'emeth\footnotemark[1]}
\author{Matt Challacombe}
\author{Christopher J. Tymczak}
\author{Valery Weber}

\affiliation{Theoretical Division,\\ Los Alamos National Laboratory, \\ Los Alamos, NM 87545, USA}

\date{\today}

\begin{abstract}
This paper presents a performance-assessment 
of the recently developed quasi-independent curvilinear coordinate
approximation (QUICCA) method [K. N\'emeth and M. Challacombe,
J. Chem. Phys. {\bf 121}, 2877, (2004)] for the optimization of crystal 
structures. 
We demonstrate that the concept of quasi-independent
curvilinear coordinates is valid also under periodic boundary 
conditions,
and it leads to highly efficient crystal structure optimizations
as illustrated by a series of test calculations. 
The asessment of QUICCA for the optimization of crystal structures
prooves that QUICCA is a generally applicable algorithm
that is highly efficient for the optimization of both isolated 
molecules and crystals while it preserves an unparalleled robustness 
and simplicity of implementation.
\end{abstract}

%\pacs{31.15.-p,31.15.Ne,02.60.Pn, 45.10.Db, 02.40.Hw} 

\maketitle

\footnotetext[1]{\tt KNemeth@LANL.Gov}

\section{Introduction}
Internal coordinates are now routinely used in the optimization of 
molecular structures by most standard quantum chemistry program 
packages. A wealth of experience in geometry optimization justifies
the application of internal coordinates and shows their
superiority as compared to Cartesians. The reason behind the
experience is that internal coordinates exhibit much less
vibrational coupling than Cartesian ones do 
\cite{PPulay69,GFogarasi79,GFogarasi92,PPulay77}.
By recent proposals of Kudin {\it et.al.} \cite{KKudin01} and of
Andzelm {\it et.al.} \cite{JAndzelm01} the foundations of 
internal coordinate
crystal structure optimizations are laid down. These outhors
pointed out how to build Wilson's B matrix \cite{EWilson55} 
when periodic boundary conditions are present. They have also
presented a few examples for succesful application of internal 
coordinates for the optimization of crystal structures. It is 
important to note that the paper by Andzelm {\it et.al.}, in our best
understanding, does not contain the definitions of 
internal coordinate derivatives with respect to lattice parameters,
neither does describe simultaneous internal coordinate relaxation
of both atomic positions and lattice parameters.
The paper by Kudin {\it et.al.} accounts for both of these
features. We find that the construction of Wilson's B matrix
is described in both papers in a less compact, more case-specific 
way, by considering each internal coordinate event by separate 
discussions. This may be instructive but less suitable for
a reader who is interested in quick implementation.
Therefore, we repeat the formulation with the aim of mathematical
compactness that may be better suited for implementation.
Our selection of symmetry unique internal coordinates and treatment
of constraints does also differ from the one proposed by the previous
authors.
More important than these interpretative differences,
our main goal was to
assess the performance of our recently developed robust internal
coordinate optimizer, that is based on the concept of
quasi-independent curvilinear coordinates (QUICCA) \cite{KNemeth04}.
Similar to Baker's test set \cite{JBaker93}, we have set up 
a test set of ten crystal structures that we believe is representative
enough to a very broad range of possible applications. We carry out
geometry optimizations on this test set with the same accuracy
that is used in Baker's test set for isolated molecules.
Test calculations are carried out with both constrained and
relaxed lattice parameters. The results indicate that
QUICCA is generally applicable to both isolated molecule and crystal
structure optimizations, with high efficiency and great robustness,
while preserving an unparalleled simplicity in the implementation.

\subsection{Lattice summation for Wilson's B matrix}
Let us consider a three-dimensional crystal with atomic 
Cartesian positions $X$ that extend the range $[-\infty,\infty]$
in all spacial directions. The periodicity of the crystal
is given by the lattice vectors $A$, $B$ and $C$.
For this infinitely large crystal, local internal coordinates can be
defined, as usual for isolated molecules. Let us now suppose,
that there is a periodicity also in the set of internal coordinates.
This is simply a consequence of applying the same rules for the 
construction of internal coordinates everywhere in the crystal.
If this periodicity exists, a unique set of primitive internal 
coordinates, $\phi^{(000)}$ can be defined. Each $i$-th element of this
set, $\phi^{(000)}_{i}$ depends on the Cartesian coordinates of
maximum four atoms: 
$\phi^{(000)}_{i}=\phi^{(000)}_{i}(X(A_{1}),X(A_{2}),X(A_{3}),X(A_{4}))$,
where $A_{j}$ denotes the serial number of atoms in the infinite
crystal, $j$ refers to a serial number in the atom-list of 
$\phi^{(000)}_{i}$, and $X(A_{j})$ denotes the Cartesian coordinates
of atom $A_{j}$. 
All internal coordinates of the crystal can be retrieved
by applying the same $T_{klm}=kA+lB+mC$ translation vector
to the elements of $\phi^{(000)}$: 
$\phi^{(klm)}_{i}=\phi^{(000)}_{i}(X(A_{1})+T_{klm},X(A_{2})+
T_{klm},X(A_{3})+T_{klm},X(A_{4})+T_{klm})$,
where $k$, $l$ and $m$ denote arbitrary integers.



\bibliography{../../Bib/mondo_new}
\end{document}

