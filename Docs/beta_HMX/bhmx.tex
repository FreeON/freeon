% uncomment the following to get a preprint
\newcommand{\commentoutA}[1]{} \newcommand{\commentoutB}[1]{#1}
% uncomment the following to get a double column version
%\newcommand{\commentoutA}[1]{#1} \newcommand{\commentoutB}[1]{}

\newcommand{\Bulk}{B_0}
\newcommand{\BPrime}{{B'_0}}
\renewcommand{\thefootnote}{\fnsymbol{footnote}}

\commentoutA{\documentclass[prl,aps,twocolumn,twocolumngrid,superbib]{revtex4}}
\commentoutB{\documentclass[11pt,prb,aps,nobibnotes,superbib,preprint]{revtex4}}
\commentoutB{\renewcommand{\baselinestretch}{1.7}}

\usepackage{graphicx}
\usepackage{amsfonts}
\usepackage{amsmath}
\usepackage{bm}
\usepackage{alltt}
\usepackage{dcolumn}
\usepackage{letterspace}
\usepackage{color}

\makeatletter
\makeatother

\begin{document}
\title[Short Title]{
All-electron density-functional studies of 
hydrostatic compression of 
$\beta$-HMX (octahydro-1,3,5,7-tetranitro-1,3,5,7-tetrazocine)\footnotemark[1]}

\author{Chee Kwan Gan\footnotemark[2]}
\author{Thomas D. Sewell\footnotemark[3]}
\author{Matt Challacombe\footnotemark[4]}

\affiliation{ Theoretical Division,\\ Los Alamos
              National Laboratory,\\ Los Alamos, New Mexico 87545}

\date{Aug 19, 2004}

\begin{abstract}
All-electron density-functional calculations of the $\beta$ phase of
HMX (octahydro-1,3,5,7-tetranitro-1,3,5,7-tetrazocine) under
hydrostatic compression have been performed with the PBE functional in
conjunction with the 6-31G** Gaussian basis set.  Full relaxation of
the monoclinic lattice parameters (i.e.,  $a$, $b$, $c$, and $\beta$)
and atomic positions have been carried out to determine the
equilibrium volume $V_0$ and the pressure-volume curve at 0~K.  The
calculated volume dependence of the pressure, linear compressibility
$a/a_0$, and $\beta/\beta_0$ (where $a_0$ and $\beta_0$ are values at
equilibrium) are found to be in good agreement with experiment.  It is
found that the calculated compressibilities of $a/a_0$, $b/b_0$, and
$c/c_0$ are essentially the same.  However, the experimental results
showed that the crystal is more compressible in the $b$ direction than
in the $c$ direction.  Since the errors due to the basis set and
Brillouin zone sampling have negligible effects on the calculated
results, our results suggest that PBE functional may not be able to
describe the dispersion interactions in $\beta$-HMX crystal
accurately. Three fitting forms have been used to deduce the bulk
modulus $\Bulk$ and $\BPrime$ of $\beta$-HMX. These results are
compared to experiment and to the results of previous molecular
simulations.

\smallskip
\noindent{\bf Keywords}:
Density-functional theory, linear scaling,
Gaussian-orbital, HMX, hydrostatic compression, explosive materials

\noindent{\bf PACS numbers}: 71.15Nc, 31.15Ew
\end{abstract}
\maketitle

\footnotetext[1]{Preprint LA-UR-}
\footnotetext[2]{ckgan@lanl.gov}
\footnotetext[3]{sewell@lanl.gov}
\footnotetext[4]{mchalla@lanl.gov}


\section{Introduction}
\label{sec:intro}
Octahydro-1,3,5,7-tetranitro-1,3,5,7-tetrazocine (HMX) is one of the
most important energetic materials. Under ambient conditions, it
exists in three pure crystalline forms, designated as $\alpha$,
$\beta$, and $\delta$ phases of HMX. The relative stability of these
forms is known to be, in descending order, $\beta$, $\alpha$, and
$\delta$.  The most stable $\beta$ phase belongs to the monoclinic
$P2_1/c$ (or the equivalent $P2_1/n$) space group, with 2 molecules
per unit cell\cite{CChoi72}.  The linear and volumetric hydrostatic
compression of $\beta$-HMX has been measured by Olinger {\it et
al.}\cite{BOlinger78} up to 7.47~GPa, and by Yoo and
Cynn\cite{Yoo98,Yoo_1999v111} up to 42.60~GPa. These two sets of data
agree with one another rather well, especially at the low pressure
regime.

A number of theoretical studies of $\beta$-HMX have been carried out
using the model potentials and rigid-molecule
approximation\cite{DSorescu98,TSewell98,DSorescu99,TSewell03}.

One of the first periodic first-principles calculations on $\beta$-HMX
(as well as other forms of HMX) has been performed by Lewis {\it et
al.}\cite{JPLewis00}. They have used a local density approximation
(LDA) in conjunction with localized ``fireball'' orbitals and a
minimal basis set. They have adopted two computational approaches,
both of which constrained the monoclinic lattice angle $\beta$ to the
experimental value. The first approach is called a ``uniform
dilation'' where the lattice lengths are increased or decreased by a
uniform percentage of the experimentally measured values. The second
approach is called an ``independent dilation' where $a$, $b$, and $c$
are independently varied. Throughout this work, we adopt and extend
the ``independent dilation'' approach by relaxing the constraint on
the lattice angle $\beta$ where $a$, $b$, $c$, and $\beta$ can be
independently varied.  This relaxation entails very extensive
calculations. However, we expect that a fully variational,
all-electron (which avoids the breakdown of pseudopotential
approximation at high compressions) approach should deliver more
accurate results than that of a constrained lattice angle approach. We
have used a PBE functional\cite{Perdew_96v77}, which is a
gradient-corrected functional that is superior than the LDA
functional.  We have also gone beyond the minimal basis set level by
using a significantly large 6-31G** basis set for the entire
calculations. This functional and basis set are chosen because they
have been shown to be accurate for the equation-of-state studies of
pentaerythritol tetranitrate (PETN)\cite{CGan04A}.

\section{Computational Framework}
We have used a parallelized version of MONDOSCF\cite{MondoSCF} to
perform all-electron calculations in this work.  MONDOSCF is a suite
of programs for linear scaling electronic-structure theory and {\it ab
initio} molecular dynamics. The code uses state-of-the-art
linear-scaling algorithms to calculate the
Coulomb\cite{MChallacombe96B,MChallacombe97} ,
exchange-correlation\cite{MChallacombe00A}, and exact Hartree--Fock
exchange\cite{ESchwegler96,ESchwegler97,ESchwegler98A} matrices.

Throughout this work we have used the gradient-corrected PBE density
functional\cite{Perdew_96v77} and the 6-31G** basis set.  Periodicity
of the system under study is assumed and the $\Gamma$ point is used
for the Brillouin zone sampling. The $\Gamma$ point effect may be
reduced by using a larger simulation cell where necessary.

The parallelized MONDOSCF benefited from the recent development of
parallel Hamiltonian builds, where an equal-time partitioning scheme
had been used to efficiently load balance computations of the
exchange-correlation\cite{CGan03} and Coulomb\cite{CGan04B} matrices.
We used a 2.4 GHz/Myrinet P4 cluster to perform the calculations;
typically 32 processors were used.




\section{Computational Protocol and results}
Starting with the experimental crystal structure of $\beta$-HMX
reported by Kohno\cite{Kohno92}, we calculate the zero pressure unit
cell volume $V_0$ under the assumption of a monoclinic crystal system.
Since the lattice forces were not fully available when we started the
calculations, we have used successive line
minimizations\cite{WPress92} to optimize the monoclinic cell
parameters, which are $a$, $b$, $c$, and $\beta$. For the
determination of $V_0$, we have performed successive line
minimizations in the $a$, $b$, $c$, and $\beta$ variables space.
Unless otherwise stated, we have used a $P2_{1}/n$ setting in the
minimization process.  To perform a line minimization in the $c$
variable space, say, we typically generate five structures with
varying $c$ values but fixed values of $a$, $b$, and $\beta$.
Geometry optimizations were carried out for each of these structures.
A cubic fit is fitted to the $(c, E)$ data, where $E$ is the minimized
energy of the structure with a lattice parameter $c$.  Typically we
need to cycle through the variable set $(a,b,c,\beta)$ a few times to
obtain $V_0$ accurately.  Calculated and measured lattice parameters
are listed in Table~\ref{tab:lattice}.  The relative errors of our
$2\times 1 \times 2$ results compared to the experimental results of
Olinger {\it et al.}\cite{BOlinger78} are 1.2\%, 1.4\%, 0.9\%, 0.1\%,
and 3.4\% for $a_0$, $b_0$, $c_0$, $\beta_0$, and $V_0$, respectively.
These errors are slightly larger than that between theory and
experiment for PETN\cite{CGan04A}. It is interesting to note that the
$V_0$ determined by Lewis {\it et al.}\cite{JPLewis00} is smaller than
the experimental values, while our results (for both $1\times 1 \times
1 $ and $2 \times 1 \times 2$ systems) are larger than the
experimental values.

\begin{table}[p]
\caption{Equilibrium lattice parameters (in $P2_{1}/c$ setting) 
and unit cell volume for $\beta$-HMX.}
\begin{center}
\begin{tabular}{lllllll}
\hline\hline
$a$(\AA) & $b$(\AA) & $c$(\AA) & $\beta$($^{\circ}$) & $V$(\AA$^3$) & Source & Comment\\ \hline
6.619 & 11.21  &  8.777 & 124.2 & 538.9 & This work & $1\times 1 \times 1$ system, PBE/6-31G**\\
6.621 & 11.18  & 8.788 & 124.5 & 536.0 & This work & $2 \times 1 \times 2$ system, PBE/6-31G**\\
6.54  & 11.03 &  8.71  & 124.4 & 518.4 & Olinger {\it et al.}\cite{BOlinger78} & X-ray diffraction \\
6.540 & 11.050 & 8.700 & 124.30 & 519.4 & Yoo and Cynn\cite{Yoo_1999v111} & X-ray diffraction  \\
%6.674 & 11.17  & 8.95 & 124.5 & 549.3 & Sewell\cite{TSewell98} & Monte Carlo simulations \\
6.531 & 10.521  & 9.252& 125.39 & 518.3 & Sewell\cite{TSewell98} & Monte Carlo simulations \\

% 6.5347 & 11.0296 & 7.3549 & 102.69 & 517.16 & Kohno {\it et al.}\cite{Kohno92} \\
6.5347 & 11.0296 & 8.6995 & 124.43 & 517.16 & Kohno {\it et al.}\cite{Kohno92} \\
6.34 & 10.50 & 8.61 & 124.3 &  473.8 & Lewis {\it et al.}\cite{JPLewis00} & Pseudopotential LDA \\
\hline\hline
\end{tabular}
\end{center}
\label{tab:lattice}
\end{table}

For the compression studies, we arbitrarily treat $a$, $c$ and $\beta$
as independent variables for the line minimizations, while $b$ is a
dependent variable through the relation $V = abc \sin\beta$, where $V$
is the volume of the monoclinic $\beta$-HMX.  Initial guesses were
obtained by rigid translation of the molecular structure obtained from
the previous larger volume ratio via a transformation that preserves
the center-of-mass positions in the crystallographic coordinates.  The
starting guess for the first compressed structure is obtained by $a =
ra_0$, $b = r b_0$, $c = r c_0$, where $r = (V/V_0)^{1/3}$.  At the
$(k+1)$th compression, we adopt a ratio method to predict the starting
structure, where $x/x_{k} = x_{k}/x_{k-1}$ for $x = a, b, c$; $x_{k}$
is the optimized value of $x$ at the $k$th compression.  To
independently verify the calculated results, we have repeated some of
the calculations using the experimental values of Olinger {\it et
al.}\cite{BOlinger78} as a guide for the starting guesses.  We have
also changed the computational protocol where a $P2_{1}/c$ setting is
used.  In this new protocol, $a$, $b$, $c$ are treated as independent
variables and $\beta$ a dependent variable during the
constrained-volume line minimization process.  However, no discernible
differences are found between the experimentally guided approach and
the ratio method, an indication that both approaches give the same
minima.
\begin{figure}
\resizebox*{3.5in}{!}{\includegraphics[clip]{VRatio_MONDOEdiff.eps}}
\caption{
Energy difference $\Delta E$ as a function of the volume ratio. The
dot line is a sixth degree polynomial fit to the $2\times 1 \times 2$
results.
}
\label{fig:VRatio_Ediff}
\end{figure}

\begin{figure}
\resizebox*{3.5in}{!}{\includegraphics[clip]{VRatioPressure-MondoYooOlinger.eps}}
\caption{
Pressure-volume results obtained from the MONDOSCF calculations and
experiments.  The Yoo and Cynn's results are from
Ref.~\cite{Yoo_1999v111} while the Olinger {\it et al.}'s
results are from Ref.~\cite{BOlinger78}.
}
\label{fig:pressure}
\end{figure}

\begin{figure}
\resizebox*{3.5in}{!}{\includegraphics[clip]{BWvratio_fracABCbeta_MONDOuniformcompr_ORC.eps}}
\caption{
Linear compression of $\beta$-HMX. Relative linear compression of the
lattice parameters. Experimental results are from Olinger {\it et
al.}\cite{BOlinger78}. A $P2_1/c$ setting is used.
}
\label{fig:lin}
\end{figure}

In Fig.~\ref{fig:VRatio_Ediff} we show the volume dependence of the
total energy per unit cell of $\beta$-HMX at 0~K. It is found that the
finite size effect is minimal since the results of the $1\times 1
\times 1$ system agree very well with that of the
$2\times 1 \times 2 $ system. However, for the $1\times 1
\times 1 $ system, line minimizations do not produce well-defined minima
for volume ratio below 0.84.  This may due to the fact that the
$\Gamma$-point effect is more significant at higher compressions that
introduces errors in the energy differences.  When a larger $2\times 1
\times 2$ system is used, the minima become well-defined again. For
this reason, we have used a $2\times 1 \times 2 $ system to study the
hydrostatic compression of the crystal.  The volumetric hydrostatic
compression is shown in Fig.~\ref{fig:pressure}.  The agreement
between our results and that of Yoo and Cynn\cite{Yoo_1999v111} is
remarkable.

The result for the linear compressibilities of the monoclinic
$\beta$-HMX crystal is shown in Fig.~\ref{fig:lin}. For comparison, we
have used Olinger {\it et al.}'s results\cite{BOlinger78}, which agree
rather well with the results of Yoo and Cynn\cite{Yoo_1999v111}.  It
is found that $a/a_0$ and $\beta/\beta_0$ agree reasonably well with
experiment.  However, while the experiment results indicate that the
crystal is more compressible in the $b$ direction than in the $c$
direction, our results show that the compression of $\beta$-HMX is
essentially isotropic. This discrepancy may due to the fact that the
PBE functional is unable to describe the dispersion interactions in
$\beta$-HMX accurately\cite{WKohn98v80}. It should be noted, however,
that the PBE functional was found to give very accurate results for
the linear compressibilities of PETN, which has a higher tetragonal
symmetry compared to that of the monoclinic symmetry of $\beta$-HMX.
It would be interesting to use other density functionals, or even
hybrid Hartree--Fock/DFT\cite{ABecke93,CAdamo99B,XXu04}, that may
offer a better result for linear compressibilities.

%\begin{figure}
%\resizebox*{3.5in}{!}{\includegraphics[clip]{arrow_expand.eps}}
%\caption{
%The progression of the minima (denoted by the directions of
%arrows) during line minimization process is shown for $b$ and $c$ 
%specifically.
%The numbers of cycles through the $(a,b,c)$ variables space
%at $V/V_0 = 0.8788$, $0.9224$, and $0.9578$ are 5, 2, and 2, respectively.
%We have used a $P2_1/c$ setting and $a$, $b$, and $c$ are independent
%variables while $\beta$ is a dependent variable.
%}
%\label{fig:guided}
%\end{figure}

\begin{figure}
\resizebox*{3.5in}{!}{\includegraphics[clip]{split.eps}}
\caption{
The progression of the optimized value of $a$, $b$, $c$, and $\beta$
during the line minimization process for the 
experimentally guided approach, for a representative case of $V/V_0 = 0.8788$. The initial values (or guesses) 
at the $0$-th cycle are
obtained from the fitting of the experimental results of Olinger {\it et al.}\cite{BOlinger78}.
A $P2_1/c$ setting is used. The independent variables for the line minimizations are $a$, $b$, and $c$,
while $\beta$ is a dependent variable.
}
\label{fig:guided}
\end{figure}

As a side note, we have independently checked the results obtained
using the ratio method for the starting guesses by repeating some
calculations where the experimental results of Olinger {\it et al.}
have been used to provide the initial guesses.  The result of a
representative case of $V/V_0 = 0.8788$ of this `experimentally
guided'' approach is shown in Fig.~\ref{fig:guided}.  It should be
noted that the solution at the end of the first cycle is still far
from its converged solution. This highlights the importance of going
through as many cycles as necessary to fully converge the solution.

We have used three fitting forms to obtain the bulk modulus $\Bulk$
and its initial pressure derivative $\BPrime$. These three forms have
been used to analyze the $PV$ data of PETN\cite{CGan04A}.  The first
form is the Murnighan equation~\cite{Murnaghan_1951}
\begin{equation}
P=\frac{\Bulk}{\BPrime}\left(\eta^{-\BPrime}-1\right)
\label{eq:Murna}
\end{equation}
where $\eta=V/V_0$.  The second form is the third-order
Birch-Murnaghan equation~\cite{Poirier_1991} used previously in the
analysis of the isotherm for the high explosive HMX~\cite{Yoo_1999v111,Menikoff_2001v21,TSewell03}
\begin{equation}
P=\frac{3}{2}\Bulk(\eta^{-7/3}-\eta^{-5/3})
     \left[1+\frac{3}{4}(\BPrime -4)(\eta^{-2/3}-1)\right].
\label{eq:BM}
\end{equation}
The third fitting form
\begin{equation}
P=\frac{(V_0-V)c^2}{[V_0-s(V_0-V)]^2},
\label{eq:cs}
\end{equation}
is derived from the shock Hugoniot conservation equations,
\begin{equation}
U_s=\sqrt{PV_0/(1-V/V_0)},
\label{eq:hug1}
\end{equation}
and
\begin{equation}
U_p=\sqrt{PV_0(1-V/V_0)},
\label{eq:hug2}
\end{equation}
where $U_s$ and $U_p$ are pseudo-shock and pseudo-particle velocities,
respectively.  These forms were used by Olinger {\it et al.} in the
analysis of their isotherm data for
PETN~\cite{Olinger_1975v62,Olinger_1976} and other explosives
including TATB,~\cite{Olinger_1976} $\beta$-HMX and
RDX,~\cite{BOlinger78} and nitromethane.~\cite{Yarger_1986v85}
Equation (\ref{eq:cs}) is a straightforward recasting of
Eqs. (\ref{eq:hug1})--(\ref{eq:hug2}) into the $PV$ plane, assuming
that $U_s=c+sU_p$. For this equation of state, the initial bulk
modulus and its pressure derivative are given by $\Bulk=c^2/V_0$ and
$\BPrime=4s-1$, respectively.

The results of bulk modulus and its initial pressure derivative for
$\beta$-HMX are tabulated in Table~\ref{tab:Bulk}. {\bf Blah blah blah...}

\begin{table}[p]
\caption{Calculated and measured bulk moduli $\Bulk$ and pressure
derivative $\BPrime$ for $\beta$-HMX.}
\begin{center}
\begin{tabular}{lllll}
\hline\hline
$\Bulk$(GPa) &   $\BPrime$   &   Data source
&    Fitting form &      Comment \\
\hline
15.9 & 6.2 & This work, $2\times 1 \times 2$ system &  Eq.~(\ref{eq:Murna}) & Fit to PBE/6-31G** results for $P \le 13.72$~GPa \\
14.1 & 8.6 & This work, $2\times 1 \times 2$ system & Eq.~(\ref{eq:BM}) & Fit to PBE/6-31G** results for $P \le 13.72$~GPa \\
16.3 & 5.9 & This work, $2\times 1 \times 2$ system & Eq.~(\ref{eq:cs}) & Fit to PBE/6-31G** results for $P \le 13.72$~GPa \\
12.0 & 10.5 & Olinger {\it et al.}~\cite{BOlinger78} & Eq.~(\ref{eq:Murna}) & Fit to experimental isotherm for $P \le 7.47$~GPa\\
7.6& 28.0 & Olinger {\it et al.}~\cite{BOlinger78} & Eq.~(\ref{eq:BM}) & Fit to experimental isotherm for $P \le 7.47$~GPa\\
12.9 & 9.4 & Olinger {\it et al.}~\cite{BOlinger78} & Eq.~(\ref{eq:cs}) & Fit to experimental isotherm for $P \le 7.47$~GPa\\
16.8 & 6.0 & Yoo and Cynn {\it et al.}~\cite{Yoo_1999v111} & Eq.~(\ref{eq:Murna}) & Fit to experimental isotherm for $P \le 14.10$~GPa\\
15.0& 8.3 & Yoo and Cynn {\it et al.}~\cite{Yoo_1999v111} & Eq.~(\ref{eq:BM}) & Fit to experimental isotherm for $P \le 14.10$~GPa \\
17.1 & 5.8 & Yoo and Cynn {\it et al.}~\cite{Yoo_1999v111} & Eq.~(\ref{eq:cs}) & Fit to experimental isotherm for $P \le 14.10$~GPa \\
\hline\hline
\end{tabular}
\end{center}
\label{tab:Bulk}
\end{table}

\section{Conclusions}
All-electron density-functional calculations for the equation-of-state
studies of $\beta$-HMX at 0~K have been performed. To allow successful
determination of minima during line minimizations, we have used a
$2\times 1 \times 2$ system, which contains 224 atoms in the
simulation cell. To the best of our knowledge, this is one of the
largest all-electron calculations performed on $\beta$-HMX to date.
We have successfully obtained $PV$ data up to 13.72~GPa.  The volume
dependence of pressure is found to agree very well with the
experimental results of Yoo and Cynn\cite{Yoo_1999v111}. However,
while our results with PBE/6-31G** suggest that $\beta$-HMX is
isotropic under hydrostatic compression, the experimental results
showed that $b$ direction is more compressible than in the $c$
direction.  This discrepancy may be due to the use of PBE functional,
which is known to be incapable of describing the dispersion
interactions accurately\cite{WKohn98v80}.  It remains to be seen if
other functionals, especially those hybrid HF/DFT could deliver a
better description of linear compressibilities.  {\bf Conclusions on
bulk modulus and initial pressure derivatives.. blah blah blah..}

\begin{acknowledgments}
This work has been carried out under the auspices of the
U.S. Department of Energy under Contract No.~W-7405-ENG-36 and the
ASCI project.  Most work was performed on the computing resources at
the Advanced Computing Laboratory of Los Alamos National Laboratory.
\end{acknowledgments}

\bibliographystyle{apsrmp} \bibliography{mondo_new}

\end{document}
