%%%%%%%%%%%%%%%%%%%%%%%%%%%%%%%%%%%%%%%%%%%%%%%%%%%%%%%%%%%%%%%%%%%%%%%%%%%%%%%%%%%%%%%%%%%%%%%%%%%%
%
% NOTE TO AIP TYPSETERS: TO CONVERT FROM TWO-COL TO PREPRINT, SWITCH
% COMMENTOUT COMMAND FROM A TO B IE. use
% \newcommand{\commentoutA}[1]{}
% \newcommand{\commentoutB}[1]{#1}
% instead of the following
\newcommand{\commentoutA}[1]{#1}
\newcommand{\commentoutB}[1]{}
\renewcommand{\thefootnote}{\fnsymbol{footnote}}
\commentoutA{\documentclass[superbib,prb,epsfig,floats,twocolumn]{revtex4}}
\commentoutB{\documentclass[aps,nobibnotes,superbib,byrevtex,preprint]{revtex4}}
\usepackage{graphics}
\usepackage{amsfonts}
\usepackage{bm}
\usepackage{alltt}
\usepackage{epsfig}
\begin{document}
%%
%%


%\documentstyle[preprint,aps]{revtex4}
%\documentstyle[prl,aps]{revtex4}

%\draft
%\tighten
%\twocolumn

%\begin{document}

\date{\today}

\title{Data structures and error estimates for O(N) quantum chemistry}

\author{Matt Challacombe and  C.~J.~Tymczak}

\address{
Theoretical Division, Los Alamos National Laboratory,
Los Alamos, NM 87545, USA}


\begin{abstract}
gnu snort fubar

\keywords{linear scaling, electronic structure, data structures, error estimates}
\end{abstract}

\maketitle

\section{Introduction}

\section{Error estimates}

\begin{equation}
\rho_q = \sum_{lmn} d_{lmn} \Lambda^q_{lmn}
\end{equation}

\begin{equation}
\Lambda^q_{lmn}({\bf r})=\lambda^q_l(x)\lambda^q_m(y) \lambda^q_l(z) e^{-\zeta_q ({\bf r -Q})^2}
\end{equation}

\begin{equation}
\lambda^q_l(x)= \zeta^{l/2}_q H_l \left[ \sqrt{\zeta_q} \left( x-Q_x \right)\right]
\end{equation}

\begin{equation}
\lambda^q_l(x) \leq K \sqrt{ 2^l ~ l!~ \zeta^l_q  } ~ e^{\frac{\zeta_q}{2} (x-Q_x)^2 } 
\end{equation}

\begin{equation}
\rho_q \leq C_q e^{ -\frac{\zeta_q}{2} ({\bf r -Q})^2 }  
\end{equation}

\begin{equation}
C_q=\sum_{lmn} \left|d_{lmn}\right| K^3 \left[ 2^{l+m+n} ~ l! m! n! ~ \zeta^{l+m+n}_q \right]^{1/2} 
\end{equation}

\begin{equation}
C_q=\max_{lmn} \left|d_{lmn}\right| K^3 \left[ 2^{l+m+n} ~ l! m! n! ~ \zeta^{l+m+n}_q \right]^{1/2} 
\end{equation}


\subsection{Gaussian extent}

\begin{equation}
C_q e^{-\zeta_q ' R^2 } < \tau 
\end{equation}

\subsection{Potential extent}

\begin{equation}
\left(\frac{\pi}{\zeta_q'} \right)^{3/2} \frac{C_q}{R} {\rm erfc}\left({\sqrt{\zeta_q '} R }\right) \ll \tau
\end{equation}



\section{Acknowledgement}


\bibliographystyle{apsrmp}
\bibliography{mondo}

\begin{references}

\end{references}

\end{document}
