%
\documentclass[prb,aps]{revtex4}

\usepackage{graphicx}
\usepackage{amsfonts}
\usepackage{amsmath}
\usepackage{bm}
\usepackage{alltt}
\usepackage{dcolumn} 
\usepackage{amsmath} 
\usepackage{graphicx}
\makeatletter 
\makeatother

\begin{document}

\title{Data structures and error estimates for O(N) quantum chemistry}

\author{C. J. Tymczak}
\author{Matt Challacombe}

\affiliation{Theoretical Division, Los Alamos National Laboratory, Los Alamos,NM 87545, USA}

\date{\today}

\begin{abstract}
Periodic boundary conditions have been implemented in the linear scaling
Quantum Chemistry code \textbf{MondoSCF}.
\end{abstract}

\maketitle

\section{The Penetration Acceptability Criterion (PAC)}

Let us consider a generalized distribution centered at \( \mathbf{P} \)

\begin{equation}
\label{A1}
\rho _{\mathbf{P}}\left( \mathbf{r}\right) =\sum _{l+m+n=L}d_{lmn}\Lambda ^{\mathbf{P}}_{lmn}
\left( \mathbf{r}\right) 
\end{equation}
where

\begin{equation}
\label{A2}
\Lambda ^{\mathbf{P}}_{lmn}\left( \mathbf{r}\right) =\left( x-P_{x}\right) ^{l}\left( y-P_{y}
\right) ^{m}\left( z-P_{z}\right) ^{n}\exp \left( -\xi _{\mathbf{P}}\left| \mathbf{x}-\mathbf{P}\right| ^{2}
\right) 
\end{equation}
Using Cramer's Inequality we can show
\begin{equation}
\label{A3}
\rho _{\mathbf{P}}\left( \mathbf{r}\right) =\sum _{lmn}d_{lmn}\Lambda ^{\mathbf{P}}_{lmn}
\left( \mathbf{r}\right) 
\leq C_{\mathbf{P}}\exp \left( -\widetilde{\xi }_{\mathbf{P}}\left| \mathbf{x}-\mathbf{P}\right| ^{2}\right) 
\end{equation}
where

\begin{equation}
\label{A4}
C_{\mathbf{P}}=\sum _{lmn}\left| d_{lmn}\right| \sqrt{l!\, m!\, n!\, \left( 2\, \widetilde{\xi }_{\mathbf{P}}
\right) 
^{l+m+n}}
\end{equation}
\begin{equation}
\label{A5}
\widetilde{\xi }_{\mathbf{P}}=\left\{ \begin{array}{c}
\xi _{\mathbf{P}}\\
\xi _{\mathbf{P}}/2
\end{array}\right. \begin{array}{c}
\quad L=0\\
\quad {\rm otherwise}
\end{array}
\end{equation}
This allow us to treat all distributions as if they where s-type Gaussian,
which allows us to easily calculate the penetration error as,
\[
\int _{V_{\infty }}\int _{V_{\infty }}\, d{\mathbf{r}}\, d{\mathbf{r}'}\frac{C_{\mathbf{P}}\exp 
\left( -\widetilde{\xi }_{\mathbf{P}}\left| \mathbf{x}-\mathbf{P}\right| ^{2}\right) C_{\mathbf{Q}}\exp 
\left( -\widetilde{\xi }_{\mathbf{Q}}\left| \mathbf{x}-\mathbf{Q}\right| ^{2}\right) }{\left| \mathbf{r}-
\mathbf{r}'\right| }\qquad \qquad \qquad \qquad 
\]


\begin{equation}
\label{A6}
\qquad \qquad \qquad =\left( \frac{\pi }{\widetilde{\xi }_{\mathbf{P}}}\right) ^{\frac{3}{2}}\left( \frac{\pi }
{\widetilde{\xi }_{\mathbf{Q}}}\right) ^{\frac{3}{2}}\frac{C_{\mathbf{P}}C_{\mathbf{Q}}}{\left| \mathbf{P}-
\mathbf{Q}\right| }{\rm erfc}\left( \sqrt{\frac{\widetilde{\xi }_{\mathbf{P}}\widetilde{\xi }_{\mathbf{Q}}}
{\widetilde{\xi }_{\mathbf{P}}+\widetilde{\xi }_{\mathbf{Q}}}}\left| \mathbf{P}-\mathbf{Q}\right| \right) \geq \tau 
\end{equation}
However, this is prohibitively expensive to evaluate for a tree code.
Inorder to eliminate this computational expense we instead evaluate
for each distribution the penetration error considering the other
distribution as a delta function of weight one. This leads to two
distances which are tabulated beforehand, \( \mathbf{P}_{max} \)
and \( \mathbf{Q}_{max} \). Then our test simply becomes

\begin{equation}
\label{A7}
\left| \mathbf{P}-\mathbf{Q}\right| \leq \left| \mathbf{P}_{max}\right| +\left| \mathbf{Q}_{max}\right| 
\end{equation}



\section{The Multipole Acceptability Criterion (MAC)}

Let us start by defining some useful quantities\begin{equation}
\label{B1}
\widehat{O}_{l}^{m}\left[ \mathbf{R}\right] =\frac{\left| \mathbf{R}\right| ^{l}P_{l}^{m}\left( \cos \theta _{\mathbf{R}}
\right) \, e^{-im\phi _{\mathbf{R}}}}{\left( l+m\right) !}\quad \; \; 
\end{equation}
\begin{equation}
\label{B2}
M_{l}^{m}\left[ \mathbf{R}\right] =\frac{\left( l-m\right) !\, P_{l}^{m}\left( \cos \theta _{\mathbf{R}}\right) \, 
e^{-im\phi _{\mathbf{R}}}}{\left| \mathbf{R}\right| ^{l+1}}
\end{equation}
\begin{equation}
\label{B3}
{\cal O}_{l}^{m}\left[ \rho ;\mathbf{P}\right] =\int _{V_{\infty }}\, d{\mathbf{r}}\, \widehat{O}_{l}^{m}\left[ 
{\mathbf{r}-\mathbf{P}}\right] \, \rho \left( \mathbf{r}\right) \qquad \; \; \; 
\end{equation}
Next, let us consider the expansion of the coulomb integral into the
multipole basis\[
\int \, d{\mathbf{r}}\, d{\mathbf{r}'}\frac{\rho _{1}\left( \mathbf{r}\right) \: \rho _{2}\left( \mathbf{r}'\right) }
{\left| \mathbf{r}-\mathbf{r}'\right| }\qquad \qquad \qquad \qquad \qquad \qquad \qquad \qquad \qquad \qquad \qquad 
\qquad \qquad \]
\begin{equation}
\label{B4}
\approx \frac{1}{2}\sum _{l=0}^{L}\, \sum _{l'=0}^{L'}\, \sum _{m=-l}^{l}\, \sum _{m'=-l'}^{l'}\, \left( -1\right) ^{l}
\, {\cal O}_{l}^{m}[\rho _{1};\mathbf{P}]M_{l+l'}^{m+m'}[\mathbf{P}-\mathbf{Q}]\, {\cal O}_{l'}^{m'}[\rho _{2};\mathbf{Q}]
\end{equation}
Let us now assume that the primitive distribution \( \rho _{1} \)
is of a single angular momentum type and expanded about its center, 

\begin{equation}
\label{B5}
{\cal O}_{l}^{m}[\rho _{1};\mathbf{P}]={\cal O}_{l}^{m}[\rho _{1};\mathbf{P}]\delta _{l\, l_{0}}
\end{equation}
this gives us\[
\int \, d{\mathbf{r}}\, d{\mathbf{r}'}\frac{\rho _{1}\left( \mathbf{r}\right) \: \rho _{2}\left( \mathbf{r}'\right) }
{\left| \mathbf{r}-\mathbf{r}'\right| }\qquad \qquad \qquad \qquad \qquad \qquad \qquad \qquad \qquad \qquad \qquad 
\qquad \qquad \]
\begin{equation}
\label{B6}
\approx \frac{1}{2}\sum _{l'=0}^{L'}\, \sum _{m=-l_{0}}^{l_{0}}\, \sum _{m'=-l'}^{l'}\, \left( -1\right) ^{l}\, 
{\cal O}_{l_{0}}^{m}[\rho _{1};\mathbf{P}]\, M_{l_{0}+l'}^{m+m'}[{\mathbf{P}-\mathbf{Q}}]\, {\cal O}_{l'}^{m'}[\rho _{2};
\mathbf{Q}]
\end{equation}
The error in the calculation is then the multipoles that we do not
sum\begin{equation}
\label{B7}
\varepsilon \left( l_{0}\right) =\left| \frac{1}{2}\sum _{l'=L'+1}^{\infty }\, \sum _{m=-l_{0}}^{l_{0}}\, 
\sum _{m'=-l'}^{l'}\, \left( -1\right) ^{l}\, {\cal O}_{l_{0}}^{m}[\rho _{1};\mathbf{P}]\, M_{l_{0}+l'}^{m+m'}
[{\mathbf{P}-\mathbf{Q}}]\, {\cal O}_{l'}^{m'}[\rho _{2};\mathbf{Q}]\right| 
\end{equation}
 Let us define for the unsold operator\[
{\cal \widehat{O}}_{l}\left[ \left| \mathbf{P}\right| \right] =\sqrt{\sum _{m}(l-m)!(l+m)!\left| \widehat{O}_{l}^{m}
\left[ \mathbf{P}\right] \right| ^{2}}\]
\begin{equation}
\label{B8}
\qquad \qquad \qquad \quad =\left| \mathbf{P}\right| ^{l}\sqrt{\sum _{m}\frac{(l-m)!}{(l+m)!}\left| P_{l}^{m}
(\mathbf{P})\right| ^{2}}=\left| \mathbf{P}\right| ^{l}\geq 0
\end{equation}
and similarly the unsold weights\begin{equation}
\label{B9}
{\cal O}_{l}[\rho ;\left| \mathbf{P}\right| ]=\sqrt{\sum _{m}(l-m)!(l+m)!\left| {\cal O}_{l}^{m}\left[ \rho ;\left| 
\mathbf{P}\right| \right] \right| ^{2}}
\end{equation}
Let us rewrite the error bound as\begin{equation}
\label{B10}
\varepsilon \left( l_{0}\right) \leq \frac{1}{2}\sum _{l'=L'+1}^{\infty }\left| \, \sum _{m=-l_{0}}^{l_{0}}\, 
\sum _{m'=-l'}^{l'}{\cal O}_{l_{0}}^{m}[\rho _{1};\mathbf{P}]\, M_{l_{0}+l'}^{m+m'}[{\mathbf{P}-\mathbf{Q}}]\, 
{\cal O}_{l'}^{m'}[\rho _{2};\mathbf{Q}]\right| 
\end{equation}
Let us also, for compactness of notation, define 
\begin{equation}
\label{B11}
{\cal O}_{l}^{m}\equiv {\cal O}_{l}^{m}[\rho _{1};\mathbf{P}]
\end{equation}
where the placement will indicate which distribution it is derived
from. Using equations (\ref{B2}) and (\ref{B3}) we can write this
as
\[
\varepsilon \left( l_{0}\right) \leq \frac{1}{2}\sum _{l'=L'+1}^{\infty }\, \left| \sum _{m=-l_{0}}^{l_{0}}\, 
\sum _{m'=-l'}^{l'}\frac{(l_{0}+l'-(m+m'))!}{\sqrt{(l_{0}+m)!(l_{0}-m)!}\: \sqrt{(l'+m')!(l'-m')!}}\right. 
\qquad \qquad \]
\begin{equation}
\label{B12}
\qquad \left. \frac{\sqrt{(l_{0}+m)!(l_{0}-m)!}{\cal O}_{l_{0}}^{m}\, P_{l_{0}+l'}^{m+m'}\left( \cos 
\theta _{\mathbf{PQ}}\right) \, \sqrt{(l'+m')!(l'-m')!}{\cal O}_{l'}^{m'}}{\left| \mathbf{P}-\mathbf{Q}
\right| ^{l'+l_{0}+1}}\right| 
\end{equation}
for reasons which will become apparent in what follows. Let us use
the inequalities (ref)\begin{equation}
\label{B13}
\left| P_{l}^{m}\left( \cos \theta _{\mathbf{R}}\right) \right| \leq 1
\end{equation}
to simplify\[
\varepsilon \left( l_{0}\right) \leq \frac{1}{2}\sum _{l'=L'+1}^{\infty }\, \left| \sum _{m=-l_{0}}^{l_{0}}\, 
\sum _{m'=-l'}^{l'}\frac{(l_{0}+l'-(m+m'))!}{\sqrt{(l_{0}+m)!(l_{0}-m)!}\: \sqrt{(l'+m')!(l'-m')!}}\right. \qquad \qquad \]
\begin{equation}
\label{B14}
\qquad \qquad \left. \frac{\sqrt{(l_{0}+m)!(l_{0}-m)!}\, {\cal O}_{l_{0}}^{m}\, \sqrt{(l'+m')!(l'-m')!}\, 
{\cal O}_{l'}^{m'}}{\left| \mathbf{P}-\mathbf{Q}\right| ^{l'+l_{0}+1}}\right| 
\end{equation}
 Now, let use the inequality (ref)\[
\left| \mathbf{a}\cdot \mathbf{b}\right| \leq \left| \mathbf{a}\right| \left| \mathbf{b}\right| \qquad \qquad \]
\begin{equation}
\label{B15}
\left| \sum _{n}a_{n}\, b_{n}\right| \leq \sqrt{\sum _{n}\left| a_{n}\right| ^{2}\sum _{n'}\left| b_{n'}\right| ^{2}}
\end{equation}
to get\[
\varepsilon \left( l_{0}\right) \leq \frac{1}{2}\sum _{l'=L'+1}^{\infty }\sqrt{\sum _{m=-l_{0}}^{l_{0}}\, 
\sum _{m'=-l'}^{l'}\frac{\left[ (l_{0}+l'-(m+m'))!\right] ^{2}}{(l_{0}+m)!(l_{0}-m)!\, (l'+m')!(l'-m')!}}\qquad 
\qquad \qquad \qquad \qquad \]
\begin{equation}
\label{B16}
\frac{\sqrt{\sum ^{l_{0}}_{m''=-l_{0}}\sum _{m'''=-l'}^{l'}(l_{0}+m'')!(l_{0}-m'')!\left| {\cal O}_{l_{0}}^{m''}\right|
 ^{2}\: (l'+m''')!(l'-m''')!\left| {\cal O}_{l'}^{m'''}\right| ^{2}}}{\left| \mathbf{P}-\mathbf{Q}\right| ^{l'+l_{0}+1}}
\end{equation}
Which we can rewrite very compactly as 
\begin{equation}
\label{B17}
\varepsilon \left( l_{0}\right) \leq \frac{1}{2}\sum _{l'=L'+1}^{\infty }\frac{{\cal O}_{l_{0}}[\rho _{1};\left|
 \mathbf{P}\right| ]\: N\left[ l_{0},l'\right] \: {\cal O}_{l'}[\rho _{2};\left| \mathbf{Q}\right| ]}{\left| 
\mathbf{P}-\mathbf{Q}\right| ^{l'+l_{0}+1}}
\end{equation}
where\begin{equation}
\label{B18}
N\left[ l,l'\right] =\sqrt{\sum _{m=-l_{0}}^{l_{0}}\, \sum _{m'=-l'}^{l'}\frac{\left[ (l_{0}+l'-(m+m'))!
\right] ^{2}}{(l_{0}+m)!(l_{0}-m)!\, (l'+m')!(l'-m')!}}
\end{equation}
However, we still have to eliminate the infinite sum over \( l' \)
within our error expression. Let us consider the limit of

\begin{equation}
\label{B19}
\lim _{l\rightarrow \infty }\left\{ {\cal O}_{l}[\rho ;\left| \mathbf{P}\right| ]\right\} ^{1/l}
\end{equation}
 Considering the density as a sum of delta functions with arbitrary
weights. 

\begin{equation}
\label{B20}
\rho \left( \mathbf{r}\right) =\sum _{i}c_{i}\, \delta \left( \mathbf{r}-\mathbf{r}_{i}\right) 
\end{equation}
 then for the multipoles\begin{equation}
\label{B21}
{\cal O}_{l}^{m}\left[ \rho ;\mathbf{P}\right] =\int _{V_{\infty }}\, d\mathbf{r}\, \widehat{O}_{l}^{m}\left[ 
\mathbf{r}-\mathbf{P}\right] \, \rho \left( \mathbf{r}\right) =\sum _{i}c_{i}\, \widehat{O}_{l}^{m}\left[ 
\mathbf{r}_{i}-\mathbf{P}\right] 
\end{equation}
and for the unsold weights\begin{eqnarray*}
{\cal O}_{l}[\rho ;\left| \mathbf{P}\right| ] & = & \sqrt{\sum _{m}(l-m)!(l+m)!\left| {\cal O}_{l}^{m}\left[ 
\rho ;\left| \mathbf{P}\right| \right] \right| ^{2}}\\
 & = & \sqrt{\sum _{m}(l-m)!(l+m)!\left| \sum _{i}c_{i}\, \widehat{O}_{l}^{m}\left[ \mathbf{r}_{i}-\mathbf{P}
\right] \right| ^{2}}
\end{eqnarray*}
\begin{equation}
\label{B22}
\; 
\end{equation}
which we can rewrite using the addition therom of anglular momentum
as (ref)\begin{equation}
\label{B23}
{\cal O}_{l}[\rho ;\left| \mathbf{P}\right| ]=\sqrt{\sum _{ij}c_{i}\, c_{j}^{*}\, \left| \mathbf{r}_{i}-
\mathbf{P}\right| ^{l}\left| \mathbf{r}_{j}-\mathbf{P}\right| ^{l}\, P_{l}\left[ \cos \gamma _{ij}\right] }
\end{equation}
which allows us to take the limit\begin{equation}
\label{B24}
\lim _{l\rightarrow \infty }\left\{ {\cal O}_{l}[\rho ;\left| \mathbf{P}\right| ]\right\} ^{1/l}=\lim _{l\rightarrow 
\infty }\left\{ \left| c_{I}\right| \, \left| \mathbf{d}_{max}\right| ^{l}\right\} ^{1/l}=\left| \mathbf{d}_{max}\right| 
\end{equation}
where \( \left| \mathbf{d}_{max}\right|  \)is the maximum distance
of a distribution to the contraction center \( \left| \mathbf{P}\right|  \).
Let us rewrite equation (\ref{B17}) as\begin{equation}
\label{B25}
\varepsilon \left( l_{0}\right) \leq \frac{1}{2}\frac{{\cal O}_{l_{0}}\left[ \rho _{1};\left| \mathbf{P}\right| 
\right] }{\left| \mathbf{P}-\mathbf{Q}\right| ^{l_{0}+L'+2}}\sum _{l=0}^{\infty }\frac{N\left[ l_{0},l+L'+1\right]
 {\cal O}_{l+L'+1}\left[ \rho _{1};\left| \mathbf{Q}\right| \right] }{\left| \mathbf{P}-\mathbf{Q}\right| ^{l}}
\end{equation}
Analyzing the behavior of \( N\left[ l,l'\right]  \) for large \( l' \)
we find that\begin{equation}
\label{B26}
N\left[ l,l'\right] \leq \alpha _{l}\frac{\left( \left( 1+l'\right) \left( 2+l'\right) \ldots \left( l+l'\right)
 \right) }{l!}=\alpha _{l}\frac{(l+l')!}{l!\, l'!}\equiv n\left[ l,l'\right] 
\end{equation}
where \( \alpha _{l}\leq \sqrt{2} \). Using equation (\ref{B26})
this can be written as\begin{equation}
\label{B27}
\varepsilon \left( l_{0}\right) \leq \frac{1}{2}\frac{{\cal O}_{l_{0}}\left[ \rho _{1};\left| \mathbf{P}\right| 
\right] }{\left| \mathbf{P}-\mathbf{Q}\right| ^{l_{0}+L'+2}}\sum _{l=0}^{\infty }\frac{n\left[ l_{0},l+L'+1\right] 
{\cal O}_{l+L'+1}\left[ \rho _{1};\left| \mathbf{Q}\right| \right] }{\left| \mathbf{P}-\mathbf{Q}\right| ^{l}}
\end{equation}
What remains is to deterimine the upper bound to \( {\cal O}_{l}\left[ \rho _{1};\left| \mathbf{Q}\right| \right]  \).
Analyizing equation (\ref{B23}) we can show that\begin{equation}
\label{B28}
{\cal O}_{l}[\rho ;\left| \mathbf{P}\right| ]\leq C_{\rho }\left| \mathbf{d}_{max}\right| ^{l}
\end{equation}
Where \( C_{\rho } \) can be determined from\begin{equation}
\label{B28B}
C_{\rho }\equiv \max_{l=L+1} \left( \frac{{\cal O}_{l}[\rho ;\left| \mathbf{P}\right| ]}{\left| \mathbf{d}_{max}\right|
 ^{l}}\right) 
\end{equation}
Using equation (\ref{B28}) the error bound can be rewritten as
\begin{equation}
\label{B29}
\varepsilon \left( l_{0}\right) \leq \frac{1}{2}\frac{{\cal O}_{l_{0}}\left[ \rho _{1};\left| \mathbf{P}\right|
 \right] \, C_{\rho _{2}}\left| \mathbf{d}_{max}\right| ^{L'+1}}{\left| \mathbf{P}-\mathbf{Q}\right| ^{l_{0}+L'+2}}
\sum _{l=0}^{\infty }\frac{n\left[ l_{0},l+L'+1\right] \left| \mathbf{d}_{max}\right| ^{l}}{\left| \mathbf{P}-
\mathbf{Q}\right| ^{l}}
\end{equation}
This can then be summed to\begin{equation}
\label{B29B}
\varepsilon \left( l_{0}\right) \leq \frac{1}{2}\frac{{\cal O}_{l_{0}}\left[ \rho _{1};\left| \mathbf{P}\right| 
\right] \, C_{\rho _{2}}\left| \mathbf{d}_{max}\right| ^{L'+1}}{\left| \mathbf{P}-\mathbf{Q}\right| ^{l_{0}+L'+2}}n
\left[ l_{0},L'+1\right] \, _{2}F_{1}\left[ 1,\, l_{0}+L'+2,\, L'+2;\, \frac{\left| \mathbf{d}_{max}\right| }
{\left| \mathbf{P}-\mathbf{Q}\right| }\right] 
\end{equation}
To leading order, this can be simplified to\begin{equation}
\label{B30}
\varepsilon \left( l_{0}\right) \leq \frac{1}{2}\frac{{\cal O}_{l_{0}}\left[ \rho _{1};\left| \mathbf{P}\right| 
\right] \, n\left[ l_{0},L'+1\right] \, C_{\rho _{2}}\left| \mathbf{d}_{max}\right| ^{L'+1}}{\left| \left| 
\mathbf{P}-\mathbf{Q}\right| -\left| \mathbf{d}_{max}\right| \right| ^{l_{0}+1}\left| \mathbf{P}-\mathbf{Q}
\right| ^{L'+1}}
\end{equation}
which bounds equation (\ref{B29B}).

We can also derive an exact error bound for FMM methods. In the case of FMM methods, equation (\ref{B5}) 
does not apply.And therefore our equations have to be modified. Starting with equation (\ref{B2}) we can 
show that the error can be writen as

\begin{equation}
\varepsilon_{FMM}  \leq  
{\frac{1}{2}}\sum_{l=0}^{\infty} \sum _{l'=0}^{\infty }
\frac{{\cal O}_{l}[\rho _{1};\left|
\mathbf{P}\right| ]\: N\left[ l,l'\right] \: {\cal O}_{l'}[\rho _{2};\left| \mathbf{Q}\right| ]}{\left| 
\mathbf{P}-\mathbf{Q}\right| ^{l+l'+1}}
-{\frac{1}{2}}\sum_{l=0}^{L} \sum _{l'=0}^{L'}
\frac{{\cal O}_{l}[\rho _{1};\left|
\mathbf{P}\right| ]\: N\left[ l,l'\right] \: {\cal O}_{l'}[\rho _{2};\left| \mathbf{Q}\right| ]}{\left| 
\mathbf{P}-\mathbf{Q}\right| ^{l+l'+1}}
\label{B40}
\end{equation}
Using equations (\ref{B26}) and (\ref{B28B}) we can rewrite this as
\begin{equation}
\varepsilon_{FMM}  \leq 
{\frac{C_{\rho_1} C_{\rho_2}}{2}} \left[
\sum_{l=0}^{\infty} \sum _{l'=0}^{\infty }
{\frac{\left| \mathbf{d} \right|^{l} n[l,l'] \left| \mathbf{d'} \right|^{l'}}
{\left|\mathbf{P}-\mathbf{Q}\right| ^{l+l'+1}}}
-\sum _{l=0}^{L}\sum_{l'=0}^{L'}
{\frac{\left| \mathbf{d} \right|^{l} n[l,l'] \left| \mathbf{d'} \right|^{l'}}
{\left|\mathbf{P}-\mathbf{Q}\right| ^{l+l'+1}}} \right]
\label{B41}
\end{equation}
where
\begin{equation}
C_{\rho }  \equiv   \max_{l=0}^{\infty} \left( \frac{{\cal O}_{l}[\rho ;\left| \mathbf{P}\right| ]}
{\left| \mathbf{d}_{max}\right|^{l}}\right) 
\label{B42}
\end{equation}
This can be summed to get
\begin{equation}
\varepsilon_{FMM}  \leq 
{\frac{C_{\rho_1} C_{\rho_2}}{2}} \left[
\frac{1}{\left|\left|\mathbf{P}-\mathbf{Q}\right|-\left| \mathbf{d} \right|-\left| \mathbf{d'} \right| \right|}
-\sum _{l=0}^{L}\sum_{l'=0}^{L'}
{\frac{\left| \mathbf{d} \right|^{l} n[l,l'] \left| \mathbf{d'} \right|^{l'}}
{\left|\mathbf{P}-\mathbf{Q}\right| ^{l+l'+1}}} \right]
\label{B43}
\end{equation}
For the case where $L=L'$ and $|\mathbf{d}|= |\mathbf{d'}|$ this has a particularly simple form
\begin{equation}
\varepsilon_{FMM}  \leq 
\frac{C_{\rho_1} C_{\rho_2} \left| \mathbf{d} \right|^{L+1}}
{\left|\mathbf{P}-\mathbf{Q}\right|^{L+1}
\left|\left|\mathbf{P}-\mathbf{Q}\right|-2\left| \mathbf{d} \right|\right|}
\label{B44}
\end{equation}
%
%
%
\end{document}
