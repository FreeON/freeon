%
%   This file is part of the APS files in the REVTeX 4 distribution.
%   Version 4.0 of REVTeX, August 2001
%
%   Copyright (c) 2001 The American Physical Society.
%
%   See the REVTeX 4 README file for restrictions and more information.
%
% TeX'ing this file requires that you have AMS-LaTeX 2.0 installed
% as well as the rest of the prerequisites for REVTeX 4.0
%
% See the REVTeX 4 README file
% It also requires running BibTeX. The commands are as follows:
%
%  1)  latex apssamp.tex
%  2)  bibtex apssamp
%  3)  latex apssamp.tex
%  4)  latex apssamp.tex
%
%\documentclass[prb,aps,nobibnotes,twocolumn,doublespace,twocolumngrid,superbib]{revtex4}
%%\documentclass[twocolumn,showpacs,preprintnumbers,amsmath,amssymb]{revtex4}
%\documentclass[preprint,showpacs,preprintnumbers,amsmath,amssymb]{revtex4}

% Some other (several out of many) possibilities
%\documentclass[preprint,aps]{revtex4}
%\documentclass[preprint,aps,draft]{revtex4}
%\documentclass[prb]{revtex4}% Physical Review B

%%\usepackage{amsmath}
%%\usepackage{amssymb}
%%\usepackage{graphicx}% Include figure files
%%\usepackage{dcolumn}% Align table columns on decimal point
%%\usepackage{bm}% bold math

%\documentclass[pre,aps,twocolumn,showpacs,twocolumngrid,superbib]{revtex4}
%\documentclass[prl,aps,twocolumn,showkeys,twocolumngrid,superbib]{revtex4}
%\documentclass[twocolumn,showkeys,showpacs,preprintnumbers,amsmath,amssymb]{revtex4}
\documentclass[pra,twocolumn,showpacs,twocolumngrid,superbib]{revtex4}
%\documentclass[showpacs,preprint,superbib]{revtex4}

\usepackage{graphicx}
\usepackage{amsfonts}
\usepackage{amsmath}
\usepackage{bm}
\usepackage{alltt}
\usepackage{fancyhdr}
\usepackage{dcolumn} 

\pagestyle{fancy}


\def\Tr{{\rm Tr}}

%\nofiles

\begin{document}

%\preprint{APS/123-QED}

%\title{Density functional energy gradients with respect to atomic positions and cell parameters
%  within the $\Gamma$-point approximation}
\title{Energy gradients with respect to atomic positions and cell parameters
  for the density functional theory at the $\Gamma$-point}

\author{Val\'ery Weber}
\affiliation{Department of Chemistry, University of Fribourg, 1700 Fribourg, Switzerland.}%
\author{Christopher J. Tymczak}%
\email{tymczak@lanl.gov}
\author{Matt Challacombe}%
\affiliation{Los Alamos National Laboratory, Theoretical Division, Los Alamos 87545, New Mexico, USA.}%

\date{\today}% It is always \today, today,
             %  but any date may be explicitly specified

\begin{abstract}
  Recently, construction of the exact Hartree-Fock exchange
  gradients with respect to atomic positions and cell parameters
  within the $\Gamma$-point approximation
  has been introduced [J. Chem. Phys. {\bf ???}, ????, (200?)].
  In this article, the formalism is extended to the evaluation of
  analytical $\Gamma$-point density functional energy atomic and lattice gradients. 
  As an illustration, the analytical gradients have been used
  in conjunction with the QUICCA algorithm [K. N\'emeth and M. Challacombe,
  J. Chem. Phys. {\bf 121}, 2877, (2004)] to optimize 3D periodic
  systems at the DFT and hybride-HF/DFT levels.
\end{abstract}

%\pacs{Valid PACS appear here}% PACS, the Physics and Astronomy
                             % Classification Scheme.
\keywords{Periodic boundary condition, density functional theory, lattice gradients, $\Gamma$-point.}
                              %display desired
\maketitle

%\section{Don't forget to change that}
%-Add KDoll01a in mondo.bib K. Doll Computer Physics Communications 137 (2001) 74-88\\
%-See that http://www.cryst.ehu.es/\\

\section{Introduction}
In preceding papers, we have developed linear scaling quantum chemical methods
for construction of the periodic Coulomb, exchange-correlation~\cite{CTymczak04a}
and the exact Hartree-Fock exchange~\cite{CTymczak04b}
matrices within the $\Gamma$-point approximation.
In this paper, the implementation of the Coulomb and exchange-correlation 
energy gradients with respect to atomic positions and cell parameters
at the $\Gamma$-point is presented.

Kohn-Sham density functional theory has proven to be a highly 
competitive method for a wide range of applications in solid 
state physics and chemistry.
The hybrid Hartree-Fock/Density Functional Theory (hybrid-HF/DFT) model chemistries
are an important next step in accuracy beyond the Generalized Gradient
Approximation~\cite{Gill92,Becke93,VBarone96,CAdamo99}. Together with linear
scaling methods for computing the density matrix~\cite{ANiklasson02A,ANiklasson03}, these
advances provide a framework for the application of hybride-HF/DFT
models to large condensed phase systems, polymers, surfaces and wires.

While the $\Gamma$-point approximation uses only the $\mathbf{k}=0$ point to sample
the Brillouin zone, it does however converge to the
${\bf k}$-space integration limit, in the worst case with the inverse
unit cell volume (see for example Refs.~\cite{CKittel71,NAshcroft76}).
The convergence of the $\Gamma$-point approximation to
the corresponding ${\bf k}$-space limit was recently
demonstrated by us for DFT~\cite{CTymczak04a}, HF and hybrid-HF/DFT~\cite{CTymczak04b} 
level of theories as well as for the Hartree-Fock 
atomic and cell gradients~\cite{VWeber05b}.

For one dimensional fluoric acid (HF)$_n$ chains, the convergence of the 
HF-MIC $\Gamma$-point energy, atomic and cell forces to the converged 
large cell $\Gamma$-point approximation with respect to cell length 
have been explicitly shown to be exponential in the cell size~\cite{VWeber05b}. 
A fast convergence of the total energy and geometrical parameters
have also been observed for 3D systems as MgO and urea.

The $\Gamma$-point approach opens the capabilities of studying very large
complex and disordered systems such as liquids, low concentration defects, adsorption of
large molecule on surfaces, {\em etc}, where conventional methods
of sampling the Brillouin zone may become computationally too demanding, and
where the $\Gamma$-point approximation is well justified.

Finding crystal structures of condensed systems can
be formulated as a minimization of the total energy
with respect to atomic coordinates and cell vectors.
The problem is then the minimization of the total energy with $L$ degree of freedom, where
$L=3N_{atm}+3$, $N_{atm}$ is the number of atoms, $3N_{atm}-3$ is the number
of independent coordinates after the elimination of translation,
and the number of independent vector elements
after the elimination of cell rotations is 6.

This minimization can be achieved with the help of an 
efficient optimizer~\cite{KNemeth04,TBucko05,KNemeth05}
and the knowledge of the gradients with respect to atomic 
positions and cell parameters.

The analytical lattice gradient method of density functional theory using GTAO for
one dimensional extended systems was implemented by Hirata and Iwata~\cite{SHirata98}.
The three dimensional case has been implemented by Kudin and Scuseria
~\cite{KKudin00A,KKudin00B}. Their approach for the Coulomb problem is
based on the direct space fast multipole method.
Recently Doll, Dovesi and Orlando~\cite{KDoll04} presented
implementation of the Hartree-Fock cell gradients in
the {\sc Crystal03}~\cite{RDovesi00} package for three dimensional systems.
Their code is based on GTAO and the summation
of the Coulomb energy is performed with the Ewald method~\cite{PEwald21},
which is a combination of direct and reciprocal lattice summations.
%For an efficient truncation of the three infinite summations of the exchange
%series, the {\sc Crystal03} program uses in the first hand, the decay between local basis function
%products and in the second, the fact that elements of the density
%matrix of an insulator decays exponentially with inter-atomic separations.
The strategy to compute the analytic Hartree-Fock gradients for
periodic system, in the frame of the {\sc Crystal03} package has been
presented by Doll, Saunders and Harrison~\cite{KDoll01} and Doll \cite{KDoll01a}.
Their implementation is based on the Hermite Gaussian-type functions
in the context of the McMurchie-Davidson algorithm~\cite{LMcmurchie78}.

The remainder of this paper is organized as follows:
In Section~\ref{Sec:Formalism}, we introduce the formalism and discuss
the implementation of the Coulomb and exchange-correlation gradients with respect
to atomic positions and cell parameters
at the $\Gamma$-point approximation. Full optimization of several 3D
periodic systems are given
in Section~\ref{Sec:NumExamples} as an illustration of the formalism.
Finally in Section~\ref{Sec:Conclusions} we summarize our results.

\section{Formalism}\label{Sec:Formalism}
The primitive cell is given by the three vectors $\mathbf{a}$,
$\mathbf{b}$ and $\mathbf{c}$. Then $M$ is the $3\times3$ matrix composed
of the primitive lattice vectors
\begin{equation*}
  M=(\mathbf{a},\mathbf{b},\mathbf{c}).
\end{equation*}
The position of a lattice is $\mathbf{R(n)}=M\mathbf{n}$,
with $\mathbf{n}=(n_a,n_b,n_c)$ a vector of integers.
The position of atom $A$ in the cell $\mathbf{R(n)}$ is $\mathbf{A}=M(\mathbf{f}_A+\mathbf{n})$,
with $\mathbf{f}_A=(f_{Aa},f_{Ab},f_{Ac})$ the fractional coordinates of
atom $A$ in the central cell.

An unnormalized Cartesian Gaussian-type function (CGTF) centered on atom $A$ is
\begin{equation*}
  \phi_a(\mathbf{r})=(x-A_x)^{a_x}(y-A_y)^{a_y}(z-A_z)^{a_z}e^{-\zeta_a(\mathbf{r-A})^2},
\end{equation*}
where the triad $a=(a_x,a_y,a_z)$ sets the angular symmetry and the exponent $\zeta_a$
is chosen to describe a particular length scale. Gaussian basis functions are often
contracted to approximate atomic eigenfunctions.

The total energy within the $\Gamma$-point approximation~\cite{CTymczak04a,CTymczak04b} can be 
expressed as:
\begin{equation}
  \begin{split}
%%%%%%%%%%%%%%%%%%%%%%%%
%    E_{tot}(&\mathbf{f}_A,\mathbf{f}_B,\ldots,M)=\sum_{ab} P_{ab}
%    [T_{ab}+\frac{1}{2}(J_{ab}+\alpha_1 K_{ab})]\\
%    &+\alpha_2 E^{xc}+h_{nuc},
%%%%%%%%%%%%%%%%%%%%%%%%
    E_{tot}(\mathbf{f}_A,&\mathbf{f}_B,\ldots,M)=E^{k}+E^{c}+\alpha_1 E^{x}+\alpha_2 E^{xc}\\
    &=\sum_{ab} P_{ab}[T_{ab}+\frac{1}{2}(J_{ab}+\alpha_1 K_{ab})]\\
    &+\alpha_2 E^{xc}+h_{nuc},
%%%%%%%%%%%%%%%%%%%%%%%%
%%    E_{tot}(\mathbf{f}_A,\mathbf{f}_B,\ldots,M)=E^{kin}+E^{C}+E^{x}+E^{xc},
%    E_{tot}(\mathbf{f}_A,\mathbf{f}_B,\ldots,M)&=\Tr[PT]+h_{nuc}\\
%    &+\frac{1}{2} \Tr [P(J+\alpha_1 K^{xc}+\alpha_2 K)]
%%    &+\frac{1}{2}\sum_{ab} P_{ab}[J_{ab}+\alpha_1 K_{ab}^{xc}+\alpha_2 K_{ab}]+h_{nuc},
%    E(&\mathbf{f}_A,\mathbf{f}_B,\ldots,M)=
%    \sum_{\substack{\mathbf{m}\\a b}}P_{ab}((a|T|b^\mathbf{m})\\
%    &+\frac{1}{2}\sum_{\substack{\mathbf{m}\mathbf{n}\\a b c d}}P_{ab}P_{cd}( (ab|c^\mathbf{m}d^\mathbf{n})
%    -\frac{1}{2}(ac^\mathbf{m}|bd^\mathbf{n}) )+E_{nuc}
  \end{split}
\end{equation}
%where the different contributions are given
%\begin{equation}
%  \begin{split}
%    E^{kin}&=\sum_{ab}P_{ab}T_{ab}\\
%    E^{C}&=\frac{1}{2}\sum_{ab}P_{ab}J_{ab}\\
%    E^{x}&=\frac{1}{2}\sum_{ab}P_{ab}K^{x}_{ab}\\
%    E^{xc}&=,
%  \end{split}
%\end{equation}
%where $P$, $T$, $J$, $K^{xc}$ and $K$ 
%are respectively the density, kinetic energy, electron-electron and elecetron-nuclear Coulomb,
%exchange-correlation, exacte Hartree-Fock exchange matrices.
%and $h_{nuc}$ is the nuclear-nuclear repulsion energy.
where $P$ is the density matrix, $T$ is the kinetic energy matrix, $J$
is the electron-electron and electron-nuclear Coulomb matrix, 
$E^{xc}$ the exchange-correlation energy, $K$ is the exacte 
Hartree-Fock exchange matrix and $h_{nuc}$ is the nuclear-nuclear repulsion energy.
The factors $\alpha_1$ and $\alpha_2$ are mixing coefficients of the
exchange-correlation and exacte Hartree-Fock energies respectively.

%within the $\Gamma$-point approximation~\cite{CTymczak04a,CTymczak04b}
%where the indices $\mathbf{mn}$ run over the direct lattice vectors,
%$abcd$ over the basis functions,

%and the ERIs are written in the Mulliken notation and computed
%with the MIC as discussed in the following.

The derivative of the total energy with respect to an external perturbation $\lambda$ is 
\begin{equation*}
  \begin{split}
%%%   \frac{\partial E_{tot}}{\partial \lambda}&=\frac{\partial E^{kin}}{\partial \lambda}
%%% -\sum_{ab}W_{ab}\frac{\partial S_{ab}}{\partial \lambda}+\frac{\partial h_{nuc}}{\partial \lambda}\\
%%%   &+\frac{1}{2}\bigg[\frac{\partial E^{C}}{\partial \lambda}\bigg|_P
%%%    +\alpha_1 \frac{\partial E^{x}}{\partial \lambda}\bigg|_P\bigg]
%%%    +\alpha_2 \frac{\partial E^{xc}}{\partial \lambda}\bigg|_P
    \frac{\partial E_{tot}}{\partial\lambda}&=
    \frac{\partial E^{k}}{\partial\lambda}+\frac{\partial E^{c}}{\partial\lambda}+
    \frac{\partial E^{x}}{\partial\lambda}+\frac{\partial E^{xc}}{\partial\lambda}\\
    &=\sum_{ab}P_{ab}\frac{\partial T_{ab}}{\partial\lambda}
    +\frac{\partial h_{nuc}}{\partial\lambda}\\
    &+\frac{1}{2}\sum_{ab}P_{ab}\bigg[\frac{\partial J_{ab}}{\partial\lambda}\bigg|_P
     +\alpha_1 \frac{\partial K_{ab}}{\partial\lambda}\bigg|_P\bigg]\\
    &+\alpha_2 \frac{\partial E^{xc}}{\partial\lambda}\bigg|_P
    -\sum_{ab}W_{ab}\frac{\partial S_{ab}}{\partial\lambda}
%    &+\alpha_2 E^{xc}+h_{nuc},
%    \frac{\partial E_{tot}}{\partial \lambda}&=2\Tr\left[PFP\frac{\partial S}{\partial \lambda}\right]
%    +\frac{\partial h_{nuc}}{\partial \lambda}\\
%    &-\Tr\left[P\frac{\partial J}{\partial \lambda}\bigg|_P+\alpha_1 P\frac{\partial K^{xc}}{\partial \lambda}\bigg|_P+\alpha_2 
%      P\frac{\partial K}{\partial \lambda}\bigg|_P\right]
%    E(\mathbf{f}_A,\mathbf{f}_B,\ldots,M)=\sum_{a b}P_{ab}(h_{ab}+J_{ab}+K_{ab}^{xc}+K_{ab})+E_{nuc}
%    -\frac{1}{2}\sum_{\substack{\mathbf{m}\\a b}}P_{ab}((a|\nabla^2|b^\mathbf{m})-\sum_A Z_A(a|1/|r-A||b^\mathbf{m}) )\\
%    &+\sum_{\substack{\mathbf{m}\mathbf{n}\\a b c d}}P_{ab}P_{cd}( (ab|c^\mathbf{m}d^\mathbf{n})
%    -\frac{1}{2}(ac^\mathbf{m}|bd^\mathbf{n}) )+E_{nuc}
  \end{split}
\end{equation*}
where $W=PFP$ is the energy-weighted density matrix.
The gradients with respect to fractional coordinates and lattice parameters 
are simply given by $\lambda=f_{Gj}$ and $\lambda=M_{ij}$, respectively.
The Hartree-Fock exchange gradients with respect to atomic and cell parameters,
$\partial K_{ab}/\partial\lambda |_P$, have been previously derived~\cite{VWeber05b}.

The energy gradient with respect to the fractional coordinate
$f_{Gj}$ can be obtained through the linear transform
\begin{equation*}
  \frac{\partial E_{tot}}{\partial f_{Gj}}=\sum_{i=x,y,z}M_{ij}\frac{\partial E_{tot}}{\partial G_i},
\end{equation*}
where $\partial E_{tot}/\partial G_i$ is the standard gradient with respect to atomic position.
In the following, we describe the implementation of the Coulomb and exchange-correlation
gradients with respect to atomic and cell parameters.

\subsection{Coulomb integrals}
\noindent The periodic quantum Coulomb sums involve the contributions
\begin{equation}
J_{ab}=J_{ab}^{In}+J_{ab}^{PFF}+J_{ab}^{TF},
\end{equation}
where $J^{In}$ s the inner cell set sum done directly, $J^{PFF}$ is the periodic far field term and $J^{TF}$ 
is a surface term which corrects the boundary at infinity.
\subsubsection{Inner Sum}
\noindent The Inner cell set contribution to the Coulomb matrix is 
\begin{equation}
J_{ab}^{In} = \sum_{{\bf R'} \in In} \int d^3{\bf r} \, d^3{\bf r'}\, \rho_{ab}({\bf r}) | 
{\bf r}-{\bf r'}+{\bf R'}|^{-1} \rho_{tot}({\bf r'})
\end{equation}
where
\begin{eqnarray}
\rho_{ab}({\bf r})  & = & \sum_{\bf R} \phi_a({\bf r}) \phi_b({\bf r}+{\bf R}) \nonumber\\
\rho_{tot}({\bf r}) & = & \sum_{ab} \sum_{\bf R} P_{ab} \, \phi_a({\bf r}) \phi_b({\bf r}
+{\bf R}) -\sum_{a} Z_a \delta({\bf r}-{\bf a})\nonumber
\nonumber  
\end{eqnarray}
The derivative of $J_{ab}^{In}$ with respect to the cell parameter $M_{ij}$ is
\begin{eqnarray}
{{\partial J_{ab}^{In}} \over {\partial M_{ij}}} &=& \sum_{{\bf R'} \in In} 2\int d^3{\bf r} \, d^3{\bf r'}\, 
{{\partial \rho_{ab}({\bf r})}\over {\partial M_{ij}} } | 
{\bf r}-{\bf r'}+{\bf R'}|^{-1} \rho_{tot}({\bf r'}) \nonumber \\
&+& \int d^3{\bf r} \, d^3{\bf r'}\, \rho_{ab}({\bf r})
{{\partial |{\bf r}-{\bf r'}+{\bf R'}|^{-1}} \over {\partial M_{ij}}} \rho_{tot}({\bf r'})
\end{eqnarray}
\subsubsection{Periodic far field}
\noindent We use a spherical multipole method inorder to compute the periodic far field correction to the colomb matrix, which is
\begin{equation}
J_{ab}^{PFF}=\sum_{mlm'l'} (-1)^l \, \rho^{ml}_{ab} \,\, {\cal M}_{mlm'l'} \,\, \rho^{m'l'}_{tot}
\end{equation}
where
\begin{eqnarray}
\rho^{ml}_{ab}     & = & \sum_{\bf R} \int d^3{\bf r} \, \phi_a({\bf r}) \phi_b({\bf r}+{\bf R}) \, O_{ml}({\bf r}) \nonumber\\
\rho^{ml}_{tot}    & = & \int d^3{\bf r} \, \rho_{tot}({\bf r})\, O_{ml}({\bf r}) \nonumber
\end{eqnarray}
and where the ${\cal M}$ tensor is summed using a ewald like partiaion [ref]
\begin{equation}
{\cal M}_{mlm'l'} =  \sum_{\bf R} M_{mlm'l'}[{\bf R}] \nonumber
\end{equation}
The derivative of $J_{ab}^{PFF}$ with respect to the cell parameter $M_{ij}$ is
\begin{eqnarray}
{{\partial J_{ab}^{PFF}} \over {\partial M_{ij}}} &=& \sum_{mlm'l'}
2 {{\partial \rho^{ml}_{ab}} \over {\partial M_{ij}}} \,\, {\cal M}_{mlm'l'} \,\, \rho^{m'l'}_{tot} \nonumber\\
&+& \rho^{ml}_{ab} \,\, {{\partial {\cal M}_{mlm'l'}}\over {\partial M_{ij}} } \,\, \rho^{m'l'}_{tot}
\end{eqnarray}
where the calculation of the derivatives of the ${\cal M}$ tensor is shown in Appendix A in detail (???) 
\subsubsection{Tin-Foil}
The tin-foil correction to the Coulomb matrix is
\begin{equation}
J_{ab}^{TF}=\frac{2\pi}{3V}(QS_{ab}-2\mathbf{d}_{ab}\cdot\mathbf{D})
\end{equation}
where $Q$ is the trace of the system quadrupole, $S_{ab}$ is an element of the overlap matrix,
$\mathbf{D}$ the dipole moment of the system,
$\mathbf{d}_{ab}$ the dipole moment of the distribution $\rho_{ab}^\infty$.
The derivative of $J_{ab}^{TF}$ with respect to the cell parameter $M_{ij}$ is
\begin{equation}
  \begin{split}
    \frac{\partial J_{ab}^{TF}}{\partial M_{ij}}&=
    -\frac{2\pi}{3V^2}\frac{\partial V}{\partial M_{ij}}(QS_{ab}-2\mathbf{d}_{ab}\cdot\mathbf{D})\\
    &+\frac{2\pi}{3V}\bigg(\frac{\partial Q}{\partial M_{ij}}S_{ab}+Q\frac{\partial S_{ab}}{\partial M_{ij}}\\
    &-2\frac{\partial \mathbf{d}_{ab}}{\partial M_{ij}}\cdot\mathbf{D}
    -2\mathbf{d}_{ab}\cdot\frac{\partial \mathbf{D}}{\partial M_{ij}} \bigg),
  \end{split}
\end{equation}
where the derivatives ${\partial Q}/{\partial M_{ij}}$,
${\partial \mathbf{d}_{ab}}/{\partial M_{ij}}$ and
${\partial \mathbf{D}}/{\partial M_{ij}}$ are straitforward;
the term ${\partial S_{ab}}/{\partial M_{ij}}$
is given in Ref.~\cite{KDoll04}.



\subsection{Exchange-correlation integrals}
The numerical integration of the DFT
exchange-correlation term is carried out over a cuboid
integration domain $V_{\Box}$ which is equivalent to the unit
cell volume $V_{UC}$. The exchange-correlation energy is thus
\begin{equation}\label{Eq:xc}
%  K_{ab}^{xc}=\int_{V_\Box}\rho_{ab}(\mathbf{r})v_{xc}[\rho,\mathbf{r}]d^3r.
%%  K_{ab}^{xc}=\int_{V_\Box}\rho_{ab}(\mathbf{r})v_{xc}(\rho,\mathbf{r})dV
%%  =\int_{V_\Box}f_{ab}(\mathbf{r})dV,
  E^{xc}%=\int_{V_\Box}\rho(\mathbf{r})\varepsilon_{xc}(\rho)dV
  =\int_{V_\Box}f(\mathbf{r})dV.
\end{equation}
%where $v_{xc}$ is the exchange-correlation potentiel.
In the following, the subscript $\Box$ will always refer to the cuboid cell.
This simple integration volume should be contrasted with more
conventional methods for computing the exchange-correlation matrix, involving
the ``Becke weights''~\cite{ABecke88}, which requirs integration over all the space $V_\infty$.
While Eq.~\ref{Eq:xc} is written in an abstract form for simplicity, 
in practice the code relies on the Pople, Gill and Johnson 
approach~\cite{JPople92,MChallacombe00A}. 

The derivative of Eq.~\ref{Eq:xc} with respect to cell parameter $M_{ij}$ is given by
\begin{equation}\label{Eq:xcLatGrd}
  \begin{split}
    \frac{\partial E^{xc}}{\partial M_{ij}}&=
%    \int_{V_{\Box}}\frac{\partial}{\partial M_{ij}}(\rho_{ab}(\mathbf{r})v_{xc}[\rho,\mathbf{r}])dV\\
%    &+\delta_{ij}\int_{S_{j}}\rho_{ab}(\mathbf{r})v_{xc}[\rho,\mathbf{r}]dS
    \int_{V_{\Box}}\frac{\partial f(\mathbf{r})}{\partial M_{ij}}dV
%    \sum_{A}(f_{Aj}+n_j)\int_{V_{\Box}}\frac{\partial f(\mathbf{r})}{\partial A_i}dV\\
    +\delta_{ij}\int_{S_{j\Box}}f(\mathbf{r})dS.
  \end{split}
\end{equation}
%where the intergation over the surface $S_{j\Box}$ arises
%from the derivative of the limits of the integral~(\ref{Eq:xc}).
%where $S_{j\Box}$ is the surface defined by the two lattice vectors different than $j$. 
The surface integral (rightmost term in Eq.~\ref{Eq:xcLatGrd}), which is not present in 
previous derivations~\cite{MTobita03,KKudin00B}, has its origin in 
the derivative of the limits of the integral in Eq.~\ref{Eq:xc}.
%%%% Need to check SHirata97
%Similar surface integral can be obtained from the derivative with respect to 
%unit cell parameters of the grid weight encountered in methods based 
%on Becke�s original numerical quadrature scheme~\cite{MTobita03,SHirata97}.
The surface integral is approximated by 
\begin{equation*}%\label{Eq:xcLatGrd}
  \int_{S_{j\Box}}f(\mathbf{r})dS=%\approx 
%  \frac{1}{2h}\int_{-h}^h\int_{S_{j\Box}}f_{ab}(\mathbf{r})dSdw_j
  \frac{1}{2h}\int_{V_{j\Box}}f(\mathbf{r})dV+\mathcal{O}(h^2)
%  \int_{S_{j}}\rho_{ab}(\mathbf{r})v_{xc}[\rho,\mathbf{r}]dS\approx 
%  \frac{1}{2h}\int_{V_{j}}\rho_{ab}(\mathbf{r})v_{xc}[\rho,\mathbf{r}]dV
\end{equation*}
where $h$ is a small number (typically $10^{-4}\,a.u.$) and 
$V_{j\Box}=S_{j\Box}\times[-h,h]$ is a thin volume where the integration over the interval $[-h,h]$
is carried out along the normal to the surface $S_{j\Box}$. This simple domain $V_{j\Box}$ 
can be efficiently integrated with the help of the HiCu algorithm~\cite{MChallacombe00A} 
to a desired accuracy.

\section{Numerical Examples}\label{Sec:NumExamples}
All developments were implemented in the {\sc MondoSCF}~\cite{MondoSCF} suite of
linear scaling quantum chemistry programs. 
%The code was compiled using the Portland Group F90 compiler {\tt pgf90} v5.1~\cite{pgf90-v5.1} 
%with the~{\tt -O1} options and with the GNU C compiler {\tt gcc} 
%v3.2.2 using the~{\tt -O1} flag.
The code was compiled using the HP fortran compiler {\tt f95} 
v5.5A~\cite{f95-v5.5a} and the {\tt -O4} option and the Compaq 
C compiler {\tt cc} v6.5~\cite{cc-v6.5} and the {\tt -O1} flag.
%All calculations were carried out on a 1024-node (2048 processors)
%dual P4 LinuxBIOS/BProc cluster connected with Myrinet 2000 running
%Red-Hat Linux release~9 (Shrike)~\cite{RedHat90}.
All calculations were carried out on a cluster of 256 4-CPU HP/Compaq 
Alphaserver ES45s with the Quadrics QsNet High Speed Interconnect.

The {\tt TIGHT} level of numerical accuracy has been used throughout this work.  
Thresholds that define the {\tt TIGHT} accuracy level include a matrix 
threshold $\tau=10^{-6}$, as well as other numerical thresholds 
detailed in Ref.~\cite{CTymczak04a}, which deliver at least 8 digits of 
relative accuracy in the total energy and 4 digits of absolute accuracy 
in the forces. 

In order to demonstrate the capabilities of our implementation of the
Khon-Sham density functional energy atomic and lattice gradients, we present in this Section
full optimization studies of 3D periodic systems without any lattice
nor any atomic positions symmetry constraint.
Our first benchmark case is the optimization of MgO at 
the Perdew-Burke-Ernzerhof (PBE)~\cite{Perdew_96v77} 
and Becke 3-parameter Lee-Yang-Parr (B3LYP)~\cite{ABecke93} 
level of theories. 
The second benchmark is XXX and was optimized at the B3LYP level of theory. 

Table~\ref{Tab:MgO} shows the progression of the cell parameters, total energies and
fractional coordinates of oxygen computed
for various cubic MgO supercells at the $\Gamma$-point PBE~\cite{Perdew_96v77}
and B3LYP~\cite{ABecke93} level of theories using
the 8-61G basis set for magnesium and the 8-51G basis set for oxygen.
The basis sets were specially optimized at the HF level for MgO by
Caus\`a et al~\cite{CBS:861G:MgO} and were obtained from Ref.~\cite{CrystalLib}.
The primitive cubic cell coordinates used for this system are given in
Ref.~\cite{PBCCoordinates}.
For comparison, we report the optimized lattice parameter of cubic MgO
obtained with the {\sc Crystal03} code and a $8\times 8\times 8$ $k$-points integration grid.
For both PBE and B3LYP, the smallest system (MgO)$_4$ 
shows a large discrepantcy of the cell parameter, energy and fractional 
coordinate of the oxygen with respect to its ${\bf k}$-space integration 
counterpart. The larger systems give cell parameter and 
fractional coordinate of the oxygen in very good agrement 
with the {\sc Crystal03} results; and the energies systematical converge 
while the system grows.
\begin{table}[t]
  \centering
  \caption{\protect
    Progression of the cell constant $a_0$,
    total energy $E$ and fractional coordinate of the oxygen $f_O$ in the primitive cell 
    for cubic (MgO)$_n$ using the periodic $\Gamma$-point
    PBE/8-61G/8-51G and B3LYP/8-61G/8-51G level of theories and the {\tt TIGHT} thresholds.
    Lattice parameters and energies are in \AA~and atomic unit respectively.
  }\label{Tab:MgO}
  \begin{tabular}{lcrccc}
  \toprule
  & & $n$ & $a_0$ & $E/n$ & $f_O$ \\
  \hline
    {\sc MondoSCF}\footnote[1]{$\Gamma$-point.} 
    & PBE   &   4 & 4.331 & $-$275.243165 & 0.4998 \\%
    &       &  32 & 4.213 & $-$275.284428 & 0.5000 \\%
    &       & 108 & 4.212 & $-$275.284641 & 0.5000 \\%
    & B3LYP &   4 & 4.328 & $-$275.389093 & 0.4972 \\%VWN5
    &       &  32 & 4.204 & $-$275.431167 & 0.5000 \\%VWN5
    &       & 108 & 4.204 & $-$275.431343 & 0.5000 \\%VWN5
  \hline
    {\sc Crystal03}\footnote[2]{$8\times 8\times 8$ $k$-points.}
    & PBE   &   1 & 4.212 & $-$275.284731 & $1/2$ \\
    & B3LYP &   1 & 4.204 & $-$275.431235 & $1/2$ \\
  \botrule
  \end{tabular}
\end{table}
%
%MgO PBExc Crystal03 User2.bas 8x8x8 default
%4.200 -275.28468980590
%4.205 -275.28471572672
%4.208 -275.28472235701
%4.210 -275.28472911839
%4.211 -275.28473028612
%4.212 -275.28473081270 <<<
%4.213 -275.28473059552
%4.215 -275.28473044697
%4.250 -275.28445240627
%
%MgO B3LYP Crystal03 User2.bas 8x8x8 default
%4.190 -275.43120191268
%4.200 -275.43122829482
%4.202 -275.43123248175
%4.203 -275.43123398833
%4.204 -275.43123514068 <<<
%4.205 -275.43123509457
%4.210 -275.43123002554
%4.215 -275.43121357505
%4.220 -275.43118801547
%4.250 -275.43082055853
%
%MONDO PBExc User2.bas Tight
%   a     a0         E                  E/n          d       f
% 4.3307 4.3307  -1100.9726623928 -275.243165598200 2.1643  0.49976
% 8.4253 4.2126  -8809.1016996344 -275.284428113575 2.1065  0.50004
%12.6369 4.2123 -29730.7411945410 -275.284640690194 2.1062  0.50001
%
%MONDO B3LYP/VWN3 User2.bas Tight
% 4.3278 4.3278  -1101.8625722845 -275.465643071125 2.1518  0.49720
% 8.4070 4.2035  -8816.2479051094 -275.507747034669 2.1017  0.49999
%
%
%MONDO B3LYP/VWN5 User2.bas Tight
% 4.3278 4.3278  -1101.5563720224 -275.389093005600 2.1518  0.49720
% 8.4080 4.2040  -8813.7973326984 -275.431166646825 2.1020  0.50000
%12.6120 4.2040 -29746.5849x98447 -275.431342578215 2.1020  0.50000
%

In Table~\ref{Tab:XXX}, we present the optimization of the XXX
at the B3LYP/X-XXXG** $\Gamma$-point approximation.
For comparison, we report the optimized lattice parameter of XXX
obtained with the {\sc Crystal03} code and a $X\times X\times X$ 
$k$-points integration grid.

%with the constraint $\alpha=\beta=\gamma=90^\circ$. 

%
% See:
% Merawa M, Noel Y, Civalleri B, Brown R, Dovesi R, Raman and infrared vibrational 
% frequencies and elastic properties of solid BaFCl calculated with various 
% Hamiltonians: an ab initio study, 
% J. Phys-Condens. Mat. 17, 535-548 (2005). 
%
% P. Baranek and J. Schamps, Influence of electronic correlation on structural, 
% dynamic and elastic properties of MgSi, 
% J. Phys. Chem. B 103, 2601-2606 (1999).
%
% A. Lichanot, E. Apr� and R. Dovesi, Quantum mechanical Hartree-Fock study of 
% the elastic properties of LiS and NaS, 
% Phys. Stat. Sol. (b) 177, 157-163 (1993). 
%
% A. Lichanot, M. Merawa and M. Caus�, Density Functional LCAO calculation of 
% periodic systems. Effect of an ``a posteriori'' correction of the Hartree-Fock 
% energy on physical properties of ionic sulfur compounds, 
% Chem. Phys. Letters 246, 263-268 (1995). 
%
% Mian M, Harrison NM, Saunders VR, Flavell WR, An ab initio Hartree-Fock investigation 
% of galena (PbS), 
% Chem. Phys. Letters 257, 627-632 (1996). 
%
\begin{table}[t]
  \centering
  \caption{\protect
    Progression of the cell constant $a_0$ and total energy $E$ 
    for (XXX)$_n$ using the periodic $\Gamma$-point
    B3LYP/X-XXXG level of theory and the {\tt TIGHT} thresholds.
%    $n$ is the number of MgO units in the (super)cell.
    Lattice parameters and energies are in \AA~and atomic unit respectively.
  }\label{Tab:XXX}
  \begin{tabular}{crcc}
  \toprule
  & $n$ & $a_0$ & $E/n$ \\
  \hline
    {\sc MondoSCF}\footnote[1]{$\Gamma$-point.} 
    &  &  & $-$  \\%
    &  &  & $-$  \\%
    &  &  & $-$  \\%
  \hline
    {\sc Crystal03}\footnote[2]{$8\times 8\times 8$ $k$-points.}
    &  &  & $-$  \\
  \botrule
  \end{tabular}
\end{table}


\section{Conclusions}\label{Sec:Conclusions}
In a previous paper, construction of the analytical 
exact Hartree-Fock exchange gradients with respect to 
atomic and cell parameters within the $\Gamma$-point 
approximation has been introduced. 
In this article, the formalism for the evaluation of the density 
functional gradients at the $\Gamma$-point approximation for 
Cartesian Gaussian-type basis functions was presented and 
implemented in the {\sc MondoSCF} package.

As an illustration, the analytical gradients and cell gradients have been used
in conjunction with the QUICCA algorithm to optimize few 3D periodic systems at
the DFT and hybrid-HF/DFT levels.

Convergence of bond lengths, bond angles,
dihedral angles and lattice parameters within the DFT and hybrid-HF/DFT $\Gamma$-point
super cell approach and under full relaxation with no symmetry
to the converged large cell $\Gamma$-point approximation have
been demonstrated for 3D systems to better than 3 digits.

Although the convergence of the Kohn-Sham density functional $\Gamma$-point total energy to
its ${\bf k}$-space integration counterpart with respect to cell size is relatively slow,
the convergence of the geometrical parameters (cell and atomic positions)
requires much smaller cells. Thus, we could show that a relative accuracy better
than 3 digits can be already achieved with cubic cells of about $600$\AA$^3$.\\


%%%%%%%%%%%%%%%%%%%%%%%%%%%%%%%%%%%%%%%%%%%%%%%%%%%%%%%%%%%%%%%%
%%%%%%%%%%%%%%%%%%%%%%%%%%%%%%%%%%%%%%%%%%%%%%%%%%%%%%%%%%%%%%%%
%Acknowledgements
\begin{acknowledgments}
 The authors would like to thank ... and ...
 for helpful comments.
 This work has been supported by the Swiss National Science Foundation, 
 the Swiss Office for Education and Science through the European 
 COST Action D14 and the US Department of Energy 
 under contract ???????????? and the ASCI project.  
 The Advanced Computing Laboratory of Los 
 Alamos National Laboratory is acknowledged.
\end{acknowledgments}  
%%%%%%%%%%%%%%%%%%%%%%%%%%%%%%%%%%%%%%%%%%%%%%%%%%%%%%%%%%%%%%%%
%%%%%%%%%%%%%%%%%%%%%%%%%%%%%%%%%%%%%%%%%%%%%%%%%%%%%%%%%%%%%%%%
\bibliography{mondo_new}
%%%%%%%%%%%%%%%%%%%%%%%%%%%%%%%%%%%%%%%%%%%%%%%%%%%%%%%%%%%%%%%%
%%%%%%%%%%%%%%%%%%%%%%%%%%%%%%%%%%%%%%%%%%%%%%%%%%%%%%%%%%%%%%%%
%%%%%%%%%%%%%%%%%%%%%%%%%%%%%%%%%%%%%%%%%%%%%%%%%%%%%%%%%%%%%%%%
%%%%%%%%%%%%%%%%%%%%%%%%%%%%%%%%%%%%%%%%%%%%%%%%%%%%%%%%%%%%%%%%
%%%%%%%%%%%%%%%%%%%%%%%%%%%%%%%%%%%%%%%%%%%%%%%%%%%%%%%%%%%%%%%%
\end{document}
%
% ****** End of file apssamp.tex ******
%
