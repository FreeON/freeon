%
%   This file is part of the APS files in the REVTeX 4 distribution.
%   Version 4.0 of REVTeX, August 2001
%
%   Copyright (c) 2001 The American Physical Society.
%
%   See the REVTeX 4 README file for restrictions and more information.
%
% TeX'ing this file requires that you have AMS-LaTeX 2.0 installed
% as well as the rest of the prerequisites for REVTeX 4.0
%
% See the REVTeX 4 README file
% It also requires running BibTeX. The commands are as follows:
%
%  1)  latex apssamp.tex
%  2)  bibtex apssamp
%  3)  latex apssamp.tex
%  4)  latex apssamp.tex
%
%\documentclass[prb,aps,nobibnotes,twocolumn,doublespace,twocolumngrid,superbib]{revtex4}
%%\documentclass[twocolumn,showpacs,preprintnumbers,amsmath,amssymb]{revtex4}
%\documentclass[preprint,showpacs,preprintnumbers,amsmath,amssymb]{revtex4}

% Some other (several out of many) possibilities
%\documentclass[preprint,aps]{revtex4}
%\documentclass[preprint,aps,draft]{revtex4}
%\documentclass[prb]{revtex4}% Physical Review B

%%\usepackage{amsmath}
%%\usepackage{amssymb}
%%\usepackage{graphicx}% Include figure files
%%\usepackage{dcolumn}% Align table columns on decimal point
%%\usepackage{bm}% bold math

%\documentclass[pre,aps,twocolumn,showpacs,twocolumngrid,superbib]{revtex4}
%\documentclass[prl,aps,twocolumn,showkeys,twocolumngrid,superbib]{revtex4}
%\documentclass[twocolumn,showkeys,showpacs,preprintnumbers,amsmath,amssymb]{revtex4}
\documentclass[prl,twocolumn,showpacs,twocolumngrid,superbib]{revtex4}
%\documentclass[showpacs,preprint,superbib]{revtex4}

\usepackage{graphicx}
\usepackage{amsfonts}
\usepackage{amsmath}
\usepackage{bm}
\usepackage{alltt}
\usepackage{fancyhdr}
\usepackage{dcolumn} 

\pagestyle{fancy}


\def\Tr{{\rm Tr}}
%\nofiles

\begin{document}

%\preprint{APS/123-QED}

\title{blablabla  energy gradients with respect to atomic positions and cell parameters
  within the blablabla $\Gamma$-point approximation}

\author{Val\'ery Weber}
\email{valery.weber@unifr.ch}
\affiliation{Department of Chemistry, University of Fribourg, 1700 Fribourg, Switzerland.}%
\author{Christopher J. Tymczak}%
\author{Matt Challacombe}%
\affiliation{Los Alamos National Laboratory, Theoretical Division, Los Alamos 87545, New Mexico, USA.}%

\date{\today}% It is always \today, today,
             %  but any date may be explicitly specified

\begin{abstract}

\end{abstract}

%\pacs{Valid PACS appear here}% PACS, the Physics and Astronomy
                             % Classification Scheme.
\keywords{Periodic boundary condition, exact Hartree-Fock exchange, atomic gradients, 
lattice gradients.}
                              %display desired
\maketitle

\section{Introduction}


\section{Numerical Examples}\label{Sec:NumExamples}
All developments were implemented in the MondoSCF~\cite{MondoSCF} suite of
linear scaling quantum chemistry programs. The code was compiled 
using the Portland Group F90 compiler {\tt pgf90} v5.1~\cite{pgf90} 
with the~{\tt -O1} options and with the GNU C compiler {\tt gcc} 
v3.2.2 using the~{\tt -O1} flag.
All calculations were carried out on a 1024-node (2048 processors)
dual P4 LinuxBIOS/BProc cluster connected with Myrinet 2000 running
Red-Hat Linux release~9 (Shrike)~\cite{RedHat90}.

The {\tt TIGHT} level of numerical accuracy has been used throughout this work.  
Thresholds that define the {\tt TIGHT} accuracy level include a matrix 
threshold $\tau=10^{-6}$, as well as other numerical thresholds 
detailed in Ref.~\cite{CTymczak04a}, which deliver at least 8 digits of 
relative accuracy in the total energy and 4 digits of absolute accuracy 
{\bf ?? WHY ABSOLUTE ACCURACY ??} in the forces.  

\begin{table}[t]
  \centering
  \caption{\protect
    Progression of the Hartree-Fock $\Gamma$-point cell constants, 
    bond lengths, bond angles, dihedral angles 
    and total energy for (urea)$_n$ using the periodic RHF-MIC/6-21G* level of theory 
    and the {\tt TIGHT} threshold. 
%    $n$ is the number of urea units in the (super)cell.
    Lengths, angles and energies are in \AA ngstr\"oms, degrees and atomic units respectively.
  }\label{Tab:Urea}
  \begin{tabular}{lcccc}
    \toprule
    &\multicolumn{3}{c}{\sc{MondoSCF}\footnote[1]{$\Gamma$-point.}}
    &\multicolumn{1}{c}{\sc{Crystal03}\footnote[2]{$2\times 2\times 2$ $k$-points.}} \\
    \hline
    $n$             & 2 & 16 & 54 & 2 \\
    $a_0$           & & & & 5.524 \\
    $c_0$           & & & & 4.642 \\
    $E/n$           & & & &$-$447.68322 \\
    &\multicolumn{4}{c}{Bond lengths} \\
    C$=$O           & & & & \\
    C$-$N           & & & & \\
    N$-$H$_1$       & & & & \\
    N$-$H$_2$       & & & & \\
    &\multicolumn{4}{c}{Bond angles} \\
    OCN             & & & & \\
    CNH$_1$         & & & & \\
    CNH$_2$         & & & & \\
    &\multicolumn{4}{c}{Dihedral angles} \\
    OCNH$_1$        & & & & \\
    OCNH$_2$        & & & & \\
    NCNH$_1$        & & & & \\
    \botrule
  \end{tabular}
\end{table}


\begin{table}[t]
  \centering
  \caption{\protect
    Progression of the Hartree-Fock $\Gamma$-point cell constant $a_0$
    and total energies $E$ for (MgO)$_n$ using the periodic 
    RHF-MIC/8-61G/8-51G level of theory and the {\tt TIGHT} threshold. 
%    $n$ is the number of MgO units in the (super)cell.
    Lengths and energies are in \AA~and atomic unit respectively.
  }\label{Tab:MgO}
  \begin{tabular}{lrccc}
  \toprule
  & $n$ & $a_0$ & $E/n$ & $f_O$ \\
  \hline
    {\sc MondoSCF}\footnote[1]{$\Gamma$-point.}
    &  4  & 4.365 & $-$274.616533 & 0.488 \\% 4.365-274.616533623575FullOpt 2.131
    & 32  & 4.192 & $-$274.664119 & 0.500 \\% 8.385-274.664118754687FullOpt 2.096
    & 108 & 4.192 & $-$274.664299 & 0.500 \\%12.575-274.664299043364FullOpt 2.096
    %&  4\footnote[2]{Full optimization.}        &4.365&$-$274.616533\\%4.365-274.616533623575FullOpt
    %&  4\footnote[3]{Lattice optimization only.}&4.370&$-$274.615477\\%4.370-274.615477868575LattOpt
    %& 32$^b$                                    &4.192&$-$274.664119\\%8.385-274.664118754687FullOpt
    %& 32$^c$                                    &4.192&$-$274.664119\\%8.384-274.664118815087LattOpt
  \hline
    {\sc Crystal03}\footnote[2]{$8\times 8\times 8$ $k$-points.} 
    &  1 & 4.192 & $-$274.664239 & $1/2$ \\%-274.664239121540
  \botrule
  \end{tabular}
\end{table}
%
%Crystal03 HF/8-61G/8-51G 8x8x8 Default
%4.201 -274.66420524315
%4.200 -274.66420889046
%4.199 -274.66421216488
%4.195 -274.66423302172
%4.194 -274.66423838462
%4.193 -274.66423912154
%4.192 -274.66423942700 <<<
%4.191 -274.66423922829
%4.190 -274.66423867131
%4.185 -274.66423067762
%
%Crystal03 PBExc/8-61G/8-51G 8x8x8 What was that?
%4.190 -275.28462703246
%4.200 -275.28469244414
%4.205 -275.28471839588
%4.210 -275.28473181717 
%4.211 -275.28473298902
%4.212 -275.28473352160 <<<
%4.213 -275.28473330887
%4.214 -275.28473191983
%4.215 -275.28473317201
%4.218 -275.28472993937
%4.220 -275.28472509192
%4.230 -275.28467343664 
%
%MgO PBExc Crystal03 User2.bas 8x8x8 default
%4.200 -275.28468980590
%4.205 -275.28471572672
%4.208 -275.28472235701
%4.210 -275.28472911839
%4.211 -275.28473028612
%4.212 -275.28473081270 <<<
%4.213 -275.28473059552
%4.215 -275.28473044697
%4.250 -275.28445240627
%
%MgO B3LYP Crystal03 User2.bas 8x8x8 default
%4.190 -275.43120191268
%4.200 -275.43122829482
%4.202 -275.43123248175
%4.203 -275.43123398833
%4.204 -275.43123514068 <<<
%4.205 -275.43123509457
%4.210 -275.43123002554
%4.215 -275.43121357505
%4.220 -275.43118801547
%4.250 -275.43082055853
%
%MONDO PBExc User2.bas Tight
%  a0         E            d       f
%4.3307 -1100.9726623928 2.1643  0.49976
%
%
%
%
%MONDO B3LYP User2.bas Tight
%
%
%
%
%
%ICE ICE ICE ICE ICE ICE ICE ICE ICE ICE ICE ICE ICE ICE ICE 
% Ferroelectric ordered ice in Cmc21 setting (from Crystal)
% HF/6-31G**/Tight
%               E            A        C
% 2x2x1 -1216.7333268588 9.253791 7.644517   %Pink
% 3x3x2 
% 4x4x2 
%
%
%
%
\section{Conclusions}\label{Sec:Conclusions}
blablabla
\\
%%%%%%%%%%%%%%%%%%%%%%%%%%%%%%%%%%%%%%%%%%%%%%%%%%%%%%%%%%%%%%%%
%%%%%%%%%%%%%%%%%%%%%%%%%%%%%%%%%%%%%%%%%%%%%%%%%%%%%%%%%%%%%%%%
%Acknowledgements
\begin{acknowledgments}
 V.W. would like to thank K. Doll for a preprint of his manuscript and informations
 about the basis sets used in his work.
 This work has been supported by the Swiss National Science Foundation, 
 the Swiss Office for Education and Science through the European 
 COST Action D14 and the US Department of Energy 
 under contract ???????????? and the ASCI project.  
 The Advanced Computing Laboratory of Los 
 Alamos National Laboratory is acknowledged.
 The authors would like to thank K. N\'emeth and C.J. Tymczak 
 for helpful comments.
\end{acknowledgments}  
%%%%%%%%%%%%%%%%%%%%%%%%%%%%%%%%%%%%%%%%%%%%%%%%%%%%%%%%%%%%%%%%
%%%%%%%%%%%%%%%%%%%%%%%%%%%%%%%%%%%%%%%%%%%%%%%%%%%%%%%%%%%%%%%%
\bibliography{mondo_new}
%%%%%%%%%%%%%%%%%%%%%%%%%%%%%%%%%%%%%%%%%%%%%%%%%%%%%%%%%%%%%%%%
%%%%%%%%%%%%%%%%%%%%%%%%%%%%%%%%%%%%%%%%%%%%%%%%%%%%%%%%%%%%%%%%
%%%%%%%%%%%%%%%%%%%%%%%%%%%%%%%%%%%%%%%%%%%%%%%%%%%%%%%%%%%%%%%%
%%%%%%%%%%%%%%%%%%%%%%%%%%%%%%%%%%%%%%%%%%%%%%%%%%%%%%%%%%%%%%%%
%%%%%%%%%%%%%%%%%%%%%%%%%%%%%%%%%%%%%%%%%%%%%%%%%%%%%%%%%%%%%%%%
\end{document}
%
% ****** End of file apssamp.tex ******
%
