\newcommand{\commentoutA}[1]{} \newcommand{\commentoutB}[1]{#1}
%\newcommand{\commentoutA}[1]{#1} \newcommand{\commentoutB}[1]{}
\newcommand{\tTT}{t_{TT}}
\newcommand{\tTB}{t_{TB}}

\renewcommand{\thefootnote}{\fnsymbol{footnote}}
\commentoutA{\documentclass[prl,aps,twocolumn,twocolumngrid,superbib]{revtex4}}
\commentoutB{\documentclass[11pt,prb,aps,nobibnotes,superbib,preprint]{revtex4}}
\commentoutB{\renewcommand{\baselinestretch}{1.9}}

\usepackage{graphicx}
\usepackage{amsfonts}
\usepackage{amsmath}
\usepackage{bm}
\usepackage{alltt}
\usepackage{dcolumn}
\usepackage{letterspace}

\makeatletter
\makeatother

\begin{document}
\title[Short Title]{
Linear scaling computation of the Fock matrix. VII. \\
Parallel computation of the Coulomb matrix}

\author{Chee Kwan Gan\footnotemark[1]}
\author{C. J. Tymczak\footnotemark[2]}
\author{Matt Challacombe\footnotemark[3]}

\affiliation{ Theoretical Division,\\ Los Alamos
              National Laboratory,\\ Los Alamos, New Mexico 87545}

\date{May 11, 2004}

\begin{abstract}
We present parallelization of a quantum chemistry tree code
[J. Chem. Phys. {\bf 106}, 5526 (1997)] for the linear scaling
computation of the Coulomb matrix. A measurement based partition
scheme called equal time partition, originally proposed by the present
authors [J. Chem. Phys. {\bf 118}, 9128 (2003)] to parallelize the
exchange-correlation matrix calculation, is used to load balance the
computation of the Coulomb matrix elements.  Equal time partition is a
measurement based algorithm for domain decomposition that exploits the
small variation of density between iterations to achieve load
balance. The efficiency of equal time partition is illustrated by
several tests from the finite and periodic systems.  It is found that
equal time partition is able to deliver 91 -- 98 \% efficiency with
128 processors in the most time consuming part of the
% MakeJ_system_1_iter2.dat 116.0836971975998
% MakeJ_system_2_iter3.dat 119.8557216102013
% MakeJ_system_7_iter3.dat 124.7111209621147
% MakeJ_system_3_iter3.dat 118.9564771302328
% MakeJ_system_8_iter3.dat 124.9240011224113
Coulomb matrix calculation.
The current parallel QCTC is able to deliver 63 -- 81\% overall
efficiency on 128 processors at a fine grained parallelism (number of
heavy atoms per processors is less than 1.6 and 1.4 for the finite and
periodic cases, respectively).
% ParaQCTC_system_1_iter2.dat 80.31859859152611 
% ParaQCTC_system_2_iter3.dat 85.53948967877915
% ParaQCTC_system_7_iter3.dat 98.89603622895031
% ParaQCTC_system_3_iter3.dat 104.0427554330148
% ParaQCTC_system_8_iter3.dat 100.0109171052371

% 168 111DHMX : 120
% 232 212BHMX : 168

\smallskip
\noindent{\bf Keywords}:
Self-consistent field theory, linear scaling methods, $N$-body problem,
Gaussian-orbital, hierarchical methods,
load balance, parallel computation
\end{abstract}
\maketitle

\footnotetext[0]{Preprint LA-UR-.}
\footnotetext[1]{ckgan@lanl.gov}
\footnotetext[3]{tymczak@lanl.gov}
\footnotetext[2]{mchalla@lanl.gov}

\section{Introduction}
\label{sec:intro}
Self-consistent field (SCF) theories such as the density functional
theory and hybrid Hartree-Fock/density functional theory are accurate
and computationally efficient. The traditional Gaussian-orbital
quantum chemistry codes that use the conventional methods\cite{ASzabo}
are usually restricted to small systems since the methods have steep
scaling (of order $O(N^{2-3})$) with respect to the system size $N$.
Recently significant progress has been made in the development of
$O(N)$ methods that overcome the bottlenecks of the conventional
methods, where the computation time scales linearly with the system
size $N$.  These methods include the computation of the Hartree--Fock
exchange matrix
% COchsenfeld98
\cite{ESchwegler96,ESchwegler97,ESchwegler98A,ESchwegler99,ESchwegler00},
the Coulomb
matrix~\cite{CWhite94B,CWhite96A,MChallacombe96,MChallacombe96B,MStrain96,MChallacombe97},
the exchange-correlation
matrix~\cite{Jorda95,RStratmann96,CGuerra98,MChallacombe00A}, and
density matrix alternatives to eigensolutions of the SCF
equations~\cite{XLi93,MDaw93,SQiu94,Hernandez96,ADaniels97,APalser98,MChallacombe99,ANiklasson02A,ANiklasson03}.

With the advent of parallel multi-processor computers that come with
decreasing cost but increasing power, there has been a great effort in
the community to parallelize the quantum chemistry
codes\cite{Harrison_94v45,Guerra_95,Sosa_98v19,Stephan_98v108,Furlani_00v128,Sosa_00v26,Yoshihiro_01v346,Baker_02v23,HTakashima02}.
Successful parallelization of $O(N)$ methods should be fruitful given
the fact that with parallel linear scaling methods, an $n$-fold
increase in processors should lead approximately an $n$-fold increase
in simulation capability. However, this holds only for scalable
algorithms.
%The existence of linear scaling methods has been
%argued from the concept of quantum locality, which
%says that the density matrix $n({\bf r},{\bf r}') $ goes to zero as
%$|{\bf r}-{\bf r }'| \rightarrow \infty$.

Two of the most computationally demanding parts in a density
functional application are the calculations of the
exchange-correlation matrix and the Coulomb matrix. The $O(N)$
exchange-correlation matrix calculation has been efficiently
parallelized through the concept of equal time (ET)
partition\cite{CGan03}.  In this work, the equal time partition will
be further employed in parallelizing the Coulomb matrix calculation.

The linear scaling computation of the Coulomb matrix has been achieved
via the quantum chemistry tree code
(QCTC)\cite{MChallacombe96,MChallacombe96B,MChallacombe97} and fast
multipole moment based
methods\cite{CWhite94B,CWhite96A,MStrain96,Perezjorda97}.  Both the
tree codes and the fast multipole methods are originally proposed to
handle the $N$-body problem\cite{JBarnes86,LGreengard87}.
Parallelization of these $N$-body algorithms has been rather active in
the computer science
communities\cite{warren:92_article,Grama94_article,Warren95b,Singh93,Singh_95v27,Grama_98v24,Gibbon_2002,Antonuccio-Delogu03}.
Even though both the $N$-body problem and the Coulomb matrix
calculation share a lot of similarities, especially in handling the
far field contribution, little work has been done on the parallel
Coulomb matrix calculation, except the master-slave
approach\cite{Furlani_00v128,Sosa_98v19,Sosa_00v26}.  However, it
should be pointed out that since the computation of $O(N)$ Coulomb
matrix (or most other $O(N)$ methods) is highly irregular for parallel
computation due to the way $O(N)$ methods exploit the quantum locality
of the system.  It is well known that master-slave approach faces
potential contention and load imbalance
problems\cite{BWilkinson99,GWilson95}, especially for a fine grained
parallelism (a small work load but a large number of processors).
This has indeed been observed\cite{Guerra_95}, so other alternatives
need to be sought.  We have noticed the smooth variation of the work
(e.g. due to the change in the electron density) between successive
cycles during the SCF calculation (so called temporal
locality\cite{JPilkington96}) and this may be used to load balance the
work load among the processors. A measurement based load balance
scheme called equal time partition\cite{CGan03} had been proposed to
parallelize the exchange-correlation matrix. Excellent speedups had
been obtained for several test systems.
%For example, we have obtained a speedup of
%113 out of 128 for a 110 molecule water cluster at a standard density
%and with the 3-21G basis set. 
Since the Coulomb matrix is dependent on the the electron density just
as the exchange-correlation matrix, we expect that equal time
partition should also deliver good speedups for the parallelization of
the Coulomb matrix.

The remainder of this paper is organized as follows: In
Sec.~\ref{ParaQCTC} we discuss our strategies to efficiently
parallelize the computation of the Coulomb matrix $J$. In
Sec.~\ref{sec:implementation} we describe a computational
implementation of the parallel QCTC. In Sec.~\ref{results} we discuss
the results of the speedup tests performed on a few representative
finite and periodic systems. In Sec.~\ref{conclusions} we summarize
the main conclusions of the paper.

\section{Parallelization of QCTC}
\label{ParaQCTC}
The quantum chemistry tree code (QCTC) for the $O(N)$ calculation of
the Coulomb matrix has been fully described in
Ref.\cite{MChallacombe97} and Ref.\cite{CJ2004}.  Here we only
highlight the essential aspects of the algorithm so that discussions
of parallelization of QCTC may be made.

The Coulomb matrix element in a finite (or gas phase) case is given by
\begin{equation}
J_{ab} = \int d{\bf r}d{\bf r}' \frac{\rho_{ab}({\bf r})
\rho({\bf r}')}{| {\bf r} - {\bf r}'|}
\label{eq:Jab}
\end{equation}
where the charge distribution\cite{LMcmurchie78} (or simply the
distribution) $\rho_{ab}({\bf r})$, is a product of the (Gaussian)
basis functions $\phi_a({\bf r})$ and $\phi_b({\bf r}) $.  The
electron density of the system is denoted by $\rho({\bf r})$.  In
QCTC, a hierarchical multipole representation of the electron density
is stored in an advanced data structure called a $k$-d
tree\cite{Bentley79,Bentley80,Gaede98}.  A compact representation of
the density in terms of Hermite-Gaussian
(HG)\cite{Ahmadi95,MChallacombe97,MChallacombe00A} basis has been
used.  With the help of the density tree, QCTC represses the matrix
element in Eq.~(\ref{eq:Jab}) as a sum of two terms\cite{CJ2004}
\begin{eqnarray}
J_{ab} &=& \sum_{Q\in \rm FF} \sum_{l} \sum_{m} O^l_m[\rho_{ab}]
\sum_{l'} \sum_{m'} M^{l+l'}_{m+m'} O^{l'}_{m'}[\rho_Q]
\nonumber \\
&+& \sum_{q\in \rm NF} \int d {\bf r} \int d {\bf r'} \rho_{ab} ({\bf
r}) \left|{\bf r}-{\bf r'} \right|^{-1}
\rho_q({\bf r})
\end{eqnarray}
where $M^l_m$ is the irregular solid harmonics interaction tensor,
$O^l_m[f]=\int d{\bf r} O^l_m({\bf r}) f({\bf r})$ is a moment of the
regular solid harmonics, $Q$ runs over the all nodes in the density
tree as determined by PAC (penetration admissibility criterion) and
MAC (multipole admissibility criterion)\cite{MChallacombe97} and $q$
runs on the left over near-field primitive distributions in the
density.

Essential to QCTC is the construction of the electron density tree.
Once the density tree is built, the calculation of the Coulomb matrix
elements proceeds by transversing the tree and checking the PAC and
MAC. When both PAC and MAC are met, the far-field contribution is
calculated.  The near-field contribution, however, is calculated
analytically.  Since most of time spent in QCTC is mainly in
traversing of the density tree for the matrix element calculations, we
have emphasized the efficient parallelization of the tree traversal of
the code with the equal time partition, to be described fully
below. However, the Amdahl's law dictates that the performance of a
massively parallel program is ultimately determined by its
unparallelized parts, hence we also consider parallelization of other
serial parts of the code. This includes the building of the density
tree on each processor.  
%The subsequent subsections will deal with parallelization 
%of various parts in QCTC.

\subsection{Load balancing matrix element calculations}
\label{ETPartition}
For a serial program, the time to traverse the density tree, denoted
by $\tTT$, for the evaluation of the matrix elements of the $J$
matrix, constitutes the largest portion of the total time spent in
QCTC.  Hence it is important to load balance the tree traversal part
as efficient as possible in the parallel QCTC.  Since excellent load
balance has been achieved in the parallel calculation of the
exchange-correlation matrix using the concept of equal time
partition\cite{CGan03}, it is only natural to extend the capability of
equal time partition to parallel QCTC.

Equal time partition works by recording the time spent in a
computational domain during one SCF cycle. At the end of the
calculation, the time spent in each domain is used to predict a new
domain decomposition for the next SCF cycle where each decomposed
domain carries ideally the same amount of workload. Since the work
pattern usually varies slowly between SCF cycles (the so called
temporal locality\cite{JPilkington96}), we expect that the predicted
domain decomposition should deliver a good load balance in the next
SCF cycle.

To facilitate ET partition, it is important to decide what work load
information to keep without incurring too much overheads which include
the timing process itself.  For this purpose, we record the time to
traverse the tree for each possible distribution. The time and
position of each distribution are stored in an array to facilitate
partitioning of the 3-D bounding box which encloses all
distributions. Equal time partition is performed on the bounding box
in a manner similar to that outlined in Ref.\cite{CGan03} to achieve
equal time (hence the name ``equal time'' partition) or cost in all
subboxes.  It is important to point out the main difference between
our scheme and other classical $N$-body
codes\cite{warren:92_article,Singh93} is that we use an exact load
timing information rather than counting the number of interactions as
in the orthogonal recursive bisection (ORB)\cite{warren:92_article}
and Costzone methods\cite{Singh93}.

A working hypothesis of the ET partition applied to QCTC is that the
sum of distribution time for each ET subbox is almost constant
irrespective of the sectioning. For very fine grained parallelism, the
shifting of a bisecting plane may induced a small change in the total
distribution time in a subbox.  This non-conservation of work load may
increase the load imbalance, which we have experienced in the
exchange-correlation matrix parallelization\cite{CGan03}.

In the first cycle, there is no ET partition from a previous cycle. In
such case we have used a reasonable heuristic where each processor
handles an almost equal number of distributions (the total number of
distributions may not be divisible by the number of processors). Since
this serves to start an ET partition, we do not consider its
performance in this work.

\subsection{Parallel density tree build} 
\label{sec:parallelTB}
In the serial QCTC, the time to build the electron density tree
constitutes about 2\% or less of the total time spent in
QCTC. However, as we increase number of processors (say, 128), the
time to build a density tree, denoted by $\tTB$, will be large
compared to $\tTT$, the time to traverse the density tree discussed in
subsection~\ref{ETPartition}.  Hence it is also important to
parallelize the building of the density tree.

In principle, an efficient parallelization of tree build should make
use of the fact that each processor is handling only a part of the
total distributions.  Hence depending on the collection of
distributions on each processor, a locally essential tree may be
constructed which is just sufficient for the tree traversals of all
the distributions on a processor\cite{CGan03}.  If a locally essential
tree is available, then efficient tree traversals of the trees are
possible.  However, without an extensive programming task,
it may be difficult to predetermine which part of
the entire density tree is needed for the construction of the locally
essential tree. As a first attempt, we have chosen a simple approach
where the entire density will be built on each processor.

For simplicity of programming we have assumed that the number of
processors is $2^k$. We observe that there are $2^k$ subtrees at the
$k$th tier in the entire density tree in the serial case, our current
implementation adopts a simple scheme where each processor is
responsible to build one of $k$-tiered subtrees in the total density
tree. When all the processors have built the $k$-tiered subtrees, an
all-to-all exchange is carried out where all processors get the rest
of the $k$-tiered subtrees so that a final ``merging'' up of the
subtrees\cite{MChallacombe97,CJ2004} is performed to obtain an entire
density tree.

The inefficiency of the current implementation of the parallel density
tree build in the limit of a large number of processors is expected.
The all-to-all exchange of data between processors is expensive,
which does not scale with the number of processors. Also, after
collecting the subtrees from all other processors, a processor has to
``merge'' more subtrees as the number of processors is increased. This
will inevitably introduced more overheads as we use more processors.
Even if one can overcome the all-to-all exchange problem, one still
faces a problem where it may be rather wasteful in the use of memory
to store the entire density tree for each processor.  However, while
the present parallel density tree build may be replaced by a more
sophisticated scheme, the current implementation delivers very good
speedups even at the 128-processor level.

As a side note, our first implementation of ET parallel QCTC tries to
avoid the problem mentioned above by partitioning the density of the entire
systems into disjoint local densities. Each processor then builds a
local tree based on the local density. However, since a distribution
in a processor does not ``see'' other local density trees, every
processor has to loop through all distributions and the partial
Coulomb matrices have to be row-wise summed (through the use a FastMat
data structure, see \cite{CGan03}) at the end of the calculation. This
turns out to be practical only for below 64-processor level. However,
the speedup does not increase with more processors because the
total intrinsic cost (i.e. the amount of useful work) of QCTC {\it increases}
rather rapidly with the number of processors. The rapid increase in
the intrinsic cost has its root in the non-optimal use of the
multipole expansion approximation since too many local trees have been
created. For the periodic systems, where one has to visit the density
tree many more times than for the finite case, the speedup
stagnants once we pass a certain number of processors. Since
this version of the parallel QCTC does not scale with the number of
processors, we do not consider it further in this work.

\section{Implementation}
\label{sec:implementation}
We have implemented a parallel QCTC algorithm in {\sc
MondoSCF}~\cite{MondoSCF}, a suite of programs for linear scaling
electronic structure theory and {\it ab initio}\/ molecular dynamics.
{\sc MondoSCF} has been written in Fortran 90/95 with the
message-passing library MPI~\cite{mpi}.  Timings are performed using
the {\tt MPI\_WTIME} function.

\section{Results}
\label{results}
We have performed scaling tests on both finite and periodic
systems. For the finite systems, we have chosen taxol
(C$_{47}$H$_{51}$NO$_{14}$) and 2 water clusters as test cases. For
the periodic systems, we have chosen pentaerythritol tetranitrate
(PETN)\cite{Gan_2004} and $\delta$- phase of
octahydro-1,3,5,7-tetranitro-1,3,5,7-tetrazocine
(HMX)\cite{Lewis_00v104} as representative test cases. These systems
are chosen because they are highly inhomogeneous, three-dimensional
(and two are periodic) systems posing a challenge to parallel
QCTC. All runs were performed on a cluster of 256 4 CPU HP/Compaq
Alphaserver ES45s with the Quadrics QsNet High Speed Interconnect.

For the purpose of performing the scaling tests, we start the
calculation with the STO-3G basis set and a low accuracy, and switch
to the final basis set and accuracy using a mixed integral approach,
and run for three SCF cycles. The density matrix ${\bf P}$ is saved to
disk and scaling tests of parallel QCTC are performed. This procedure
may not be necessary. However, we are confident that the timings are
representatives of a routine calculation.

The result of the taxol scaling tests is shown in
Fig.~\ref{fig:taxol}.  The calculations are performed with the 6-31G
and 6-31G** basis sets, and at a ``good'' level of accuracy.  The
results of two different speedups are presented.  The first speedup,
called the equal time speedup, measures the efficiency of the equal
time partition for the Coulomb matrix element calculation by
traversing the density tree (see Section~\ref{ETPartition}) and is
defined by
\begin{equation}
S_{ET} = \frac{2\tTT^{(n)}}{\tTT^{(2)}}
\end{equation}
where $\tTT^{(n)}$ is the time to evaluate the matrix elements by
traversing the density tree with $n$ processors. Notice that the
speedups are relative to a 2-processor calculation.  The second
speedup, called the QCTC speedup, measures the overall efficiency of
parallel QCTC.  From Fig.~\ref{fig:taxol} it is observed that the
equal time speedup is excellent up to 64 processors. The efficiency at
the 64-processor level (where the number of heavy per processor is
less than 1) is at least 94\% efficient.
% 60.46298
The overall parallel QCTC speedup is very good up to 32 processors but
degrades slightly at the 64-processor level.  The overall parallel
QCTC efficiencies are
% 49.64853 and 53.10910520061477
77.6\% and 83.0\% for the 6-31G and 6-31G** basis sets, respectively.

The loss of efficiency is due to the fact that the time for parallel
density tree build does not decrease at the same rate (i.e.  divided
by the number of processors) as the tree traversal part.  In
Fig.~\ref{fig:time_ratio} we present the $\tTB/\tTT$ ratio as a
function of the number of processors for calculations on different
systems for comparisons.  We note that if the time for parallel tree
build $\tTB$ were to decrease at the same rate (i.e. divided by the
number of processors) as the time for tree traversal $\tTT$ as we
increase the number of processors, then the $\tTB/\tTT$ ratio would
remain nearly constant.  However, Fig.~\ref{fig:time_ratio} shows that
the $\tTB/\tTT$ ratio increases steadily as the number of processors
is increased in all cases, a fact that has been anticipated from the
discussion in Section~\ref{sec:parallelTB}.  Since the slope for the
6-31G** case is smaller than that for the 6-31G case, this explains
the slight increase in the overall parallel QCTC performance of the
6-31G** case over the 6-31G case, as shown in Fig.~\ref{fig:taxol}.

We note that our results of parallel QCTC
% 7.7995003 and 7.772217865
speedup of 7.80 (with the 6-31G and 6-31G** basis sets) with 8
processors compares favorably with the speedup of about 6.0 of Sosa
{\it et al.}\cite{Sosa_00v26}, which is for an entire single-point
energy calculation with HF/3-21G.

%\commentoutA{
\begin{figure}[t]
\resizebox*{3.5in}{!}{\includegraphics[clip]{taxol-6-31g-6-31gss.eps}}
\caption{ 
Scaling of parallel QCTC on taxol (C$_{47}$H$_{51}$NO$_{14}$)
BLYP/6-31G and BLYP/6-31G**. Speedups are relative to a 2-processor
calculation.  The label ET and QCTC denotes equal time and overall
parallel QCTC speedups, respectively, as explained in the main text.
}
\label{fig:taxol}
\end{figure}
%}

%\commentoutA{
\begin{figure}[t]
\resizebox*{3.5in}{!}{\includegraphics[clip]{NProc_TTBTTTRatio.eps}}
\caption{ 
The ratio of the time to build the density tree ($\tTB$) to the time
to traverse the density tree ($\tTT$), as a function of the number of
processors.  }
\label{fig:time_ratio}
\end{figure}
%}

%\commentoutA{
\begin{figure}[t]
\resizebox*{3.5in}{!}{\includegraphics[clip]{sys1Iter2_sys2Iter3.eps}}
\caption{ 
Scaling of parallel QCTC on 2 cluster of water molecules with
BLYP/6-31G**. Speedups are relative to a 2-processor calculation.  }
\label{110And200WaterOnGood}
\end{figure}
%}

%\commentoutA{
\begin{figure}[t]
\resizebox*{3.5in}{!}{\includegraphics[clip]{sys1Iter2_sys7Iter3.eps}}
\caption{ 
Scaling of parallel QCTC on 110-molecule water cluster with
BLYP/6-31G** on ``good'' and ``tight'' accuracies.  Speedups are
relative to a 2-processor calculation.  }
\label{fig:110WaterOnGoodAndTight}
\end{figure}
%}

%\commentoutA{
\begin{figure}[t]
\resizebox*{3.5in}{!}{\includegraphics[clip]{NProc_TRatio_110H2O_GoodAndTight.eps}}
\caption{ 
The ratio of the time to build the density tree ($\tTB$) to the the
time to traverse the density tree ($\tTT$), as a function of the
number of processors, for 110-molecule water cluster (BLYP/6-31G**)
calculations with a ``good'' and ``tight'' accuracy.  }
\label{fig:NProc_TRatio_110H2O_GoodAndTight}
\end{figure}
%} 

Similar scaling tests have been performed on a 110-molecule and
200-molecule water clusters with BLYP/6-31G** at a ``good'' level of
accuracy.  The result of the scaling tests is shown in
Fig.~\ref{110And200WaterOnGood}. It is found that the equal time
speedups are rather good for both cases. The overall parallel QCTC
speedups are
% 80.31859859152611 and 85.53948967877915
80.3 and 85.5 for the 128-processor calculations for 110-molecule and
200-molecule water clusters, respectively.
%62.7\% and 66.8\% for the 110-molecule and 200-molecule 
The decrease in the parallel QCTC efficiency is again due to the
% 28.80799713967989 and 26.56818727311716
high $\tTB/\tTT$ ratio at the 128-processor level, which are 28.8\%
and 26.6\% for the 110-molecule and 200-molecule water clusters,
respectively (see Fig.~\ref{fig:time_ratio}).

To investigate the performance of parallel QCTC at a higher accuracy
level, we have performed the scaling tests on a 110-molecule water
cluster but with a ``tight'' accuracy. The results for both ``good''
and ``tight'' accuracies are presented in
Fig.~\ref{fig:110WaterOnGoodAndTight} for comparison.  It is seen that
the equal time speedup is better for the ``tight'' case than for the
``good'' case, which is anticipated since increasing the accuracy
level increase the granularity, which leads to a better performance in
equal time partition\cite{CGan03}.  The overall parallel QCTC has
increased its efficiency from ``good'' to ``tight'', which is mainly
due to the decrease in the $\tTB/\tTT$ ratio, as shown in
Fig.~\ref{fig:NProc_TRatio_110H2O_GoodAndTight}.


%\commentoutA{
\begin{figure}[t]
\resizebox*{3.5in}{!}{\includegraphics[clip]{Sys3Iter3_Sys8Iter3.eps}}
\caption{ 
Scaling of parallel QCTC on $\beta$-HMX and PETN with PBE/6-31G**.
Speedups are relative to a 2-processor calculation.  }
\label{fig:111DHMX212PETN}
\end{figure}
%}

Finally, for the periodic systems Fig.~\ref{fig:111DHMX212PETN} shows
that the overall parallel QCTC with PBE/6-31G** at the ``good'' level
of accuracy is excellent. At the 128-processor level, the $1\times
1\times 1$ $\delta$-HMX (168 atoms per simulation cell) delivers 104.0
fold speedup, while the $2\times 1 \times 2$ PETN (232 atoms per
simulation cell) delivers 100.0 fold speedup. These performances are
better compared to the 110-molecule (a speedup of 80.3) or
200-molecule (a speedup of 85.5) water cluster calculations. This is
due to the smaller $\tTB/\tTT$ ratio (see Fig.~\ref{fig:time_ratio})
for the periodic cases compared to the finite cases, which mainly
results from the increase in the time spent in the tree traversal part
(e.g. 87.7 and 64.0 secs for the $2\times 1 \times 2$ PETN and 200-molecule
water clusters, respectively).

\section{Conclusions}
\label{conclusions}
We have proposed an efficient method of parallelizing the calculation
of the Coulomb matrix. The concept of equal time has been proven to be
fruitful for load balancing the most time consuming part of QCTC,
which is the traversal of the density tree for the matrix element
calculation. 
Equal time exploits the temporal locality between SCF iterations 
to overcome strong spatial irregularities. It is expected that equal
time should exploit this property between geometry steps in an
optimization or molecular dynamics run.
The efficiency of the equal time partition ranges from 91
-- 98 \% for all the test case presented in this work at a
128-processor level. The overall QCTC speedup, however, ranges from 63
-- 81 \% overall efficiency on a 128-processor level. The slight
decrease in efficiency is mainly due to the parallel tree build
process.  While the current simple implementation of the parallel tree
build should eventually replaced by a more sophisticated data-parallel
version, the current implement implementation has enable us to run
routine calculations to address a wide range of interesting problems.

\begin{acknowledgments}
This work has been carried out under the auspices of the
U.S. Department of Energy under Contract No.~W-7405-ENG-36 and the
ASCI project.  Most work was performed on the computing resources at
the Advanced Computing Laboratory of Los Alamos National Laboratory.
\end{acknowledgments}

\bibliographystyle{apsrmp} \bibliography{mondo}

\end{document}
