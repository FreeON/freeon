\documentstyle[10pt,letterpaper]{letter}
\begin{document}
\pagestyle{empty}

\signature{Chee Kwan Gan}

\address{
Dr.~Chee Kwan Gan \\
(ckgan@lanl.gov) \\
Theoretical Division \\
Group T-12, MS B268 \\
Los Alamos National Laboratory \\
Los Alamos, NM 87545 }

\date{June 3, 2004}

\begin{letter}{
Professor Donald H. Levy, Editor\\
Journal of Chemical Physics\\
Department of Chemistry\\
5735 S. Ellis Ave.\\
The University of Chicago\\
Chicago, IL 60637}

\opening{Dear Professor Levy,}

Please find a manuscript {\em Linear Scaling Computation of the Fock
Matrix.~IX.~Parallel Computation of the Coulomb Matrix} by Gan,
Tymczak, and Challacombe, which I am submitting for publication in the
Journal of Chemical Physics (J. Chem. Phys.). This article follows up two
previous J. Chem. Phys. publications, one describing a linear scaling method for
Coulomb matrix computation [J. Chem. Phys. {\bf 106}, 5526 (1997)] and
another one which describes how to efficiently parallelize the calculation
of the exchange-correlation matrix [J. Chem. Phys. {\bf 118}, 9128
(2003)].

This paper establishes a
strong connection between linear scaling methods and methods that are
efficient in parallel. In particular, this contribution achieves the following:


\begin{itemize}
\item  It goes well beyond the typical 8-processor mark
       for parallel quantum chemistry.  Together with methods for 
       exchange-correlation [J. Chem. Phys. {\bf 118} 9128
       (2003)], this work establishes linear scaling Density
       Functional  methods that scale up to 128 processors with fine 
       grained parallelism. 

\item It bridges the gap between the quantum Coulomb problem and 
      well established methods that achieve massive parallelism for the
      classical $N$-body problem.

\item  While not incorporating all the techniques used in parallelization
       of the classical N-body problem, it does overcome the most
       serious challenge to the quantum Coulomb problem, involving 
       irregularity in the domain decomposition. 
  
\item  The success of the Equal Time partition method, applied now to 
       both the Coulomb and exchange-correlation problem provides
       impetus for its continued development in electronic structure
       theory.  Also, because it is a general framework for the domain
       decomposition of irregular problems, we believe it will find
       application in other areas of computational chemistry and
       beyond. 

\end{itemize}

The Journal of Chemical Physics has a rich history of publishing work
on numerical methods for Electronic Structure Theory.  More and more,
the success of these numerical methods are determined by their ability
to map efficiently onto parallel platforms.  This is clearly reflected
in a number of recent J. Chem. Phys. articles on parallel methods, notably {\em
Parallel filter diagonalization: A novel method to resolve quantum
states in dense spectral regions} and {\em Multigrid methods for
classical molecular dynamics simulations of biomolecules}.

With this in mind, I hope you will consider this submission for
publication in the Journal of Chemical Physics.  Suggestions for
reviewers with expertise in both parallel methods and Quantum
Chemistry include: 

\begin{itemize}
\item Robert J. Harrison at ORNL (harrisonrj@ornl.gov)
\item Michael E. Colvin at LLNL (colvin2@llnl.gov)
\item C. L. Janssen at SNL (cljanss@ca.sandia.gov)
\item M. W. Schmidt at Iowa State University (mike@si.fi.ameslab.gov)
\item Mark S. Gordon at Ames (mark@si.fi.ameslab.gov)
\end{itemize}


\closing{With best regards,}
\end{letter}
\end{document}
