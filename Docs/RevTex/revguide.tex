%% ****** Start of file revguide.tex ****** %
%%
%%   This file is part of the APS files in the REVTeX 4 distribution.
%%   Version 4.0 beta 2 of REVTeX, July xx, 1999.
%%
%%   Based on the online documentation files of REVTeX 3.1,
%%   on notes by David Carlisle,
%%   as modified by Arthur Ogawa.
%%
%%   Copyright (c) 1999 The American Physical Society.
%%
%%   See the REVTeX 4 README file for restrictions and more information.
%%
\documentclass[%
prl%
%,preprint%
,twocolumngrid%
,secnumarabic%
,amssymb]{revtex4}
\usepackage{acrofont}%NOTE: Comment out this line for the release version!
\usepackage{amsmath}%
\usepackage{bm}%

\usepackage{hyperref}%
%\nofiles

\begin{document}

\title{\revtex~4 Author's Guide}%

\author{American Physical Society}%
\thanks{%
  \revtex~3.1 portions by APS;
  V4 notes by David Carlisle (mailto:david@carlisle.demon.co.uk), March 31, 1999;
  V4 guide by Arthur Ogawa (mailto:ogawa@teleport.com)
}%
\affiliation{Ridge, Woodbury, Washington, DC}
\date{13 September 1999}%
\maketitle
\tableofcontents

\section{Introduction}

This is the author's guide to \revtex,
a system for preparing journal submissions in both print and electronic form,
which is implemented as a document class for the \LaTeX\ document preparation system.
An electronic document created in \revtex\ can be 
typeset in formats suitable for journal submission
or for circulation by the author as a manuscript or reprint,
but most importantly,
it can be used for direct submission as an electronic manuscript, or \emph{compuscript}.

\subsection{Choosing \revtex}
You will want to use \revtex\ to prepare
%If you are preparing 
a paper for submission to an academic journal,
%the following factors should incline you toward using the \revtex\ system.
if:
\begin{itemize}
\item
The journal or its society is a participant in the \revtex\ project.

\item
The journal has a compuscript submission program that is consistent with \revtex.

\item
Your paper makes significant use of mathematical notation or is highly technical in nature.

\item
You are familiar with and use the \TeX\ typesetting system,
or the \LaTeX\ document perparation system for \TeX.

\item
Your document's intended use extends to electronic publishing.

\item
Your document is destined to be translated to XML or another descriptive markup system.

\item
%In writing your paper,
You wish to get the most value from your time and effort as an author
by focusing on the content and structure of your paper
without undue concern for
%the concretes of its formatted presentation.
format details such as margins, fonts, and so on.

\item
You wish to typeset your document in a number of different formats depending on the requirements of the recipient.

\item
You wish to get the most value from your computer system in using it as a platform for document preparation.

\end{itemize}

Note that, although \LaTeX\ is ultimately a required part of the \revtex\ system, 
you do not need to be an expert user of \LaTeX\ in order to be an effective user of \revtex.

If you adopt \revtex, you should expect to benefit in the following ways:
%
\begin{itemize}
\item
\revtex\ provides all the markup elements needed for the preparation of your manuscript,
so you will not need to develop special tags.

\item
\revtex\ markup is designed to be acceptable for manuscript submission,
so you will not need to be concerned about proper format for
editorial offices (double spacing, margin requirements, etc.).

\item
\revtex\ macros accommodate many presubmission distribution needs:
you can, for example, assign preprint numbers to your manuscripts
or
easily change to single-spaced copy to save paper before submission to
editorial offices.

\item
Since \revtex\ macros are recognized by numerous physics organizations as
a \TeX\ standard for manuscript preparation,
you can enjoy the benefits of electronic submission programs.

\item
\revtex\ compuscript files can be used by a variety of publishers to create author proofs,
giving you less proofreading,
accelerated production schedules, or reduced cost-per-page.
\end{itemize}

\subsection{Historical}

The \revtex\ system for \LaTeX,
so named for the \emph{Physical Review} journals,
began its development in 1986,
was first released in 1988,
revised to version 2 in 1990,
and to version 3.1 in 1996.
In its earliest incarnations, it was both an authoring tool and a production tool
and was based on \LaTeX2.09.

These earlier versions of \revtex\ were restrictive of
what authors were allowed to do
and
were incompatible with packages that authors wanted to use.
\revtex~3 did not keep pace with the
advances of the \LaTeX\ community and thus became
inconvenient to work with.

\subsection{Design Principles of \revtex~4}

\revtex~4 is designed to
bring \revtex\ up to date and make
it a more valuable tool for the production process of the American Physical Society
and for authors who
circulate their work on their own.
This version of \revtex\ is a complete rewrite,
with the following set of design goals:

\begin{itemize}
\item
Make \revtex\ fully compatible with \LaTeXe; it is now
a \LaTeXe\ document class, similar in function to the standard \classname{article} class.

\item
Relax the restrictions in \revtex\ that had only
been necessary for 
typesetting journal camera-ready copy.

\item
Rely on standard \LaTeXe\ packages for common tasks, e.g,
\classname{graphicx},
\classname{color},
\classname{multicol},
\classname{hyperref},
and
\classname{longtable}.

\item
Add macros to support translation to SGML.

\item
Improve frontmatter macros for tagging author names and affiliations.

\item
Improve back matter macros for tagging references;
actively promote the use of Bib\TeX.

\item
Provide a closer approximation of the pages of \emph{Physical Review} and other journals
so authors can use \revtex\ to check their adherence to length requirements.

\item
Incorporate new features, such as hypertext, to make \revtex\ a desirable e-print format.
\end{itemize}

The improved tagging will to aid the peer-review and
publication process from the moment a \revtex\ paper is submitted.

\subsection{Status of \revtex~4}
\revtex~4 is in beta testing. Papers that use \revtex~4 are not yet
eligible for the compuscript program (described in \revtex\ Input
Guide for \revtex~3.1).
The Americal Physical Society is making this beta
release to get feedback on the features and to track down bugs.
Please send any comments and bug reports concerning \revtex~4 to
\url{mailto:revtex4@aps.org}.

\subsection{Documentation Roadmap}

This manual applies to version 4 of the \revtex\ document class for \LaTeX.
In this manual:
\begin{itemize}
\item
We give a \hyperlink{Tsec:quick}{quickstart} guide for experienced users in Section~\ref{sec:quick}.

\item
We describe \revtex's system requirements and explain 
\hyperlink{Tsec:getting}{how to get} and use the \revtex\ tools and documentation
in Section~\ref{sec:getting}.

\item
We give \hyperlink{Tsec:instruct}{instructions on preparing} a \revtex\ compuscript (i.e., an instance of
the \classname{revtex4} document class) in Section~\ref{sec:instruct}.

\item
We provide a reference manual to the \hyperlink{Tsec:markup}{\revtex\ markup system} and
illustrate how it applies to scientific papers in Sections~\ref{sec:class}--\ref{sec:body}.

\item
We describe how to \hyperlink{Tsec:usepackage}{add other \LaTeX\ packages} to the \revtex\ system,
so you can exploit their capabilities in your document in Section~\ref{sec:usepackage}.

\item
We give pointers for
\hyperlink{Tsec:debug}{troubleshooting} in Section~\ref{sec:debug}.

\item
We describe the requirements of the \hyperlink{Tsec:compuscript}{compuscript program} in Section~\ref{sec:compuscript}.

\item
We detail your \hyperlink{Tsec:resources}{resources for help} in Section~\ref{sec:resources}.

\item
We list books on the \hyperlink{Tsec:TeXbooks}{use of \TeX\ and \LaTeX} in the Bibliography.

\end{itemize}

The appendices to this manual contain reference information and information of
interest to a restricted audience:
%
\begin{itemize}
\item
In Appendix~\ref{sec:diff31},
we summarize the \hyperlink{Tsec:diff31}{differences in the markup}
between \revtex~4 and the previous release, \revtex~3.1.

\item
In Appendix~\ref{sec:conv31},
we describe how to convert a \revtex~3.1 document into a \revtex~4 document.

\item
In Appendix~\ref{sec:diffart},
we summarize the \hyperlink{Tsec:diffart}{differences in the markup}
between \revtex~4 and the standard \LaTeX\ article class.

\item
In Appendix~\ref{sec:chars}, we list the 
\hyperlink{Tsec:chars}{special characters} obtainable through \revtex.

\item
In Appendix~\ref{sec:markup}, we summarize the \hyperlink{Tsec:markup}{\revtex\ markup} needed for
a typical document.

\end{itemize}


\section{Quick Start}\label{sec:quick}

\hypertarget{Tsec:quick}{This section} is for readers impatient to create their first \revtex~4 document.
In order to jump right in, you must:
\begin{itemize}
\item
Be familiar with \LaTeX\ and, ideally, BiB\TeX.

\item
Have available to you a working \TeX\ installation,
complete with \LaTeX, BiB\TeX, makeindex, previewer, printer, etc.

\item
Either have \revtex\ installed, possess the distribution media,
or have access to the Internet.

\end{itemize}

Furthermore, to use the sophisticated length-checking capabilities of \revtex,
you must either possess the requisite fonts,
or you must install whatever fonts are required.

To quickstart \revtex, follow these steps:
\begin{enumerate}
\item
Pick up the \revtex\ document class for \LaTeX\ and associated files:
see \url{http://publish.aps.org/revtex4/}.

\item
Install the necessary components
by putting all of the \file{.cls}, \file{.sty}, and \file{.rtx} files
into a location within your filesystem where they will be available to \LaTeX.

Note: under the TDS, they would be placed into \url{textmf/tex/latex/revtex}.

\item
Put all \file{.bst} files where they can be found by BiB\TeX; under the TDS, this 
would be \url{textmf/bibtex/bst/revtex}.

\item
Make note of the \file{.dvi} and \file{.pdf} files in the distribution; they
are the \revtex\ online documentation.
Please make yourself familiar with their contents.

If you wish to move these files into your documentation tree under the TDS, put them in
\url{texmf/doc/latex/revtex}.

\item
The file \file{template.aps} is a boilerplate for creating a \revtex\ document.
Under the TDS, it belongs in \url{texmf/doc/latex/revtex}.

Clone this file under a new name, say \file{mypaper.tex}, in your personal area of your filesystem,
and typeset that new file.

\item
Alter the document to suit your purposes,
using the sample markup and embedded comments as a guide.

\item
You are on your way!

\end{enumerate}

\section{Getting Started With \revtex}\label{sec:getting}

\subsection{Site Preparation}\label{sec:prep}

To use \revtex, you must have available to you a working \TeX\ installation,
complete with \LaTeX, BiB\TeX, makeindex, text editor, previewer, printer, and any ancillary
applications needed to operate it.
Most new computers sold today are capable of serving your authoring needs.

Commercial and shareware \TeX\ distributions for most computers can
be found through the \TeX\ Users Group (\url{http://www.tug.org}),
in particular,
the very powerful and convenient \TeX\ Live CD-ROM (\url{http://www.tug.org/texlive})
has runnable binaries for many UNIX flavors, Windows 9x and Windows 2000, and MacOS.
All these distributions contain the \LaTeX\ document preparation system upon which
\revtex\ is based.

Follow the installation instructions for your \TeX\ software included with the distribution.
Confirm your \TeX\ installation by typesetting, previewing, and printing some
sample documents.
Then process the following short document to confirm that your
system will run \revtex:
\begin{verbatim}
%This is la-test.tex
\NeedsTeXFormat{LaTeX2e}[1996/06/01]%
\documentclass{article}
\begin{document}
 Hello, world!
\end{document}
\end{verbatim}


\subsection{Installation of \revtex}

\revtex~4 is incorporated into many commercial and shareware \TeX\ distributions,
%so there is a possibility that you yourself will have no need to install \revtex.
so you may find it unnecessary to install it.
To determine if such is the case, create and typeset the \file{rev-test.tex} document below.
If it compiles successfully, you have a working \revtex\ and can skip the
rest of this section.

The definitive \hypertarget{Tsec:getting}{distribution point} for \revtex~4 is
\url{http://publish.aps.org/revtex4/}.
It is also available on the Comprehensive \TeX\ Archive Network,
at \url{ftp://ctan.tug.org/tex-archive/macros/latex/contrib/supported/revtex}.

Full installation instructions for \revtex\ are in the \file{README} file distributed
with \revtex.

To confirm the integrity of your \revtex\ installation,
create and typeset the following \TeX\ document:
\begin{verbatim}
%This is rev-test.tex
\documentclass{revtex4}
\begin{document}
 Hello, world!
\end{document}
\end{verbatim}

Note: if you encounter difficulties with \revtex, the output from the
\file{la-test.tex} job in section~\ref{sec:prep} and the above \file{rev-test.tex}
can help diagnose installation problems.

\subsection{Your First \revtex\ Document}

Let's create a \revtex\ document that can ultimately be developed 
into a full-fledged journal submission.

\begin{enumerate}
\item
Start by making a copy of the \revtex-distributed file \file{template.aps}
under a new name, such as \file{mypaper.tex}. 
Put this file into a portion of your filesystem where your own documents are stored.

\item
Typeset and preview \file{mypaper.tex} and examine the formatted output.
The document is almost devoid of content.

\item
Open \file{mypaper.tex} in your text editor and locate the line
\begin{verbatim}\title{}\end{verbatim}
Change this line so that it reads:
\begin{verbatim}
\title{%
 A Proposal for the
 Routing of Public Rail Service
}
\end{verbatim}

\item
Locate the line
\begin{verbatim}\author{}\end{verbatim}
and change it to read:
\begin{verbatim}\author{Hedley Lamarr}\end{verbatim}
(or insert your own name here).

\item
Locate the line
\begin{verbatim}\affiliation{}\end{verbatim}
and change it to read:
\begin{verbatim}
\affiliation{%
 B. J. La Petomaine Institute,
 Rock Ridge AZ 12345
}
\end{verbatim}
(or insert your own institution here).

\item
Locate the line
\begin{verbatim}\section{}\end{verbatim}
and change it to read:
\begin{verbatim}
\section{%
 A Cautionary Note About Quicksand
}
\end{verbatim}
(or insert your own title here).
Likewise insert titles into the \cmd\subsection\ and \cmd\subsubsection\ commands
on the following lines.

\item
Follow the \cmd\subsubsection\ command with some general text of your own choosing.

\item
Save the file and typeset it.

\item
Congratulations, you have broken the ice with \revtex.
\end{enumerate}


\section{Creating Your \revtex\ Document}\label{sec:instruct}

Your \revtex\ document is a \LaTeX\ document
(specifically of the \classname{revtex} class),
and you create and process it like any other \LaTeX\ document. 

This section takes you through the steps of creating a \revtex\ document
in enough detail to allow you to create a full journal submission.

If you are familiar with earlier versions of \revtex,
please read Appendices~\ref{sec:diff31} and \ref{sec:conv31}, which
show how to convert from that version.
If you are familiar with the \LaTeX\ article class, upon which \revtex\ is based,
you can get a quick overview of \revtex's distinctive features by reading Appendix~\ref{sec:diffart}.
If you are unfamiliar with \LaTeX, you are advised to
obtain and refer to the manual, the \LUG\cite{LaTeXman}.

\subsection{Class options}\label{sec:task.opt}
Your document consists of \emph{preamble}\index{preamble} and body,
the latter delimited by \envb{document} and \enve{document} statements,
and the former consisting of all statements preceding the \enve{document}.

Start your document with a basic shell as follows:
%
\begin{quote}
\cmd\documentclass\oarg{options}\aarg{\classname{revtex4}}\\
\cmd\usepackage\marg{package}\\
\envb{document}\\
\meta{content}
\enve{document}
\end{quote}

The document class is \classname{revtex4};
%with \classname{aps} a mandatory class option for papers to be submitted to the APS.
class \meta{options} are separated by commas and include
\classoption{eqsecnum} (to number equations by section),
\classoption{preprint} (to get double-spaced output for submission purposes),
\classoption{tightenlines} (to get single-spaced output with the preprint style),
and \classoption{amsfonts} and \classoption{amssymb} (see Sec.~\ref{sec:fonts}).

There are class options for specific societies, called the \emph{society substyle},
such as 
\classoption{aps} for a genera American Physical Society, 
\classoption{aip} for the AMerican Institute of Physics,
\classoption{osa} for the Optical Society of America,
and
\classoption{seg} for the Society of Exploration Geophysicists.
There are class options for specific journals, called the \emph{journal substyle}.
Those relating to the APS are
\classoption{pra}, \classoption{prb}, \classoption{prc}, \classoption{prd},
\classoption{pre}, \classoption{prl}, \classoption{prstab}, and \classoption{rmp}
for \emph{Physical Review} \emph{A}, \emph{B}, \emph{C}, \emph{D}, \emph{E}, \emph{Letters},
\emph{Special Topics---Accelerators and Beams}, and \emph{Reviews of Modern Physics}, respectively.

Under the \classname{aps} society substyle, the journal substyle
\classoption{pra} is the default.
The \classoption{prb} journal substyle gives superscript reference citations,
as is the style for \emph{Physical Review B}.
The \classoption{prl} substyle yields
the slightly different line spacing of \emph{Letters} (use for accurate
length estimates).
Other than this, there are no substantial differences in the APS journal options.
%Please do not use the \classoption{prb} option unless you
%will be submitting to \emph{Physical Review B}.

The \classoption{floats} class option enables \LaTeX{}-style floating figures and tables.
%---it is for an author's personal use, and is \emph{not} for use with files to be
%submitted to the APS.
%FIXME: is this policy still the case?
%All files submitted to the APS should have figures and tables at the end of the file.
Alternatively, the \classoption{nofloats} class option automatically 
moves the figures and tables to the end of the formatted document.
%Other arrangements may not be accommodated by the compuscript program.
The \classoption{twocolumn} class option typesets the document in
a two-column layout for your convenience in creating a reprint format.
%is for personal use, and not for use in submitted files.

Please refer to the file \file{apssamp.tex} for an example of how to 
invoke these options.
Numerous other class options are available; please see Section~\ref{sec:class} for details.

The document preamble can have any number of \cmd\usepackage\ statements;
see Section~\ref{sec:usepackage} for information about \revtex's compatibility with
other \LaTeX\ packages. 

\subsection{Front matter}\label{sec:front}

The document body begins with the frontmatter statements,
all of which absorb data for use by the \cmd\maketitle\ command
that ends the frontmatter.
Continue your document with a \cmd\maketitle\ command, preceding
that command with frontmatter statements as described below.
\begin{quote}
\envb{document}
\cmd\title\marg{title}\\
\cmd\author\marg{author}\\
\meta{frontmatter}
\cmd\maketitle
\end{quote}

Enter the title with the \cmd\title command:
\begin{quote}
\title\oarg{short title}\marg{title text}
\end{quote}
If your document's title is sufficiently long, you may need to provide a truncated
title for the purposes of the page running header; enter that as the optional argument to
the \cmd\title\ command.

\subsubsection{Author and Affiliation}%

Next enter the authors and affiliations.
For an article with a single author, give the \cmd\author\ and \cmd\affiliation\ commands,
for example:
\begin{verbatim}
\author{Jackson P. Jones}
\affiliation{321 Main Street, Everville,
          Illinois 12345-6789}
\end{verbatim}

For multiple authors at a single institution, put each author into a separate \cmd\author\ command,
and follow with the \cmd\affiliation\ statement:
\begin{verbatim}
\author{Jackson P. Jones}
\author{Joan Q. Johnson}
\affiliation{321 Main Street, Everville,
          Illinois 12345-6789}
\end{verbatim}
This arrangement is called an \emph{author group};
it has one or more \cmd\author\ commands followed by one or more \cmd\affiliation\ commands
(each author is understood to be affiliated with all of the specified affiliations).

Your frontmatter itself may have more than one author group;
this is how you accomodate a mixture of authors and affiliations.

For each individual author, you may give any combination of
\cmd\email, \cmd\homepage, \cmd\thanks, or \cmd\altaffiliation\ statements:
\begin{verbatim}
\author{Jackson P. Jones}
 \email{JackP@Jones.org}
 \email{JPJ@ev.il.us}
 \homepage{www.jones.org}
 \thanks{Work supported by Jenny Jones}
 \altaffiliation{Everville Institute}
\affiliation{321 Main Street, Everville,
         Illinois 12345-6789}
\end{verbatim}
These author attributes are formatted either as title page footnotes or in the title block itself,
depending on the requirements of the journal substyle.

Complex arrangements of authors and affiliations are possible with \revtex;
see Appendix~\ref{sec:authorgroup} for more details.

\subsubsection{Other Front Matter}%

Enter the \cmd\date\marg{date} command to have the date printed on the manuscript.
Using \cmd\today\ will cause \LaTeX{} to insert the current date whenever the file is run:
\begin{verbatim}
\date{\today}
\end{verbatim}

Next enter your abstract in the \env{abstract} environment:
\begin{verbatim}
\begin{abstract}
In this paper we show the result of...
\end{abstract}
\end{verbatim}

The final element of the frontmatter data is the \cmd\pacs\marg{pacs numbers} command.
\begin{verbatim}
\pacs{23.23.+x, 56.65.Dy}
\end{verbatim}

The \cmd\maketitle\ command must be entered last of all.
Note: If you omit this command, your formatted output will have no title block at all.
\begin{verbatim}
\maketitle
\end{verbatim}

Please see Section~\ref{sec:ref:front} for more information about frontmatter commands,
and the author/affiliation commands in particular.

\subsection{Section headings}

Section headings are input as in \LaTeX.
The output is similar, with a few extra features.

Four levels of headings are available in \revtex{}:
\begin{quote}
\cmd\section\oarg{short title}\marg{title text}\\
\cmd\subsection\marg{title text}\\
\cmd\subsubsection\marg{title text}\\
\cmd\paragraph\marg{title text}
\end{quote}

Provide the \meta{short title} if needed for the sake of the running header
(required only by some journal substyles).

Use the starred form of the command to suppress the automatic numbering; e.g.,
\begin{verbatim}
\section*{Introduction}
\end{verbatim}

To label a section heading for cross referencing use the \cmd\label\marg{key}
command \emph{after} the heading; e.g.,
\begin{verbatim}
\section{Introduction}
\label{sec:intro}
\end{verbatim}

In the some journal substyles, such as those of the APS,
all text in the \cmd\section\ command is automatically set uppercase.
If a lowercase letter is needed, use \cmd\lowercase\aarg{x}.
For example, to use ``He'' for helium in a \cmd\section\marg{title text} command, type
\verb+H+\cmd\lowercase\aarg{e} in \marg{title text}.

The \cmd\appendix\ command signals that all following sections are
appendices, so \cmd\section\marg{title text} after \cmd\appendix\ will set
\marg{title text} as an appendix heading (an empty \marg{title text} is permitted).
For a single appendix, use a \cmd\section\verb|*|\marg{title text} command to suppress the
appendix letter in the section heading.

Use \cmd\protect\verb+\\+ to force a line break in a section heading.
(Fragile commands must be protected in section headings and captions, and \verb+\\+ is a fragile command.)

%FIXME: PRL should set the \secnumdepth=-\maxdimen
%Compuscript note:
%{\bf Note\/}: For \emph{Physical Review Letters},
%if there are to be section headings, use only the fourth-level type, \cmd\paragraph\marg{title text}.
%Use the stared form of the command (\cmd\paragraph\verb+*+\marg{title text}) to avoid the
%numbering that is normally attached [(a), (b), $\ldots$].

\subsection{General Text}

\paragraph{Paragraphs}\ always end with a blank input line.
Because \TeX\ automatically calculates linebreaks and word hyphenation in 
a paragraph, you should not force linebreaks or hyphenation in your document.
Of course, you nonehteless continue to explicitly hyphenate, e.g., ``author-prepared copy.''

\paragraph{Use directional quotes}\ for quotation marks around quoted text
(\texttt{``xxx''}), not straight double quotes (\texttt{"xxx"}).
(For opening quotes, this is two octal 140 (hexadecimal 60) characters;
for  closing quotes, this is two octal 047 (hexadecimal 27) characters.)

You can control the width of the text across the page in two-column layout:
the \env{widetext} environment will set the text across the full width of the typing area.
This may be needed to set very long equations. 
See Section~\ref{sec:dispmath}.
The \env{widetext} environment has no effect on the output
if you have invoked the \classoption{preprint} class option.
The preprint style is a uniform width throughout.

Don't use \cmd\vspace, \cmd\smallskip, \cmd\bigskip, or any other vertical motion commands.
Likewise, horizontal motion commands like \cmd\hspace, should be avoided.

\LaTeX's standard \cmd\footnote\ command is available in \revtex{}.
Your target journal, however, may effectively invoke the \classoption{endnotes} class option;
these notes will then be placed at the end of the bibliography element.
%Use recent issues of the particular journal as a guide.

Note that in such a case, the argument of
the \cmd\footnote\ command is a moving argument in the sense of the \LUG, Appendix~C.1.3:
any fragile command within that argument must be preceded by a \cmd\protect\ command.

\subsection{Math in text}

\revtex\ uses the \TeX\ markup \verb+$+ for math, e.g.,
\begin{quote}
the quantity $a^{z}$
\end{quote}
is obtained from the input
\begin{verbatim}
the quantity $a^{z}$
\end{verbatim}

Within math mode, use \verb+^+\marg{math} for superscripts (and \verb+_+\marg{math} for subscripts),
as you see in the source for this guide.
If you omit the braces after the \verb+^+, \TeX{} will superscript the next \emph{token}
(generally a single character or command),
but it is safest to use explicit braces \verb+{}+.

As with text, your math should not require vertical or horzontal motion commands, 
because \TeX\ calculates math spacing itself automatically.
In particular, please \emph{do not} insert explicit spacing around relations (e.g., $=$) or operators (e.g., $+$).
These suggestions notwithstanding, some fine-tuning of math is required in specific
cases, see Chapter~18 in the \TeX book\cite{TeXbook}.

\subsection{Text in math}\label{sec:textinmath}

There are times when you need to insert text into math,
but there are more and less satisfactory ways of doing so.

The \cmd\rm\ command only switches to Roman font for math letters.
It does not, for example, let you print a normal text hyphen:
\verb+${\rm e-p}$+ gives ``$\rm e-p$''.
Using an \cmd\mbox\marg{text} will give you normal text, including a hyphen,
but will not scale correctly in superscripts:
\verb+$x_{\mbox{e-p}}$+ gives ``$x_{\mbox{e-p}}$''.

The \cmd\text\marg{text} command is the preferred method of setting text within math mode.
It gives you regular text \emph{and} scales correctly in superscripts:
\verb+$y=x \text{ for } x_{\text{e-p}}$+ gives ``$y=x \text{ for } x_{\text{e-p}}$''.

To use the \cmd\text\ command, you must load the \classname{amsmath} package:
include a \cmd\usepackage\aarg{\classname{amsmath}} command in your document preamble.

\subsection{Displayed equations}\label{sec:dispmath}

Equations are set centered in the column width or flush left depending on the selected journal substyle.

For the simplest type of displayed equation, a numbered, one-line equation,
use the \env{equation} environment.
\revtex\ takes care of the equation number%
---the number will be set below the equation if necessary.
Use \cmd\[\dots\cmd\] for a single, one-line unnumbered display equation.

Use the \env{eqnarray} environment when more than one consecutive equation occurs,
putting each equation in a separate row of the environment, and using \cmd\nonumber
before the row end (\cmd\\) to suppress the equation number where necessary.
If the equations are related to each other, align each on the respective relation operator (such as $=$).

When an equation is broken over lines or is continued over multiple relation operators,
it is called a multi-line or continued equation, respectively;
here, too, use the \env{eqnarray} environment.

For a continued equation, align each row on the relation operator just as with
multiple equations, and use the \cmd\nonumber\ command to suppress auto-numbering on broken lines.
Also, use the starred form of the row end (\cmd\\\verb+*+) to prevent a pagebreak at that juncture.

Short displayed equations that can appear together on a single line
separated by \cmd\qquad\ space, may be placed in a single \env{equation} environment.

In two-column mode, if an equation needs to be broken into many lines,
for ease of reading set it in a wide column using the \env{widetext} environment.
Then return to the normal text width as soon as possible.
However short pieces of paragraph text
and/or math between nearly contiguous wide equations should be incorporated into the surrounding wide sections.

In \file{apssamp.tex}, we illustrate how to obtain each of the above effects.

\subsubsection{Numbering displayed equations}

The \revtex{} macro package allows two methods for numbering equations:
you can allow \revtex{} to automatically number for you,
or you can assign your own equation numbers.

For automatically numbered single-line and multi-line equations, use the
\env{equation} and \env{eqnarray} environments as described above.
For unnumbered single-line equations, use the \verb+\[+\dots\verb+\]+ construction.
The command \cmd\nonumber\ will suppress the numbering on a single line of an
\env{eqnarray}.
For a multi-line equation with no equation numbers at all,
use the \env{eqnarray*} environment.

If you wish a series of equations to be a lettered sequence,
e.g., (3a), (3b), and (3c),
put the respective \env{equation} or \env{eqnarray} environment within the
\env{subequations} environment. 
You must load the \classname{amsmath} package for this capability;
include the statement \cmd\usepackage\aarg{\classname{amsmath}} in your document preamble.

Use the command \cmd\tag\marg{number} to produce an idiosyncratic equation number:
$(1')$, for example.
Numbers assigned by \cmd\tag\ are
completely independent of \revtex's automatic numbering.
The package \classname{amsmath} is required if you use the \cmd\tag\ command:
put the statement \cmd\usepackage\aarg{\classname{amsmath}} in your document preamble.

To have \revtex{} number equations by section,
use the \classoption{eqsecnum} class option in your document preamble. 

See \file{apssamp.tex} for examples.

\subsubsection{Cross-referencing displayed equations}

To refer to a numbered equation, use
the \cmd\label\marg{key} and \cmd\ref\marg{key} commands.
The \cmd\label\marg{key} command is used within the referenced equation
(on the desired line of the \env{eqnarray}, if a multi-line equation):

\paragraph{input:}\leavevmode\nopagebreak
\begin{verbatim}
\begin{equation}
 A=B \label{pauli}
\end{equation}
 ... It follows from Eq.~(\ref{pauli})
that this is the case ...
\begin{eqnarray}
 A & = &B,\label{pauli2}\\
 A'& = &B'
\end{eqnarray}
\end{verbatim}

\paragraph{output:}\leavevmode\nopagebreak
\begin{equation}
A=B \label{pauli}
\end{equation}
 ... It follows from Eq.~(\ref{pauli})
that this is the case ...
\begin{eqnarray}
A & = &B,\label{pauli2}\\
A'& = &B'
\end{eqnarray}

Please note the parentheses surrounding the \cmd\ref\ command.
these are \emph{not} provided automatically; you must incorporate them
into your electronic document if you want them.

Numbers produced with \cmd\tag\ can also be cross-referenced:
follow the \cmd\tag\ command with a \cmd\label\ command.

Using a \cmd\label\ after \envb{subequations} will allow you
to reference the \emph{general} number of the equations in the
\env{subequations} environment. For example, if
\begin{verbatim}
\begin{subequations}
 \label{allequations} % notice location
 \begin{eqnarray}
  E&=&mc^2,\label{equationa}
 \\
  E&=&mc^2,\label{equationb}
 \\
  E&=&mc^2,\label{equationc}
 \end{eqnarray}
\end{subequations}
\end{verbatim}
%
gives the output
\begin{subequations}
\label{allequations} % notice location
\begin{eqnarray}
E&=&mc^2,\label{equationa}
\\
E&=&mc^2,\label{equationb}
\\
E&=&mc^2,\label{equationc}
\end{eqnarray}
\end{subequations}
%
then \verb+Eq.~(\ref{allequations})+ gives ``Eq.~(\ref{allequations})''.

{\bf Note:} incorrect cross-referencing will result if
\cmd\label\ is used in an unnumbered single-line equation
(i.e., within the \verb+\[+ and \verb+\]+ commands),
or if \cmd\label\ is used on a line of an eqnarray that is not being numbered
(i.e., a line that has a \cmd\nonumber).

Please see Sec.~\ref{sec:xrefs} for further information about
cross-referencing.


\subsection{Special characters}

If you intend to submit your document to a compuscript program, 
it would be best to avoid the use of specially defined characters;
instead choose symbols from those shown in the \LUG{} or in Section~\ref{sec:chars}.
These characters are supported by the software that converts your
\revtex\ document to SGML or other format.

See Appendix~\ref{sec:chars} for
a list of standard \LaTeX{} symbols,
a list of symbols available when the \classoption{amsfonts} and \classoption{amssymb} options are used,
and
a list of extra symbols made available by \revtex.

\subsection{Citations and References}\label{sec:endnotes}

References are cited in text using the \cmd\cite\marg{key} command
and are listed in the bibliography using the \cmd\bibitem\marg{key} command.
Put the list of references after the main body of the paper using
one of two alternative methods.

If you are using \BibTeX, give the command
\begin{quote}
\cmd\bibliography\marg{bib files}
\end{quote}
where \meta{bib files} is a comma-separated list of \BibTeX\ bibliography database files,
each with a \file{.bib} extension.
See Section~\ref{sec:use-bib} for further instructions on using \BibTeX.

Alternatively, you may use an explict \env{thebibliography} environment:
%
\begin{verbatim}
\begin{thebibliography}{}
\bibitem[Tal(1982)]{tal82}
Y. Tal and L. J. Bartolotti,
J. Chem. Phys. {\bf 76}, 4056 (1982).
\end{thebibliography}
\end{verbatim}

In either case output looks like:
%
\begingroup\savenofiles\makeatletter\let\@endnotesinbib\relax
\begin{thebibliography}{}
\bibitem[Tal(1982)]{tal82}
Y. Tal and L. J. Bartolotti,
J. Chem. Phys. {\bf 76}, 4056 (1982).
\end{thebibliography}
\endgroup
%
The \cmd\bibitem\ command's optional argument specifies
information that is used to cite the reference when using author/year citation style.
The required argument, here \verb+tal82+ is a tag.
If you compile your \env{thebibliography} environment by hand,
you can chose the tag for each bibliographic entry as any string of letters and numbers.
If using \BibTeX, the tag must match that of the desired entry in your 
bibliographic database.

You use the tag in the \cmd\cite\ command to indicate which reference you want to cite.
For example,
%
\paragraph{input:}\leavevmode\nopagebreak
\begin{verbatim}
As has been noted previously~\cite{tal82}.
\end{verbatim}
%
\paragraph{output:}\leavevmode\nopagebreak
\begin{quote}
As has been noted previously~\cite{tal82}.
%Klootch for manual
\makeatletter
\immediate\write\@auxout{%
\string\bibcite{tal82}{{1}{1982}{{Tal}}{{Y. Tal and L. J. Bartolotti}}}%
}%
\end{quote}

In journal substyles using superscript reference citations,
such as \emph{Physical Review~B},
you need an alternative command to get on-line citations;
the command \cmd\onlinecite\marg{key} is available for this purpose.
For example, \verb+Ref.~\onlinecite{tal82}+ will give the output ``Ref.~\onlinecite{tal82}''.

When the citation constitutes part of the grammar of the sentence,
you use the \cmd\textcite\marg{key} command, for example,
\verb+\textcite{tal82} has shown+ will give the output ``\textcite{tal82} has shown''.

A \cmd\cite\ command with multiple keys is formatted with
consecutive reference numbers collapsed; e.g., [1,2,3,5,7,8,9] will be
output as [1--3,5,7--9].
If you need to split the list over more than one line, use a \verb+%+
character immediately following a comma;
thereby ensuring that the list will be processed correctly.
\begin{verbatim}
. . . as shown in \cite{a,b,c,d,e,f,%
g,h,i,j,k,l,m,n,o,p,q,r,s,t,u,v,w,x,y,z}
\end{verbatim}
Note the \verb+%+ inserted after the comma on the first line,
which avoids unwanted spaces.

\subsubsection{Using \BibTeX}\label{sec:use-bib}
The \BibTeX\ application is an adjunct to \TeX\ that aids in the preparation of your
bibliography.

To use \BibTeX\ with \revtex, you must
select an appropriate journal substyle,
issue the \cmd\bibliography\ command as described above,
give \cmd\cite\marg{key}\ commands
(using as \meta{key} that of the deisred entry in your bibliographic database),
and of course prepare your \file{.bib} bibliographic databases.
In this section, we use the \cmd\cite\ command to stand also for
\cmd\textcite\ and \cmd\onlinecite.
\begin{itemize}
\item
Selecting a journal substyle automatically invokes the necessary 
\cmd\bibliographystyle\ command with the appropriate argument.
For instance, for APS journals in general, this argument is \file{revtex4},
but is \file{rmp} in the particular case of the \classoption{rmp} (\emph{Reviews of Modern Physics}) journal substyle.
Your selected jounal substyle must do likewise.

\item
As explained above, the \cmd\bibliography\ command performs double duty by
specifying both the location within your document where the list of references
is to appear, and the set of \BibTeX\ bibliography database files to be used when
\BibTeX\ prepares your \file{.bbl} file.

\item
Each \cmd\cite\ command in your document automatically records its citation key
in your document's \file{.aux} file, for later use by \BibTeX.

\item
An appropriate bibliographic database is required as well. 
You may have created one of your own, or you may have access to one of the 
compiled databases, depending on your field of research.
\end{itemize}

With the above requirements met, you carry out the following steps:
(we take the name of your document to be \file{myfile}\file{.tex})
\begin{enumerate}
\item
Process your document once under \revtex\ as specified elsewhere in this guide,
and ignore any \LaTeX\ reports of undefined citations.
\LaTeX\ compiles a list of needed references in the \file{myfile}\file{.aux} file
from each instance of a \cmd\cite\ command in your document.

\item
Run \BibTeX\ on the \file{myfile}\file{.aux} file,
thereby creating the \file{myfile}\file{.bbl} file.
To run \BibTeX\ on a command-line operating system,
you might give a shell command like \file{bibtex} \file{myfile}.

\item
Process your document a second time under \revtex,
still ignoring any \LaTeX\ reports of undefined citations.
\LaTeX\ typesets the bibliography and, for each \cmd\bibitem\ statement therein,
records the meaning of each reference key in the \file{.aux} file for use when the key is cited.

\item
Process your document a third time under \revtex.
This time a reports of an undefined citation indicates that you have either
failed to correctly enter the citation key in your \cmd\cite\ command that
matches the key in the \file{.bib} file, or that the \file{.bib} file lacks
any entry with that key.

\item
Repair any problems and repeat the whole process from step~1.

\item
If you have no reports of undefined citations, your \BibTeX\ work is complete.
\end{enumerate}

For more information on using \BibTeX\ with \LaTeX, see Sections~4.3.1 and~C.11.3
of the \LUG\cite{LaTeXman}, Section~13.2 of \cite{Compan}, or the online \BibTeX\ manual
\url{http:ctan.tug.org/tex-archive/help/Catalogue/entries/bibtex.html}.


\subsubsection{References by Hand}
If you are not using \BibTeX, please bear in mind the following when
preparing your \cmd\bibitem s.

\begin{itemize}
\item
The \cmd\bibitem\oarg{bib text}\marg{key} command begins each reference item.

\item
References should be listed in the reference section in the order in
which they are first cited in the text if using numerical citations,
in alphabetical order if using author/year citations.

\item
Numerical references are automatically numbered by \revtex{} in the order
in which they occur in the reference section.

\item
The \meta{key} in \cmd\bibitem\marg{key} is a tag; you can choose any string
of letters and numbers to associate with the reference.
This tag is used with the \cmd\cite\marg{key} command when citing the reference.

\item
The \meta{bib text} in \cmd\bibitem\oarg{bib text} is only used in the case of
author/year citations; it should have the structure
\begin{quote}
\cmd\bibitem\verb+[+\meta{short-name}\verb+(+\meta{year}\verb+)+\meta{long-name}\verb+]+
\end{quote}
where \meta{short-name} is the author name used in a parenthetical citation, 
\meta{long-name} that used in a textual citation, and 
\meta{year} is the year.

\item
If you wish to prepare a bibliography that can serve as the basis for a document
using either author/year or numerical citations, then prepare it for the former.
If you later choose a journal substyle using numerical citations you need make no changes to your bibliography.

\end{itemize}

\subsubsection{The \file{reftest} Tool}

\revtex\ includes a tool for authors who prepare their bibliographies by hand,
called \file{reftest.tex}.
It will check to make sure that you have
(1) no uncited references,
(2) no undefined citations, and
(3) your references are in the same order as your citations.
Using \file{reftest}, an author can put the citations in
the correct order once, after writing the paper, by using the correct order
reported by \file{reftest.tex}.

This process only works if you use \LaTeX{}'s
\cmd\bibitem\marg{key} and \cmd\cite\marg{key} mechanisms.

To check the references for the file \file{myfile}\file{.tex},
\begin{enumerate}
\item
Run \file{myfile}\file{.tex} through \LaTeX{} as usual,
thereby creating an up-to-date auxiliary file \file{myfile}\file{.aux}.
(\file{reftest.tex} uses that file to analyze your references.)

\item
Run \LaTeX{} on \file{reftest.tex}: it prompts for the name of the file you wish to check.
Answer \file{myfile} at the prompt (\emph{not} \file{myfile}\file{.tex} or \file{myfile}\file{.aux}).

\item
Note messages on your console and in the log file (\file{reftest.log})
that tell you of any problems.
Correct them.

\item
Preview or print the file \file{reftest.dvi} to see
the correct order of your references.
Note that this information does \emph{not} appear in the log file.

\end{enumerate}



\subsection{Figures and Artwork}\label{sec:figures}

Figures are part of the compuscript and
should be input using the \env{figure} environment as illustrated below;
\LaTeX{}  will label and automatically number the captions FIG.~1, FIG.~2, etc.,
or in whatever format required by the chosen journal substyle.
%
Note how the \cmd\label\marg{key} command is used to
cross-reference figures in text. The \cmd\label\marg{key} command should be
inserted inside or after the figure caption, before the end of the
figure environment.

\paragraph{input:}\leavevmode\nopagebreak
\begin{verbatim}
\begin{figure}
 \caption{Text of first caption.}
 \label{fig1}
\end{figure}

\begin{figure}
 \caption{%
  This is the second caption:
  comparison of the differential cross
  sections for the subprocess
  $qg \rightarrow qggg$ of our
  approximation (dotted line)}
 \label{fig2}
\end{figure}
\end{verbatim}

\paragraph{output:}\leavevmode\nopagebreak
\begingroup\makeatletter\def\fps@figure{h}
\begin{figure}
\caption{Text of first caption.}
\label{fig1}
\end{figure}

\begin{figure}
\caption{This is the second caption:
comparison of the differential cross
sections for the subprocess
$qg \rightarrow qggg$ of our
approximation (dotted line)}
\label{fig2}
\end{figure}
\endgroup

Figures are cited in text with the use of  the \cmd\ref\marg{key} command:

\paragraph{input:}\leavevmode\nopagebreak
\begin{verbatim}
 ...It can be seen from Fig.~\ref{fig1} 
that the data are inconsistent...
\end{verbatim}

\paragraph{output:}\leavevmode\nopagebreak
\begin{quote}
 ...It can be seen from Fig.~\ref{fig1} 
that the data are inconsistent...
\end{quote}
%
Further information on cross-referencing can be found in Sec.~\ref{sec:xrefs}.

\subsubsection{Artwork}
Use the standard \LaTeX\ \cmd\includegraphics\ command,
as enhanced by the \classname{graphicx} package,
to import an electronic art file into your document,
most commonly into a \env{figure}.
\begin{quote}
\envb{figure}\\
\cmd\includegraphics\oarg{key-vals}\marg{filename}\\
\cmd\caption\marg{title text}\\
\cmd\label\marg{key}\\
\enve{figure}
\end{quote}

For more information on the enhancements of the \classname{graphicx} package, 
see \cite{CompanG} or
\url{ftp://ctan.tug.org/tex-archive/macros/latex/required/graphics/grfguide.ps}.

\subsubsection{Figure Placement}

As with tables (cf. Section~\ref{sec:tables}), 
figures float to the top or bottom of the page if not otherwise specified,
using the standard \LaTeX\ float placement mechanism.
Initially, you should put each \env{figure} environment immediately following its first reference in the text;
this will usually result in satisfactory placement on the page.
Use the optional argument of the \env{figure} environment to
make adjustments to your float placement
\begin{quote}
\envb{figure}\oarg{placement}\\
\dots\\
\enve{figure}
\end{quote}
where \meta{placement} can be any combination of \verb|htbp!|, signifying
``here'', ``top'', ``bottom'', ``page'', and ``as soon as possible''.
For more details about float placement, 
please study the instructions in the \LUG, Appendix~C.9.1.

Invoking the \revtex\ \classoption{preprint} class option changes \LaTeX's
float behavior: all figures are automatically printed out at the end of your document.
This arrangement may be required by your journal's compuscript program.


\subsection{Tables and Alignments}\label{sec:tables}

Tables are part of the compuscript and
should be input using the \env{table} environment as detailed below;
\LaTeX{}  will label and number the captions TABLE~1, TABLE~2, etc.
or in whatever format required by the chosen journal substyle.

Each table must begin with \envb{table}, end with \enve{table}, and
have a caption (using the \cmd\caption\marg{text} command).
The optional \cmd\label\marg{key} command follows the \cmd\caption\ 
and is used for cross-referencing.
Use the \cmd\ref\marg{key} command to cite tables in text.

The content of the \env{table} environment should be 
a \env{tabular}\marg{preamble} environment.
Please refer to Section~3.6.3 and Appendix~C.10.2 of the \LUG{} 
for more details about the \env{tabular} environment.

Use the commands \cmd\toprule, \cmd\colrule, and \cmd\botrule\ to
structure your \env{tabular} into the column heads (those rows between \cmd\toprule\ and \cmd\colrule)
and the alignment body (those rows between \cmd\colrule\ and \cmd\botrule).
Follow current journal style concerning placement of other table rules.

\paragraph{input:}\leavevmode\nopagebreak
\begin{verbatim}
\begin{table}
 \begin{tabular}{ll}
 \toprule
  Column 1&Column 2\\
 \colrule
  Cell 1&Cell 2\\
 \botrule
 \end{tabular}
 \caption{Text of table caption.}
 \label{tab1}
\end{table}
\end{verbatim}

\paragraph{output:}\leavevmode\nopagebreak
\begingroup\makeatletter\def\fps@table{h}
\begin{table}
 \begin{tabular}{ll}
 \toprule
  Column 1&Column 2\\
 \colrule
  Cell 1&Cell 2\\
 \botrule
 \end{tabular}
 \caption{Text of table caption.}
 \label{tab1}
\end{table}
\endgroup

\subsubsection{Some special table considerations}
\begin{itemize}
\item
Use the correct number of descriptive column headings.

\item
\emph{Numerical columns} should align on the decimal point (or
decimal points if more than one is is present).
The column specifier \texttt{d},
should be used for simple numeric data with a \emph{single} decimal point.
Material without a decimal point is simply set in math mode, centered.

To use the \texttt{d} column specifier, you must load the \classname{dcolumn} package;
put \cmd\usepackage\aarg{\classname{dcolumn}}
in your document preamble.
%
The entry of a \texttt{d} column is typeset in math mode;
do note insert any \verb+$+ math shift characters into a \texttt{d} column.
If text is required in the column, use \cmd\text\ or \cmd\mbox\ as appropriate.

If multiple decimal points are present then the last is
used for alignment. To escape from the \texttt{d} column use
\cmd\multicolumn\ as usual. See \file{apssamp.tex} for examples.

\item
Use \verb+$+ delimiters for all math in a table;
do not put a displayed equation in a table.

\item
\emph{Footnotes} in a table are labeled \emph{a}, \emph{b}, \emph{c}, etc.;
use the \LaTeX\ \cmd\footnote\ command.
%Tablenote commands that act just like regular footnotes have been added.
See \file{apssamp.tex} for examples and explanations of use.

\item
Use the \cmd\squeezetable\ command with tables that do not otherwise fit on the page:
placing this command before your \envb{tabular} statement
makes the fonts in the body of the \env{tabular} smaller, allowing
larger tables to fit onto the page.

\end{itemize}

\subsubsection{Table Placement}

Like figures (cf. Section~\ref{sec:figures}), 
tables float to the top or bottom of the page if not otherwise specified,
using the standard \LaTeX\ float placement mechanism.
Initially, you should put each \env{table} environment immediately following its first reference in the text;
this will usually result in satisfactory placement on the page.
Use the optional argument of the \env{table} environment to
make adjustments to your float placement
\begin{quote}
\envb{table}\oarg{placement}\\
\dots\\
\enve{table}
\end{quote}
where \meta{placement} can be any combination of \verb|htbp!|, signifying
``here'', ``top'', ``bottom'', ``page'', and ``as soon as possible''.
For more details about float placement, 
please study the instructions in the \LUG, Appendix~C.9.1.

Invoking the \revtex\ \classoption{preprint} class option changes \LaTeX's
float behavior: all tables are automatically printed at the end of your document.
This arrangement may be required by your journal's compuscript program.

\subsection{Cross-referencing}\label{sec:xrefs}

\revtex{} has built-in features for labeling and cross-referencing
section headings, equations, tables, and figures. This section contains a
simplified explanation of cross-referencing features.  The format for using
these features with section headings, equations, tables, and figures is
discussed in the appropriate section.

Cross-referencing depends upon the use of ``tags,'' which are defined by
the user.  The \cmd\label\marg{key} command is used to identify tags for
\revtex . Tags are strings of characters that serve to label section
headings, equations, tables, and  figures, so that you don't need to know
what number \revtex{} has assigned to the item in order to talk about it in
text.

You will need to process your file through \revtex\ twice to ensure that
the tags have been properly linked to appropriate numbers.  If you add any
tags in subsequent editing sessions, you will need to repeat this process:
\LaTeX{} will display a warning message in the log file that ends with
\texttt{... Rerun to get cross-references right}.
If you see that message, run the file through \revtex\ again.

If the error message persists, please check your labels;
you may have labelled more than one object with the same \m{key}.

Another \LaTeX\ warning is \texttt{There were undefined references}, which
signifies that you have used a key in a \cmd\ref\ without ever using it in a \cmd\label\ statement.
If you encounter this message after running your document through \LaTeX\ twice,
search your document for the \m{key} in question: it must appear as the argument of a \cmd\label\ command.

\revtex{} performs autonumbering exactly as in standard \LaTeX:
when you process your file for the first time,
\LaTeX\ creates an auxiliary file (with the \file{.aux} extension) that 
records the value of each \meta{key}.
Each subsequent run retrieves the proper number from the auxiliary file 
and updates the auxiliary file.
At the end of each run,
any change in the value of a \meta{key} produces a \LaTeX\ warning message.

\subsection{Fonts}\label{sec:fonts}

\revtex{} has been set up to give good results on standard \LaTeX{} installations,
but we cannot guarantee that you will be able to access all the font options---%
memory and font restrictions vary in \TeX{} implementations and computers.

\subsubsection{Bold symbols in math}\label{sec:bboxamsfonts}

If you require bold symbols in math, particularly in superscripts or subscripts,
use the \cmd\bm\marg{symbol}\ command.
You must have the AMS fonts installed and invoke the \classoption{amsfonts} class option.
You must also load the \classname{bm} package:
place the command \cmd\usepackage\aarg{\classname{bm}} in your document preamble.

The \cmd\bm\ command makes the symbol bold in math mode,
and it ensures that it is the correct size, even in
superscripts. If the correct font in the correct size is not available, then
you get \marg{symbol} at the correct size in lightface and \LaTeX{} will issue
a warning that says \texttt{No \cmd\boldmath\ typeface in this size}.
You can also use \cmd\bm\ to get bold greek characters---upper- and
lowercase---and other symbols.

The following will come out bold with \cmd\bm:
normal math italic letters, numbers,
Greek letters (uppercase and lowercase),
small bracketing and operators, and \cmd\mathcal.

Note that \cmd\bm\marg{math} is a fragile command.

\subsubsection{Extra typefaces in math: \classoption{amsfonts} option}

In addition to the extra bold capabilities you get in math with the
\classoption{amsfonts} option, you also gain access to the Fraktur and Blackboard
Bold typefaces.
You select these with normal font-switching commands:
\verb+${\mathfrak{G}}$+ gives a Fraktur ``$\mathfrak{G}$''
and \verb+${\mathbb{Z}}$+ gives a Blackboard Bold ``$\mathbb{Z}$''.
Fraktur will become bold in a \cmd\bm; there is no bold version of Blackboard Bold.
%If you have the AMS fonts installed and have the
%\classoption{amsfonts} option selected, example output can be found in Appendix~\ref{sec:chars}.

\subsubsection{Extra symbols in math: \classoption{amssymb} option}

Many new symbols are available to you if you have the AMS fonts installed.
The \classoption{amssymb} class option gives you all the font capabilities of the
\classoption{amsfonts} class option and further defines the commands to get the
symbols shown in Appendix~\ref{sec:chars}, which contains examples of the symbols and
for instructions on use. These characters will scale correctly in superscripts and heads.
%Note that the symbols and typefaces
%in Appendix~\ref{sec:chars} will not be printed unless you have the AMS fonts installed
%and have selected either the \classoption{amsfonts} or \classoption{amssymb} class option.

\subsubsection{AMS fonts}\label{AMSFonts}

The AMS fonts, developed by the American Mathematical Society,
are available free of charge at \url{ftp://ctan.tug.org/fonts/amsfonts}.
Most \LaTeX\ installations incorporate the AMS fonts in many formats,
including ATM-compatible Type 1 PostScript fonts.
There are two class options for accessing the AMS fonts:
\classoption{amsfonts} and \classoption{amssymb}.

The \classoption{amsfonts} option defines the \cmd\mathfrak\ and \cmd\mathbb\
commands to switch to the Fraktur and Blackboard Bold fonts, respectively.
Fraktur characters will come out bold in a \cmd\bm, Blackboard Bold will not.
The \classoption{amsfonts} option also adds support for bold math letters and
symbols in smaller sizes and in superscripts when a
\cmd\bm\marg{symbol} is used.  For example, \verb+$^{\bm{\pi}}$+ gives a bold
lowercase pi in the superscript position: $^{\bm{\pi}}$.

\classoption{amssymb} gives the
capabilities of the \classoption{amsfonts} option and additionally defines many
new characters for use in math.

\revtex{} does not support the use of the extra Euler fonts (the AMS fonts
starting with \texttt{eur} or \texttt{eus}) or the Cyrillic fonts (the AMS fonts
starting with \texttt{w}).



\section{A \revtex\ Command Reference}

This section is a systematic reference to all \revtex-specifc commands.
Please see the \LUG\ for complete information about \LaTeX\ commands.

\subsection{Document Class Declaration and Options}\label{sec:class}%

All \revtex\ documents must start with the declaration:
\begin{quote}
\cmd\documentclass\oarg{options}\aarg{\classname{revtex4}}
\end{quote}

There are numerous \emph{options}, as listed below.

\subsubsection{The Document Substyle}

Among your document class options will be exactly one \emph{substyle},
an option specifying the society or the journal
to which your article will be submitted.
One such society is the American Physical Society, hence
the document class option \classoption{aps} signifies that your article is to 
be submitted to one of the APS journals. 
Alternatively, you can specify a particular journal.
Select a substyle from the following list:
%
\begin{quote}
\begin{tabular}{@{}ll@{}}
\toprule
\textbf{substyle}&\textbf{Journal}\lrstrut\\
\colrule
\classoption{aps}&American Physical Society\\
\classoption{pra}&Physical Review A\\
\classoption{prb}&Physical Review B\\
\classoption{prc}&Physical Review C\\
\classoption{prd}&Physical Review D\\
\classoption{pre}&Physical Review E\\
\classoption{prl}&Physical Review Letters\\
\classoption{prstab}&Physical Review Special Topics---Accelerators and Beams\\
\classoption{rmp}&Reviews of Modern Physics\lrstrut\\
\botrule
\end{tabular}
\end{quote}

%Note: The \classoption{prb} journal substyle uses superscript citation format.

Another possible society is the OSA, selected with
the \classoption{osa} substyle; currently unimplemented.

If you invoke a class option that \revtex\ does not otherwise know about, 
it looks for a journal substyle with the corresponding name (with a \file{.rtx} extension).
If no such substyle file exists, that option is made available as a
global class option for other packages to use as appropriate.

You should examine your log file for any messages of the sort:
\begin{verbatim}
LaTeX Warning: Unused global option(s):
\end{verbatim}
to see what options you have invoke that are not defined or ever used.
If you see on that list the name of a journal substyle, you will know that
the corresponding \file{.rtx} file was not found.

Correct the situation by installing the indicated \file{.rtx} file in
a location on your file system where \TeX\ can find it.
Under the TDS, it would be placed into \url{textmf/tex/latex/revtex}.


\subsubsection{Type Size Options}
You may select a type size from among the following.
Note that selecting a type size is optional;
your selected journal has a default type size.
\begin{description}
\item[\classoption{10pt}]
 The default size.
\item[\classoption{11pt}]
 Alternative size for author drafts.
\item[\classoption{12pt}]
 The default size in the \classoption{preprint} option described below.
\end{description}

\subsubsection{Media Size Options}

The media size options of the standard \LaTeX\ classes are available.
Note that selecting the media size does not affect the text area of 
your formatted article.

\subsubsection{AMS Font Options}
You may specify one of the following two options:
\begin{description}
\item[\classoption{amsfonts}]
Load the AMS font package.
(Equivalent to putting \cmd\usepackage\aarg{\classname{amsfonts}}
in the document preamble.)

\item[\classoption{noamsfonts}]
Don't load the AMS fonts package (even if a journal option loads \classname{amsfonts} by default).
\end{description}

You may specify one of the following two options:
\begin{description}

\item[\classoption{amssymb}]
Load the AMS symbols package. (Equivalent to putting
\cmd\usepackage\aarg{\classname{amssymb}} in the document preamble.)

\item[\classoption{noamssymb}]
Don't load the AMS symbols package (even if a journal option loads \classname{amssymb} by default).
\end{description}

\subsubsection{Author and Address Options}
The following four options, all relating to how the authors and affiliations
are formatted in the title block, are mutually exclusive.
You may have only one of them in effect at one time.

\begin{description}
\item[\classoption{groupedaddress}]
List each group of authors with shared addresses separately,
followed by the addresses.
Each shared address will only be
typeset once and all authors that share an address
will be typeset in the same group.

\item[\classoption{unsortedaddress}]
List the authors in exactly the order specified
even if this means typesetting some addresses more than once.

\item[\classoption{runinaddress}]
List authors similarly to \classoption{groupedaddress},
except that the authors are formatted in a paragraph instead of
on separate lines.

\item[\classoption{supercriptaddress}]
List all authors in a single list. Author
addresses are indicated by superscript markers which index into a
numbered list of addresses typeset after the author list.
\end{description}

Note that your chosen journal substyle will make a default choice 
of one of the above four options, and you may override this choice in your document.

\subsubsection{One- or Two-Column Layout}
\begin{description}
\item[\classoption{twocolumn}]
  Selects two-column layout. Unlike the option in the standard
  classes, the columns on the final page will be balanced.
  (This is implemented using Frank Mittelbach's \classname{multicol} package.)
\item[\classoption{onecolumn}]
  A single column across the full page width will be
  used. This is the default for the \classoption{preprint} option.
\end{description}

\subsubsection{Preprint and Other Options}
\begin{description}

\item[\classoption{preprint}]
Sets the article in single column at 12pt with
enlarged interline spacing and makes minor layout changes.
%This option may also be invoked as \classoption{manuscript}.
%This is a preferred option for submitting papers to \emph{Physical Review}. 
This option is intended for use when the formatted document is to be copyedited,
and it is activated by default.

\item[\classoption{galley}]
Sets the article in a single, narrow column approximating the format of journal article.
In \classoption{galley} format, the \env{widetext} environment sets its content using the full page width (over twice the width of general text).
This formatting option is one of two ways to gauge the length of a journal article; the other is \classoption{lengthcheck}.

\item[\classoption{tightenlines}]
If used in conjunction with the above options, this produces normal single spaced documents.

\item[\classoption{draft}]
This option marks overset lines (\texttt{Overfull} \cmd\hbox\ \texttt{in paragraph}),
as in the standard classes.

\item[\classoption{showpacs} and \classoption{noshowpacs}]
These options determine whether the Physics and Astronomy Classification Scheme data appear in the formatted output.

\item[\classoption{final}]
This item is the opposite of \classoption{draft}.

\item[\classoption{lengthcheck}]
This class option specifies that the formatted document 
should approach as closely as possible the
formatting of an actual journal article, thereby
facilitating performance of a length check.
Note that particular font requirements may be in effect for this option.

\item[\classoption{byrevtex}]
Using the \classoption{byrevtex} class option signifies that you want the
``Typeset by \revtex'' tagline to appear on your output.
\end{description}

\subsubsection{Footnote and Bibliography Options}

\begin{description}
\item[\classoption{bibnotes}]
Instead of putting remarks (\cmd\thanks, \cmd\email,
\cmd\homepage, and \cmd\altaffiliation) associated with authors as
footnotes on the title page, put them at the beginning of the
bibliography as unnumbered entries.

\item[\classoption{nobibnotes}]
Nullifies the effect of the \classoption{bibnotes} option.
If the journal substyle effectively invokes that option by default,
you can invoke \classoption{nobibnotes} to override that choice.

\item[\classoption{footinbib}]
Put all footnotes as numbered entries at the end of
the bibliography.
(Footnotes in the frontmatter are controlled independently
by the \classoption{bibnotes} option.)

\item[\classoption{nofootinbib}]
Nullifies the effect of the \classoption{footinbib} option.
If the journal substyle effectively invokes that option by default,
you can invoke \classoption{nofootinbib} to override that choice.

\item[\classoption{superbib}]
Number the entries in the bibliography with
superscripts rather than with numbers in square brackets.
(this is, e.g., the style of \emph{Phys.\ Rev.~B}.)
\end{description}

\subsubsection{Equation numbers}
\begin{description}
\item[\classoption{eqsecnum}]
Number equations within sections.

\item[\classoption{fleqn}]
Typeset equations flush left.
\end{description}

\subsubsection{Section Numbering Option}
The \classoption{secnumarabic} class option specifies that 
you want the sectioning commands to have arabic numbering.

\subsubsection{Floats Option}\label{ref:nofloats}
The \classoption{nofloats} option specifies that
floating elements such as figures and tables are to be set at the end of 
the formatted document (end floats).

Specifying the \classoption{floats} option means normal \LaTeX\ float behavior
and will override those journals which would by default have end floats.

If you specify neither option, then the selection will be made by the journal substyle;
usually \classoption{floats}.

These options are described in more detail below.

\subsubsection{Title Page Options}
It should not be necessary to use these options in your document,
because the journal substyle sets them as appropriate.
\begin{description}
\item[\classoption{titlepage}]
Start a new page after typesetting the title block.

\item[\classoption{notitlepage}]
Typeset the title block above the body of the text.
\end{description}

\subsubsection{Formatting for Duplex Printing}
The options \classoption{twoside} (the default) and \classoption{oneside} work
as in standard \LaTeX\ classes.

\subsubsection{Hypertext Option}
Use the option \classoption{hyperref} if you want your formatted document 
to have hypertext capabilities. This option implies the use of 
the \classname{hyperref} package, available from
\url{ftp://ctan.tug.org/macros/latex/contrib/supported/hyperref},
which is automatically loaded.

\subsubsection{Job Macro Package}
You can create a ``job macro package'' for your document
that will be read in automatically every time
your document is processed.
Thus, if your job is a file called \file{myarticle.tex},
then the file \file{myarticle.rty} will be read in
just the same as if you had placed a 
\cmd\usepackage\aarg{\classname{myarticle.rty}}
statement immediately following your \cmd\documentclass\ statement.

Within your \file{.rty} file,
you can define and use control sequence names that contain the \verb$@$ character,
and you can override any of the definitions or assignments made 
by the \revtex\ document class or the selected journal substyle.
That is, you have the power to make a mess.

If you choose to have a job macro package,
be sure to read the \LaTeX\ guide to document classes (\file{clsguide.tex}) or
read up on the subject of packages and classes in
\emph{The \LaTeX\ Companion} \cite{Compan} or a similar book.

The file \file{template.rty} contains a template for
creating your own job macro package.

\paragraph{Example}\
Here is a code fragment suitable for inclusion in your 
job macro package that defines the sectioning counters
to produce arabic numbers instead of the default roman numbers,
and which numbers the sectioning commands to the level of
\cmd\subparagraph.
\begin{verbatim}
\def\thesection{%
 \arabic{section}}%
\def\thesubsection{%
 \arabic{subsection}}%
\def\thesubsubsection{%
 \arabic{subsubsection}}%
\def\theparagraph{%
 \arabic{paragraph}}%
\def\thesubparagraph{%
 \theparagraph.\arabic{subparagraph}}%
\setcounter{secnumdepth}{5}%
\end{verbatim}


\subsection{Frontmatter Commands}\label{sec:ref:front}
As in the standard classes, the frontmatter is specified by a sequence
of declarations that gather information (data commands).
The \cmd\maketitle\ command then uses this information to typeset the title block.

\subsubsection{Data Commands}

\paragraph{Title}\ \cmd\title\oarg{short title}\marg{title}
The optional \emph{short title} will be used in running heads. If it
is not specified, then it defaults to the same value as \emph{title}.

\paragraph{Keywords}\ \cmd\keywords\marg{keyword list}
A comma-separated list of keywords (as used by subject review or
abstract publications).

\paragraph{PACS}\ %
\cmd\pacs\marg{PACS numbers}
PACS Subject classification numbers.
You must specify \cmd\pacs\ before the \cmd\maketitle\ command. 

\paragraph{Abstract}\ \envb{abstract}\emph{abstract}\enve{abstract}
The abstract is considered part of the frontmatter, and thus the
abstract environment must come before the \cmd\maketitle\ command in the
source file.

\paragraph{Dates and Numbers}\ 
The following commands specify the
volume, issue,  year, and electronic identifier
of the article,
as well as the dates received, revised, accepted, and published.

With the exception of the \LaTeX\ standard \cmd\date\ command,
these commands are more likely to be used by journal staff than by the
author of the document. The argument of each should be in the final
typeset form; the class does not parse these arguments.
%
\begin{quote}
\cmd\volumeyear\marg{year}\\
\cmd\volumenumber\marg{number}\\
\cmd\issuenumber\marg{number}\\
\cmd\eid\marg{identifier}\\[\baselineskip]
\cmd\date\oarg{text}\marg{date}\\
\cmd\received\oarg{text}\marg{date}\\
\cmd\revised\oarg{text}\marg{date}\\
\cmd\accepted\oarg{text}\marg{date}\\
\cmd\published\oarg{text}\marg{date}
\end{quote}
%
In the latter five commands, \oarg{text} signifies an alternative value for the
text that is produced just before the date, e.g., in the case of \cmd\received, it
might be ``Received''. You can use the optional argument to override the value 
chosen by the journal substyle.

\LaTeX\ will calculate page numbering from information taken from the previous run's \file{.aux} file,
if not otherwise specified:
%
\begin{quote}
\cmd\startpage\marg{number}\\
\cmd\endpage\marg{number}
\end{quote}

\paragraph{Preprint command}\ 

\cmd\preprint\marg{text} has no effect unless the preprint option has
been specified, in which case it adds identifying text to the page headline.


\subsubsection{Author/Affiliation Data Commands}

The most significant new feature in \revtex~4 concerns the commands used for
specifying author names, affiliations, and other author-related information.
They are designed to better mark up the information
(e.g., \cmd\email\ rather than \cmd\thanks)
for use in the editorial and production processes.

These data are organized into one or more ``author groups'',
each comprised of one or more authors followed by one or more affiliations:
the given authors are understood to share all of the given affiliations.
Furthermore each author can possess
any number of email, homepage, alternative affiliation, and general thanks.

Following an author group is an optional collaboration specification,
which is taken to apply to all of the preceding author groups up to the most 
recent collaboration specification.
A collaboration, like an individual author, can have
any number of email, homepage, alternative affiliation, and general thanks.

\paragraph{Author}\ 
\cmd\author\marg{author name}
Contrary to the usage of the \cmd\author\ commands in standard \LaTeX\ 
classes, each author should be specified in a \emph{separate}
\cmd\author\ command.

You may assist your journal in dealing with unusual names by specifying
the author's first name, or, independently, surname:
\begin{quote}
\cmd\author\verb|{|\\
 \cmd\firstname\marg{first-name}\\
 \cmd\surname\marg{surname}\\
\verb|}|
\end{quote}

Either one or both may be used. For example:
\begin{verbatim}
\author{Andrew \surname{Lloyd Weber}}
\author{\firstname{Yo yo} Ma}
\end{verbatim}

Note: The command \cmd\and\ used in the standard \LaTeX\ classes is not
supported by this class, and simply generates an error message.

The \cmd\author\ command may be followed by any combination of
author data
commands specifying email address, general URL, alternative affiliation, and
``thanks''. 
These commands are all implicitly subsidiary to the immediately preceding \cmd\author\ command
and may be repeated, if so desired, to give, e.g., multiple email addresses.

\paragraph{Email}\ \cmd\email[\textit{text}]\marg{email address}
Specify the electronic mail address of the immediately preceding \cmd\author.
The \meta{text} phrase is prepended to the email address.
Only the actual address should appear in the argument;
the \texttt{mailto:} is understood.

\paragraph{Homepage}\ \cmd\homepage[\textit{text}]\marg{URL}
Specify a URL for the immediately preceding \cmd\author.
This acts in the same way as \cmd\email, and may refer to a WWW homepage
of an author.

\paragraph{Alternative Address}\leavevmode\ 

\cmd\altaffiliation\oarg{comment}\marg{address}

Specify an alternative address for the immediately preceding \cmd\author.
This command produces a footnote with text constructed from the two
arguments, so the \meta{comment} argument will be something
like ``Currently at'' or ``Work undertaken while visiting'' or other
explanatory text to be placed in front of the address in the footnote.

\paragraph{Thanks}\ 
\cmd\thanks[\textit{text}]\marg{Extra remarks}

In the standard classes \cmd\thanks\ is used inside the argument
of \cmd\author, but in this class \cmd\thanks\ must \emph{follow} the
\cmd\author\ command.

Email addresses, URL's, and alternate affiliations
should be typeset with the appropriate command above and \emph{not} with the
\cmd\thanks\ command.
The latter should only be used when the other, more specific, choices are not appropriate.

\paragraph{Affiliation}\leavevmode\ 

\cmd\affiliation\marg{affiliation}

The affiliation (or address) of an author (or group of authors)
is specified using this command. All authors given since the
previous \cmd\affiliation\ command (or the start of the document) will be
taken as being at this address.

Some journal classes distinguish between ``affiliation'', which is
usually just the name of the department or institution where the work
was undertaken, and ``address'', which is a full postal address.
Currently \revtex\ does not make this distinction.
%, but does offer an
%\cmd\address\ command that is currently an alias for \cmd\affiliation.

If the \classoption{supercriptaddress} option is invoked, affiliations will be
numbered in the order they appear in the source file. This order is
effectively determined by the order in which the authors are listed,
and may not be the desired ordering.

To control the numbering, you may give the \cmd\affiliation\ commands
\emph{before} any authors are specified. 
This forces the numbering to follow the order of the listed \cmd\affiliation\ commands.
The addresses can then be
re-specified after the relevant authors.
In any case, if an address is
specified more than once it is only allocated one number, and, except
with the \classoption{unsortedaddress} option, it will be typeset once.

\paragraph{Collaboration}\leavevmode\ 

\cmd\collaboration\marg{collaboration}
Specify a collaboration applying to all prior author groups 
up to the most recent \cmd\collaboration.

This command will work only in the \classoption{superscriptaddress} mode.
The collaboration name will be typeset within parentheses following the
list of authors and can have \cmd\email, \cmd\homepage, \cmd\altaffiliation, and \cmd\thanks
commands associated with it. The \cmd\collaboration\ command should be followed
by a \cmd\noaffiliation\ command.

See Appendix~\ref{sec:authorgroup} for examples and more details
about author/affiliation data commands.

\subsubsection{Table of Contents}
As with standard \LaTeX,
you use the \cmd\tableofcontents\ command to mark the place in your
document where the table of contents is to appear, typically immediately after 
the \cmd\maketitle\ command.

Note that you will have to typeset your document at least three times before 
the information in the contents is valid: twice to obtain a contents of the correct number of lines 
and a third time for the pagination therein to be valid.

If using the \classoption{rmp} journal substyle, you see proper indentation on the contents 
only after the third typesetting run.


\subsection{Body Commands}\label{sec:body}%

\subsubsection{Bibliographies with Bib\TeX}
\revtex\ facilitates using Bib\TeX\ for compiling the bibliography.
During the editorial and production processes, it is useful to be able to
extract the bibliographic information to check it against definitive databases.
This will allow us to catch errors early in the life of the manuscript
and to add hyperlinks so that referees can locate electronic versions
of cited papers.

\paragraph{Reference component tagging}\leavevmode\ 

\cmd\bibinfo\marg{label}\marg{text}

The extra tagging is achieved by
using a \cmd\bibinfo\ command that takes a \meta{label} argument to
identify what is being tagged.
The labels correspond, for the most part, to the field names in a \file{.bib} file.
For instance, the
author of a cited paper would be tagged with \cmd\bibinfo\marg{author} and
the journal would be tagged with \cmd\bibinfo\marg{journal}. The
\textit{text} argument contains the corresponding string from the
\BibTeX\ file (suitably processed by \BibTeX\ of course).

The \cmd\bibinfo\ command does not affect the typesetting of the
information; rather, it is purely informative. Authors may choose to
add the \cmd\bibinfo\ commands by hand, but this rapidly becomes
tedious. To avoid the tedium, we have created a new \revtex\ Bib\TeX\
file, \file{revtex.bst}. This style file will automatically add the correct
\cmd\bibinfo\ tagging. Futhermore, the style file has been expanded to
handle items like URLs and e-prints which now frequently appear in
citations. Authors can now add this information to their \file{.bib} files in
a standard manner.

For more details on the Bib\TeX\ style files, please see the manual
\file{revbib.tex}, included with the \revtex~4 distribution.

\paragraph{Limitations in Bib\TeX}\ 
The advantages of Bib\TeX\ notwithstanding,
there are certain common constructions you cannot readily achieve through its use:
multiple references and references with lead-in text.
The following \env{thebibliography} environment illustrates each.
\begin{verbatim}
\begin{thebibliography}{}
\bibitem[Weinberg and Tomozawa(1966)]{Tom66}
 S. Weinberg, 
   \prl{\bf 17}, 616 (1966);
 Y. Tomozawa, 
   Nuovo Cimento A {\bf 46}, 707 (1966).
\bibitem[Moravcsik and Noyes(1961)]{Mor61}
 For early developments, see:
  M.J. Moravcsik and H.P. Noyes,
  Ann. Rev. Nucl. Sci. 
    {\bf 11}, 95 (1961).
\end{thebibliography}
\end{verbatim}
The first item gives two citations under a single \cmd\bibitem, i.e., 
a multiple reference.
The second gives a reference preceded by lead-in text.
In both cases you can achieve the effect only by manually editing the 
\file{.bbl} file.
The author of Bib\TeX\ is Oren Patashnick.


\subsubsection{Acknowledgments}
If your document has an acknowledgments section, use
the \env{acknowledgments} environment as its container.
Depending on the journal substyle, this element
may be formatted as an unnumbered section.

\subsubsection{Float processing}\label{ref:printfig}
Environments such as \env{figure} and \env{table} (and potentially other
similar environments defined by loaded packages or journal options)
may be positioned using \LaTeX's standard float placement
algorithm (the default),
or they may be held back (using an external file)
and set at the end of the document (end floats).

Invoke the commands \cmd\printtables\ and \cmd\printfigures\ at
the end of the document where the tables and figures should be printed
(as with the standard \cmd\printtindex\ command).

When floats are positioned in the document
body by the float placement system, these two commands are
silently ignored, so it is always safe to  use them and to switch between
different journal styles that may change the behavior of the formatter.

If the \cmd\printtables\ command is missing, the tables will be 
printed at the end of the document. Likewise, if \cmd\printfigures\ is missing,
the figures will be printed at the end of the document.
Therefore it is safe to omit these commands as long as 
you are satisfied with \revtex's default choices.

We recommend that you use explicit \envb{table} and \enve{table}
markup in your document (likewise with \env{longtable} and \env{figure}).
Moreover, if you use the \classoption{nofloats} option, or if your chosen journal substyle makes this selection,
then you \emph{must} use this explicit markup scheme.
In particular, please do \emph{not} follow the practice of defining typing shortcuts
for table and figure environments, like
\begin{verbatim}
\def\bt{\begin{table}}% Incompatible!
\def\et{\end{table}}%
\end{verbatim}
Such commands will be incompatible with generating end floats.

\subsubsection{Tables}\label{ref:table}

The following commands affect the \env{table} environment. They do not
apply to tables set directly in the text with a \env{tabular} environment
not enclosed in a \env{table}. They do however apply to \env{longtable}
environments if that environment (from the \classname{longtable} package) is used.

By default, tables are set in a smaller size than the text body
(\cmd\small). The \cmd\squeezetable\ declaration makes them smaller
(\cmd\scriptsize).

In general you can locally redefine \cmd\tabbodyfont\ to be whatever you
like. (\cmd\Huge\cmd\color\verb|{magenta}|\ldots?) 

\cmd\footnote\ works in table environments, producing the text at the
end of the table, not at the bottom of the page
(as if the body of the environment were enclosed in a \env{minipage} environment,
which is essentially how this feature is implemented).

\paragraph{Using the \env{tabular} environment}\ 
\revtex\ introduces three commands to help structure your
alignments, \cmd\toprule, \cmd\colrule, and \cmd\botrule;
use these commands after the row end (\cmd\\), 
similar to \cmd\hline.

The \cmd\toprule\ command starts off your \env{tabular}, and
all table rows down to the \cmd\colrule\ are understood to
comprise the table column heads.
The \cmd\botrule\ command comes last in your \env{tabular},
and all table rows below the \cmd\colrule\ command are understood
to comprise the table body.

\paragraph{Using the \classname{longtable} package}\ 

The \revtex\ document class is specifically designed to be compatible with 
the \classname{longtable} package. 
If any of your tables is so long as to require setting on multiple pages, you
are advised to use that package and its \env{longtable} environment.

To load the \classname{longtable} package, insert a
\cmd\usepackage\aarg{\classname{longtable}} 
command in your document preamble.

For more documentation on the \env{longtable} environment
and on the package options of the \classname{longtable} package,
please see the documentation thereof at \url{ftp://ctan.tug.org/macros/latex/required/tools/longtable.dtx}
or refer to the \LaTeX\ Companion.

Note that the \classname{longtable} package does not allow use 
of the \env{longtable} environment on multicolumn pages.
%
%Note that if you use the \classoption{twocolumn} option together with 
%the \env{longtable} environment, you get:
%\begin{verbatim}
% ! Package longtable Error: longtable not in 1-column mode.
%\end{verbatim}
%Therefore, using the \env{longtable} environment and the \classoption{twocolumn} option
%must needs be mutually exclusive.
%
%This a package limitation built into these two
%components of the standard \LaTeX\ system.
If you prefer to see this limitation lifted, please correspond directly with
\url{mailto:bugs@latex-project.org}.


\subsubsection{\revtex~4 symbols and the \classname{revsymb} package}

Symbols made available in earlier versions of \revtex\ are
defined in a separate package, \classname{revsymb},
so that they may be used with other classes.
(This might be useful if, say, copying text from a \revtex\ document
to a set of slides being produced with a class such as \classname{slides},
\classname{seminar} or \classname{foiltex}.)

The following are defined in this package:
\cmd\lambdabar, \cmd\openone, \cmd\corresponds, 
%\cmd\slantfrac, 
\cmd\succsim,
\cmd\precsim, \cmd\lesssim, \cmd\vereq, \cmd\gtrsim, \cmd\tensor, \cmd\overstar,
\cmd\overdots, \cmd\overcirc, \cmd\loarrow, \cmd\roarrow.
See Section~\ref{sec:revtexnotations} for examples.

\subsubsection{Bold Math}
The Bold Math (\classname{bm}) package is now the basis for creating
bold symbols in math mode.
The command \cmd\bm\marg{symbol} 
makes \marg{symbol} bold in math mode, ensuring that it
is the correct size, even in superscripts. If the correct font in the
correct size is not available then you get \marg{symbol} at the correct
size in lightface and \LaTeXe\ will issue a warning that says
``\texttt{No boldmath typeface in this size}\dots''.

\subsubsection{\env{widetext} environment}
Text that is too wide to fit the narrow measure of the two-column or
galley layouts may be placed in a \env{widetext} environment by using 
\envb{widetext} and \enve{widetext}.

In two-column mode, this will temporarily return to one-column mode,
balancing the text before the environment into two short columns,
and returning to two-column mode after the environment has finished. 

In galley mode \env{widetext} increases the measure allowing the text to
extend into the (otherwise empty) space at the right-hand side of the page. 

In one-column mode the environment has no effect.


\subsection{Using \LaTeX\ packages with \revtex}\label{sec:usepackage}%

\LaTeX\ users often employ add-in software packages in order
to use higher-level markup than is available with the standard
\LaTeX\ document classes, or to achieve particular formatting within
their document.

Such packages are available, for instance, on CTAN at
\url{ftp://ctan.tug.org/tex-archive/macros/latex/required/}
and at
\url{ftp://ctan.tug.org/tex-archive/macros/latex/contrib/}
or may be available on your distribution media,
such as the \TeX\ Live CD-ROM \url{http://www.tug.org/texlive}.

Some of these packages are automatically loaded by \revtex\ when you
select certain class options;
these are ``required'' packages (see Section~\ref{sec:pkg,req}).
They will either be distributed with \revtex\ or will be a required part of 
your \LaTeX\ distribution.

Others are declared to be ``compatible'' with \revtex\ (see Section~\ref{sec:pkg,cmp});
we anticipate your need to use these packages, have tested \revtex's compatibility with them,
and are committed to maintaining compatibility.

Still others are declared to be ``deprecated,'' see Section~\ref{sec:pkg,dep};
their use with \revtex\ is discouraged.
A package may be included in this category because it establishes markup
that is incompatible with the electronic submissions scheme of the APS,
or because its definitions are incompatible with those of \revtex\ (they ``break'' \revtex).

The customary way to load a package is through the \cmd\usepackage\ command;
simply invoke this command just after your \cmd\documentclass\ statement.
For instance, if you wish to load the \classname{longtable} package, your
document preamble might look like:
\begin{verbatim}
\documentclass{revtex}
\usepackage{longtable}
\end{verbatim}

Required packages are automatically loaded by \revtex\ on an as-needed basis
and do not need an explicit \cmd\usepackage\ statement in your document.

\subsubsection{Required Packages}\label{sec:pkg,req}

In order to use some of the advanced functions in \revtex~4, you will
have to install certain \LaTeXe\ packages. Most of these packages are
standard in any \LaTeXe\ distribution, but some are not. If you have
problems obtaining any of these packages, please contact \revtex\
support for help. 

\paragraph{\classname{natbib}}\ 
The \classname{natbib} package,
available at \url{ftp://ctan.tug.org/tex-archive/macros/latex/contrib/supported/natbib/},
provides the general framework for citations and references within \revtex,
regardless of the journal substyle.

\paragraph{multicol}\ 
The \classname{multicol} package, a required part of the \LaTeX\ distribution,
is used to balance columns when the \classoption{twocolumn}
option is in effect.
The file \file{mulitcol.sty} is loaded automatically when the
\classoption{twocolumn} option is specified.
Note that this package places limitations on your use of the
\classname{longtable} package, also a required component of \LaTeX,
and is incompatible with the \revtex\ \classoption{floats} option.

\paragraph{graphics/graphicx}\ 
Graphics inclusion should use
the \LaTeX\ \classname{graphicx} packages and the standard \LaTeX\ command 
\cmd\includegraphics. This package is required in all
\LaTeX\ distributions. To load the package, put the line:
\begin{verbatim}
\usepackage{graphicx}
\end{verbatim}
in your document preamble.


\subsubsection{Compatible Packages}\label{sec:pkg,cmp}%
Of the many packages available for use with \LaTeX, only a small subset
are tested for compatibility with \revtex, and they are documented
in this section.
%
If you encounter a bug stemming from the use of one of these packages in
conjunction with any of the APS journals, please contact \revtex\ support.

\paragraph{AMS packages}\
\revtex\ is compatible with and depends upon the AMS packages
\classname{amsfonts},
\classname{amssymb}, and
\classname{amsmath}.

\paragraph{longtable}\ 

\file{longtable.sty} is used for large tables that will span more than one
page and must be loaded using the \cmd\usepackage\
command. 

\paragraph{hyperref}\ 

\file{hyperref.sty} is a package by Sebastian Rahtz that is
used for putting hypertext links into \LaTeXe documents.
\revtex~4 has hooks to allow e-mail addresses and URL's to become hyperlinks.

\paragraph{bm (Bold Math)}\ 

\classname{bm} is used for creating bold symbols in math mode.
It is loaded by using the \cmd\usepackage\ command and is distributed
with \revtex~4.

 
\subsubsection{Deprecated Packages}\label{sec:pkg,dep}%
Because the APS does not have control over the functions of packages, 
it cannot commit to making \revtex\ work with all available packages.
Furthermore, some packages may establish markup conventions that do not
work well with the electronic submissions scheme of the APS.
Therefore, the use of certain packages may be deprecated.

At present we know of no packages in this category.
%But watch this space!



\section{Troubleshooting and Other Questions}\label{sec:debug}

This section is intended to help authors with problems  and common
questions that arise when using \revtex{}.

\paragraph{Question:}\ How do I get lowercase letters
in the \cmd\section\marg{title text} command?

In the APS journal substyles, text in the \cmd\section\marg{title text} command
is automatically set uppercase.
For a lowercase letter use \cmd\lowercase\verb+{x}+.
For example, to use ``He'' for helium in a \cmd\section\marg{title text} command,
type \verb+H+\cmd\lowercase\verb+{e}+ in \marg{title text}.
This also works in math mode:
\verb+$+\cmd\lowercase\verb+{e}^2+\verb+$+ in a \cmd\section\marg{title text} command will output
$e^2$.

\paragraph{Problem:}\ I am getting error messages from my \cmd\section\marg{title text},
\cmd\subsection\marg{title text},
\cmd\subsubsection\marg{title text},
\cmd\footnote\marg{text}, or
\cmd\caption\marg{text} commands, and I can't understand why!

You may have a so-called ``fragile'' command in a section heading or
caption. This is solved in \LaTeX{} by immediately preceding the fragile
command with \cmd\protect. Some common fragile commands include:
\begin{quote}
  \cmd\footnote\ \cmd\footnotemark\ \cmd\footnotetext\\
  \cmd\nocite\\
  \cmd\( \cmd\) \cmd\[ \cmd\] \cmd\\
\end{quote}
as well as any command with an optional argument.
Moreover, \cmd\verb\ must \emph{never} appear in the argument of any command.

If you have one of these commands, or another fragile command (check
\LUG), precede it with \cmd\protect\ and try running the file again.
For example, if you have
\begin{verbatim}
\section{The result:\\Results in an error!}%
\end{verbatim}
change it to
\begin{verbatim}
\section{The result:\protect\\This is OK.}%
\end{verbatim}

\paragraph{Problem:}\ I have tables that do not fit into the preprint width.

Try putting the \cmd\squeezetable\ command right after the
\envb{table} command. This will reduce the size of the type in the
body of the table, thus allowing more data to fit.

\paragraph{Problem:}\ \TeX{} (or my device driver) runs out of font space.

Try removing the \classoption{amsfonts} and \classoption{amssymb} class options. \TeX{}
implementations vary, and some implementations will be unable to provide
the resources needed to run these options.

\paragraph{Problem:}\ \TeX{} runs out of string space (\verb+pool_size+ is too small).

Remove the \classoption{amssymb} class option. It defines hundreds of
symbol names. Some \TeX{} implementations will be unable to provide the
resources needed to run this option.

\paragraph{Problem:}\ (a) The text immediately following an equation is ``outdented''.
That is, indented into the margin.
(b) I get a \verb+missing+ error in the references, but the input is OK.
If I let \TeX{} run through, the output is OK, too.

\revtex{} is having a bad interaction with an older version of \LaTeX{}.
Upgrading to a newer \LaTeX{} has cured these problems in the past.

\paragraph{Problem:}\ One (or more) of my equations is being cross-referenced incorrectly.

Make sure that you have run \LaTeX{} at least twice since the
equation numbering was last disturbed by an input change. Also note that
incorrect cross-referencing will result if \cmd\label\marg{key} is used in an
unnumbered single line equation (i.e., within the \verb+\[+ and \verb+\]+
commands), or if \cmd\label\marg{key} is used on a line of an eqnarray that is
not being numbered (i.e., a line that has a \cmd\nonumber).

\paragraph{Problem:}\ I get a \LaTeX{} message at the end of the run that tells me
that the references may have changed, no matter how many times I run
\LaTeX{}.

Make sure that you have not used the same tag to label two
different things. This will produce this effect, but will also produce a
warning during the run and is therefore easy to detect. Also make sure that
you have not used the same tag for two different \cmd\bibitem s. That is, make
sure that two different \cmd\bibitem\marg{key} commands do not use the same
text for \marg{key}. You will probably \emph{not} get a warning for this,
so this a  more subtle error.


\section{The Compuscript Program}\label{sec:compuscript}
The bright promise of \revtex\ is, of course, that your electronic document can 
qualify for the compuscript program of a participating journal.
This manual does not attempt to cover any aspects of such programs except to
encourage you to ensure that your document's markup is of the highest quality.

You may obtain further information about the compuscript program of
the American Physical Society at \url{http://publish.aps.org/ESUB/}, 
the American Institute of Physics at \url{http://www.aip.org}, 
the Optical Society of America at \url{http://www.osa.org}, 
the Society of Exploration Geologists at \url{http://www.seg.org}.

\section{Contact Information}\label{sec:resources}%

Should you find any bugs, problems or inconsistencies, contact
\revtex\ support at \url{mailto:revtex4@aps.org}.
Please try to include information on what you were doing at the time
and if possible, a small sample document that manifests the problem. 

\begin{thebibliography}{}\label{sec:TeXbooks}
\bibitem[Knuth(1986)]{TeXbook} Knuth, D.E., \href{http://www.awl.com/cseng}{\emph{The \TeX book}}, Addison Wesley Longman, 1986.
\bibitem[Lamport(1996)]{LaTeXman}Lamport, L., \href{http://www.awl.com/cseng}{\emph{\LaTeX, a Document Preparation System}}, Addison Wesley Longman, 1996.
\bibitem[Goossens(1994)]{Compan}  Goossens, M. et al., \href{http://www.awl.com/cseng}{\emph{The \LaTeX\ Companion}}, Addison Wesley Longman, 1994.
\bibitem[Goossens(1997)]{CompanG} Goossens, M. et al., \href{http://www.awl.com/cseng}{\emph{The \LaTeX\ Graphics Companion}}, Addison Wesley Longman, 1997.
\bibitem[Rahtz(1999)]{CompanW} Rahtz, S. et al., \href{http://www.awl.com/cseng}{\emph{The \LaTeX\ Web Companion}}, Addison Wesley Longman, 1999.
\end{thebibliography}

\appendix

\section{Differences From \revtex~3.1}\label{sec:diff31}
If you are already an experienced user of \revtex\ version~3.1 under \LaTeXe,
and have installed \revtex~4, you can immediately start using the new system.
Please take note of the following \hypertarget{Tsec:diff31}{differences}

\subsection{Platform Required}
\revtex~4 works solely with \LaTeXe; it is not useable as a \LaTeX2.09 package.
Furthermore, \revtex~4 requires an up-to-date \LaTeX\ installation (1996/06/01 or later);
its use under older versions is not supported.

\subsection{Markup Differences}%

Documentation of \revtex~3.1 (\url{ftp://aps.org/revtex/manend.tex})
mentions a number of commands particular to that document style
(that is, extensions to the \LaTeX\ article style).
Some of these commands have changed, as noted in Table~\ref{tab:diff31}, and new extensions to
the \LaTeXe\ article class have been introduced with \revtex~4.
%
Furthermore, \revtex~4 uses certain \LaTeX\ commands in a different way than in the
\classname{article} class. These are also noted in Section~\ref{sec:diffart}.

In any case, simply making the transition from using
the \classname{article} document style under \LaTeX2.09
to
using the \classname{article} document class under \LaTeXe\
mandates changes to your legacy document.
You are responsible for such required changes; see Appendix~D of the \LUG\ for details.

\begin{table*}
\begin{tabular}{lp{330pt}}
\toprule
\textbf{\revtex~3.1 command}&\textbf{\revtex~4 replacement}
\lrstrut\\
\colrule
\cmd\documentstyle\oarg{options}\aarg{\classname{revtex}}&\cmd\documentclass\oarg{options}\aarg{\classname{revtex4}}
\\
option  \classoption{aps}      & is now the default
\\
options \classoption{aps}, \classoption{osa}, \classoption{seg}&the society is now implied by the selection of the journal
\\
option  \classoption{manuscript}& \classoption{preprint}
\\
\cmd\tighten\ preamble command & \classoption{tightenlines} class option
\\
\cmd\draft\ preamble command   & \classoption{draft} class option
\\
\cmd\title                     & \cmd\title\ can take an optional argument signifying an alternative title
\\
\cmd\author                    & \cmd\author\marg{name} may appear multiple times; each signifies a new author name.\\
                               & \cmd\lastname\marg{surname} lets you mark up the author's surname\\
                               & \cmd\firstname\marg{firstname} lets you mark up the author's first name\\
                               & \cmd\homepage\marg{URL} gives a URL for the above author\\
                               & \cmd\email\marg{email} gives an email address for the above author\\
\cmd\and                       & obsolete, remove this command\\
\cmd\address                   & \cmd\affiliation\marg{institution}\ gives the affiliation for the group of authors above\\
                               & \cmd\affiliation\oarg{note} lets you specify a footnote to this institution\\
                               & \cmd\noaffiliation\ signifies that the above authors have no affiliation\\
\cmd\altaddress                & \cmd\altaffiliation; applies to a single \cmd\author\\
\cmd\preprint                  & \cmd\preprint\marg{number} can appear multiple times, and must precede \cmd\maketitle\\
\cmd\pacs                      & \cmd\pacs\ must precede \cmd\maketitle\\
\env{abstract} environment     & \env{abstract} environment must precede \cmd\maketitle\\
\cmd\maketitle                 & \cmd\maketitle\ must follow \emph{all} frontmatter data commands\\
\cmd\narrowtext                & obsolete, remove this command\\
\cmd\mediumtext                & obsolete, remove this command\\
\cmd\widetext                  & obsolete, replace with \env{widetext} environment\\
\cmd\FL                        & obsolete, remove this command\\
\cmd\FR                        & obsolete, remove this command\\
\cmd\eqnum                     & replace with \cmd\tag, load \classname{amsmath}\\
\env{mathletters}              & replace with \env{subequations}, load \classname{amsmath}\\
\env{quasitable} environment   & replace with \env{longtable}, load \classname{longtable}\\
\env{references} environment   & replace with \env{thebibliography}\verb+{}+\\
\cmd\case                      & replace with \cmd\textstyle\cmd\frac\\
\cmd\slantfrac                 & replace with \cmd\frac\\
\cmd\tablenote                 & replace with \cmd\footnote\\
\cmd\tablenotemark             & replace with \cmd\footnotemark\\
\cmd\tablenotetext             & replace with \cmd\footnotetext\lrstrut\\
\botrule
\end{tabular}
\caption{Differences between \revtex~3.1 and \revtex~4 markup}\label{tab:diff31}
\end{table*}


\section{Converting a \revtex~3.1 Document to \revtex~4}\label{sec:conv31}%

To convert a \revtex\ 3 document to one compatible with \revtex\ 4, carry out the following actions:
\begin{itemize}
\item
Change \cmd\documentstyle\verb+{revtex}+ to \cmd\documentclass\verb+{revtex4}+,
and run the document under \LaTeXe\ instead of \LaTeX2.09.

\item
Replace the \cmd\draft\ command with the \classoption{draft} class option.

\item
Replace the \cmd\tighten\ command with the \classoption{tightenlines} class option.

\item
For each \cmd\author\ command, split the multiple authors into
individual \cmd\author\ commands. Remove any instances of \cmd\and.

\item
Use \cmd\affiliation\ instead of \cmd\address.

\item
Move \cmd\maketitle\ downstream of all \cmd\pacs\ commands and downstream of any \env{abstract} environment instance.

\item
Convert \env{quasitable} to \env{longtable}, and load the \classname{longtable} package.

\item
Remove all obsolete commands:
\cmd\FL, \cmd\FR, \cmd\narrowtext, and \cmd\mediumtext\ (see Table~\ref{tab:diff31}).

\item
Replace \cmd\case\ with \cmd\frac. If you need the fraction to be set in text style
despite being in a display equation, use the construction \cmd\textstyle\cmd\frac.
Note that \cmd\frac\ does not support the syntax \cmd\case\verb+1/2+.

\item
Replace \cmd\slantfrac\ with \cmd\frac.

\item
Change \cmd\frak\ to \cmd\mathfrak\marg{char} and \cmd\Bbb\ to \cmd\mathbb\marg{char},
and invoke one of the class options \classoption{amsfonts} or \classoption{amssymb}.

\item
Replace environment \env{mathletters} with environment \env{subequations} and load the \classname{amsmath} package.

\item
Replace \cmd\eqnum\ with \cmd\tag\ and load the \classname{amsmath} package.

\item
Replace \cmd\bbox\ with \cmd\bm\ and load the \classname{bm} package.

\item
If using the \cmd\text\ command, load the \classname{amsmath} package.

\item
If using the \verb+d+ column specifier in \env{tabular} environments,
load the \classname{dcolumn} package,
and be aware that the content of each cell in the column is implicitly in math mode:
remove any \verb+$+ math shift characters appearing in a \verb+d+ column.

\item
Replace \cmd\tablenote\ with \cmd\footnote, \cmd\tablenotemark\ with \cmd\footnotemark, and \cmd\tablenotetext\ with \cmd\footnotetext.

\item
Replace \envb{references} with \envb{thebibliography}\verb+{}+;
\enve{references} with \enve{thebibliography}.

\end{itemize}



\section{Differences between \revtex~4 and the standard \LaTeX{} article class}\label{sec:diffart}%

%FIXME: revise for \revtex~4.

If you are familiar with the standard \LaTeX\ \classname{article} document class,
you will find that \revtex\ provides a familiar environment in which to prepare your article.
However, \revtex\ is different from the \classname{article} class, as noted here.

In some respects, \revtex\ simply extends the \classname{article} class
the same way many users do: it incorporates packages from among the \LaTeX\ 
required suite of packages, such as the AMS-authored packages \classname{amsfonts}, \classname{amssymb}, and \classname{amsmath}.
These packages introduce the ability to typeset many math symbols not otherwise available to \LaTeX.
The \classname{amsmath} package provides the \env{subequations} environment and the the \cmd\tag\ command.

Other packages from the the required suite of \LaTeX\ packages include
\classname{bm}, which gives access to bold math through the \cmd\bm\ command;
\classname{longtable}, which lets you create tables that can break over pages;
and
\classname{multicol}, which forms the basis of \revtex's two-column capabilities.

In other respects, \revtex\ simply extends the \classname{article} class.
It defines new class options, such as the many journal substyles,
and defines its own new math symbols, such as \cmd\tensor,
and it defines new commands, such as \cmd\bibinfo,
that let you mark up your document in a way that enhances its value as an electronic document.

However, using \revtex\ will also force you to relearn certain commands and environments,
such as the new markup rules for your frontmatter and bibliography.
In these incompatible extensions to the standard \classname{article} class, 
\revtex\ either gives you a somewhat more convenient way of marking up your paper,
or gives you the ability to do something that is not provided for in the standard \classname{article} class.

\begin{itemize}
\item
The document class declaration is different:
the document class is \classoption{revtex4}.

There is a class option for each APS journal (they are collectively called ``journal substyles''):
\classoption{pra}, \classoption{prb},
\classoption{prc}, \classoption{prd},
\classoption{pre}, \classoption{prl},
\classoption{prstab}, and \classoption{rmp} for
\emph{Physical Review} \emph{A}, \emph{B}, \emph{C}, \emph{D}, \emph{E}, \emph{Letters},
\emph{Special Topics---Accelerators and Beams}, and \emph{Reviews of Modern Physics}, respectively.
The chosen journal substyle may in turn make default selections of a number of class options;
an explicit document class option always overrides this.

New class options are
\classoption{eqsecnum} (number equations by section),
\classoption{preprint} (double-spaced output for submission purposes),
\classoption{tightenlines}  (single-spaced output with the preprint option),
and \classoption{amsfonts} and \classoption{amssymb} (extra font capabilities, see Sec.~\ref{sec:fonts}).

%FIXME: \classoption{pra} is the default.
The \classoption{prb} option gives superscript reference citations, as is the style for \emph{Physical Review B}.
%please use this journal substyle only if you will be submitting to this journal.
The \classoption{prl} option yields a slightly different line spacing, giving more
accurate PRL length estimates.
%and \classoption{prl} causes your paper to be formatted in two-column layout.
Apart than this, there are no substantial differences between the substyles for \emph{Physical Review A--E}.

The \classoption{floats} class option enables \LaTeX{}-style floating figures and tables.
%---it is \emph{not} for use with files to be submitted to the APS: instead, use 
The \classoption{nofloats} option causes floating elements to be formatted at the end of the document.

The \classoption{twocolumn} class option causes the document to be formatted in a 
two-column layout; \classoption{onecolumn} in a one-column layout.
%for personal use, and not for use in submitted files. 

\item
The frontmatter is different in \revtex;
a simple one might look like (cf.~\file{template.aps})
\begin{verbatim}
\documentclass[draft,pra,aps]{revtex4}
\begin{document}
\title{Title here}
\author{Author(s) here}
 \affiliation{Address(es) here}
\author{Another author(s) here}
 \affiliation{Another address(es) here}
\date{\today}
\begin{abstract}
Abstract here.
\end{abstract}
\pacs{PACS numbers here}
\maketitle
\end{verbatim}
Note the \cmd\affiliation\marg{text}, and \cmd\pacs\marg{pacs number} commands are new,
and the \cmd\maketitle\ command \emph{follows} the \env{abstract}.
Also, each author appears in a separate \cmd\author\ command; the \cmd\and\ command is not used.
See Sec.~\ref{sec:front} for details.

\item
Figures and tables are input the same as in \LaTeX{}, however, with the 
\classoption{nofloats} option they are automatically moved to the end of the document;
see Sections~\ref{ref:nofloats} and~\ref{ref:printfig} for more details.
%While you may enable floats with the \classoption{floats} class option,
%you \emph{may not} use this for files that you submit to the APS:
%it is only for your personal use.
%Floating tables and figures do not break across pages.
%All tables expand to fill the column width.
%FIXME: tables break? and fill the column width?

\item
The \cmd\text\marg{text} command formats \meta{text} in text mode within math.
In particular, you get hyphens instead of minus signs.
Used in a superscript, you get the correct size. See Sec.~\ref{sec:textinmath}.

\item
Using a \cmd\label\marg{key} within the \envb{subequations} environment allows
you to reference the \emph{general} number of the equations in the
\classoption{subequations} environment. For example:
%
\begin{verbatim}
\begin{subequations}
 \label{alleqs}  % observe location
 \begin{eqnarray}
  E    & = &mc^{2},\label{eqa}\\
  c^{2}& = &a^{2} + b^{2},\label{eqb}\\
  E    & = &m(a^{2} + b^{2}),\label{eqc}
 \end{eqnarray}
\end{subequations}
\end{verbatim}
%
gives the output
%
\smallskip\hrule\smallskip
\begin{subequations}
\label{alleqs}  % observe location
\begin{eqnarray}
E    & = &mc^{2},\label{eqa}\\
c^{2}& = &a^{2} + b^{2},\label{eqb}\\
E    & = &m(a^{2} + b^{2}),\label{eqc}
\end{eqnarray}
\end{subequations}
\smallskip\hrule\smallskip
%
and \verb+Eq.\ (\ref{alleqs})+ gives ``Eq.\ (\ref{alleqs})''.

\item
Using \texttt{d} in a tabular specification creates a column centered on
the decimal points of the entries. See Sec.~\ref{sec:tables} for details;
see \file{apssamp.tex} for examples.

\item
These additional diacritics are available:
\cmd\tensor\ (double-headed overarrow),
\cmd\overdots\ (triple overdots),
\cmd\overstar\ (star),
\cmd\overcirc\ (circle),
\cmd\loarrow\ (left-going overarrow), and
\cmd\roarrow\ (right-going overarrow).
They scale correctly in superscripts.
See Appendix~\ref{sec:chars} for examples.

%\item
%\cmd\case\marg{num}\marg{den} gives text-style fractions (smaller) in display-style math.

\item
Style files for use with \BibTeX{} are bundled with the various journal substyles.
The journal substyle automatically issues the needed \cmd\bibliographystyle\ command.

\item
For hand-prepared bibliographies, \file{reftest.tex}
checks that your document has
(1)~no uncited bibitems,
(2)~no undefined citations, and
(3)~its \cmd\bibitem s in the same order as its citations.
%These are all requirements in \emph{Physical Review} style.
See Sec.~\ref{sec:endnotes}.

%\item
%The \LaTeX{} command \cmd\extracolsep\marg{dimension} sets extra intercolumn
%spacing, but this extra spacing has already been set in \revtex{} to allow
%the columns in the table to expand out to fill the text width. Therefore,
%\cmd\extracolsep\marg{dimension} will not work in \revtex{}.
%Use the \verb+@+\marg{spacing} expression to specify spacing between two adjacent columns,
%for personal files. 
%See Appendix C.9.2 of Lamport for a full explanation of \verb+@+ expressions.
%An example has been given in \file{apssamp.tex}.
%Note that final journal table layout is set by production software. 
%%FIXME: relates to implicit tabular*

\end{itemize}

The American Physical Society intends for \revtex{} to be
as compatible as possible with \LaTeX{} and with 
packages that can be used with \LaTeX.
%You \emph{cannot} use these options for files that you submit to the APS;
%they are added for your personal use.
Please let us know of any \LaTeX\ commands incompatible with \revtex,
or of any packages useable with the \LaTeX\ article class that 
are incompatible with \revtex.

\section{Specifying Authors and Affiliations}\label{sec:authorgroup}
This section provides more detail on how to specify authors and affiliations
for your document, and shows how to obtain various title block formatting 
effects with the class options.

The following examples exhibit a representative cross section of frontmatter blocks.
They are taken from actual journal papers; the journal involved is indicated.

[to come]


\section{Adding New Journal Styles}%

Earlier versions of \revtex\ provided formatting for a
large group of societies and journals.
\revtex\ 4 establishes a new, open architecture
for adding journal substyles.

To add a new journal substyle to \revtex: 
Create a file with a \file{.rtx} extension and 
put into it whatever macro definitions or parameter assignments are required.
%
To use the journal substyle, your document should invoke a corresponding document class option,
causing your \file{.rtx} file to be read in.

For instance, in the case of a fictitious publication called the
``Journal of Irreproducible Results'',
you could create a file called \file{jir.rtx} and invoke that substyle 
via a \cmd\documentclass\ statement like 
\begin{verbatim}
\documentclass[jir]{revtex4}
\end{verbatim}

To create a useful substyle \file{.rtx} file, you might want to use as a model
the American Physical Society substyle \file{aps.rtx}.

Notes:
\begin{itemize}

\item
Journal substyles should ideally not create new markup syntax.
All document-level environments and commands should be defined in \revtex\ itself.

If your journal requires markup (compuscript structure) that goes beyond that 
supplied by \revtex, please contact the maintainers of \revtex.

\item
The file \file{aps.rtx} has specific code at the beginning that insists on 
being run under \revtex; your substyle should do likewise.

\item
Your journal substyle, like \file{aps.rtx}, is read in after all of the 
code of the \file{revtex.cls}; it can depend on all of the definitions
in that file to be in effect, and can redefine them as needed.

\item
Your journal substyle, like \file{aps.rtx}, can invoke certain formatting options,
but may do so only if the document's options do not specify a preference:
the document's options must override any choices made by the journal substyle.

\item
In some cases, journal-specifc code is sufficiently extensive that it is
useful to break it out into a separate file, as in the case of \file{rmp.rtx}.
This file has code that insists that it run under \file{aps.rtx}; your journal-specific
substyle should do likewise.

\item
Hint: If your journal style has no head above the abstract, you can simply
define the procedure \cmd\frontmatter@abstractheading\ to do nothing:
\begin{verbatim}
\def\frontmatter@abstractheading{}%
\end{verbatim}

\item
If the journal involved has a compuscript program whose requirements bear on
documents prepared according to your journal substyle, 
the documentation for your substyle should include those requirements
(or a pointer to them).

\end{itemize}

\section{Character Set Listing}\label{sec:chars}

This appendix provides tables showing all of the \hypertarget{Tsec:chars}{special characters}
and mathematical symbols that are available within \revtex{}.
Some of these symbols require the AMS fonts to be available.

If you are preparing a paper for submission to a journal,
you should check that journal's preferences in using special symbols.
Typically, a journal will prefer that you use a symbol command taken from the following lists
and will deprecate your inventing new command names.

%\revtex{} version~4 supports an extensive set of symbols, alphabets, and
%special fonts. Their availability does not relieve an author (or editor)
%of considering whether a chosen notation or symbol will  convey the
%intended meaning, and whether there is a more conventional alternative.
%As always, for the benefit of the reader, notation should be clear, as simple
%as possible, and consistent with standard usage. Nonstandard symbols should
%only be used if necessary; their meaning should be explained in the paper
%at the first occurrence.
%
%Editorial policy on this issue may vary from journal to journal. Check recent
%issues of a given journal and/or query the editor. APS authors
%may also consult the journal's
%``Information for Contributors'' as well as the \SNG{}.
%In preparing an accepted paper for publication, the
%editor may require the use of alternative notation.

\subsection{\LaTeX notations}

\subsubsection{Standard \LaTeX symbols}
Tables~\ref{tab:ltx.accents} through \ref{tab:ltx.misc} show
the standard symbols for \LaTeX{} users.

\begin{table}
\caption{Text accents with letter a.}\label{tab:ltx.accents}
\def\xxx{4pt}
\begin{tabular}{c@{\hspace{\xxx}}lc@{\hspace{\xxx}}lc@{\hspace{\xxx}}%
                lc@{\hspace{\xxx}}l}
\toprule
\`{a} & \verb+\`{a}+&
\'{a} & \verb+\'{a}+&
\^{a} & \verb+\^{a}+&
\"{a} & \verb+\"{a}+\\
\~{a} & \verb+\~{a}+&
\={a} & \verb+\={a}+&
\.{a} & \verb+\.{a}+&
\u{a} & \cmd\u\verb+{a}+\\
\v{a} & \cmd\v\verb+{a}+&
\H{a} & \cmd\H\verb+{a}+&
\t{aa}& \cmd\t\verb+{aa}+&
\c{a} & \cmd\c\verb+{a}+\\
\d{a} & \cmd\d\verb+{a}+&
\b{a} & \cmd\b\verb+{a}+\lrstrut\\
\botrule
\end{tabular}
\end{table}

\begin{table}
\caption{Math accents with letter a.}
\def\xxx{4pt}
\begin{tabular}%
  {c@{\hspace{\xxx}}lc@{\hspace{\xxx}}lc@{\hspace{\xxx}}lc@{\hspace{\xxx}}l}
\toprule
$\hat{a}$       &\cmd\hat\verb+{a}+ &
$\check{a}$     &\cmd\check\verb+{a}+ &
$\dot{a}$       &\cmd\dot\verb+{a}+ &
$\ddot{a}$      &\cmd\ddot\verb+{a}+ \\
$\breve{a}$     &\cmd\breve\verb+{a}+ &
$\tilde{a}$     &\cmd\tilde\verb+{a}+ &
$\grave{a}$     &\cmd\grave\verb+{a}+ &
$\acute{a}$     &\cmd\acute\verb+{a}+ \\
$\bar{a}$       &\cmd\bar\verb+{a}+ &
$\vec{a}$       &\cmd\vec\verb+{a}+ &\lrstrut\\
\botrule
\end{tabular}
\end{table}

\begin{table}
\caption{Special symbols; any mode.}
\def\xxx{4pt}
\begin{tabular}{c@{\hspace{\xxx}}lc@{\hspace{\xxx}}lc@{\hspace{\xxx}}l}
\toprule
$\dagger$ & \cmd\dagger& \S & \cmd\S& \copyright & \cmd\copyright\\
$\ddagger$ & \cmd\ddagger& \P & \cmd\P& \pounds & \cmd\pounds\lrstrut\\
\botrule
\end{tabular}
\end{table}

\begin{table}
\caption{Other special (foreign) symbols; text mode.}
\def\xxx{4pt}
\begin{tabular}%
  {c@{\hspace{\xxx}}lc@{\hspace{\xxx}}lc@{\hspace{\xxx}}lc@{\hspace{\xxx}}l}
\toprule
\aa & \cmd\aa&
\AA & \cmd\AA&
\ae & \cmd\ae&
\AE & \cmd\AE\\
 \o & \cmd\o&
 \O & \cmd\O&
\oe & \cmd\oe&
\OE & \cmd\OE\\
 \l & \cmd\l&
 \L & \cmd\L&
 ?` & \verb+?`+&
 !` & \verb+!`+\\
 \ss & \cmd\ss\lrstrut\\
\botrule
\end{tabular}
\end{table}

\begin{table}
\caption{Greek letters; used in math mode.}
\def\xxx{4pt}
\begin{tabular}%
  {c@{\hspace{\xxx}}lc@{\hspace{\xxx}}lc@{\hspace{\xxx}}lc@{\hspace{\xxx}}l}
\hline\hline
\multicolumn{4}{c}\emph{Lowercase}\frstrut\lrstrut\\
\colrule
$\alpha$            &\cmd\alpha&
$\beta$             &\cmd\beta&
$\gamma$            &\cmd\gamma&
$\delta$            &\cmd\delta\\
$\epsilon$          &\cmd\epsilon&
$\varepsilon$       &\cmd\varepsilon&
$\zeta$             &\cmd\zeta&
$\eta$              &\cmd\eta\\
$\theta$            &\cmd\theta&
$\vartheta$         &\cmd\vartheta&
$\iota$             &\cmd\iota&
$\kappa$            &\cmd\kappa\\
$\lambda$           &\cmd\lambda&
$\mu$               &\cmd\mu&
$\nu$               &\cmd\nu&
$\xi$               &\cmd\xi\\
$o$                 &\texttt{o} &
$\pi$               &\cmd\pi&
$\varpi$            &\cmd\varpi&
$\rho$              &\cmd\rho\\
$\varrho$           &\cmd\varrho&
$\sigma$            &\cmd\sigma&
$\varsigma$         &\cmd\varsigma&
$\tau$              &\cmd\tau\\
$\upsilon$          &\cmd\upsilon&
$\phi$              &\cmd\phi&
$\varphi$           &\cmd\varphi&
$\chi$              &\cmd\chi\\
$\psi$              &\cmd\psi&
$\omega$            &\cmd\omega&
\\[\baselineskip]
\multicolumn{4}{c}\emph{Uppercase}\lrstrut\\
\colrule
$\Gamma$            &\cmd\Gamma&
$\Delta$            &\cmd\Delta&
$\Theta$            &\cmd\Theta&
$\Lambda$           &\cmd\Lambda\\
$\Xi$               &\cmd\Xi&
$\Pi$               &\cmd\Pi&
$\Sigma$            &\cmd\Sigma&
$\Upsilon$          &\cmd\Upsilon\\
$\Phi$              &\cmd\Phi&
$\Psi$              &\cmd\Psi&
$\Omega$            &\cmd\Omega\lrstrut\\
\botrule
\end{tabular}
\end{table}

\begin{table}
\caption{Binary operation symbols; used in math mode.}
\def\xxx{3.2pt}
\begin{tabular}%
  {c@{\hspace{\xxx}}lc@{\hspace{\xxx}}lc@{\hspace{\xxx}}lc@{\hspace{\xxx}}l}
\toprule
$\pm$               &\cmd\pm&
$\mp$               &\cmd\mp&
$\times$            &\cmd\times&
$\div$              &\cmd\div\\
$\ast$              &\cmd\ast&
$\star$             &\cmd\star&
$\circ$             &\cmd\circ&
$\bullet$           &\cmd\bullet\\
$\cap$              &\cmd\cap&
$\cup$              &\cmd\cup&
$\uplus$            &\cmd\uplus&
$\cdot$             &\cmd\cdot\\
$\sqcap$            &\cmd\sqcap&
$\sqcup$            &\cmd\sqcup&
$\vee$              &\cmd\vee&
$\wedge$            &\cmd\wedge\\
$\oplus$            &\cmd\oplus&
$\ominus$           &\cmd\ominus&
$\otimes$           &\cmd\otimes&
$\oslash$           &\cmd\oslash\\
$\bigtriangleup$    &\cmd\bigtriangleup&
$\odot$             &\cmd\odot&
$\lhd$              &\cmd\lhd&
$\dagger$           &\cmd\dagger\\
$\bigtriangledown$  &\cmd\bigtriangledown&
$\bigcirc$          &\cmd\bigcirc&
$\rhd$              &\cmd\rhd&
$\ddagger$          &\cmd\ddagger\\
$\triangleleft$     &\cmd\triangleleft&
$\diamond$          &\cmd\diamond&
$\unlhd$            &\cmd\unlhd&
$\setminus$         &\cmd\setminus\\
$\triangleright$    &\cmd\triangleright&
$\wr$               &\cmd\wr&
$\unrhd$            &\cmd\unrhd&
$\amalg$            &\cmd\amalg\lrstrut\\
\botrule
\end{tabular}
\end{table}

\begin{table}
\caption{Relation symbols; used in math mode. }\label{tab:ltx.rels}
\def\xxx{4pt}
\begin{tabular}%
  {c@{\hspace{\xxx}}lc@{\hspace{\xxx}}lc@{\hspace{\xxx}}lc@{\hspace{\xxx}}l}
\toprule
$\leq$              &\cmd\leq&
$\geq$              &\cmd\geq&
$\ll$               &\cmd\ll&
$\gg$               &\cmd\gg\\
$\equiv$            &\cmd\equiv&
$\asymp$            &\cmd\asymp&
$\neq$              &\cmd\neq&
$\doteq$            &\cmd\doteq\\
$\subset$           &\cmd\subset&
$\supset$           &\cmd\supset&
$\subseteq$         &\cmd\subseteq&
$\supseteq$         &\cmd\supseteq\\
$\sqsubset$         &\cmd\sqsubset&
$\sqsupset$         &\cmd\sqsupset&
$\sqsubseteq$       &\cmd\sqsubseteq&
$\sqsupseteq$       &\cmd\sqsupseteq\\
$\models$           &\cmd\models&
$\perp$             &\cmd\perp&
$\mid$              &\cmd\mid&
$\parallel$         &\cmd\parallel\\
$\prec$             &\cmd\prec&
$\succ$             &\cmd\succ&
$\preceq$           &\cmd\preceq&
$\succeq$           &\cmd\succeq\\
$\sim$              &\cmd\sim&
$\simeq$            &\cmd\simeq&
$\approx$           &\cmd\approx&
$\cong$             &\cmd\cong\\
$\bowtie$           &\cmd\bowtie&
$\Join$             &\cmd\Join&
$\smile$            &\cmd\smile&
$\frown$            &\cmd\frown\\
$\in$               &\cmd\in&
$\ni$               &\cmd\ni&
$\vdash$            &\cmd\vdash&
$\dashv$            &\cmd\dashv\\
$\propto$           &\cmd\propto\lrstrut\\
\botrule
\end{tabular}
\end{table}

Negated relations can sometimes
be constructed with \cmd\not. For example,
\[
\hbox{\verb+If $x \not< y$ then $x \not\leq z$.+}
\]
gives
\[
\hbox{If $x \not< y$ then $x \not\leq z$.}
\]
The AMS fonts have many negated relations already constructed.
See Appendix~\ref{AMSFonts,notations}.

%mtp

\begin{table}
\caption{Arrow symbols; used in math mode.}
\def\xxx{4pt}
\begin{tabular}{c@{\hspace{\xxx}}lc@{\hspace{\xxx}}l}
\toprule
$\leftarrow$        &\cmd\leftarrow&
$\rightarrow$       &\cmd\rightarrow\\
$\longleftarrow$    &\cmd\longleftarrow&
$\longrightarrow$   &\cmd\longrightarrow\\
$\Leftarrow$        &\cmd\Leftarrow&
$\Rightarrow$       &\cmd\Rightarrow\\
$\Longleftarrow$    &\cmd\Longleftarrow&
$\Longrightarrow$   &\cmd\Longrightarrow\\
$\hookleftarrow$    &\cmd\hookleftarrow&
$\hookrightarrow$   &\cmd\hookrightarrow\\
$\leftharpoonup$    &\cmd\leftharpoonup&
$\rightharpoonup$   &\cmd\rightharpoonup\\
$\leftharpoondown$  &\cmd\leftharpoondown&
$\rightharpoondown$ &\cmd\rightharpoondown\\
$\rightleftharpoons$&\cmd\rightleftharpoons&
$\leadsto$          &\cmd\leadsto\\
$\leftrightarrow$   &\cmd\leftrightarrow&
$\longleftrightarrow$&\cmd\longleftrightarrow\\
$\Leftrightarrow$   &\cmd\Leftrightarrow&
$\Longleftrightarrow$&\cmd\Longleftrightarrow\\
$\mapsto$           &\cmd\mapsto&
$\longmapsto$       &\cmd\longmapsto\\
\multicolumn{4}{c}{%
\begin{tabular}{c@{\hspace{\xxx}}l}
$\uparrow$          &\cmd\uparrow\\
$\downarrow$        &\cmd\downarrow \\
$\Uparrow$          &\cmd\Uparrow\\
$\Downarrow$        &\cmd\Downarrow\\
$\updownarrow$      &\cmd\updownarrow\\
$\Updownarrow$      &\cmd\Updownarrow\\
$\nearrow$          &\cmd\nearrow\\
$\searrow$          &\cmd\searrow\\
$\swarrow$          &\cmd\swarrow\\
$\nwarrow$          &\cmd\nwarrow
\end{tabular}%
} % end of multicolumn
\lrstrut\\
\botrule
\end{tabular}
\end{table}

\begin{table}
\caption{Miscellaneous symbols; used in math mode.}
\def\xxx{4pt}
\begin{tabular}%
  {c@{\hspace{\xxx}}lc@{\hspace{\xxx}}lc@{\hspace{\xxx}}lc@{\hspace{\xxx}}l}
\toprule
$\flat$             &\cmd\flat&
$\natural$          &\cmd\natural&
$\sharp$            &\cmd\sharp&
$\prime$            &\cmd\prime\\
$\backslash$        &\cmd\backslash&
$\forall$           &\cmd\forall&
$\infty$            &\cmd\infty&
$\exists$           &\cmd\exists\\
$\emptyset$         &\cmd\emptyset&
$\Box$              &\cmd\Box&
$\nabla$            &\cmd\nabla&
$\neg$              &\cmd\neg\\
$\Diamond$          &\cmd\Diamond&
$\surd$             &\cmd\surd&
$\triangle$         &\cmd\triangle&
$\|$                &\verb+\|+ \\
$\clubsuit$         &\cmd\clubsuit&
$\aleph$            &\cmd\aleph&
$\wp$               &\cmd\wp&
$\top$              &\cmd\top\\
$\diamondsuit$      &\cmd\diamondsuit&
$\Re$               &\cmd\Re&
$\ell$              &\cmd\ell&
$\bot$              &\cmd\bot\\
$\heartsuit$        &\cmd\heartsuit&
$\Im$               &\cmd\Im&
$\imath$            &\cmd\imath&
$\partial$          &\cmd\partial\\
$\spadesuit$        &\cmd\spadesuit&
$\hbar$             &\cmd\hbar&
$\jmath$            &\cmd\jmath&
$\angle$            &\cmd\angle\\
$\mho$              &\cmd\mho\lrstrut\\
\botrule
\end{tabular}
\end{table}

\begin{table}
\caption{%
Log-like functions; used in math mode
(for example, \cmd\log\ \texttt{x} gives $\log x$).
}
\begin{tabular}{llllllll}
\toprule
\cmd\arccos&
\cmd\arcsin&
\cmd\arctan&
\cmd\arg&
\cmd\cos\\
\cmd\cosh&
\cmd\cot&
\cmd\coth&
\cmd\csc&
\cmd\deg\\
\cmd\det&
\cmd\dim&
\cmd\exp&
\cmd\gcd&
\cmd\hom\\
\cmd\inf&
\cmd\ker&
\cmd\lg&
\cmd\lim&
\cmd\liminf\\
\cmd\limsup&
\cmd\ln&
\cmd\log&
\cmd\max&
\cmd\min\\
\cmd\Pr&
\cmd\sec&
\cmd\sin&
\cmd\sinh&
\cmd\sup\\
\cmd\tan&
\cmd\tanh\lrstrut\\
\botrule
\end{tabular}
\end{table}

\begin{table}
\caption{Delimiters; used in math mode.}
\def\xxx{4pt}
\begin{tabular}{c@{\hspace{\xxx}}lc@{\hspace{\xxx}}lc@{\hspace{\xxx}}l}
\toprule
$($                 &\verb+(+ &
$)$                 &\verb+)+ &
$/$                 &\verb+/+ \\
$[$                 &\verb+[+ &
$]$                 &\verb+]+ &
$\backslash$        &\cmd\backslash\\
$\{$                &\verb+\{+ &
$\}$                &\verb+\}+ &
$|$                 &\verb+|+ \\
$\langle$           &\cmd\langle&
$\rangle$           &\cmd\rangle&
$\|$                &\verb+\|+ \\
$\uparrow$          &\cmd\uparrow&
$\Uparrow$          &\cmd\Uparrow&
$\lfloor$           &\cmd\lfloor\\
$\downarrow$        &\cmd\downarrow&
$\Downarrow$        &\cmd\Downarrow&
$\rfloor$           &\cmd\rfloor\\
$\updownarrow$      &\cmd\updownarrow&
$\Updownarrow$      &\cmd\Updownarrow&
$\lceil$            &\cmd\lceil\\
&                   &
&                   &
$\rceil$            &\cmd\rceil\lrstrut\\
\botrule
\end{tabular}
\end{table}

\begin{table}
\caption{Miscellaneous symbols; used in math mode.}\label{tab:ltx.misc}
\def\xxx{4pt}
\begin{tabular}{c@{\hspace{\xxx}}lc@{\hspace{\xxx}}lc@{\hspace{\xxx}}l}
\toprule
$\textstyle\sum$       $\displaystyle\sum$          &\cmd\sum&
$\textstyle\prod$      $\displaystyle\prod$         &\cmd\prod&
$\textstyle\coprod$    $\displaystyle\coprod$       &\cmd\coprod\\
$\textstyle\int$       $\displaystyle\int$          &\cmd\int&
$\textstyle\oint$      $\displaystyle\oint$         &\cmd\oint&
$\textstyle\biguplus$  $\displaystyle\biguplus$     &\cmd\biguplus\\
$\textstyle\bigcap$    $\displaystyle\bigcap$       &\cmd\bigcap&
$\textstyle\bigcup$    $\displaystyle\bigcup$       &\cmd\bigcup&
$\textstyle\bigsqcup$  $\displaystyle\bigsqcup$     &\cmd\bigsqcup\\
$\textstyle\bigodot$   $\displaystyle\bigodot$      &\cmd\bigodot&
$\textstyle\bigotimes$ $\displaystyle\bigotimes$    &\cmd\bigotimes&
$\textstyle\bigoplus$  $\displaystyle\bigoplus$     &\cmd\bigoplus\\
$\textstyle\bigvee$    $\displaystyle\bigvee$       &\cmd\bigvee&
$\textstyle\bigwedge$  $\displaystyle\bigwedge$     &\cmd\bigwedge\lrstrut\\
\botrule
\end{tabular}
\end{table}


\subsubsection{Standard \LaTeX typefaces}

\LaTeX\ provides a pair of special typefaces, \cmd\mathcal\ and \cmd\mathsf.

Use the \cmd\mathcal\ command for script (calligraphic) letters (note the $\mathcal{L}$):
\begin{verbatim}
\mathcal{L}_{\mathrm{int}} = e F^{3}_{\pi} r^{2}
  B^{0}(r,t) \epsilon \sin(\Omega t)
  \exp(\eta t),
\end{verbatim}
gives
\[
\mathcal{L}_{\mathrm{int}} = eF^{3}_{\pi} r^{2}
 B^{0}(r,t)\epsilon\sin(\Omega t)\exp(\eta t),
\]
Only uppercase letters are available in the \cmd\mathcal\ font.

You can switch to sans serif letters by using the \cmd\mathsf\
command (note the $\mathsf{M}$):
\begin{verbatim}
R(\mathcal{Q}-\mathcal{Q}_{0})
=
R_{0} \exp\left(-\frac{1}{2}\Delta \mathcal{Q} \cdot \mathsf{M}
\cdot \Delta \mathcal{Q}\right).
\end{verbatim}
gives
\[
  R(\mathcal{Q}-\mathcal{Q}_{0}) =
  R_{0} \exp\left(-\frac{1}{2}\Delta \mathcal{Q} \cdot \mathsf{M}
  \cdot \Delta \mathcal{Q}\right).
\]
Both uppercase and lowercase letters are available with \cmd\mathsf.

\subsubsection{Other notations}

% math mode only?
The \cmd\overline\ command puts a horizontal line above its argument
in math mode:
\begin{verbatim}
$\overline{x}+\overline{y}$
\end{verbatim}
gives
\[
\overline{x}+\overline{y}
\]

There is an analogous \cmd\underline\ command that works in text
or math mode:
\begin{verbatim}
The equation \underline{is} $\underline{x+y}$.
\end{verbatim}
gives
\[
\hbox{The equation \underline{is} $\underline{x+y}$.}
\]

% math mode?
Horizontal braces are put above or below an expression with the
\cmd\overbrace\ and \cmd\underbrace\ commands:
\begin{verbatim}
\[
\underbrace{a_{1} + \overbrace{a_{2}+a_{3}} + a_{4}}
\]
\end{verbatim}
gives
\[
\underbrace{a_{1} + \overbrace{a_{2}+a_{3}} + a_{4}}
\]
and in displayed math, a subscript or a superscript puts a label on
the brace:
\begin{verbatim}
\[
\underbrace{
 a_{1} + \overbrace{a_{2}+\cdots+a_{n-1}}^{n-2}
 + a_{n}
}_{n}
\]
\end{verbatim}
gives
\[
\underbrace{
 a_{1} + \overbrace{a_{2}+\cdots+a_{n-1}}^{n-2}
 + a_{n}
}_{n}
\]

Wide versions of the \cmd\hat\ and \cmd\tilde\ commands are available.
They are called \cmd\widehat\ and \cmd\widetilde, respectively.
Here is an example:
\begin{verbatim}
\[
\widehat{a} + \widehat{ab} + \widehat{abc} + \widehat{abcd}
\]
\end{verbatim}
gives
\[
\widehat{a} + \widehat{ab} + \widehat{abc} + \widehat{abcd}
\]


\subsection{AMS fonts notations}\label{AMSFonts,notations}

The AMS fonts are fonts that were developed by the American Mathematical Society
and are now made available free of charge by the AMS.
The METAFONT source files for these
fonts are freely available, as are precompiled \file{.pk} files
and ATM-compatible Type 1 PostScript fonts.
There are two class options that can be used to invoke the AMS fonts:
\classoption{amsfonts} and \classoption{amssymb}.
Not distributed
with \revtex{} are the files \file{amsfonts.sty} and \file{amssymb.sty} of the
\AmSLaTeX{} distribution.

\subsubsection{Using the \classoption{amsfonts} option}\label{sec:AMSFonts,amsfonts}

The \classoption{amsfonts} class option will give you
access to the \cmd\mathfrak\ and \cmd\mathbb\ fonts and will also
use the extra Computer Modern fonts from the AMS in order to provide
better access to bold math characters at smaller sizes and in
super- and subscripts.

\paragraph{AMS fonts typefaces.}\ 
With the AMS fonts installed and in use through either the \classoption{amsfonts} or
\classoption{amssymb} class option, the \cmd\mathfrak\ and \cmd\mathbb\ commands
are available.
The command \cmd\mathfrak\ switches to the AMS Fraktur font, while
\cmd\mathbb\ switches to the so-called ``Blackboard Bold'' font.
Only uppercase letters are available in Blackboard Bold, and there is
no bold version of the font. Fraktur has both uppercase and lowercase letters
and will become bold in a bbox.

Here are the letters ``ABCDE'' from \cmd\mathfrak: $\mathfrak{ABCDE}$.
And here are the letters ``RIZN'' from \cmd\mathbb: $\mathbb{RIZN}$.

Here is some math with superscripts and \cmd\mathfrak. It demonstrates
the output of \cmd\bm\marg{symbol}.
%FIXME: no difference!
\begin{quote}
 Normal:  $\mathfrak{E}=mc^{2\pi}$, \qquad
 \cmd\bm: $\bm{\mathfrak{E}}=mc^{2\bm{\pi}}$
\end{quote}

\subsubsection{Using the \classoption{amssymb} option}\label{sec:AMSFonts,symb}

The \classoption{amssymb} class option gives all the font capabilities of the
\classoption{amsfonts} option. It also defines names for many extra symbols that
are present in the AMS fonts. The names are the same as those the AMS uses.
These symbols and their names are shown below, given that you have the AMS fonts
installed and the \classoption{amssymb} option selected.

% some of these characters are very high
\renewcommand{\arraystretch}{1.2}

Please be aware that no bold versions are available for any of the characters in this subsection.

% Lowercase Greek
\begin{table}
\caption{Extra lowercase Greek letters available with \classoption{amssymb} option selected.}
\def\xxx{4pt}
\begin{tabular}{c@{\hspace{\xxx}}lc@{\hspace{\xxx}}l}
\toprule
$\digamma$  & \cmd\digamma &
$\varkappa$ & \cmd\varkappa\lrstrut\\
\botrule
\end{tabular}
\end{table}

% Hebrew letters
\begin{table}
\caption{Extra Hebrew letters available with \classoption{amssymb} selected.}
\def\xxx{4pt}
\begin{tabular}{c@{\hspace{\xxx}}lc@{\hspace{\xxx}}l}
\toprule
$\beth$   & \cmd\beth &
$\gimel$  & \cmd\gimel\\
$\daleth$ & \cmd\daleth\lrstrut\\
\botrule
\end{tabular}
\end{table}


% Bin Rel
\begin{table}
\caption{Binary relations available with \classoption{amssymb} selected.}
\def\xxx{4pt}
\begin{tabular}{c@{\hspace{\xxx}}lc@{\hspace{\xxx}}l}
\toprule
$\leqq$                         &\cmd\leqq&
$\geqq$                         &\cmd\geqq\\
$\leqslant$                         &\cmd\leqslant&
$\geqslant$                         &\cmd\geqslant\\
$\eqslantless$                         &\cmd\eqslantless&
$\eqslantgtr$                         &\cmd\eqslantgtr\\
$\lesssim$                         &\cmd\lesssim&
$\gtrsim$                         &\cmd\gtrsim\\
$\lessapprox$                         &\cmd\lessapprox&
$\gtrapprox$                         &\cmd\gtrapprox\\
$\approxeq$                         &\cmd\approxeq\\
$\lessdot$                         &\cmd\lessdot&
$\gtrdot$                         &\cmd\gtrdot\\
$\lll$                         &\cmd\lll, \cmd\llless &
$\ggg$                         &\cmd\ggg, \cmd\gggtr \\
$\lessgtr$                         &\cmd\lessgtr&
$\gtrless$                         &\cmd\gtrless\\
$\lesseqgtr$                         &\cmd\lesseqgtr&
$\gtreqless$                         &\cmd\gtreqless\\[4pt]
$\lesseqqgtr$                         &\cmd\lesseqqgtr&
$\gtreqqless$                         &\cmd\gtreqqless\\
$\preccurlyeq$                         &\cmd\preccurlyeq&
$\succcurlyeq$                         &\cmd\succcurlyeq\\
$\curlyeqprec$                         &\cmd\curlyeqprec&
$\curlyeqsucc$                         &\cmd\curlyeqsucc\\
$\precsim$                         &\cmd\precsim&
$\succsim$                         &\cmd\succsim\\
$\precapprox$                         &\cmd\precapprox&
$\succapprox$                         &\cmd\succapprox\\
$\subseteqq$                         &\cmd\subseteqq&
$\supseteqq$                         &\cmd\supseteqq\\
$\Subset$                         &\cmd\Subset&
$\Supset$                         &\cmd\Supset\\
$\sqsubset$                         &\cmd\sqsubset&
$\sqsupset$                         &\cmd\sqsupset\\
$\backsim$                         &\cmd\backsim&
$\thicksim$                         &\cmd\thicksim\\
$\backsimeq$                         &\cmd\backsimeq&
$\thickapprox$                         &\cmd\thickapprox\\
$\doteqdot$                         &\cmd\doteqdot, \cmd\Doteq &
$\eqcirc$                         &\cmd\eqcirc\\
$\risingdotseq$                         &\cmd\risingdotseq&
$\circeq$                         &\cmd\circeq\\
$\fallingdotseq$                         &\cmd\fallingdotseq&
$\triangleq$                         &\cmd\triangleq\\
$\vartriangleleft$                         &\cmd\vartriangleleft&
$\vartriangleright$                         &\cmd\vartriangleright\\
$\trianglelefteq$                         &\cmd\trianglelefteq&
$\trianglerighteq$                         &\cmd\trianglerighteq\\
$\vDash$                         &\cmd\vDash&
$\Vdash$                         &\cmd\Vdash\\
$\Vvdash$                         &\cmd\Vvdash\\
$\smallsmile$                         &\cmd\smallsmile&
$\smallfrown$                         &\cmd\smallfrown\\
$\shortmid$                         &\cmd\shortmid&
$\shortparallel$                         &\cmd\shortparallel\\
$\bumpeq$                         &\cmd\bumpeq&
$\Bumpeq$                         &\cmd\Bumpeq\\
$\between$                         &\cmd\between&
$\pitchfork$                         &\cmd\pitchfork\lrstrut\\
\botrule
\end{tabular}
\end{table}

\begin{table}
\caption{Miscellaneous symbols available with \classoption{amssymb} selected.}
\def\xxx{4pt}
\begin{tabular}{c@{\hspace{\xxx}}lc@{\hspace{\xxx}}l}
\toprule
$\hbar $                         &\cmd\hbar&
$\hslash$                         &\cmd\hslash\\
$\backprime $                         &\cmd\backprime&
$\varnothing$                         &\cmd\varnothing\\
$\vartriangle$                         &\cmd\vartriangle&
$\blacktriangle$                         &\cmd\blacktriangle\\
$\triangledown$                         &\cmd\triangledown&
$\blacktriangledown$                         &\cmd\blacktriangledown\\
$\square$                         &\cmd\square&
$\blacksquare$                         &\cmd\blacksquare\\
$\lozenge$                         &\cmd\lozenge&
$\blacklozenge$                         &\cmd\blacklozenge\\
$\circledS$                         &\cmd\circledS&
$\bigstar$                         &\cmd\bigstar\\
$\angle $                         &\cmd\angle&
$\sphericalangle$                         &\cmd\sphericalangle\\
$\measuredangle$                         &\cmd\measuredangle\\
$\nexists$                         &\cmd\nexists&
$\complement$                         &\cmd\complement\\
$\mho$                         &\cmd\mho&
$\eth$                         &\cmd\eth\\
$\Finv$                         &\cmd\Finv&
$\Game$                         &\cmd\Game\\
$\diagup$                         &\cmd\diagup&
$\diagdown$                         &\cmd\diagdown\\
$\Bbbk$                         &\cmd\Bbbk\lrstrut\\
\botrule
\end{tabular}
\label{tab:a}
\end{table}


% Bin Op
\begin{table}
\caption{Binary operators available with \classoption{amssymb} selected.}
\def\xxx{4pt}
\begin{tabular}{c@{\hspace{\xxx}}lc@{\hspace{\xxx}}l}
\toprule
$\dotplus$                         &\cmd\dotplus&
$\ltimes$                         &\cmd\ltimes\\
$\smallsetminus$                         &\cmd\smallsetminus&
$\rtimes$                         &\cmd\rtimes\\
$\barwedge$                         &\cmd\barwedge&
$\curlywedge$                         &\cmd\curlywedge\\
$\veebar$                         &\cmd\veebar&
$\curlyvee$                         &\cmd\curlyvee\\
$\doublebarwedge$                         &\cmd\doublebarwedge\\
$\Cap$                         &\cmd\Cap, \cmd\doublecap &
$\leftthreetimes$                         &\cmd\leftthreetimes\\
$\Cup$                         &\cmd\Cup, \cmd\doublecup &
$\rightthreetimes$                         &\cmd\rightthreetimes\\
$\boxtimes$                         &\cmd\boxtimes&
$\circledast$                         &\cmd\circledast\\
$\boxminus$                         &\cmd\boxminus&
$\circleddash$                         &\cmd\circleddash\\
$\boxplus$                         &\cmd\boxplus&
$\centerdot$                         &\cmd\centerdot\\
$\boxdot$                         &\cmd\boxdot&
$\circledcirc$                         &\cmd\circledcirc\\
$\divideontimes$                         &\cmd\divideontimes&
$\intercal$                         &\cmd\intercal\lrstrut\\
\botrule
\end{tabular}
\end{table}


% other junk
\begin{table}
\caption{Other miscellaneous symbols available with \classoption{amssymb} selected.}
\def\xxx{4pt}
\begin{tabular}{c@{\hspace{\xxx}}lc@{\hspace{\xxx}}l}
\toprule
$\varpropto$                         &\cmd\varpropto&
$\backepsilon$                         &\cmd\backepsilon\\
$\blacktriangleleft$                         &\cmd\blacktriangleleft&
$\blacktriangleright$                         &\cmd\blacktriangleright\\
$\therefore$                         &\cmd\therefore&
$\because$                         &\cmd\because\lrstrut\\
\botrule
\end{tabular}
\end{table}

% negated relations
\begin{table}
\caption{Negated relations available with \classoption{amssymb} selected.}
\def\xxx{4pt}
\begin{tabular}{c@{\hspace{\xxx}}lc@{\hspace{\xxx}}l}
\toprule
$\nsim$                         &\cmd\nsim&
$\ncong$                         &\cmd\ncong\\
$\nless$                         &\cmd\nless&
$\ngtr$                         &\cmd\ngtr\\
$\nleq$                         &\cmd\nleq&
$\ngeq$                         &\cmd\ngeq\\
$\nleqslant$                         &\cmd\nleqslant&
$\ngeqslant$                         &\cmd\ngeqslant\\
$\nleqq$                         &\cmd\nleqq&
$\ngeqq$                         &\cmd\ngeqq\\
$\lneq$                         &\cmd\lneq&
$\gneq$                         &\cmd\gneq\\
$\lneqq$                         &\cmd\lneqq&
$\gneqq$                         &\cmd\gneqq\\
$\lvertneqq$                         &\cmd\lvertneqq&
$\gvertneqq$                         &\cmd\gvertneqq\\
$\lnsim$                         &\cmd\lnsim&
$\gnsim$                         &\cmd\gnsim\\
$\lnapprox$                         &\cmd\lnapprox&
$\gnapprox$                         &\cmd\gnapprox\\
$\nprec$                         &\cmd\nprec&
$\nsucc$                         &\cmd\nsucc\\
$\npreceq$                         &\cmd\npreceq&
$\nsucceq$                         &\cmd\nsucceq\\
$\precneqq$                         &\cmd\precneqq&
$\succneqq$                         &\cmd\succneqq\\
$\precnsim$                         &\cmd\precnsim&
$\succnsim$                         &\cmd\succnsim\\
$\precnapprox$                         &\cmd\precnapprox&
$\succnapprox$                         &\cmd\succnapprox\\
$\ntriangleleft$                         &\cmd\ntriangleleft&
$\ntriangleright$                         &\cmd\ntriangleright\\
$\ntrianglelefteq$                         &\cmd\ntrianglelefteq&
$\ntrianglerighteq$                         &\cmd\ntrianglerighteq\\
$\nshortmid$                         &\cmd\nshortmid&
$\nmid$                         &\cmd\nmid\\
$\nshortparallel$                         &\cmd\nshortparallel&
$\nparallel$                         &\cmd\nparallel\\
$\nvdash$                         &\cmd\nvdash&
$\nvDash$                         &\cmd\nvDash\\
$\nVdash$                         &\cmd\nVdash&
$\nVDash$                         &\cmd\nVDash\\
$\nsubseteq$                         &\cmd\nsubseteq&
$\nsupseteq$                         &\cmd\nsupseteq\\
$\nsubseteqq$                         &\cmd\nsubseteqq&
$\nsupseteqq$                         &\cmd\nsupseteqq\\
$\varsubsetneq$                         &\cmd\varsubsetneq&
$\varsupsetneq$                         &\cmd\varsupsetneq\\
$\subsetneq$                         &\cmd\subsetneq&
$\supsetneq$                         &\cmd\supsetneq\\
$\varsubsetneqq$                         &\cmd\varsubsetneqq&
$\varsupsetneqq$                         &\cmd\varsupsetneqq\\
$\subsetneqq$                         &\cmd\subsetneqq&
$\supsetneqq$                         &\cmd\supsetneqq\lrstrut\\
\botrule
\end{tabular}
\end{table}

\begin{table}
\caption{Yet more
miscellaneous symbols available with \classoption{amssymb} selected.}
\def\xxx{4pt}
\begin{tabular}{c@{\hspace{\xxx}}lc@{\hspace{\xxx}}l}
\toprule
$\dashrightarrow$                         &\cmd\dashrightarrow&
$\dashleftarrow$                         &\cmd\dashleftarrow\\
$\dasharrow$                         &\cmd\dasharrow\\
$\ulcorner$                         &\cmd\ulcorner&
$\urcorner$                         &\cmd\urcorner\\
$\llcorner$                         &\cmd\llcorner&
$\lrcorner$                         &\cmd\lrcorner\\
$\yen$                         &\cmd\yen&
$\checkmark$                         &\cmd\checkmark\\
$\circledR$                         &\cmd\circledR&
$\maltese$                         &\cmd\maltese\lrstrut\\
\botrule
\end{tabular}
\end{table}

% Negated arrows
\begin{table}
\caption{Extra negated arrows available with \classoption{amssymb} selected.}
\def\xxx{4pt}
\begin{tabular}{c@{\hspace{\xxx}}lc@{\hspace{\xxx}}l}
\toprule
$\nleftrightarrow$                         &\cmd\nleftrightarrow&
$\nLeftrightarrow$                         &\cmd\nLeftrightarrow\\
$\nleftarrow$                         &\cmd\nleftarrow&
$\nrightarrow$                         &\cmd\nrightarrow\\
$\nLeftarrow$                         &\cmd\nLeftarrow&
$\nRightarrow$                         &\cmd\nRightarrow\lrstrut\\
\botrule
\end{tabular}
\end{table}

%%  Arrows
\begin{table}
\caption{Extra arrows available with \classoption{amssymb} selected.}
\def\xxx{4pt}
\begin{tabular}{c@{\hspace{\xxx}}lc@{\hspace{\xxx}}l}
\toprule
$\leftrightarrows$                         &\cmd\leftrightarrows&
$\rightleftarrows$                         &\cmd\rightleftarrows\\
$\leftleftarrows$                         &\cmd\leftleftarrows&
$\rightrightarrows$                         &\cmd\rightrightarrows\\
$\leftrightharpoons$                         &\cmd\leftrightharpoons&
$\rightleftharpoons$                         &\cmd\rightleftharpoons\\
$\Lleftarrow$                         &\cmd\Lleftarrow&
$\Rrightarrow$                         &\cmd\Rrightarrow\\
$\twoheadleftarrow$                         &\cmd\twoheadleftarrow&
$\twoheadrightarrow$                         &\cmd\twoheadrightarrow\\
$\leftarrowtail$                         &\cmd\leftarrowtail&
$\rightarrowtail$                         &\cmd\rightarrowtail\\
$\looparrowleft$                         &\cmd\looparrowleft&
$\looparrowright$                         &\cmd\looparrowright\\
$\Lsh$                         &\cmd\Lsh&
$\Rsh$                         &\cmd\Rsh\\
$\upuparrows$                         &\cmd\upuparrows&
$\downdownarrows$                         &\cmd\downdownarrows\\
$\upharpoonleft$                         &\cmd\upharpoonleft&
$\upharpoonright$                         &\cmd\upharpoonright,\\
&&&\hskip1pc\cmd\restriction\\
$\downharpoonleft$                         &\cmd\downharpoonleft&
$\downharpoonright$                         &\cmd\downharpoonright\\
$\curvearrowleft$                         &\cmd\curvearrowleft&
$\curvearrowright$                         &\cmd\curvearrowright\\
$\circlearrowleft$                         &\cmd\circlearrowleft&
$\circlearrowright$                         &\cmd\circlearrowright\\
$\multimap$                         &\cmd\multimap&
$\rightsquigarrow$                         &\cmd\rightsquigarrow\\
$\leftrightsquigarrow$                         &\cmd\leftrightsquigarrow\lrstrut\\
\botrule
\end{tabular}
\end{table}

% other stuff

%\widehat
%\widetilde
%\eqsim

\subsection{\revtex{} notations}\label{sec:revtexnotations}

An openface numeral ``1'' is available; it does not change size in superscripts.
Here is an example: \verb+$+\cmd\openone\verb+$+ gives $\openone$.
%FIXME: review all fragile

Bold large bracketing is also available. The normal commands
\cmd\Biggl,\cmd\Bigl,$\ldots$, when used with an extra ``b'' on the
end of the command, come out bold:
\begin{verbatim}
\[
\Biggl(\biggl(\Bigl(\bigl(
(x)
\bigr)\Bigr)\biggr)\Biggr)
\]
\end{verbatim}
gives
\[
\Biggl(\biggl(\Bigl(\bigl(
(x)
\bigr)\Bigr)\biggr)\Biggr)
\]
while
\begin{verbatim}
\[
\Bigglb(\bigglb(\Biglb(\biglb(
(x)
\bigrb)\Bigrb)\biggrb)\Biggrb)
\]
\end{verbatim}
gives
%FIXME: size
\[
\Bigglb(\bigglb(\Biglb(\biglb(
(x)
\bigrb)\Bigrb)\biggrb)\Biggrb)
\]

The commands \cmd\lesssim, \cmd\gtrsim\ give the output
$\lesssim,\gtrsim$, even without the \classoption{amssymb} class option.
(The commands \cmd\alt, \cmd\agt, respectively, may also be used.)
These commands will be fragile if you are not using the \classoption{amssymb} option.

Some extra diacritics have been provided. They scale correctly in
superscripts. Some examples follow.
\verb+$+\cmd\tensor  \verb+{x}+\verb+$+ gives $\tensor  {x}$.
\verb+$+\cmd\overstar\verb+{x}+\verb+$+ gives $\overstar{x}$.
\verb+$+\cmd\overdots\verb+{x}+\verb+$+ gives $\overdots{x}$.
\verb+$+\cmd\overcirc\verb+{x}+\verb+$+ gives $\overcirc{x}$.
\verb+$+\cmd\loarrow \verb+{x}+\verb+$+ gives $\loarrow {x}$.
\verb+$+\cmd\roarrow \verb+{x}+\verb+$+ gives $\roarrow {x}$.
These commands all work correctly in superscripts.

%\cmd\slantfrac\marg{num}\marg{den} produces a slanted fraction in math mode:
%        $\slantfrac{1}{2}$.
% This command should not be used in files destined to be submitted to
% the APS (normal upright fractions will be produced).

 \cmd\corresponds\ produces the symbol $\corresponds$ math mode,
 \cmd\precsim\ produces $\precsim$ in math mode, and
 \cmd\succsim\ produces $\succsim$ in math mode. The AMS fonts will be used
 for these symbols if you have them, but are not necessary.
%FIXME: \corresponds didn't work

 \cmd\lambdabar\ produces ``lambda-bar'' in math mode: $\lambdabar$.
 

\section{Markup List}\label{sec:markup}

In the following pages are \hypertarget{Tsec:markup}{brief descriptions} of some necessary commands.
Those commands that are unique to \revtex{} are noted with (R).
Please consult the \LUG{} if you have further questions regarding \LaTeX{} commands.

If commands have arguments, they are so noted with \oarg{text}, or \marg{key}, as the case may be.
The commands are in order of their likely occurrence in a document.

\begin{description}
\makeatletter
\def\lbracett{\texttt{\expandafter\@gobble\string\{}}%
\def\rbracett{\texttt{\expandafter\@gobble\string\}}}%
\makeatother
\item[
\cmd\documentclass\oarg{options}\lbracett\classname{revtex4}\rbracett
]
  \oarg{options} is a comma-separated list of option names;
  see Sections~\ref{sec:task.opt} and~\ref{sec:class} for complete
  option lists and explanations.

  You usually select a journal substyle option, e.g., \classoption{aps}. 

  Use the \classoption{preprint} option to force formatted output to the 
  ``preprint'' style, suitable for copyediting.
  Otherwise, the chosen journal substyle selects a default.
              
  If output is in the preprint style, you can select
  the \classoption{tightenlines} class option to force 
  single line spacing.

  To number equations by section, use the \classoption{eqsecnum} option.
              
  Use the \classoption{showpacs} option to produce the PACS numbers.

\item[
\envb{document}
]
        Begins the body of the \revtex{} document.

\item[
\cmd\preprint\marg{text}
]
           When appearing within the front matter of a document, places
           \meta{text} at the top right corner of the first page in
           preprint style. Used for site-specific preprint
           numbers.
           (R)

\item[
\cmd\title\oarg{short title}\marg{title text}
]
           \meta{title text} is the title of the paper;
           \meta{short title} optionally specifies a title suitable for the page running head.
           The title should be broken with the \cmd\protect\cmd\\ command.

\item[
\cmd\author\marg{name}
]
           \meta{name} represents an author name.
%          Use \verb+\\+ to force linebreaks.

\item[
\cmd\affiliation\marg{text}
]
            \meta{text} represents an author's address (institution).  The
            address should be broken with \verb+\\+ if necessary.
            (R)

\item[
\cmd\date\marg{date}
]
            lets you specify a date to be formatted in the title block.

\item[
$\vcenter{\hbox{\envb{abstract}}%
          \hbox{\dots}
          \hbox{\enve{abstract}}}$
]
           Signals the beginning and end of the abstract, respectively.

\item[
\cmd\pacs\marg{pacs number}
]
          \meta{pacs number} represents valid PACS numbers.
          Invoke the \classoption{showpacs} option to have \meta{pacs number} printed.
          (R)

\item[
\cmd\maketitle\
]
              Prints the material contained in the \cmd\title\marg{title text},
              \cmd\author\marg{name}, \cmd\affiliation\marg{text} and
              \cmd\date\marg{date} commands.

%\item[
%\cmd\narrowtext
%]
%            For galley style, will set all text that follows into
%            a 3.4-in.\ column. Does not affect preprint output.
%            (R)

%\item[
%\cmd\mediumtext
%]
%             For galley style, will set figure captions and tables
%                        5.5-in.\ wide. Does not affect preprint output.
%             (R)

\item[
$\vcenter{\hbox{\envb{widetext}}
          \hbox{\dots}
          \hbox{\enve{widetext}}}$
]
  Sets all enclosed text on the full page width;
  only effective in a two-column layout.
  (R)

\item[
\cmd\section\marg{title text}
]
  \meta{title text} represents a primary heading.
  Fragile commands should be preceded by \cmd\protect.

\item[
\cmd\subsection\marg{title text}
]
         \meta{title text} represents a secondary heading.
         Fragile commands should be preceded by \cmd\protect.

\item[
\cmd\subsubsection\marg{title text}
]
      \meta{title text} represents a third-level heading.
      Fragile commands should be preceded by \cmd\protect.

\item[
\cmd\paragraph\marg{title text}
]
          \meta{title text} represents a fourth-level heading.
          Fragile commands should be preceded by \cmd\protect.

\item[
\cmd\cite\marg{key}
]
  Sets a reference or byline footnote citation.
  \meta{key} represents a list of reference keys used with
  \cmd\bibitem\marg{key}.
  Lists of consecutive numbers will be collapsed; e.g.,
  [1,2,3] will become [1--3].
  The style of citation in your output will depend on the chosen journal substyle.
  Fragile.

\item[
\cmd\textcite\marg{key}
]
  Sets a reference citation just like \cmd\cite\marg{key} does,
  except the citation is part of the text (as, e.g., the subject of the sentence).
  Fragile.
  (R)

\item[
\cmd\onlinecite\marg{key}
]
  Sets a reference citation just like \cmd\cite\marg{key} does,
  except that it places the citation on the baseline of the text
  even in styles where the citations are otherwise superscripts.
  Fragile.
  (R)

%\item[
%\cmd\case\marg{num}\marg{den}
%]
%      Sets textstyle (smaller) fractions in displayed equations.
%      \meta{num} is the numerator, \meta{den} is the denominator.
%%     An optional \verb+/+ may be added between \meta{num} and \meta{den}.
%      (R)

\item[
\cmd\openone\
]
      Produces an openface one ($\openone$).
%     Fragile.
      (R)

\item[
\cmd\precsim, \cmd\succsim\
]
       Produce the signs $\precsim$ and $\succsim$, respectively, in math mode.

\item[
\cmd\lesssim, \cmd\gtrsim
]
               Produce ``approximately less than'' and ``approximately
               greater than'' signs \hbox{($\lesssim,\gtrsim$)},
               respectively, in math mode.
%      Fragile.

\item[
\cmd\tensor\marg{math}
]
        \verb+$\tensor{x}$+ gives $\tensor{x}$.
        (R)

\item[
\cmd\loarrow\marg{math}
]
        \verb+$\loarrow{x}$+ gives $\loarrow{x}$.
        (R)

\item[
\cmd\roarrow\marg{math}
]
        \verb+$\roarrow{x}$+ gives $\roarrow{x}$.
        (R)

\item[
\cmd\overstar\marg{math}
]
     \verb+$\overstar{x}$+ gives $\overstar{x}$.
     (R)

\item[
\cmd\overcirc\marg{math}
]
     \verb+$\overcirc{x}$+ gives $\overcirc{x}$.
     (R)

\item[
\cmd\biglb\texttt{(}, etc.
]
        Commands to produce large bold bracketing.
        (R)

\item[
\cmd\corresponds\
]
       Produces ``corresponds'' sign in math mode: $\corresponds$.

%\item[
%\cmd\slantfrac\marg{num}\marg{den}
%]
%       Produces a slanted fraction in math mode:
%        $\slantfrac{1}{2}$. Should not be used for APS files.
%       (R)

\item[
\cmd\lambdabar\
]
        Produces ``lambda-bar'' in math mode: $\lambdabar$.
        (R)

%\item[
%\cmd\FL\
%]
%                     Sets the displayed equation that follows flush left
%                        with the margin. Only works in galley style.
%                        (R)

%\item[
%\cmd\FR
%]
%                     Sets the displayed equation that follows flush right.
%                        Only works in galley style.
%                        (R)

\item[
\cmd\[, \cmd\]
]
                   Signals beginning and end of unnumbered displayed
                        equation.

\item[
$\vcenter{\hbox{\envb{equation}}%
          \hbox{\dots}
          \hbox{\enve{equation}}}$
]
       Signals beginning and end of single-line displayed equation.

\item[
$\vcenter{\hbox{\envb{eqnarray}}%
          \hbox{\dots}
          \hbox{\enve{eqnarray}}}$
]
       Signals beginning and end of multi-line displayed equation.

\item[
\cmd\nonumber
]
               Suppresses the numbering of a single line in a
                        eqnarray environment.

\item[
\cmd\tag\marg{number}
]
  Provides an idiosyncratic number for
  a single line of an eqnarray. The number can be cross-referenced with
  \cmd\ref\marg{key} when \cmd\label\marg{key} is used right after \cmd\tag\marg{number}.
  Numbers set with \cmd\tag\marg{number} are completely independent of the
  automatic numbering.
  (R)

\item[
\envb{longtable}\dots\enve{longtable}
]
     Environment to produce tables that can break over pages.
     Requires the \classname{longtable} package;
     see Section~\ref{ref:table},
     and see \file{apssamp.tex} for an example.
     (R)

\item[
\cmd\label\marg{key}
]
    defines a tag. This command appears in
    displayed equations that need cross-referencing, all
    tables, and all figure captions. Also used following
    section headings that need cross-referencing.

\item[
\cmd\ref\marg{key}
]
    references a tag. Use this command in text
    wherever sections, numbered equations, tables, or figures are cited.

\item[
\env{acknowledgments} environment
]
        A container foracknowledgment section, complete with head.
        (R)

\item[
\cmd\appendix
]
      After using this command, all \cmd\section\marg{title text} commands
      will set \meta{title text} as an appendix heading.
      \cmd\section\verb+*+\marg{title text} will set \meta{title text}
      as an appendix heading without a letter (A, B, etc.)
      and should be used when there is only one appendix.

\item[
$\vcenter{\hbox{\envb{thebibliography}}%
          \hbox{\dots}
          \hbox{\enve{thebibliography}}}$
]
         Signals beginning and end of the list of references.
         (R)

\item[
\cmd\bibitem\oarg{symbol}\marg{key}
]
        Sets a reference in the reference section.  \meta{symbol}
                        represents an optional, author-specified reference
                        symbol. 
                        \meta{key} represents the reference tag.

\item[
\envb{figure}
]
          Begins the environment for a numbered figure.

\item[
\cmd\includegraphics\oarg{key-vals}\marg{filename}
]
          Import the given graphics file into the document.
          You must \cmd\usepackage\aarg{\classname{graphicx}}
          in order to be able to use the \cmd\includegraphics\
          command with the \texttt{key-vals} syntax.

\item[
\cmd\caption\marg{caption title}
]
        \meta{caption title} represents the text of the caption.
      Fragile commands must be preceded by \cmd\protect.

\item[
\cmd\label\marg{key}
]
        \meta{key} represents the figure caption tag.

\item[
\enve{figure}
]
        Ends the environment for the figure.

\item[
\envb{table}
]
           Signals the beginning of a table.

\item[
\cmd\squeezetable
]
           Used immediately after \envb{table}, shrinks tables that would not otherwise fit.
           (R)

\item[
\cmd\caption\marg{caption title}
]
      Sets the table caption. \meta{caption title} represents the
      text of the caption.
      Fragile commands must be preceded by \cmd\protect.

\item[
\envb{tabular}\marg{preamble}
]
     Signals the beginning of the tabular material.  \meta{preamble}
                        represents formatting commands for the columns.

\item[
\cmd\hline
]
                  Sets a horizontal rule, separating column headings
                        from data. \cmd\tableline\ may also be used.

\item[
\enve{tabular}
]
           Signals end of tabular material.

\item[
\enve{table}
]
             Signals the end of a table.

\item[
\enve{document}
]
          Ends the body of the \revtex{} document.
\end{description}


\end{document}

Author-prepared compuscripts should include the following parts
(preferably) in this
order:  title, author, address, abstract, suggested PACS numbers (use
current Physics and Astronomy Classification Scheme), main manuscript body,
references, figure captions, and tables.  The production staff will add
verified PACS numbers and manuscript receipt date. Specific instructions
pertaining to various parts of the compuscript are listed below as well as
a short annotated example of compuscript input.  Proper tagging
commands and coding are indicated where necessary.

In general, compuscripts should not contain any author-defined macros.
However, macros which are simple text substitutions can be
``expanded'' by the production software and thus are permissible.
More complicated macros will create problems when the file is converted
and must be avoided.
%See Sec.~\ref{sec:macros} for further information.

This guide assumes some familiarity with \LaTeX,
specifically the \classname{article} class, upon which \revtex{} is based.
The notations \meta{text}, \meta{key}, etc.\ are used throughout this guide to denote user-supplied arguments.
Commands will be shown in their full form; i.e., with their mandatory arguments.

Authors should also print the file \file{apssamp.tex} and study the input for
further instruction and detailed examples. This guide and the sample file
depend upon each other to cover all features of \revtex. The file
\file{template.aps} may be copied to another name to use as a basis for creating a
new \revtex{} document.

\subsection{Galley style and preprint style}

The \revtex{} macro package has been developed to accommodate the preprint
needs of the author as well as the production needs of the APS. If you use
\revtex{} to prepare a manuscript for electronic submission to the editorial office
of the American Physical Society
and participation in the compuscript program, please follow these
steps:
\begin{enumerate}
\item%(a)
Review the galley-format output (the default style), %FIXME
which mimics final journal output. Carefully check the content of
text, equations, references, captions (and figures), and tables, as if
reading the journal. This format is often preferred for distribution
of preliminary versions to colleagues.

\item%(b)
In the case of papers subject to
length restrictions, estimate the overall length by directly measuring the
journal text. Add in the space that will be occupied by any figures (at
their final reduction). On the journal page, the two-column area available
for text and figures is nominally 9.5 in.\ (24 cm) deep; nominally 6
text lines occupy 1 in. of vertical space. Alternatively, produce mock
page output by using the [twocolumn] class option, and perhaps
even include EPS figures.

Alternatively, you can use the \classoption{lengthcheck} to cause \revtex\ to 
format your paper in a format most closely appoximating the target journal.

\item%(c)
Switch to
preprint style (see below) and review the output, which is at a larger type
size. It is this format that is presented to reviewers and used by
editorial staff. Check that equations and tables remain satisfactory. A \revtex{}
facility for ``squeezing'' preprint-style tables is described later.

\item%(d)
Submit the preprint-style file electronically to the Editorial Offices.
\end{enumerate}

