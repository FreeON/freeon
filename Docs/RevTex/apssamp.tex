% ****** Start of file apssamp.tex ******
%
%   This file is part of the APS files in the REVTeX 4 distribution.
%   Version 4 of REVTeX, July 1999.
%
%   Copyright (c) 1999 The American Physical Society.
%
%   See the REVTeX 4 README file for restrictions and more information.
%
%\documentclass[preprint,eqsecnum,aps]{revtex4}
%\documentclass[eqsecnum,aps,draft]{revtex4}
 \documentclass[prb]{revtex4}

\usepackage{dcolumn}
\usepackage{amsmath}

% NOTICE: the following definitions are only for the sake of formatting
% the LaTeX commands incorporated into this particular document. 
% You will not need them for a typical Physical Review paper;
% you should *not* include them in your own documents.
\makeatletter
\def\btt#1{\texttt{\@backslashchar#1}}%
\DeclareRobustCommand\bblash{\btt{\@backslashchar}}%
\makeatother

%\nofiles

\begin{document}

\preprint{HEP/123-qed}

\title[Short Title]{Manuscript Title:\\
with Forced Linebreak}% Force line breaks with \\

\author{Ann  Author}
 \thanks{Also at Physics Department, XYZ University.}%Lines break automatically or can be forced with \\
\author{Second Author}%
 \email{Second.Author@institution.edu}
\affiliation{%
Authors' institution and/or address\\
This line break forced% with \\
}%

\author{Charlie Author}
 \homepage{http://www.Second.institution.edu/~Charlie.Author}
\address{
Second institution and/or address\\
This line break forced% with \\
}%

\date{\today}% It is always today \today.

\begin{abstract}
An article usually includes an abstract,
a concise summary of the work covered at length in the main body of the article.
It is used for secondary publications and for information retrieval purposes.
Valid PACS numbers may be entered using the \verb+\pacs{#1}+ command.
\end{abstract}

\pacs{Valid PACS appear here}%

\maketitle

\tableofcontents

\section{First-level heading:\protect\\ The line break was forced via \bblash}
\label{sec:level1}

Here is the first sentence in Sec.~\ref{sec:level1}, demonstrating
section cross-referencing. Note that this sample file was run
with the eqsecnum option selected.
Here is an open-face one: $\openone$.

\subsection{Second-level heading:\protect\\ The line break was forced via \bblash}
\label{sec:level2}

Here is the first sentence in Sec.~\ref{sec:level2}, demonstrating
section cross-referencing.
The environment \texttt{widetext}
will make the text the width of the full page, as on page~\pageref{wideeq}.
A blank input line tells \TeX\ that the paragraph has ended.

The width-changing commands only take effect in twocolumn style; the default
is preprint style, which gives output of a constant width.

This file may be run in both preprint and twocolumn styles. Preprint
format is used for submission purposes. Twocolumn format is used to mimic
final journal output. 

When commands are referred to in this example file, they are always shown
with their required arguments, using normal \TeX{} format. In this format,
\verb+#1+, \verb+#2+, etc. stand for required author-supplied arguments to commands.
For example, in
\verb+\section{#1}+ the \verb+#1+ stands for the title text of the author's section
heading, and in \verb+\title{#1}+ the \verb+#1+ stands for the
title text of the paper.

Reference citations in text use the commands
\verb+\cite{#1}+ or \verb+\onlinecite{#1}+.
\verb+#1+ may contain letters and numbers.
In the reference section of this paper
each reference is ``tagged'' by the \verb+\bibitem{#1}+ command.
\verb+#1+ should be \emph{identical} in both commands.

The form for citing in text is \verb+\cite{#1}+,
and the result is shown here \cite{smith82,jones78}.
When needing to explicitly use the word, ``Reference'', e.g., at the 
beginning of a sentence, use \verb+\onlinecite{#1}+
(Refs.~\onlinecite{smith82} and \onlinecite{jonessmith80}).
It is worth mentioning that REV\TeX{} ``collapses'' lists
of reference numbers where possible.
We now cite everyone together \cite{smith82,jones78,jonessmith80}, and once again
(Refs.~\onlinecite{smith82,jones78,jonessmith80}).

When the {\tt prb} option is used, the command \verb+\onlinecite{#1}+ will
put the reference citations on-line: this effect appears in the preceding paragraph.
Note that the location of citations must be adjusted to the reference style:
the superscript references in {\tt prb} style must appear after punctuation;
other styles must appear before any punctuation%
\footnote{%
 Authors are encouraged to use
 {\rm B{\sc ib}\TeX}\/ and prsty.bst
 to create their reference list in proper APS style.
 Instructions can be requested by e-mail (\protect\url{mailto:mis@aps.org}).
}.
This sample was written
for the regular (non-{\tt prb}) citation style, but invoking the
{\tt prb} option will show the results of  the command \verb+\onlinecite{#1}+
in the preceding paragraph.

\section{Displayed equations}

\subsection{Another second-level heading}

\subsubsection{Third-level heading:\protect\\ The line break was forced via \bblash}
\label{sec:level3}

Here is the first sentence in Sec.~\ref{sec:level3}, demonstrating section cross-referencing.
In \LaTeX\ there are many different ways to display equations, and a few preferred ways are noted below.
Displayed math will center by default.

\paragraph{Fourth-level heading: Single-line equations.}
Below we have numbered single-line equations; this is the most common type of equation in {\it Physical Review\/}:
\begin{eqnarray}
\chi_+(p)\alt{\bf [}2|{\bf p}|(|{\bf p}|+p_z){\bf ]}^{-1/2}
\left(
\begin{array}{c}
|{\bf p}|+p_z\\
px+ip_y
\end{array}\right)\;,
\\
\left\{\openone234567890abc123\alpha\beta\gamma\delta%
1234556\alpha\beta{1\sum^{a}_{b}\over A^2}\right\}\label{one}.
\end{eqnarray}
Note the open one in Eq.~(\ref{one}).

Not all numbered equations will fit
within a narrow column this way. The equation number will move down
automatically if it cannot fit on the same line with a one-line equation:
\begin{equation}
\left\{ab12345678abc123456abcdef\alpha\beta\gamma\delta%
1234556\alpha\beta{1\sum^{a}_{b}\over A^2}\right\}.
\end{equation}

When the \verb+\label{#1}+ command is used [cf. input
for Eq. (\ref{one})],
the equation can be referred to in text without your knowing the
equation number that \TeX\ will assign to it. Just use
\verb+\ref{#1}+, where \verb+#1+ is the same name that you used in the
\verb+\label{#1}+ command.

If you have a single-line equation that you don't want
numbered, you can use the \btt{[}, \btt{]} format:
\[g^+g^+ \rightarrow g^+g^+g^+g^+ \dots ~,~~q^+q^+\rightarrow
q^+g^+g^+ \dots ~. \]

\subsubsection{Multiline equations}

Multiline equations are obtained by using the
\btt{begin$\{$eqnarray$\}$}, \btt{end$\{$eqnarray$\}$} format.
Use the \btt{nonumber}
command at the end of each line where you do not want a number:
\begin{eqnarray}
{\cal M}=&&ig_Z^2(4E_1E_2)^{1/2}(l_i^2)^{-1}
\delta_{\sigma_1,-\sigma_2}
(g_{\sigma_2}^e)^2\chi_{-\sigma_2}(p_2)\nonumber\\
&&\times
[\epsilon_jl_i\epsilon_i]_{\sigma_1}\chi_{\sigma_1}(p_1),
\end{eqnarray}
\begin{eqnarray}
\sum \vert M^{\rm viol}_g \vert ^2&=&g^{2n-4}_S(Q^2)~N^{n-2}
        (N^2-1)\nonumber \\
 & &\times \left( \sum_{i<j}\right)
  \sum_{\rm perm}
 {1 \over S_{12}}
 {1 \over S_{12}}\sum_\tau c^f_\tau~.
\end{eqnarray}
{\bf Note:} do not use \verb+\label{#1}+ on a line of a multiline
equation if \verb+\nonumber+ is also used on that line. Incorrect
cross-referencing will result.

If you wish to set a multiline equation without \emph{any} equation numbers,
you can use the \verb+\begin{eqnarray*}+,
\verb+\end{eqnarray*}+ format:
\begin{eqnarray*}
\sum \vert M^{\rm viol}_g \vert ^2&=&g^{2n-4}_S(Q^2)~N^{n-2}
        (N^2-1)\\
 & &\times \left( \sum_{i<j}\right)
 \left( \sum_{\rm perm}
 {1 \over S_{12}S_{23}S_{n1}}\right)
 {1 \over S_{12}}~.
\end{eqnarray*}
To obtain numbers not normally produced by the automatic numbering,
use the \verb+\tag{#1}+ command, where \verb+#1+ is the desired
equation number. For example, to get an equation number of
(\ref{eq:mynum}),
\begin{equation}
g^+g^+ \rightarrow g^+g^+g^+g^+ \dots ~,~~q^+q^+\rightarrow
q^+g^+g^+ \dots ~. \tag{2.6$'$}\label{eq:mynum}
\end{equation}

{\it A few notes on} \verb=\eqnum{#1}=.
The \verb+\eqnum{#1}+ must come before the \verb+\label{#1}+, if any.
The numbering set with \verb+\eqnum{#1}+ is {\it transparent} to the
automatic numbering in REV\TeX{}; therefore,
you must know the number ahead of time, and {\it must\/} make
sure that the number set with \verb+\eqnum{#1}+ stays in step
with the automatic numbering.
\verb+\eqnum{#1}+ works with both single-line and multiline equations.
You could, if you wished, do all the numbering in a paper
manually with \verb+\eqnum{#1}+.

Enclosing single-line and multiline equations in
\verb+\begin{subequations}+ and \verb+\end{subequations}+ will produce
a set of equations that are ``numbered'' with letters, as shown
in Eqs.~(\ref{mlett:1}) and (\ref{mlett:2}) below:
\begin{subequations}
\label{generallabel}
\begin{equation}
\left\{abc123456abcdef\alpha\beta\gamma\delta%
1234556\alpha\beta{1\sum^{a}_{b}\over A^2}\right\},\label{mlett:1}
\end{equation}
\begin{eqnarray}
{\cal M}=&&ig_Z^2(4E_1E_2)^{1/2}(l_i^2)^{-1}
(g_{\sigma_2}^e)^2\chi_{-\sigma_2}(p_2)\nonumber\\
&&\times
[\epsilon_i]_{\sigma_1}\chi_{\sigma_1}(p_1).\label{mlett:2}
\end{eqnarray}
\end{subequations}
If you use a \verb+\label{#1}+ command right after the
\verb+\begin{subequations}+, then \verb+\ref{#1}+ can be used to reference
all the equations in a subequations environment. For example, the equations
in the preceding subequations environment were Eqs.~(\ref{generallabel}).

\subsubsection{Wide equations}
The equation that follows is set in a wide format, i.e., it
spans across the full page.  The wide format is reserved for
long equations that cannot be easily broken into four lines or less:
%\widetext
\begin{equation}
{\cal R}^{(\rm d)}=
 g_{\sigma_2}^e\left({[\Gamma^Z(3,21)]_{\sigma_1}\over
Q_{12}^2-M_W^2}+{[\Gamma^Z(13,2)]_{\sigma_1}\over Q_{13}^2-M_W^2}
\right) +x_WQ_e\left({[\Gamma^\gamma(3,21)]_{\sigma_1}\over
Q_{12}^2-M_W^2}+{[\Gamma^\gamma(13,2)]_{\sigma_1}
\over Q_{13}^2-M_W^2} \right)\;. \label{wideeq}
\end{equation}
This is typed so you can see that the output
is in wide format.  (Since
there is no input line between \btt{end$\{$equation$\}$}
and this paragraph,
there is no paragraph indent for this paragraph.) We also have
\begin{equation}
{\cal R}^{(f)}=-g^3\delta_{\sigma_1,\sigma_2}
\left( {g^e_{\sigma_2}D_Z\over\cos\theta_W}-Q_eD_\gamma
\cos\theta_W \right)
\left( {[\epsilon_3]_{\sigma_1}\over
Q^2_{12}-M^2_W/\xi}\epsilon_1\cdot\epsilon_2+
{[\epsilon_2]_{\sigma_1}\over
Q^2_{13}-M^2_W/\xi}\epsilon_1\cdot\epsilon_3 \right)\;.
\end{equation}

%\narrowtext
\section{Cross-referencing}
REV\TeX{} will automatically number sections, equations,
figure captions, and tables. In order to
reference them in text, use the \verb+\label{#1}+ and \verb+\ref{#1}+
commands.

The \verb+\label{#1}+ command appears following a section heading;
within an equation; or within a figure
or table environment, inside of or following the caption.
The \verb+\ref{#1}+ command appears in text
where citation is to occur.  We will refer to the first
figure (Fig.~\ref{autonum})%
\begin{figure}
\caption{A figure caption.  The figure captions are automatically
numbered.}
\label{autonum}
\end{figure}
here.
We can refer to the ``later figure'' also (Fig.~\ref{latefigure}).%
\begin{figure}
\caption{The ``late figure.'' This figure was inserted when the paper
was finished.  Since the figures are automatically numbered,
no renumbering in text was necessary. All that
needed to be done was to type the caption in the
proper place and cite the figure in text.\label{latefigure}}
\end{figure}

References to figures: Fig.~\ref{autonum}, Fig.~\ref{latefigure},
Fig.~\ref{reduced},%
\begin{figure}
\caption{A figure caption. Figures will be reduced to an appropriate
size by  the production
staff upon receipt.}
\label{reduced}
\end{figure}
and Fig.~\ref{fig4}.%
\begin{figure}
\caption{A figure caption.  The labels you give tables and figures
can be descriptive (as that of Fig.~\ref{autonum}, which has
a  \btt{label}$\{${\tt autonum}$\}$) or they can
reflect their numerical
order, as that of this figure
(\btt{label}$\{${\tt fig4}$\}$).\label{fig4}}
\end{figure}

References to tables:
Table \ref{table1},%
\begin{table}
\caption{This is a narrow table, which
occupies the width of a narrow
column. The table captions are automatically numbered.
This table shows left-aligned, centered, and right-aligned columns. It also
shows one of two possible methods of setting tablenotes (footnotes within
tables). In this table the tablenotes are numbered and set automatically.
All the author need do is use \btt{tablenote$\{$\#1$\}$} to set a tablenote
mark and its text.
\label{table1}}
\begin{tabular}{lcr}
One\footnote{Note a.}&Two\footnote{Note b.}&Three\\
\colrule
one&two&three\\
one&two&three\\
\end{tabular}
\end{table}
Table \ref{table2},%
\begin{table}
\caption{This is a table of medium width.
This table shows tablenotes where the author has numbered the tablenotes
by hand. In this approach, \btt{tablenotemark[\#1]} is used to produce the
tablenote mark. {\tt\#1} is a numeric value. Each time the same value
for {\tt\#1} is used,
the same mark is produced in the table. After the end of the tabular
environment,  \btt{tablenotemark[\#1]$\tt\{$\#2$\tt\}$}  commands are used:
 {\tt\#1} represents the same numbers used in \btt{tablenotemark[\#1]}
and {\tt\#2} represents the text of the tablenote. Using these two commands
will allow the author to number tablenotes by hand.
 Inspecting the input for this table should clarify any questions.
\label{table2}}
\begin{tabular}{cccccccc}
 &$r_c$ (\AA)&$r_0$ (\AA)&$\kappa r_0$&
 &$r_c$ (\AA) &$r_0$ (\AA)&$\kappa r_0$\\
\colrule
Cu& 0.800 & 14.10 & 2.550 &Sn\footnotemark[1]
& 0.680 & 1.870 & 3.700 \\
Ag& 0.990 & 15.90 & 2.710 &Pb\footnotemark[2]
& 0.450 & 1.930 & 3.760 \\
Au& 1.150 & 15.90 & 2.710 &Ca\footnotemark[3]
& 0.750 & 2.170 & 3.560 \\
Mg& 0.490 & 17.60 & 3.200 &Sr\footnotemark[4]
& 0.900 & 2.370 & 3.720 \\
Zn& 0.300 & 15.20 & 2.970 &Li\footnotemark[2]
& 0.380 & 1.730 & 2.830 \\
Cd& 0.530 & 17.10 & 3.160 &Na\footnotemark[5]
& 0.760 & 2.110 & 3.120 \\
Hg& 0.550 & 17.80 & 3.220 &K\footnotemark[5]
&  1.120 & 2.620 & 3.480 \\
Al& 0.230 & 15.80 & 3.240 &Rb\footnotemark[3]
& 1.330 & 2.800 & 3.590 \\
Ga& 0.310 & 16.70 & 3.330 &Cs\footnotemark[4]
& 1.420 & 3.030 & 3.740 \\
In& 0.460 & 18.40 & 3.500 &Ba\footnotemark[5]
& 0.960 & 2.460 & 3.780 \\
Tl& 0.480 & 18.90 & 3.550 & & & & \\
\end{tabular}
\footnotetext[1]{Here's the first, from Ref.~\onlinecite{smith82}.}
\footnotetext[2]{Here's the second.}
\footnotetext[3]{Here's the third.}
\footnotetext[4]{Here's the fourth.}
\footnotetext[5]{And etc.}
\end{table}
Table \ref{table3},%
\begin{table}
\caption{A wide table.  Two alternative occupations of special
positions
by KMnCL$_3$ ions in the two space groups $D_{4h}^1$ and $D_{4h}^1$.
  For a special value of the $x$ and $y$ parameters, a
set of special positions may
split into two sets of special positions of higher symmetry.}
\begin{tabular}{ccccc}
 &\multicolumn{2}{c}{$D_{4h}^1$}&\multicolumn{2}{c}{$D_{4h}^5$}\\
 Ion&1st alternative&2nd alternative&lst alternative
&2nd alternative\\ \colrule
 K&$(2e)+(2f)$&$(4i)$ &$(2c)+(2d)$&$(4f)$ \\
 Mn&$(2g)$\footnote{The $z$ parameter of these positions is $z\sim\frac{1}{4}$.}
 &$(a)+(b)+(c)+(d)$&$(4e)$&$(2a)+(2b)$\\
 Cl&$(a)+(b)+(c)+(d)$&$(2g)$\footnotemark[1]
 &$(4e)^{\rm a}$\\
 He&$(8r)^{\rm a}$&$(4j)^{\rm a}$&$(4g)^{\rm a}$\\
 Ag& &$(4k)^{\rm a}$& &$(4h)^{\rm a}$\\
 \end{tabular}
 \label{table3}
 \end{table}
Table \ref{table4},%
\begin{table}
\caption{Another wide table. Numbers in columns Three--Five have been aligned
by using the ``d'' column specifier. Non-numeric entries (those entries
without a ``.'') are centered in ``d'' columns.}
\begin{tabular}{ccddd}
One&Two&Three&Four&Five\\
\colrule
one&two&three&four&five\\
He&2& 2.77234 & 45672. & 0.69 \\
C\footnote{Some tables require footnotes.}
  &C\footnote{Some tables need more than one footnote.}
  & 12537.64 & 37.66345 & 86.37 \\
\end{tabular}
\label{table4}
\end{table}
Table \ref{latetable},%
\begin{table}
\caption{A ``late table.''  This table was added after most of the
paper had been completed. Since the tables are
automatically numbered, no renumbering in text was necessary. This
table
was added to show the use of the the ``d'' column and the
@ specifier for lining things up. The ``d'' column is useful for simpler
columns of numerical data, but it may be necessary to use multiple columns
and the @ specifier for more complex alignments.}
\begin{tabular}{dr@{}l@{${}\pm{}$}r@{}l}
%% NOTE, multicolumn NEEDED in next line
\multicolumn{1}{c}{Align by .}&
  \multicolumn{4}{c}{Multiple alignments}\\
\colrule
23.890\,12        &23&.890\,12&    0&.002\\
12\,323.          &123\,223&&    344& \\
0.834\,390\,12    &80&.80&        45&.3416\\
\end{tabular}
\label{latetable}
\end{table}
and Table \ref{table6}.%
\begin{table}
\caption{The Poisson ratio defined as the ratio of lateral
contraction to longitudinal expansion for uniaxial stress.
 Experimental values are given for comparison.}
\begin{tabular}{cddcdd}
  &\multicolumn{2}{c}{$\sigma$}& 
  &\multicolumn{2}{c}{$\sigma$}
\\
  &\multicolumn{1}{c}{Predicted}
  &\multicolumn{1}{c}{Observed$^{\rm a}$}
  &
  &\multicolumn{1}{c}{Predicted}
  &\multicolumn{1}{c}{Observed$^{\rm a}$}
\\
\colrule
Cu &  0.48 & 0.36 &Al& 0.47 & 0.33 \\
Ag &  0.48 & 0.37 &Tl& 0.47 & 0.35\\
Au &  0.48 & 0.36 &Sn& 0.46 & 0.33\\
Mg &  0.47 & 0.35 &Pb& 0.46 & \multicolumn{1}{c}{0.40--0.45}\\
Zn &  0.47 & 0.25 &Pb& 0.49 & 0.43 \\
   &       &      & K& 0.49 & 0.44 \\
\end{tabular}
\label{table6}
\end{table}

{\it Physical Review}
style requires that the initial citation of figures or tables
be in numerical order in text, so don't cite Fig.~\ref{reduced}
until you've cited Fig.~\ref{latefigure}.
See {\it Style and Notation
Guide}.

\begin{acknowledgments}
%
We wish to acknowledge the support of the author community in using
REV\TeX{}, offering suggestions and encouragement, testing new versions,
$\ldots$ .

If a section does not have a number (like the Acknowledgments
section), use the so-called ``star version'' of the command. That is,
insert a star between the command and its arguments:
\verb+\section*{#1}+, \verb+\subsection*{#1}+, etc.
For the Acknowledgments section you can also use the command
\verb+\acknowledgments+ to produce the heading.
%
\end{acknowledgments}

\appendix
\section{Appendixes}

To start the appendixes, you should use the \verb+\appendix+ command.
This signals that all following section commands refer to appendixes instead
of regular sections.
Therefore, the \verb+\appendix+ command should be used only once---to setup
the section commands to act as appendixes. Thereafter normal section commands
are used.
The heading for a section can be left empty. For example,
\begin{verbatim}
\appendix
\section{}
\end{verbatim}
will produce an appendix heading that says ``APPENDIX A'' and
\begin{verbatim}
\appendix
\section{Background}
\end{verbatim}
will produce an appendix heading that says ``APPENDIX A: BACKGROUND'' (note
that the colon is set automatically).

 If there is only
one appendix, then the letter ``A'' should not appear. This is suppressed by
using the star version of the section command (\verb+\section*{#1}+).

\section{A little more on appendixes}

Observe that this appendix was started by using
\begin{verbatim}
\section{A little more on appendixes}
\end{verbatim}

Note the equation number in an appendix:
\begin{equation} E=mc^2. \end{equation}

\subsection{A subsection in an appendix}
\label{app:subsec}

You can use a subsection or subsubsection in an
appendix. Note the numbering: we are now in Appendix \ref{app:subsec}.

Note the equation numbers in this appendix, produced
with the subequations environment:
\begin{subequations}
\begin{eqnarray}
E&=&mc, \label{appa}
\\
E&=&mc^2, \label{appb}
\\
E&\agt& mc^3. \label{appc}
\end{eqnarray}
\end{subequations}
They turn out to be Eqs.~(\ref{appa}), (\ref{appb}), and (\ref{appc}).

\begin{thebibliography}{}
\bibitem[Smith (1982)]{smith82}Smith, A., and B. Doe, 1982, J. Chem.\  Phys.\ {\bf 76}, 4056.
\bibitem[Jones (1978)]{jones78}Jones, C., 1978, J. Chem.\ Phys. {\bf 68}, 5298.
\bibitem[Jones and Smith (1980)]{jonessmith80}Jones, C., and A. Smith, 1980, J. Chem.\ Phys.\ {\bf 72,} 3416; {\bf 73,} 5168 (1980); {\bf72,} 4009 (1980).
\end{thebibliography}

\end{document}
%
% ****** End of file apssamp.tex ******
